\chapter{\label{c:et-lf-control}Control of the low frequency Einstein Telescope detector}

\begin{itemize}
  \item Build upon the PDH stuff laid out in Chapter 3: describe the evolution of the ET-LF interferometer in order to control it: the need for a Schnupp asymmetry (to provide the equivalent of a PDH-style controller), DARM offset, etc...
  
  \item Tobin's talk at G0900745 is really good for explaining DC readout's benefits
  
  \item Talk about the need for an OMC?
  
  \item Also see scribblings / emails from Ken etc. around the time of the Florence meeting where we discussed the low frequency sensing problem
  
  \item p255 of design study shows why we need 2 filter cavities for ET-LF and 1 for ET-HF
  
  \item See noise budget from before Florence
  
  \item DRFPMI uses both a Schnupp asymmetry (for RF sidebands) and a DARM offset (for carrier LO)
  
  \item DC readout: see Ward R L et al 2008 Class. Quantum Grav. 25 114030 and Hild S et al 2009 Class. Quantum Grav. 26 055012
  
  \item Show plots of sideband powers not entering the cavities
  
  \item Explicitly state that the tuned mode is just for lock acquisition, and that an optimised tuned mode would probably require different sideband frequencies and/or SRCL (since SRCL optimised for detuned is pretty rubbish for tuned mode in terms of sideband resonance)

  \item The Einstein telescope facility: ET-LF and ET-HF, the xylophone, etc.
    \begin{itemize}
      \item Unprecedented LF sensitivity: opens up universe
      \item Pushing warm technology to the limit with ET-HF
      \item Challenges: control at low frequencies with such detuning
      \item ET-LF layout
      \item Predicted sensitivity vs Optickle calculated sensitivity
      \item ISC stuff...
	\begin{itemize}
	  \item Consider only plane waves - justification: only want to control lengths for now. Angles are not considered a challenging aspect as nothing much has changed since aLIGO.
	  \item Optical response: with and without mechanical TFs - show the difference it makes to the response at low frequencies, and why it is necessary to turn them off in the case when you're computing a sensing matrix
	  \item From sensing matrix to control matrix (control loops, locking order, bandwidth, etc.)
	  \item Dynamic range of sensors and actuators in ET-LF
	  \item Problem with dynamic range, need local control or SPI or similar
	\end{itemize}
      \end{itemize}
      
  \item See p5 of VIR-0449D-11 (2011), Advanced Virgo steady state length sensing and control simulation, for description of why DARM offset is better than MICH
\end{itemize}
  
See Bryan's paper for sidebands on sidebands in the context of the Caltech prototype for aLIGO, to describe the use of beats between sidebands: https://iopscience.iop.org/article/10.1088/0264-9381/23/18/010/pdf

\section{The Einstein Telescope Facility}
In \checkme{2011} a group of scientists primarily based in Europe completed a design study to examine the requirements for a gravitational wave observatory that pushes the \MI{} topology to its limits, while keeping any newly built facility generic enough to allow for the implementation of new topologies as the state of the art evolves. In this study they laid out the expected improvements in technologies to mitigate technical noise sources currently challenging the sensitivity of the third generation of detectors, with a number of practical differences to existing facilities.

The proposed \ET{} facility composes six interferometers split between the three corners of a triangle with length \SI{10}{\kilo\meter} long. The idea is to exploit the geometry to implement \MI{} with \SI{10}{\kilo\meter} at the three vertices of the triangle to benefit from the colocation of multiple interferometers. The extended arm length over existing facilities provides less susceptibility to displacement noise sources by a factor of \cite{Dwyer2015, aligocosmic2016},
\begin{equation}
  \frac{h_{\SI{10}{\kilo\meter}}}{h_{\SI{4}{\kilo\meter}}} = \sqrt{\frac{\SI{10}{\kilo\meter}}{\SI{4}{\kilo\meter}}} \approx 1.58,
\end{equation}
translating into an increase in the observable volume by around $3.95$, assuming that displacement noise sources stay at the same magnitude.

Seismic noise limits the sensitivity of current generation detectors below \SI{10}{\hertz}, and there are astrophysical advantages to being able to achieve good sensitivity at these frequencies \cite{Sathyaprakash2012}. The majority of spinning neutron stars discovered via optical techniques have had orbital frequencies $f$ below \SI{10}{\hertz}. With enhanced low frequency sensitivity, \gls{ET} would have greater sensitivity to such ``chirp'' signals, and as these signals increase in frequency with time proportional to $\frac{1}{f^{8/5}}$ the ability to see them above the noise from an earlier time would assist in being able to characterise the source parameters. Signals at \SI{2}{\hertz} stay in \ETLF{}'s sensitive band for hours instead of milliseconds as witnessed for instance with GW150914 \cite{ET2011} \note{cite detection paper}. Having extra time to observe chirps also provides the possibility to track the signal evolution with corresponding changes in the signal recycling cavity tuning \cite{Simakov2014}.

\subsection{New facility}
Seismic noise in current detectors limits the sensitivity at low frequencies, and creates challenging control requirements due to the \gls{RMS} motion of the mirrors due to the motion of the ground. The interferometer beam tubes will be underground \note{how far?} to mitigate seismic noise coupling.

The suspension systems for the main test masses are based on those of the superattenuator in \AVIRGO{}. The \checkme{\SI{17}{\meter}} long pendulum stage \note{which stage?} pushes the longitudinal resonant frequency down to \checkme{\SI{0.1}{\hertz}}, providing good attenuation of seismic noise above \SI{2}{\hertz}.

\subsection{Xylophone configuration}
To provide maximum astrophysical reach the facility is intended to provide sensitivity across an unprecedented bandwidth, from around \SI{2}{\hertz} to \SI{10}{\kilo\hertz}\textemdash a bandwidth significantly larger than that of existing detectors. It was realised that the most technically feasible option would be to implement a number of colocated, moderate bandwidth detectors, each optimised to provide good sensitivity in a certain band \cite{Hild2010}, an idea first proposed for \ALIGO{} \cite{Conforto2004}. In the \emph{ET-C xylophone} configuration \cite{Hild2010}, a low power, cryogenic interferometer optimises sensitivity to reduce radiation pressure noise at the expense of shot noise, whilst a high power interferometer optimises high frequency sensitivity through the reduction of shot noise.

\subsection{Colocated detectors}

\subsubsection{Antenna pattern}
As the configuration is arranged into a triangle with the interferometers at each of the three vertices, the combined readout signals from each of the interferometers can be combined in such a way as to achieve optimal sensitivity in all directions.

\subsubsection{Null streams}
The use of three separate xylophone interferometers sensing the

The facility will have a triangle of tunnels, each \SI{10}{\kilo\meter} long, with each of the six interferometers inhabiting two of the three;

Each corner of the triangle will contain one low and one high frequency interferometer in a pair to allow for 


\subsection{ET-LF}
The low frequency detector consists of a \DRFPMI{} configuration with a detuned signal recycling cavity. This detuning allows for enhanced sensitivity at the signal recycling cavity pole, where the light-mirror dynamics create an optical spring that provides sensitivity below the \gls{SQL} at the spring frequency (see Section\,\ref{sec:signal-recycling}). The cavities will have \SI{18}{\kilo\watt} of light power, which is considerably lower than that of \ALIGO{} at design sensitivity (approximately \SI{800}{\kilo\watt}); cryogenic test masses to facilitate a reduction in thermal noise and frequency dependent squeezing to reduce quantum noise.

The sensitivity of \ETLF{} is shown in Figure\,XXX

\subsection{ET-HF}

The sensitivity of \ETHF{} is shown in Figure\,XXX (same plot as ET-LF)

\section{Control in the advanced detectors}

%CHECKME At the output port, light from one arm is transmitted through the beam splitter while the light from the other arm is reflected, and so a reflection phase convention applies (see Appendix\,\ref{a:reflection-phase}). The arm lengths are therefore offset by $\frac{\lambda}{4}$ with respect to one another.

% Section 1.3.1 of Gabriele Vajente's thesis covers this in more detail.

\subsection{\label{sec:et-lf-control-challenges}Control challenges with the \ET{}}
Both \ETLF{} and \ETHF{} will present new challenges to the control of large-scale \DRFPMI{}s.

\ETLF{} employs a detuned signal recycling cavity to increase the sensitivity of the interferometer at low frequencies, and this introduces control challenges. The ability for a detuned signal recycling to shift the most sensitive frequency has been demonstrated in \GEO{} between \SI{200}{\hertz} and \SI{1}{\kilo\hertz} \cite{Hild2006}. \ETLF{} plans to detune the signal recycling cavity to around \SI{25}{\hertz}, which has never before been demonstrated in suspended audio-band detectors. With tuned signal recycling, the \gls{RF} sidebands used for control of the differential arm cavity mode are present within the signal recycling cavity with equal amplitude. In detuned operation, the sidebands no longer have equal amplitude and so some of the interferometer's noise couplings that otherwise cancel at the output in the tuned case no longer cancel \cite{Hild2007}. The phase modulation of the control sidebands (created by an \gls{EOM} on the input path) also gets partially converted to amplitude modulation by the detuning, and this can lead to issues with the dynamic range of the photodetector to be used to sense the readout \cite{Grote2007}. Finally, the asymmetric sidebands create additional zero crossings in the error signal from the signal recycling cavity which for the tuned case do not appear. Often the control of a detuned signal recycling cavity involves locking to the zero crossing of a sideband instead of the carrier, but for low detunings the slope of the sideband error signal is very shallow and it is therefore difficult to distinguish small changes in cavity length from noise.

In \ETHF{}, the presence of \SI{3}{\mega\watt} light power in the arm cavities will lead to \emph{parametric instabilities}, where coupling between the mechanical modes of a mirror and the scattered light creates positive (unstable) mechanical feedback \cite{Braginsky2001}. A single mode has so far been observed in \ALIGO{} \cite{Evans2015} where it was mitigated by applying a small thermal lens to the cavity mirrors to move the radius of curvature away from the position of an unstable mode. Another option for \ALIGO{} is to use its \glspl{ESD} to apply active damping to the test masses, though for its design input power the number of predicted modes (more than 40) may make active damping extremely difficult. More research will need to be done to develop mitigation schemes for \ETHF{} where the cavity power will be higher still.

\section{Longitudinal control of an interferometer with multiple degrees of freedom}
A successful control scheme for an interferometer must satisfy a number of requirements. When the interferometer is in its uncontrolled state, the control scheme must be able to bring it to the operating point (\emph{lock acquisition}). Once at the operating point, it must be able to keep it there for a long period of time by controlling the impact of noise. Finally, the signal that represents the gravitational wave channel must have low enough noise, and therefore high enough sensitivity, to meet the design goals of the scheme.

The final lock acquisition scheme is inextricably linked to the technical environment in which the interferometer will operate, and so it is inappropriate to discuss this while \gls{ET}'s technical design is unknown. We will focus our efforts, therefore, on the second and third challenges. The low-noise readout in both \ETLF{} and \ETHF{} will be \gls{DC} readout, which is the standard technique for \GEOHF{} and the advanced detectors.

\subsection{Low-noise readout of the gravitational wave channel}
In Equation\,\ref{eq:mich-p-out} we see that a static field $\frac{P_{\text{in}}}{2}$ is present upon the photodetector, independent of the arm length change. In experiments where sensitivity can be sacrificed for simplicity, often it is practical to keep the interferometer at an operating point commonly referred to as ``half way up the fringe''. Here, the interferometer's mirrors are nominally positioned such that the output signal is oscillating about the midpoint between crest and trough (see Figure\,\ref{fig:optimal-operating-point}). As the gradient is steepest at this point, any small changes to the relative arm length of the Michelson interferometer result in a significant difference in power at the photodetector. This operating point, however, is not optimal in terms of \emph{sensitivity} to arm length fluctuations.

%% FIXME: change this plot's x-labels to use wavelength, to fit with the conclusion in the text.
\begin{figure}
  \centering
  \includegraphics[width=\columnwidth]{graphics/generated/from-python/70-optimal-operating-point.pdf}
  \caption[Fringe]{\label{fig:optimal-operating-point}Optimal operating point.}
\end{figure}

By inspecting Equation\,\ref{eq:mich-p-out}, it is clear to see that there must exist, in cases where there is a signal due to a difference in arm length, a static photodetector power independent of the arm length. This does not contribute any displacement information to the measurement, but does contribute shot noise:
\begin{equation}
  P_{\text{shot, out}} = \sqrt{2 h f_0 P_{\text{in}}},
\end{equation}
where $h$ is Planck's constant, $f_0$ is the light frequency and $P_{\text{in}} = A_{\text{in}}^2$, the power entering the interferometer at the beam splitter. The optimally sensitive operating point is therefore not simply one which maximises the signal gradient, but rather one which maximises the \gls{SNR}. The \gls{SNR} is:
\begin{equation}
  \text{SNR} = \frac{P_{\text{out}}}{P_{\text{shot, out}}} = \sqrt{\frac{P_{\text{in}}}{4 h f_0}} \left( 1 + \cos \left(k \Delta x \right) \right).
\end{equation}

The $\Delta x$ term in Equation\,\ref{eq:mich-p-out} is a combination of a static arm length \emph{detuning}\textemdash representing the arm length mismatch required to reach the desired operating point\textemdash and a differential gravitational wave signal $\Delta x_{\text{GW}}$. A suitable choice of $ x_{\text{tune}}$ can remove the majority of the static power present at the output. Setting the slope of the \gls{SNR} with respect to the tuning to zero,
\begin{equation}
  \frac{\Delta \text{SNR}}{\Delta x_{\text{GW}}} = -k \sqrt{\frac{P_{\text{in}}}{4 h f_0}} \sin \left(k \Delta x\right) = 0,
\end{equation}
we find that maximum \gls{SNR} is achieved for static tunings 
\begin{equation}
  \Delta x \text{ mod } \lambda = 0.
\end{equation}
This result shows that the optimal operating point in terms of \gls{SNR} is at the point where the light from the two arms interferes destructively. While any multiple of $\lambda$ will satisfy the \gls{SNR} condition as defined, in reality we have not considered laser noise coupling. The more matched the arm lengths are, the lower the laser noise couples to the output port. In reality there are also mismatches in the reflectivities of the mirrors in the arms: this creates an asymmetry called a \emph{contrast defect} which leads to additional shot noise at the output port.

\subsubsection{The dc readout technique}
The \gls{DC} readout technique is a form of homodyne readout (see Section\,\ref{sec:homodyne-readout}) that involves a compromise between the best sensitivity and the technical complexity. The operating point is kept close to the dark fringe to optimise shot noise, but a slight offset is introduced between the differential round trip phase of the arms in order to allow some of the carrier light to enter the readout port where it acts as a homodyne local oscillator to the signal sidebands. In practice, this detuning \textemdash of the order \SI{}{\pico\meter} at the arm cavities in \ALIGO{}\textemdash is sufficiently small to allow the required sensitivities to be reached. This provides improved sensitivity over heterodyne techniques \cite{Fricke2012}, and in squeezed interferometers such as \ETLF{} and \ETHF{} it avoids the need to inject squeezing at \gls{RF} sideband frequencies in addition to the carrier frequency.

\subsection{\label{sec:dofs-of-drfpmi}Degrees of freedom of the \DRFPMI{}}
As \ETLF{} and \ETHF{} will employ \gls{DC} readout, there is no local oscillator phase or alignment to control (see Section\,\ref{sec:homodyne-readout}). The degrees of freedom that must be controlled in order to reach the design sensitivity are the arm cavity differential and common modes, the length between the beam splitter and the \glspl{ITM} and the recycling cavity lengths. These are defined in the following sections.

\subsubsection{Differential arm cavity length}
\checkme{Should be held at the dark fringe by means of differential feedback to the end test masses...}

\subsubsection{Power recycling cavity length}
The power recycling cavity should be resonant for the input light in order to optimally recycle light reflected from the beam splitter back towards the laser. The power recycling cavity length can be defined in terms of the average distance between the power recycling mirror and the two \glspl{ITM}:

equation...

\subsubsection{Michelson length}
The length between the \glspl{ITM} and the beam splitter should be held constant to keep the amount of carrier and sideband power in the signal recycling cavity stable, which avoids the need for complicated time-varying control signals. This is kept constant by feeding back to the position of the beam splitter or with differential actuation upon to the \glspl{ITM}.

\note{Lots of useful info in Heinzel 2002}

\subsubsection{Signal recycling cavity length}
The signal recycling length, along with the signal recycling mirror transmissivity, determines the bandwidth of the signal extraction and therefore needs controlled in order to keep the signal to noise stable...

\subsection{Sensing matrices for an interferometer's degrees of freedom}
\note{ideally diagonal, in practice not - the job of the designer is to make them as decoupled as possible. feedback paths can be used to suppress cross-couplings.}

\section{Modelling effort for the \ET{} interferometers}
\note{See email sent to Andreas about outcomes of ASPERA. Basically, we had to model the interferometer controls and higher order modes, so we did it simultaneously with two tools. Discuss how this was achieved, via a shared parameter set, etc. Say that we started with ET-LF, but the framework is developed for both...}

\begin{figure}
  \centering
  \includegraphics[width=\columnwidth]{graphics/generated/from-svg/70-darm-schnupp-offsets.pdf}
  \caption[Differential arm and Schnupp offsets in a \DRFPMI{}]{\label{fig:darm-schnupp-offsets}\note{Make this a generic diagram of ETLF with the ports and readouts defined: output port in particular, as it is mentioned in text}}
\end{figure}

\section{Control of ET-HF}
\ETHF{} can to some extent be seen as a version of \ALIGO{} and \AVIRGO{} that pushes the state of the art of high power, heavy mirror, room temperature detectors, and so it may be possible to adapt much of the advance detectors' strategies for both longitudinal and angular control. Some aspects such as the use of LG33 modes and the presence of parametric instabilities, however, require extensive research to understand the implications they may have on control, and so much work remains to be done before a technical design for the \ET{} facility will be ready. This chapter will primarily discuss the longitudinal control of \ETLF{}, focusing in particular on the challenges arising from the large detuning of its signal recycling cavity.

\section{Control of ET-LF}
For the control of \ETLF{} we loosely follow the approach taken in \ALIGO{} \cite{Abbott2010} and \AVIRGO{} \cite{Vajente2008}, given that these represent the most sensitive \DRFPMI{}s built to date. The least constrained parameters are defined first, such as the control sideband frequencies, and parameters which depend on previously defined parameters are determined sequentially. Given the constraints from the design study, we begin with the transmissivity of the power recycling mirror and the frequency of the control sidebands, and then use these to define auxiliary lengths within the interferometer and finally calculate the arm cavity offset required for the \gls{DC} readout to work. With these parameters set, we can calculate the error signals present at various points within the interferometer and determine a control matrix to show the best ports to use for each degree of freedom.

\subsection{Optimal input coupling}
As discussed in Section\,\ref{sec:power-recycling}, placing a power recycling mirror in an \MI{} creates an additional cavity between the input light and the arms. The intention of the power recycling mirror is to minimise the light reflected back towards the laser, and in order to do this the cavity it creates should be \emph{impedance matched} (see, for example, Section\,5.1 of \cite{Freise2010}). The transmissivity of the mirror determines the impedance matching, and it can be determined using knowledge of the light lost within the interferometer. The design study defines the loss per optic to be \SI{35}{\ppm}, and the transmissivity of the \glspl{ETM} is \SI{6}{\ppm}; both contribute to the impedance matching. Figure\,\ref{fig:reflected-power-vs-prm-trans} shows the light power leaving the power recycling mirror heading back towards the laser. The minimum reflected power corresponds to a power recycling mirror transmissivity of \num{0.046}, which is comparable to that of the \glspl{ITM}.

\begin{figure}
  \centering
  \includegraphics[width=\columnwidth]{graphics/generated/from-python/70-reflected-power-vs-prm-transmissivity.pdf}
  \caption[Reflected power from \ETLF{} as a function of power recycling mirror transmissivity]{\label{fig:reflected-power-vs-prm-trans}Reflected power from \ETLF{} as a function of power recycling mirror transmissivity. For optimal coupling of the input laser light to the interferometer, the transmissivity of the power recycling mirror must be set to balance the input light with the total loss from the interferometer. For \ETLF{} with loss as per the design study, this transmissivity should be \num{0.046}.}
\end{figure}

For a future technical design study, additional allowance should be made for scattering loss and other mechanisms which will affect impedance matching. A slight overestimate of the 

\subsection{\label{sec:decoupled-sidebands}Decoupled control sidebands}
\ETLF{} will require at least two \gls{RF} sideband frequencies in order to control the common mode arm, power and signal recycling and Michelson cavity lengths. These frequencies must be chosen to be outside an integer multiple of the free spectral range (\gls{FSR}) of the arm cavities in order to prevent them from entering the arm cavities and getting resonantly enhanced by the high arm cavity finesse, and should ideally be \gls{RF} to benefit from the noise advantages presented in Section\,\ref{sec:pdh}. An upper limit of \SI{100}{\mega\hertz} is reasonable, as such high frequency demodulations are difficult to implement in photodetector electronics. Suitably decoupled sideband resonances provide error signals representing the cavities to be controlled. Often, demodulation at the sideband frequency itself contains useful signal content representing a cavity with minimal cross-couplings from other cavities; however, sometimes a signal with greater decoupling can be found by demodulating the light at some combination of sideband frequencies. In \ALIGO{}, for example, some of the control signals used in the operating mode involve demodulation at the sum or difference of the two sideband frequencies.

As described in Section\,\ref{sec:laser-noise}, laser frequency noise in a \MI{} is best suppressed when the arm lengths are well matched. In a \DRFPMI{}, however, the arm lengths must be slightly mismatched to allow for the control sidebands to enter the signal recycling cavity and therefore provide an error signal for the signal recycling cavity length. This is called the \emph{Schnupp asymmetry}, which comes from the Schnupp \emph{modulation} scheme developed for \MI{}s as an evolution upon the Pound-Drever-Hall technique in \FP{} cavities \note{cite something}.

For a Schnupp asymmetry $L_{\text{asy}}$, the two arms have length $L + \frac{L_{\text{asy}}}{2}$ and $L - \frac{L_{\text{asy}}}{2}$ where the nominal length is $L$. This mismatch is ideally as small as possible, to allow for the greatest suppression of laser noise, but large enough to allow for sufficient sideband power in the desired recycling cavities.

The choice can be made as to whether both sidebands should be resonant in both recycling cavities, or to have one cavity without a sideband resonance, and this is governed by the Schnupp asymmetry $L_{\text{asy}}$. The transmissivity of the power and signal recycling mirrors determine the Schnupp lengths corresponding to each case. A small offset of a few \SI{}{\centi\meter} between the Michelson arm lengths allows for the coupling of both sidebands into both recycling cavities, whilst a larger offset of a few tens of \SI{}{\centi\meter} prevents one sideband from entering the signal recycling cavity \cite{Vajente2008}. Both methods of control are possible, with the former being implemented in \KAGRA{} \note{reference for KAGRA?} and the latter in \ALIGO{} \cite{Abbott2010}. We choose to follow the \ALIGO{} approach so that we can optimise one sideband frequency ($f_1$) to be anti-resonant in the signal recycling cavity, to provide a maximum decoupling between the recycling cavity error signals to assist with the challenging control of the detuned signal recycling cavity.

\subsubsection{Sideband frequencies}

\paragraph{Control sideband resonance in the recycling cavities}
The \SI{310}{\meter} recycling cavity lengths are have \gls{FSR} \SI{483.5}{\kilo\hertz}. We therefore want to find a frequency $f_2$ which approximately satisfies:
\begin{equation}
  \label{eq:src-fsr}
  \begin{split}
    L_{\text{SRC}} &= A \frac{c_0}{2 f_2} \\
                   &\neq B \frac{c_0}{2 f_1},
  \end{split}
\end{equation}
where $A$ and $B$ are positive integers. We must also ensure that the sidebands are resonant within the power recycling cavity. As the arm cavities at the operating point reflect the light back towards the power recycling cavity, the resonant condition is instead a half-integer multiple of the power recycling cavity \gls{FSR}, i.e.:
\begin{equation}
  \label{eq:prc-fsr}
  L_{\text{PRC}} = \left(C + \frac{1}{2} \right) \frac{c_0}{2 f_1},
\end{equation}
for positive integer $C$.

\paragraph{Avoiding resonance of the control sidebands in the arm cavities}
In addition to optimising the control sideband frequencies using Equations \ref{eq:src-fsr} and \ref{eq:prc-fsr}, we must prevent the sidebands from entering into the arm cavities. The arm cavity \gls{FSR} is given by Equation\,\ref{eq:fsr} to be \SI{14.99}{\kilo\hertz} for the \SI{10}{\kilo\meter} arms. As an integer multiple of the arm cavity \gls{FSR} would allow optimal coupling of the sidebands into the arm cavities, one might assume that an odd half-integer multiple would be optimally anti-resonant; however, in this scenario the lower higher-order control sidebands, necessarily created by the phase modulation upon the \gls{EOM} (see Appendix\,\ref{eq:field-phase-bessel}), would become resonant, and so we choose to offset the sideband frequency slightly from the anti-resonant condition. We therefore stipulate two further requirements:
\begin{equation}
  \label{eq:arm-fsr}
  \begin{split}
    L_{\text{Arm}} &= \left(D + \frac{1}{2} \right) \frac{c_0}{2 f_1} \\
                   &= \left(D + \frac{1}{2} \right) \frac{c_0}{2 f_2},
  \end{split}
\end{equation}
for positive integer $D$.

\paragraph{Control sideband frequencies}
For \ETLF{} we chose the first sideband frequency $f_1$ to be \SI{11363101}{\hertz} which satisfies Equations \ref{eq:src-fsr}, \ref{eq:prc-fsr} and \ref{eq:arm-fsr}. We can then choose the second sideband to be an integer multiple of the first in order to allow both to be resonant within the power recycling cavity. We chose the second sideband frequency to be $f_2 = 5f_1 = \SI{56815505}{\hertz}$. We also assume a modulation depth of \num{0.1} for $f_1$ and $f_2$ to minimise higher order modulation sidebands while keeping a reasonable amount of light power in the sidebands.

\subsubsection{Schnupp asymmetry}
Figure\,\ref{fig:sideband-powers-vs-schnupp} shows the power of each sideband field in the recycling cavities of \ETLF{} given the Schnupp asymmetry and recycling cavity lengths, for both tuned and detuned signal recycling. The optimal Schnupp asymmetry is different for each case due to the asymmetric sidebands in detuned operation. Since the Schnupp asymmetry is a macroscopic length, it is not easily adjusted during operation, and so we choose an asymmetry, \SI{0.08}{\meter}, that gives good separation between the recycling cavities in the detuned state since it will be \ETLF{}'s primary mode of operation.

\begin{figure}
  \centering
  \begin{subfigure}{.49\textwidth}
    \includegraphics[width=\columnwidth]{graphics/generated/from-python/70-sideband-powers-vs-schnupp-tuned.pdf}
    \caption[Power of the control sidebands in the cavities of \ETLF{} with tuned signal recycling]{\label{fig:sideband-powers-vs-schnupp-tuned}Tuned signal recycling.}
  \end{subfigure}
  \hfill
  \begin{subfigure}{.49\textwidth}
    \includegraphics[width=\columnwidth]{graphics/generated/from-python/70-sideband-powers-vs-schnupp-detuned.pdf}
    \caption[Power of the control sidebands in the cavities of \ETLF{} with tuned signal recycling]{\label{fig:sideband-powers-vs-schnupp-detuned}Detuned signal recycling.}
  \end{subfigure}
  \caption[Power of the control sidebands in the cavities of \ETLF{}]{\label{fig:sideband-powers-vs-schnupp}Power of the control sidebands in the cavities of \ETLF{} during tuned and detuned operation. A small offset, called the \emph{Schnupp asymmetry}, is intentionally introduced to the Michelson length in order to allow the sidebands $f_1$ and $f_2$ to couple to the signal recycling cavity for the purposes of control. Here, $f_2$ is resonant but $f_1$ is not, and so there are discriminating signals for the power and signal recycling cavity. The optimal Schnupp asymmetry is one which provides the best decoupling of sideband error signals representing the lengths of the power and signal recycling cavities. The power is a good estimate for what the relative sensitivity of the sidebands in the recycling cavities will be.}
\end{figure}

\subsubsection{Optimisation of the signal recycling cavity length}

\paragraph{Detuned operation}
Note the discrepancy between two of the control sideband frequency constraints: the power and signal recycling cavities cannot both be simultaneously resonant and anti-resonant to $f_1$ and $f_2$ given that $f_2 = 5 f_1$. To resolve this discrepancy we can scan the length of the signal recycling cavity in order to find where the next \gls{FSR} is encountered for $f_2$, and this is shown in Figure\,\ref{fig:sideband-powers-srcl}. From Figure\,\ref{fig:sideband-powers-srcl-detuned}, we can see that changing the signal recycling cavity length from \SI{310}{\meter} to \SI{311.585}{\meter} results in the desired sideband resonance condition for the upper $f_2$ sideband. The power of $f_2$ in the power recycling cavity drops as the lower and upper sidebands get critically coupled into the signal recycling cavity. As the detuning in \ETLF{} is so large, the signal recycling cavity is not resonant for both the upper and lower $f_2$ sidebands and so the control of the power recycling cavity must be achieved before that of the signal recycling cavity to minimise the signal content in $f_2$ arising from the former.

In the signal recycling cavity, $f_2$ provides an error signal for the length that is \num{650} times larger than the equivalent for $f_1$. Meanwhile, $f_1$ provides an error signal for the power recycling cavity that is a factor of \num{13} larger than that of $f_2$. If this difference proves not to be high enough, it can be increased at the expense of the difference between the two in the signal recycling cavity by scaling the modulation depths of $f_1$ and $f_2$.

\paragraph{Tuned operation}
The equivalent powers in the tuned configuration is shown in Figure\,\ref{fig:sideband-powers-srcl-tuned}. Here, we see that the chosen signal recycling cavity length is not optimal, with a length of \SI{312.8}{\meter} seeming more appropriate. This length is far above the limit of the actuators that one can reasonably expect on the suspension platforms, and so this cannot easily be changed after commissioning; we must investigate another method to separate the error signals in the tuned configuration. Figure\,\ref{fig:sideband-powers-vs-f2-tuned} shows the effect that changing the frequency of $f_2$ has on the powers within the cavities, and we see that moving $f_2$ from \SI{56.815}{\mega\hertz} to \SI{57.008}{\mega\hertz} obtains a more appropriate decoupling. Note that the power in $f_1$ is still larger than the power in $f_2$ in the tuned case, even at this new frequency. As described in Section\,\ref{sec:decoupled-sidebands}, we can investigate the use of beats between sidebands to retrieve appropriate error signals and so this is not necessarily a problem.

\begin{figure}
  \centering
  \begin{subfigure}{.49\textwidth}
    \includegraphics[width=\columnwidth]{graphics/generated/from-python/70-sideband-powers-vs-srcl-detuned.pdf}
    \caption[Power of the control sidebands in the signal recycling cavity as a function of length of \ETLF{} in the detuned configuration]{\label{fig:sideband-powers-srcl-detuned}Sideband powers for detuned signal recycling.}
  \end{subfigure}
  \hfill
  \begin{subfigure}{.49\textwidth}
    \includegraphics[width=\columnwidth]{graphics/generated/from-python/70-sideband-powers-vs-srcl-tuned.pdf}
    \caption[Power of the control sidebands in the signal recycling cavity as a function of length of \ETLF{} in the tuned configuration]{\label{fig:sideband-powers-srcl-tuned}Sideband powers for tuned signal recycling.}
  \end{subfigure}
  \caption[Power of the control sidebands in the signal recycling cavity as a function of length of \ETLF{}]{\label{fig:sideband-powers-srcl}Blah.}
\end{figure}

\begin{figure}
  \centering
  \includegraphics[width=\columnwidth]{graphics/generated/from-python/70-sideband-powers-vs-second-sideband-tuned.pdf}
  \caption[Power of the control sidebands in the signal recycling cavity as a function of the second sideband frequency of \ETLF{} in the tuned configuration]{\label{fig:sideband-powers-vs-f2-tuned}Power of the control sidebands in the signal recycling cavity as a function of the second sideband frequency of \ETLF{} in the tuned configuration. This shows only the power in the lower sidebands, because for the tuned configuration the upper and lower sideband signals are identical.}
\end{figure}

\subsection{Dark fringe offset}
As described in Section\,\ref{sec:homodyne-readout}, \gls{DC} readout at the output port of a \DRFPMI{} requires carrier light to be present to act as a local oscillator (phase reference) for the signal sidebands. In a perfectly controlled interferometer there is no classical light at the output port and so this offset must be introduced by differentially detuning the interferometer's arms by a small amount to create the appropriate dark fringe offset. In practice, asymmetries which create differential detuning are already present in the arms, for example arising from mismatched arm cavity finesse or asymmetric beam splitter reflectivity.

For our simulations, we do not assume any asymmetries and so we can intentionally introduce a dark fringe offset with an offset in the \gls{MICH} or \gls{DARM} degrees of freedom. While a \gls{DARM} offset has the disadvantage that it involves the creation of an optical spring due to the high light power in the arms, it has favourable noise couplings compared to a \gls{MICH} offset \cite{Vajente2011}.

Figure\,\ref{fig:sideband-powers-vs-darm-offset-detuned} shows the power at the output port and in the arm cavities as a function of \gls{DARM} offset in the detuned configuration. For an offset of \SI{12}{\pico\meter}, the power at the output can be set to around \SI{25}{\milli\watt} with a difference of about 3\% in the power in the arms. Standard photodetectors used in \ALIGO{} and \AVIRGO{} can usually handle around \SI{100}{\milli\watt} but in \ETLF{} the \gls{DARM} offset required to reach this figure would reduce the power in the Y arm cavity significantly, affecting sensitivity.

Figure\,\ref{fig:sideband-powers-vs-darm-offset-tuned} shows the power at the output port for the tuned configuration, where a \gls{DARM} offset of around \SI{3}{\pico\meter} produces approximately the same power as in the detuned case. The tuned case requires a smaller offset because of the larger effective transmissivity of the signal recycling cavity. \note{Fix this to make it reach the same power as detuned}

\begin{figure}
  \centering
  \begin{subfigure}{.49\textwidth}
    \includegraphics[width=\columnwidth]{graphics/generated/from-python/70-sideband-powers-vs-darm-offset-detuned.pdf}
    \caption[Carrier power at the output port of \ETLF{} in detuned configuration as a function of differential arm cavity offset]{\label{fig:sideband-powers-vs-darm-offset-detuned}Detuned.}
  \end{subfigure}
  \hfill
  \begin{subfigure}{.49\textwidth}
    \includegraphics[width=\columnwidth]{graphics/generated/from-python/70-sideband-powers-vs-darm-offset-tuned.pdf}
    \caption[Carrier power at the output port of \ETLF{} in tuned configuration as a function of differential arm cavity offset]{\label{fig:sideband-powers-vs-darm-offset-tuned}blah.}
  \end{subfigure}
  \caption[Carrier power at the output port of \ETLF{} as a function of differential arm cavity offset]{\label{fig:sideband-powers-vs-darm-offset}Tuned.}
\end{figure}

\note{This creates an optical spring: see ``Optomechanical issues in the gravitational wave detector Advanced VIRGO'' - S.D. talked about this - it occurs at 0.2 Hz or so?}

\section{Conceptual control scheme}

To estimate a control scheme for \ETLF{}, we must decide upon the ports at which we will sense signals from the interferometer and calculate the response of its degrees of freedom to these sensors.

\subsection{Readout ports}
\note{Figure xx} shows some available readout ports for \ETLF{} where the sidebands and carrier may be measured for the purposes of control:
\begin{itemize}
  \item \textbf{REFL} (\emph{reflected}) senses the light reflected from the interferometer back towards the input laser.
  \item \textbf{POP} (\emph{pick off \gls{PRCL}}) senses the light in the power recycling cavity. Although the diagram shows a beam splitter, in reality this will probably be obtained via a secondary mirror reflection (such as the anti-reflective coating on the power recycling mirror).
  \item \textbf{AS} (\emph{asymmetric}) senses the light at the output port of the \DRFPMI{}.
\end{itemize}

For the purposes of the control simulations, each readout port contains photodetectors demodulating the signals at \gls{DC}, $\pm f_1$ (\SI{\pm11}{\mega\hertz}), $\pm f_2$ (\SI{\pm57}{\mega\hertz}), $-f_1 - f_2$ (\SI{-68}{\mega\hertz}), $-f_1 + f_2$ (\SI{-45}{\mega\hertz}), $f_2 - f_1$ (\SI{45}{\mega\hertz}) and $f_2 + f_1$ (\SI{68}{\mega\hertz}). There are therefore \num{9} classical light fields propagated through the interferometer. The power in each field within each relevant space or cavity of the interferometer is shown in Tables\,\ref{tab:et-lf-detuned-dc-powers} and \ref{tab:et-lf-tuned-dc-powers} for detuned and tuned operation, respectively.

\begin{table}
  \centering
  \resizebox{\textwidth}{!}{%
    \begin{tabular}{l|ccccccccc|c}
      & \textbf{\SI{-68}{\mega\hertz}} & \textbf{\SI{-57}{\mega\hertz}} & \textbf{\SI{-45}{\mega\hertz}} & \textbf{\SI{-11}{\mega\hertz}} & \textbf{Carrier} & \textbf{\SI{11}{\mega\hertz}} & \textbf{\SI{45}{\mega\hertz}} & \textbf{\SI{57}{\mega\hertz}} & \textbf{\SI{68}{\mega\hertz}} & \textbf{Total} \\
      \hline
      Input & 0 & 0 & 0 & 0 & \SI{3}{\watt} & 0 & 0 & 0 & 0 & \\
      After modulators & \SI{19}{\micro\watt} & \SI{7.4}{\milli\watt} & \SI{19}{\micro\watt} & \SI{7.4}{\milli\watt} & \SI{3}{\watt} & \SI{7.4}{\milli\watt} & \SI{19}{\micro\watt} & \SI{7.4}{\milli\watt} & \SI{19}{\micro\watt} & \\
      Power recycling cavity & \SI{1}{\milli\watt} & \SI{521}{\milli\watt} & \SI{1}{\milli\watt} & \SI{410}{\milli\watt} & \SI{65}{\watt} & \SI{407}{\milli\watt} & \SI{1}{\milli\watt} & \SI{34}{\milli\watt} & \SI{1}{\milli\watt} & \\
      Michelson cavity & \SI{1}{\milli\watt} & \SI{228}{\milli\watt} & \SI{1}{\milli\watt} & \SI{208}{\milli\watt} & \SI{33}{\watt} & \SI{204}{\milli\watt} & \SI{1}{\milli\watt} & \SI{27}{\milli\watt} & \SI{1}{\milli\watt} & \\
      Arm cavity X & \SI{1}{\nano\watt} & \SI{557}{\micro\watt} & \SI{1}{\nano\watt} & \SI{9}{\milli\watt} & \SI{18}{\kilo\watt} & \SI{8.8}{\milli\watt} & \SI{1}{\nano\watt} & \SI{65}{\micro\watt} & \SI{1}{\nano\watt} & \\
      Arm cavity Y & \SI{1}{\nano\watt} & \SI{772}{\micro\watt} & \SI{1}{\nano\watt} & \SI{8.7}{\milli\watt} & \SI{18}{\kilo\watt} & \SI{8.8}{\milli\watt} & \SI{1}{\nano\watt} & \SI{68}{\micro\watt} & \SI{1}{\nano\watt} & \\
      Signal recycling cavity & \SI{1}{\nano\watt} & \SI{3.8}{\milli\watt} & \SI{1}{\nano\watt} & \SI{69}{\micro\watt} & \SI{15}{\milli\watt} & \SI{41}{\micro\watt} & \SI{1}{\nano\watt} & \SI{26}{\milli\watt} & \SI{1}{\nano\watt} & \\
      Output & \SI{1}{\nano\watt} & \SI{756}{\micro\watt} & \SI{1}{\nano\watt} & \SI{14}{\micro\watt} & \SI{3}{\milli\watt} & \SI{8.3}{\micro\watt} & \SI{1}{\nano\watt} & \SI{5.1}{\milli\watt} & \SI{1}{\nano\watt} & \\
    \end{tabular}
  }
  \caption{\label{tab:et-lf-detuned-dc-powers}Powers in various parts of \ETLF{} in the detuned configuration. The input light is passed through \glspl{EOM} which impart control sidebands at two pairs of frequencies offset from the carrier. As the arm cavity \gls{FSR} is \checkme{higher} than that of the control sideband frequencies, the sideband light power in the arms is vastly smaller than the carrier power.}
\end{table}

\begin{table}
  \centering
  \resizebox{\textwidth}{!}{%
    \begin{tabular}{l|ccccccccc|c}
      & \textbf{\SI{-68}{\mega\hertz}} & \textbf{\SI{-57}{\mega\hertz}} & \textbf{\SI{-46}{\mega\hertz}} & \textbf{\SI{-11}{\mega\hertz}} & \textbf{Carrier} & \textbf{\SI{11}{\mega\hertz}} & \textbf{\SI{46}{\mega\hertz}} & \textbf{\SI{57}{\mega\hertz}} & \textbf{\SI{68}{\mega\hertz}} & \textbf{Total} \\
      \hline
      Input & 0 & 0 & 0 & 0 & \SI{3}{\watt} & 0 & 0 & 0 & 0 & \SI{3}{\watt} \\
      After modulators & \SI{19}{\micro\watt} & \SI{7.4}{\milli\watt} & \SI{19}{\micro\watt} & \SI{7.4}{\milli\watt} & \SI{3}{\watt} & \SI{7.4}{\milli\watt} & \SI{19}{\micro\watt} & \SI{7.4}{\milli\watt} & \SI{19}{\micro\watt} & \SI{3}{\watt} \\
      Power recycling cavity & \SI{2.2}{\micro\watt} & \SI{105}{\micro\watt} & \SI{2.1}{\micro\watt} & \SI{398}{\milli\watt} & \SI{64}{\watt} & \SI{398}{\milli\watt} & \SI{2.1}{\micro\watt} & \SI{105}{\micro\watt} & \SI{2.2}{\micro\watt} & \\
      Michelson cavity & \SI{1}{\micro\watt} & \SI{85}{\micro\watt} & \SI{1}{\micro\watt} & \SI{190}{\milli\watt} & \SI{32}{\watt} & \SI{190}{\milli\watt} & \SI{1}{\micro\watt} & \SI{85}{\micro\watt} & \SI{1}{\micro\watt} & \\
      Arm cavity X & \SI{10}{\nano\watt} & \SI{1}{\micro\watt} & \SI{10}{\nano\watt} & \SI{8.2}{\milli\watt} & \SI{18}{\kilo\watt} & \SI{8.2}{\milli\watt} & \SI{10}{\nano\watt} & \SI{1}{\micro\watt} & \SI{10}{\nano\watt} & \\
      Arm cavity Y & \SI{10}{\nano\watt} & \SI{1}{\micro\watt} & \SI{10}{\nano\watt} & \SI{9.1}{\milli\watt} & \SI{18}{\kilo\watt} & \SI{9.1}{\milli\watt} & \SI{10}{\nano\watt} & \SI{1}{\micro\watt} & \SI{10}{\nano\watt} & \\
      Signal recycling cavity & \SI{10}{\nano\watt} & \SI{80}{\micro\watt} & \SI{10}{\nano\watt} & \SI{304}{\micro\watt} & \SI{192}{\milli\watt} & \SI{304}{\micro\watt} & \SI{10}{\nano\watt} & \SI{80}{\micro\watt} & \SI{10}{\nano\watt} & \\
      Output & \SI{1}{\nano\watt} & \SI{16}{\micro\watt} & \SI{1}{\nano\watt} & \SI{61}{\micro\watt} & \SI{38}{\milli\watt} & \SI{61}{\micro\watt} & \SI{1}{\nano\watt} & \SI{16}{\micro\watt} & \SI{1}{\nano\watt} & \\
    \end{tabular}
  }
  \caption{\label{tab:et-lf-tuned-dc-powers}Powers in various parts of \ETLF{} in the tuned configuration. Note that the control sideband frequencies involving $f_2$ are different from those in the detuned case. In this case, the power in the upper and lower sidebands is identical everywhere because they have there are no detuned cavities to separate their resonant conditions (see Section\,\ref{sec:et-lf-control-challenges}). The light incident upon the interferometer from the \glspl{EOM} has the same power, but the power in the rest of the interferometer changes due to the different coupling between the interferometer's cavities.}
\end{table}

\subsection{Driving coefficients}
In order to calculate the response of each degree of freedom as defined in Section\,\ref{sec:dofs-of-drfpmi}, we must define the mirrors that we will actuate upon to create this motion. \gls{CARM} and \gls{DARM} are defined in terms of the motion of the \glspl{ETM}, with the difference between the two degrees of freedom being whether the motion of the two \gls{ETM} is in-phase or out-of-phase, respectively. The \gls{PRCL} and \gls{SRCL} degrees of freedom are defined as motion of the power and signal recycling mirrors, respectively. \gls{MICH} is a little more tricky, as simply moving the \glspl{ITM} results in a signal arising from the arms. We can either move both the \gls{ITM} and \gls{ETM} in each arm together, and differentially between the two cavities, or we can move the beam splitter and the recycling mirrors differentially. We choose the latter option. In this case, the beam splitter must be moved by a smaller amount than the recycling mirrors to account for its non-zero angle of incidence. The driving coefficients are shown in Table\,\ref{tab:et-lf-driving-coefficients}.

\begin{table}
  \centering
  \begin{tabular}{lccccccc}
    \textbf{Degree of freedom} & \textbf{\gls{ITM} X} & \textbf{\gls{ETM} X} & \textbf{\gls{ITM} Y} & \textbf{\gls{ETM} Y} & \textbf{\gls{BS}} & \textbf{\gls{PRM}} & \textbf{\gls{SRM}} \\
    \gls{CARM} & \num{0} & \num{0.5} & \num{0} & \num{0.5} & \num{0} & \num{0} & \num{0} \\
    \gls{DARM} & \num{0} & \num{0.5} & \num{0} & \num{-0.5} & \num{0} & \num{0} & \num{0} \\
    \gls{MICH} & \num{0} & \num{0} & \num{0} & \num{0} & \num{0.15} & \num{-0.5} & \num{0.5} \\
    \gls{PRCL} & \num{0} & \num{0} & \num{0} & \num{0} & \num{0} & \num{1} & \num{0} \\
    \gls{SRCL} & \num{0} & \num{0} & \num{0} & \num{0} & \num{0} & \num{0} & \num{1} \\
  \end{tabular}
  \caption{\label{tab:et-lf-driving-coefficients}blah...}
\end{table}

\subsection{Control matrices}

\subsubsection{Detuned configuration}
\note{Describe the wedge product}

\subsubsection{Tuned configuration}

\section{Control noise}
\note{Highlight the dynamic range problem with photodetectors sensing the low frequency motion (poster from Florence), and introduce the basic control loop developed with SimulinkNb - discuss the noises included, the assumptions made, etc. Finish with the list from the poster of what has to be done: further modelling of suspension SPIs, or better sensors, or both, etc...}

\subsection{Seismic noise in ET-LF}
\note{Take transfer function through S.A. from lowest noise site measured...}
\note{Assume optimal worst contribution of noise to ETMs...}

\section{Updated parameters for ET-LF}
\note{Copy table from design study, but with updated values from simulations}
\note{Italicise the parameters unchanged from the design study?}

\section{Future work}

\subsection{Switching between tuned to detuned operation}
Transition from tuned to detuned signal recycling operating points and vice versa involves a technique which can maintain control of the interferometer as it transitions between two desired set points. When dual recycling was first demonstrated in suspended optics in the Garching prototype, it involved a varying frequency offset applied to the \gls{RF} modulation sidebands as the tuning was changed \cite{Freise2000}. In \GEO{} the control technique involved a complicated sequence of events \cite{Grote2004} including an uncontrolled ``jump'' between two operating points \cite{Hild2007} to reached tuned mode. In \ETLF{} it is expected that the signal recycling cavity finesse will be too high to allow for a previously demonstrated transition scheme. Instead, investigations are underway to model the impact that single control sidebands added to the input light or subcarriers added to the squeezing injection port have on the controllability of the signal recycling cavity at arbitrary detunings. Another possibility is to adapt the \emph{arm length stabilisation} system developed for the lock acquisition of \ALIGO{} \cite{Mullavey2012, Staley2014}, whereby a second carrier at a different wavelength is used to lock cavities. This takes advantage of the lower finesse of the second carrier's wavelength in the cavities, allowing for a wider locking range. The cavities are first pre-stabilised using this second carrier before the main carrier is brought to resonance.

The transmissivity of the signal recycling mirror in \ETLF{} allows for a reasonably wide capture range as the signal recycling mirror swings through the resonant condition and so tuned signal recycling is straightforward to control given the prior experience from the advanced detectors... however...

Future work will address the problem of transitioning \ETLF{} from tuned to detuned operation.

\subsection{Testing of control concepts}
A possible test for \GEOHF{} to conduct in the future is to detune its signal recycling cavity to \SI{25}{\hertz} to test the ability to control the interferometer at such an extreme.

\subsection{Future upgrades to ET}
\note{SSM, Stefan D's idea for triangular speed meter, etc...}