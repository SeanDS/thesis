\chapter{\label{c:et-lf-control}Conceptual sensing scheme for the low frequency Einstein Telescope detector}

% sinks
\newcommand{\AS}{AS}
\newcommand{\POP}{POP}
\newcommand{\REFL}{REFL}

% readout ports, I and Q
\newcommand{\ASDC}{$\text{\AS}_{\text{DC}}$}
\newcommand{\ASFIRSTI}{$\text{\AS}_{\num{11}}^{\text{I}}$}
\newcommand{\ASFIRSTQ}{$\text{\AS}_{\num{11}}^{\text{Q}}$}
\newcommand{\ASSECONDI}{$\text{\AS}_{\num{57}}^{\text{I}}$}
\newcommand{\ASSECONDQ}{$\text{\AS}_{\num{57}}^{\text{Q}}$}
\newcommand{\ASSUMI}{$\text{\AS}_{\num{68}}^{\text{I}}$}
\newcommand{\ASSUMQ}{$\text{\AS}_{\num{68}}^{\text{Q}}$}
\newcommand{\ASDIFFI}{$\text{\AS}_{\num{45}}^{\text{I}}$}
\newcommand{\ASDIFFQ}{$\text{\AS}_{\num{45}}^{\text{Q}}$}
\newcommand{\POPDC}{$\text{\POP}_{\text{DC}}$}
\newcommand{\POPFIRSTI}{$\text{\POP}_{\num{11}}^{\text{I}}$}
\newcommand{\POPFIRSTQ}{$\text{\POP}_{\num{11}}^{\text{Q}}$}
\newcommand{\POPSECONDI}{$\text{\POP}_{\num{57}}^{\text{I}}$}
\newcommand{\POPSECONDQ}{$\text{\POP}_{\num{57}}^{\text{Q}}$}
\newcommand{\POPSUMI}{$\text{\POP}_{\num{68}}^{\text{I}}$}
\newcommand{\POPSUMQ}{$\text{\POP}_{\num{68}}^{\text{Q}}$}
\newcommand{\POPDIFFI}{$\text{\POP}_{\num{45}}^{\text{I}}$}
\newcommand{\POPDIFFQ}{$\text{\POP}_{\num{45}}^{\text{Q}}$}
\newcommand{\REFLDC}{$\text{\REFL}_{\text{DC}}$}
\newcommand{\REFLFIRSTI}{$\text{\REFL}_{\num{11}}^{\text{I}}$}
\newcommand{\REFLFIRSTQ}{$\text{\REFL}_{\num{11}}^{\text{Q}}$}
\newcommand{\REFLSECONDI}{$\text{\REFL}_{\num{57}}^{\text{I}}$}
\newcommand{\REFLSECONDQ}{$\text{\REFL}_{\num{57}}^{\text{Q}}$}
\newcommand{\REFLSUMI}{$\text{\REFL}_{\num{68}}^{\text{I}}$}
\newcommand{\REFLSUMQ}{$\text{\REFL}_{\num{68}}^{\text{Q}}$}
\newcommand{\REFLDIFFI}{$\text{\REFL}_{\num{45}}^{\text{I}}$}
\newcommand{\REFLDIFFQ}{$\text{\REFL}_{\num{45}}^{\text{Q}}$}

% readout ports, collapsed
\newcommand{\ASFIRST}{$\text{\AS}_{\num{11}}$}
\newcommand{\ASSECOND}{$\text{\AS}_{\num{57}}$}
\newcommand{\ASSUM}{$\text{\AS}_{\num{68}}$}
\newcommand{\ASDIFF}{$\text{\AS}_{\num{45}}$}
\newcommand{\POPFIRST}{$\text{\POP}_{\num{11}}$}
\newcommand{\POPSECOND}{$\text{\POP}_{\num{57}}$}
\newcommand{\POPSUM}{$\text{\POP}_{\num{68}}$}
\newcommand{\POPDIFF}{$\text{\POP}_{\num{45}}$}
\newcommand{\REFLFIRST}{$\text{\REFL}_{\num{11}}$}
\newcommand{\REFLSECOND}{$\text{\REFL}_{\num{57}}$}
\newcommand{\REFLSUM}{$\text{\REFL}_{\num{68}}$}
\newcommand{\REFLDIFF}{$\text{\REFL}_{\num{45}}$}

\section{The \ET{} facility}
In \checkme{2011} a group of scientists primarily based in Europe completed a design study \cite{ET2011} to examine the requirements for a gravitational wave observatory that pushes the \MI{} topology to its limits, while making any newly built facility generic enough to allow for the implementation of new topologies as the state of the art evolves. In this study they laid out the expected improvements in technologies to mitigate fundamental and technical noise sources currently limiting the sensitivity of the third generation of detectors, with a number of practical differences to existing facilities.

The proposed \ET{} facility composes six \DRFPMI{}s split between the three corners of a triangle with length \SI{10}{\kilo\meter} long. The design exploits the geometry to implement interferometers with \SI{10}{\kilo\meter} arms meeting at the three vertices to benefit from the colocation of multiple interferometers. The extended arm length over \ALIGO{} provides less susceptibility to displacement noise sources \cite{Dwyer2015, aligocosmic2016} by a factor of:
\begin{equation}
  \frac{h_{\SI{10}{\kilo\meter}}}{h_{\SI{4}{\kilo\meter}}} = \sqrt{\frac{\SI{10}{\kilo\meter}}{\SI{4}{\kilo\meter}}} \approx 1.58,
\end{equation}
translating into an increase in the observable volume of the universe by around $3.95$, assuming that displacement noise sources stay the same.

Seismic noise limits the sensitivity of current generation detectors below \SI{10}{\hertz}, and there are astrophysical advantages to being able to achieve good sensitivity at these frequencies \cite{Sathyaprakash2012}, particularly in the ability to see the inspiral and merger of high mass black hole binary coalescences. The majority of spinning neutron stars discovered via optical techniques have also had orbital frequencies below \SI{10}{\hertz} where fundamental noise limits the ability for the current generation of detectors to see such signals. Signals at \SI{2}{\hertz} would stay in \ETLF{}'s sensitive band for hours instead of milliseconds as witnessed for instance with \GWFIRSTEVENT{} \cite{Abbott2016}. Having extra observation time also provides the possibility to track the signal evolution with corresponding changes in the signal recycling cavity tuning \cite{Heinzel2002, Simakov2014}. With greatly enhanced low frequency sensitivity, the \ET{} would have unprecedented sensitivity to such signals.

\subsection{New facility}
As discussed in Section\,\ref{sec:seismic-noise}, seismic noise in current detectors limits the sensitivity at low frequencies and creates challenging control requirements due to the mirror motion created by ground vibrations. The \ET{} interferometers will be \num{100} to \SI{200}{\meter} underground to mitigate seismic noise. This location also helps to limit the impact of gravity gradient noise, as discussed in Section\,\ref{sec:gravity-gradient-noise}, which is expected to become a problem as seismic noise is mitigated.

\subsection{Colocated xylophone configuration}
To provide maximum astrophysical reach the facility is intended to provide sensitivity across an unprecedented bandwidth, from around \SI{2}{\hertz} to \SI{10}{\kilo\hertz}\textemdash a bandwidth significantly larger than that of existing detectors. It was realised that the most technically feasible option would be to implement a number of colocated detectors, each optimised to provide good sensitivity in either low or high frequencies \cite{Hild2010}, an idea first proposed for \ALIGO{} \cite{Conforto2004}. In the proposed \emph{ET-C xylophone} configuration \cite{Hild2010}, a low power, cryogenic interferometer optimises sensitivity to reduce radiation pressure noise at the expense of shot noise, whilst a high power interferometer optimises high frequency sensitivity through the reduction of shot noise. This design was later updated to the \emph{ET-D xylophone} configuration \cite{Hild2011}, the most up-to-date at the time of writing, which improved upon the design's loss and noise assumptions.

There are a number of benefits to having multiple identical interferometers located in the same facility. As the low and high frequency xylophones at each vertex will have the same design, the signal and noise properties across the facility will have a similar impact and so it should be possible to combine the signals from each interferometer in such a way as to generate a \emph{null stream} that contains noise but not signal \cite{Hewitson2005, Ajith2006}. This will be useful for the characterisation of noise sources and will be particularly beneficial for the new noise sources that may be interrogated due to the increased sensitivity. The arrangement of three detectors in a triangle also allows the facility to be optimally sensitive to gravitational waves from all directions \cite{Winkler1985}, whereas existing single-interferometer detectors are sensitive only to incident signals in the plane of the detector.

% data from http://www.et-gw.eu/etsensitivities
\begin{figure}
  \centering
  \includegraphics[width=\columnwidth]{graphics/generated/from-python/70-et-d-sensitivity-curves.pdf}
  \caption[Sensitivity curves for the Einstein Telescope]{\label{fig:et-d-sensitivity}Sensitivity of the \ET{} detectors, based on the ET-D design \cite{Hild2011}. \ETLF{} is optimised for low frequencies, \ETHF{} is optimised for high frequencies, and the combination yields sensitivity between \SI{2}{\hertz} and \SI{10}{\kilo\hertz}.}
\end{figure}

\subsubsection{ET-LF}
The low frequency detector consists of a \DRFPMI{} configuration as introduced in Section\,\ref{sec:signal-recycling}, but with a detuned signal recycling cavity. This detuning allows for enhanced sensitivity at the signal recycling cavity pole where the light-mirror dynamics create an optical spring that provides sensitivity below the \gls{SQL} at the spring frequency. The cavities will have \SI{18}{\kilo\watt} of light power, which is considerably lower than that of \ALIGO{} at design sensitivity (\SI{800}{\kilo\watt}), facilitating reduced quantum radiation pressure noise at low frequencies. Cryogenic test masses are to be used to facilitate a reduction in thermal noise, and the wavelength of the carrier will be changed from the standard \SI{1064}{\nano\meter} to \SI{1550}{\nano\meter} to utilise lower noise materials at such temperatures. Two filter cavities to facilitate frequency dependent squeezing for the further suppression of quantum noise are also included.

The suspension systems for the main test masses are based on those of the \emph{superattenuator} in \VIRGO{} \cite{Acernese2010}. The \checkme{\SI{17}{\meter}} long pendulum stage \note{which stage?} pushes the longitudinal resonant frequency down from around \SI{1}{\hertz} in existing detectors to \checkme{\SI{0.1}{\hertz}}, providing better attenuation of seismic noise above \SI{2}{\hertz}.

The sensitivity of \ETLF{} is shown in \checkme{blue} in Figure\,\ref{fig:et-d-sensitivity}.

\subsubsection{ET-HF}
\ETHF{} takes the designs of \ALIGO{} and \AVIRGO{} and pushes them to their high frequency limit, and adds new technologies such as LG33 modes \cite{Carbone2013} and frequency dependent squeezing \cite{Kimble2001} to reduce coating thermal and quantum noise. The combination of greater arm cavity power, heavier test masses, squeezing and improved coatings and materials will increase sensitivity at frequencies above a few hundred \SI{}{\hertz} beyond the current generation by a factor of around \checkme{\num{10}}.

The sensitivity of \ETHF{} is shown in \checkme{orange} in Figure\,\ref{fig:et-d-sensitivity}.

\section{\label{sec:et-lf-control-challenges}Control challenges with the \ET{}}
Both \ETLF{} and \ETHF{} will present new challenges to the control of large-scale \DRFPMI{}s. Although \ETHF{} can to some extent be seen as a larger version of \ALIGO{} and \AVIRGO{}, and it may therefore be possible to adapt much of the advance detectors' strategies for both longitudinal and angular control, some aspects such as the use of LG33 modes on a large scale and the presence of parametric instabilities at such high arm cavity powers \cite{Evans2015} require extensive research to understand the implications they may have on control. \ETLF{} will also use a topology that resembles existing generation detectors, but it pushes the sensitivity at low frequencies further down and this presents additional challenges with sensing and noise. This chapter will discuss the longitudinal control of \ETLF{}, focusing in particular on the challenge of controlling the interferometer in its detuned state.

\ETLF{} employs a detuned signal recycling cavity to increase the sensitivity of the interferometer at low frequencies. The ability for a detuned signal recycling to shift the most sensitive frequency has been demonstrated in \GEO{} between \SI{200}{\hertz} and \SI{1}{\kilo\hertz} \cite{Hild2006}, however the plan for \ETLF{} is to detune the signal recycling cavity to around \SI{25}{\hertz} which has not been achieved before in a suspended audio-band detector. With tuned signal recycling, where the signal recycling cavity is resonant for the carrier, the sidebands used for control of the differential arm cavity mode are present within the signal recycling cavity with equal amplitude. In detuned operation, where the cavity is not resonant for the carrier, the sidebands have unequal amplitude and some of the interferometer's noise couplings that otherwise cancel at the output in the tuned case no longer cancel \cite{Hild2007}. The phase modulation of the control sidebands, created by \glspl{EOM} on the input path, also gets partially converted to amplitude modulation by the detuning, and this can lead to issues with the dynamic range of any photodetectors used to sense the readout \cite{Grote2007}. Finally, the asymmetric sidebands create additional zero crossings and a \gls{DC} offset in the error signal from the signal recycling cavity which for the tuned case do not appear. Often it can be difficult to find a port at which to sense the motion of the signal recycling cavity decoupled from other cavities, and this effect can be exacerbated by large detunings such as in \ETLF{} where noise cross-couplings can become more significant \cite{Hild2007}.

In the \ET{} design study the discussion for \ETLF{} stopped short of a control scheme. The rest of this chapter will discuss a conceptual approach to the control of \ETLF{} and will highlight the future work that must be undertaken before a technical design for the control of \ETLF{} can be produced.

\section{\label{sec:multi-dof-control}Longitudinal control of an interferometer with multiple degrees of freedom}
A successful control scheme for an interferometer must satisfy a number of requirements. When the interferometer is in its uncontrolled state, the control scheme must be able to bring it to the \emph{operating point} where it has the desired response to incident gravitational wave signals in a process called \emph{lock acquisition}. Once at the operating point, it must be robust against small perturbations by controlling the impact of noise and signal nonstationarities. Finally, the signal that represents the gravitational wave channel must have low noise, and therefore high sensitivity, to meet the design goals of the scheme.

The lock acquisition scheme is inextricably linked to the technical environment in which the interferometer will operate, and so it is inappropriate to discuss this while \gls{ET}'s technical design is subject to ongoing research. We will focus our efforts, therefore, on the second and third challenges above. Both \ETLF{} and \ETHF{} will utilise \gls{DC} readout to sense the differential motion of the arm cavities in order to sense the effect of gravitational waves.

\subsection{The dc readout technique}
\gls{DC} readout is the standard technique for \GEOHF{} and the advanced detectors \cite{Hild2007, Ward2008, Fricke2012}. It is a form of homodyne readout that involves a compromise between the best sensitivity and technical complexity. The operating point is kept close to the \emph{dark fringe} (see Section\,\ref{sec:operating-point}) to optimise shot noise, but a slight offset is introduced between the differential round trip phase of each arm in order to allow some of the carrier light to enter the readout port where it acts as a homodyne local oscillator to the signal sidebands (see Section\,\ref{sec:homodyne-readout}). In practice, this detuning\textemdash of the order \SI{}{\pico\meter} at the arm cavities in \ALIGO{}\textemdash is sufficiently small to allow the required sensitivities to be reached. This provides improved sensitivity over heterodyne techniques \cite{Fricke2012}, and in squeezed interferometers it avoids the need to inject squeezing at \gls{RF} sideband frequencies in addition to the carrier frequency.

\subsection{\label{sec:dofs-of-drfpmi}Degrees of freedom of the \DRFPMI{}}
The \emph{degrees of freedom} of an interferometer are the non-degenerate ways in which the interferometer's mirrors may move away from the operating point. Each degree of freedom has a different precision requirement, with the most stringent typically being the degree of freedom corresponding to the gravitational wave channel. With \gls{DC} readout there is no local oscillator phase or alignment to control, and so in a \DRFPMI{} the main degrees of freedom to consider are the arm cavity differential and common modes, the length between the beam splitter and the \glspl{ITM} and the recycling cavity lengths. These are defined in the following sections and the lengths that compose each degree of freedom are shown in Figure\,\ref{fig:et-lf-cavity-lengths}.

\begin{figure}
  \centering
  \includegraphics[width=\columnwidth]{graphics/generated/from-svg/70-cavity-lengths.pdf}
  \caption[Cavity lengths in \ETLF{}]{\label{fig:et-lf-cavity-lengths}Cavity lengths in \ETLF{}. The differential and common arm cavity modes compose $L_{\text{X}}$ and $L_{\text{Y}}$, while the auxiliary power and signal recycling and Michelson cavity lengths are composed of a subset of the distances between the beam splitter and each \gls{ITM} and the distances between each recycling mirror and the beam splitter.}
\end{figure}

The motion of each degree of freedom must be witnessed by a sensor and fed back to actuators to control the relevant length. This is called \emph{linear inverting feedback} and the concept of shaping feedback dynamics is discussed in greater detail in the context of suspended interferometers in Chapter\,\ref{c:speedmeter-control}.

\subsubsection{Common arm cavity length}
In-phase changes in the length of the arms of a \MI{} do not primarily couple to the gravitational wave channel, but it is crucial to control this degree of freedom in order to keep it in its most sensitive state. In a \DRFPMI{} this is the state in which the power in the arm cavities is held near maximum. Noise, particularly from the laser's amplitude and frequency fluctuations as discussed in Section\,\ref{sec:laser-noise}, can lead to a loss of light power in the arm cavities and so this degree of freedom must be controlled. Typically the actuator used to control common arm length changes is the laser's crystal, whereupon the application of strain changes the laser's frequency, though it is also possible to correct slower, larger drifts with common feedback some other input optics or to the \glspl{ETM}.

We can define the common arm cavity length, \emph{\gls{CARM}}, in terms of the average length of the arms:
\begin{equation}
  L_{\text{CARM}} = \frac{L_{\text{X}} + L_{\text{Y}}}{2}.
\end{equation}

\subsubsection{Differential arm cavity length}
Gravitational waves change the length of the arms in a \MI{} differentially and due to the presence of the arm cavities in a \DRFPMI{} the differential arm cavity length represents the degree of freedom the main readout is most sensitive to. This length is ideally held at the dark fringe to maximise sensitivity but for the purposes of control in \ETLF{} it will be contain a small amount of carrier light. The differential arm cavity length signal can be fed back to the \glspl{ETM} differentially to hold the length at the desired operating point.

We define the differential arm cavity length, \emph{\gls{DARM}}, in terms of the average differential length of the arms:
\begin{equation}
  L_{\text{DARM}} = \frac{L_{\text{X}} - L_{\text{Y}}}{2}.
\end{equation}

Any arm cavity length change can be expressed in terms of a linear combination of \gls{CARM} and \gls{DARM}.

\subsubsection{Michelson length}
The length between the beam splitter and the \glspl{ITM} should be held in the dark fringe condition for the carrier to correctly couple the common and differential arm cavity modes to the input and output port of the beam splitter, respectively. In a \DRFPMI{} it must also be held constant to keep the amount of carrier and sideband power in the signal recycling cavity stable, which avoids the need for complicated time-varying control signals. This length, \emph{\gls{MICH}}, can be expressed as the differential length between the beam splitter and the \glspl{ITM}:
\begin{equation}
  L_{\text{MICH}} = \frac{l_{\text{X}} - l_{\text{Y}}}{2}.
\end{equation}

The \gls{MICH} length is kept constant by feeding back to the position of the beam splitter and recycling mirrors or to the \glspl{ITM}. Note that the common equivalent of the Michelson mode, $\frac{l_{\text{X}} + l_{\text{Y}}}{2}$, is present but to first order has no impact on the interferometer control and so is not considered.

\subsubsection{Power and signal recycling cavity lengths}
The power recycling cavity should be resonant for the input light in order to optimally recycle light reflected from the beam splitter back towards the laser, to allow the arm cavity power with respect to the interferometer's input power to be maximised. The power recycling cavity length, \emph{\gls{PRCL}} can be defined in terms of the average distance between the power recycling mirror (\gls{PRM}) and the two \glspl{ITM}:
\begin{equation}
  \label{eq:prcl-length}
  \begin{split}
    L_{\text{PRCL}} &= l_{\text{P}} + \frac{l_{\text{X}} - l_{\text{Y}}}{2} \\
		    &= l_{\text{P}} + L_{\text{MICH}}.
  \end{split}
\end{equation}

The signal recycling length, along with the signal recycling mirror (\gls{SRM}) transmissivity, determines the bandwidth of the signal extraction and therefore needs controlled in order to keep the interferometer's sensitivity stationary. This length, \emph{\gls{SRCL}}, is defined similarly to \gls{PRCL} in terms of the position of the signal recycling mirror:
\begin{equation}
  \label{eq:srcl-length}
  \begin{split}
    L_{\text{SRCL}} &= l_{\text{S}} + \frac{l_{\text{X}} - l_{\text{Y}}}{2} \\
		    &= l_{\text{S}} + L_{\text{MICH}}.
  \end{split}
\end{equation}

Control of \gls{PRCL} and \gls{SRCL} can be achieved via corrective feedback to the position of the power and signal recycling mirrors, respectively.

\section{Longitudinal sensing scheme for ET-LF}

\subsection{Scope and method}
The simulations required to devise a longitudinal sensing scheme are conducted with Optickle (see Appendix\,\ref{a:simulation-tools}) and use the plane wave approximation. Angular control of the interferometer is not expected to present a challenge greater than that already seen in the advanced detectors, although future angular sensing and control simulations will be possible with the model developed over the course of this work.

The sensing scheme assumes that the interferometer has been brought close to its operating point by a lock acquisition routine, and here we primarily consider the control of the interferometer in terms of its response. It is well known that noise coupling between longitudinal degrees of freedom of detuned \DRFPMI{}s can be significant \cite{Hild2007}, but a crucial initial step before any control noise simulations can be undertaken is the development of a provisional sensing scheme. For \ETLF{} we loosely follow the approach taken for \ALIGO{} \cite{Abbott2010} and \AVIRGO{} \cite{Vajente2008} given that these represent the most sensitive \DRFPMI{}s built to date. To simplify the steps required to produce the scheme the least constrained parameters are defined first and parameters which depend on previously defined parameters are then determined sequentially. With these parameters set, we can search for ports at which sensible error signals representing each degree of freedom can be extracted and we can then define a \emph{sensing matrix} that highlights cross-couplings between each of the degrees of freedom. This is the starting point for control loop noise studies which will be the subject of future work.

\subsection{\label{sec:decoupled-sidebands}Control of the interferometer's degrees of freedom}
In gravitational wave detectors the error signals representing each degree of freedom are usually derived using a form of the Pound-Drever-Hall scheme discussed in Section\,\ref{sec:pdh} for \MI{}s called \emph{Schnupp modulation}. In prototype facilities this control method was extended to power- \cite{Regehr1995} and \DRFPMI{}s \cite{Heinzel1998}, and it is a standard technique used in gravitational wave detectors and remains suitable for \ETLF{}.

As discussed in Section\,\ref{sec:multi-dof-control}, the control of a \DRFPMI{} involves the extraction of error signals representing each degree of freedom from photodetectors placed at the ports of the interferometer; however it is not necessary to obtain error signals that contain solely information representing one degree of freedom as long as the individual constituents can be separated after detection. By employing \emph{gain hierarchy} it can be possible to suppress signal content at a particular readout from another degree of freedom, and this has been demonstrated in \LIGO{} \cite{Fritschel2001}. The most problematic extraneous signal coupling in signals demodulated at the main sideband modulation frequencies tend to arise from \gls{CARM} and \gls{DARM}, but this tends to be suppressed in the error signals after these degrees of freedom are controlled. Common mode fluctuations arising from laser noise tend to appear across the sensitive band of the detector and so \gls{CARM} control requires a wide bandwidth servo. Control of \gls{DARM} and the auxiliary modes \gls{PRCL}, \gls{SRCL} and \gls{MICH} tends to be driven by the presence of low frequency seismic noise and so typically requires much smaller servo bandwidth. Through the appropriate choice of servo hierarchy the cross-couplings present at each sensor from secondary degrees of freedom can be minimised.

This section will discuss the creation and extraction of error signals for each degree of freedom discussed in Section\,\ref{sec:dofs-of-drfpmi}. Given the constraints from the design study, we begin with the transmissivity of the power recycling mirror and the frequency of the control sidebands, and then use these to define auxiliary lengths within the interferometer and finally calculate the arm cavity offset required for the \gls{DC} readout to work.

\subsubsection{Optimal input coupling}
Placing a power recycling mirror in an \MI{}, as discussed in Section\,\ref{sec:power-recycling}, creates an additional cavity between the input light and the arms. The intention of the power recycling mirror is to minimise the light reflected back towards the laser, and in order to do this the cavity it creates should be \emph{impedance matched} (see, for example, Section\,5.1 of \cite{Freise2010}). The transmissivity of the power recycling mirror, $T_{\text{PRM}}$, determines the impedance matching and it can be determined using knowledge of the light lost within the interferometer, and in the \ET{} design study the value for \ETLF{} is suggested to be \SI{4.6}{\percent}. The design study also defines the loss per optic to be \SI{35}{\ppm}, and the transmissivity of the \glspl{ETM} is \SI{6}{\ppm}; both contribute to the impedance matching. Figure\,\ref{fig:reflected-power-vs-prm-trans} shows the ratio of the light power leaving the power recycling mirror heading back towards the laser to the input light power. The minimum reflected power corresponds to a power recycling mirror transmissivity of \SI{4.6}{\percent}, validating the choice from the design study. If the scatter or substrate loss on any optic is changed in the future, for instance due to the development of new coatings, this model can recalculate the optimum power recycling mirror transmissivity.

\begin{figure}
  \centering
  \includegraphics[width=\columnwidth]{graphics/generated/from-python/70-reflected-power-vs-prm-transmissivity.pdf}
  \caption[Reflected power from \ETLF{} as a function of power recycling mirror transmissivity]{\label{fig:reflected-power-vs-prm-trans}Reflected power from \ETLF{} as a function of power recycling mirror transmissivity. For optimal coupling of the input laser light to the interferometer, the transmissivity of the power recycling mirror must be set to balance the input light with the total loss from the interferometer. For \ETLF{} with loss as per the design study, this transmissivity should be \SI{4.6}{\percent}.}
\end{figure}

\subsubsection{\label{sec:multiple-control-sidebands}Multiple control sidebands}
Any control sideband frequencies used in \ETLF{} must be outside an integer multiple of the free spectral range (\gls{FSR}, see Appendix\,\ref{sec:cavity-fom}) of the arm cavities in order to allow them to act as a phase discriminant for \gls{CARM} and \gls{DARM}. They should also ideally be \gls{RF}, at least around \SI{10}{\mega\hertz}, to benefit from the noise advantages discussed in Section\,\ref{sec:pdh}. An upper limit of \SI{100}{\mega\hertz} is reasonable, as higher frequency demodulations are difficult to implement successfully in photodetector electronics.

\ETLF{} will require at least two sideband frequencies in order to control the five longitudinal degrees of freedom because it is not trivial to decouple the motion of multiple degrees of freedom when they are sensed by the same light. Suitably decoupled sideband resonances provide error signals representing the cavities to be controlled, although it is not necessary to use a different modulation source for each cavity; indeed it is beneficial from the point of view of quantum shot noise on the sensors to try to find unique combinations of the two sideband frequencies to sense each of the five degrees of freedom.

Sometimes demodulation at the sideband frequency at a particular port of the interferometer contains useful signal content representing one of the degrees of freedom with minimal cross-couplings from other cavities; however, sometimes a signal with greater decoupling, particularly for the inner degrees of freedom, can be found by demodulating the light at some combination of sideband frequencies \cite{Strain2003, Barr2006}. For example, some of the control signals used in \ALIGO{} involve demodulation at the sum or difference of the two sideband frequencies \cite{Abbott2010}.

For a single control sideband scheme it is possible to calculate a set of possible frequencies given the approximate lengths of the cavities in which it is supposed to be resonant or anti-resonant; however, with multiple sets of sideband frequencies some fine tuning is necessary. An approach involving two control sideband frequencies is discussed in the following subsection.

\subsubsection{\label{sec:control-sideband-freqs}Control sideband frequencies}

\paragraph{Control sideband resonance in the recycling cavities}
The \SI{310}{\meter} recycling cavity lengths defined in the design study have \gls{FSR} $\frac{c_0}{2 L_{\text{PRCL}}} = \frac{c_0}{2 L_{\text{SRCL}}} = \SI{483.5}{\kilo\hertz}$. To start, we can try to make the first sideband frequency resonate in the power recycling cavity. As the arm cavities at the operating point reflect the light back towards the power recycling cavity, the resonant condition is a half-integer multiple of the power recycling cavity \gls{FSR}, i.e.:
\begin{equation}
  \label{eq:prc-fsr}
  f_1 = \left(A + \frac{1}{2} \right) \frac{c_0}{2 L_{\text{PRC}}},
\end{equation}
for positive integer $A$.

We can repeat this step for the signal recycling cavity, instead making the second sideband frequency resonant and the first anti-resonant. In this case the resonant condition is an integer multiple of the cavity's \gls{FSR}, i.e.:
\begin{align}
  \label{eq:src-fsr}
  f_1 &\neq B \frac{c_0}{2 L_{\text{SRC}}} \\
  f_2 &= C \frac{c_0}{2 L_{\text{SRC}}},
\end{align}
where $B$ and $C$ are again positive integers.

\paragraph{Avoiding resonance of the control sidebands in the arm cavities}
In addition to optimising the control sideband frequencies using Equations \ref{eq:prc-fsr} and \ref{eq:src-fsr}, we must prevent the sidebands from entering into the arm cavities. The arm cavity \gls{FSR} is \SI{14.99}{\kilo\hertz} for the \SI{10}{\kilo\meter} arms. As an integer multiple of the arm cavity \gls{FSR} would allow optimal coupling of the sidebands into the arm cavities, one might assume that an odd half-integer multiple would be optimally anti-resonant; however, in this scenario the lower higher-order control sidebands, necessarily created by the phase modulation upon the \gls{EOM} (see Appendix\,\ref{eq:field-phase-bessel}), would become resonant, and so we choose to offset the sideband frequency slightly from the anti-resonant condition. We therefore stipulate two further requirements:
\begin{align}
  \label{eq:arm-fsr}
  f_1 &= \left(D + \delta \right) \frac{c_0}{2 L_{\text{Arm}}} \\
  f_2 &= \left(D + \delta \right) \frac{c_0}{2 L_{\text{Arm}}},
\end{align}
for positive integer $D$ and small perturbation $\left| \delta \right| < \frac{1}{2}$.

\paragraph{Control sideband frequencies}
For \ETLF{} we chose the first sideband frequency $f_1$ to be \SI{11363101}{\hertz} which for $A = 23$, $D = 758$ satisfy the requirements and falls within the range defined in Section\,\ref{sec:multiple-control-sidebands}. As the light in the signal recycling cavity must first pass through the power recycling cavity, we must also ensure that $f_2$ resonates in the power recycling cavity too. This is achieved by choosing the second sideband to be an integer multiple of the first, which is already resonant in the power recycling cavity. To provide ample difference between the first and second sideband frequencies, we chose $f_2 = 5f_1 = \SI{56815505}{\hertz}$. This is separated far enough in frequency from $f_1$ that we can investigate the use of beats between $f_1$ and $f_2$ for control purposes as discussed in Section\,\ref{sec:etlf-readout-ports}.

We assume a modulation depth of \num{0.1} for $f_1$ and $f_2$ to keep the power in higher modulation orders low while allowing for a reasonable amount of light power in the first order. This parameter has little impact on others and can be later tuned to provide for better separation between the two sideband signals on the sensors.

\subsubsection{Schnupp asymmetry}
For the control sidebands to enter the signal recycling cavity to provide an error signal for \gls{SRCL}, the arm lengths of a \DRFPMI{} must be slightly mismatched; this is called the \emph{Schnupp asymmetry}. For a Schnupp asymmetry $\Delta l_{\text{SCH}}$, the two arms have length $l + \frac{\Delta l_{\text{SCH}}}{2}$ and $l - \frac{\Delta l_{\text{SCH}}}{2}$ where the nominal length is $l$. This arrangement is shown in Figure\,\ref{fig:schnupp-darm-offsets}.

\begin{figure}
  \centering
  \includegraphics[width=\columnwidth]{graphics/generated/from-svg/70-schnupp-darm-offsets.pdf}
  \caption[Schnupp asymmetry and differential arm cavity offset in a \DRFPMI{}]{\label{fig:schnupp-darm-offsets}The Schnupp asymmetry and differential arm cavity offset in a \DRFPMI{}. The Schnupp asymmetry allows the control sidebands resonant in the Michelson to leave the beam splitter's output port where they may sense the motion of the signal recycling cavity. This is typically an offset of the order of \SI{}{\centi\meter}. The differential arm cavity (\gls{DARM}) offset is a microscopic detuning of the arm cavities to allow a small amount of light to interfere constructively at the beam splitter in order for it to act as a local oscillator to the signal sidebands at the output port. This detuning is typically of the order of \SI{}{\pico\meter}.}
\end{figure}

As described in Section\,\ref{sec:laser-noise}, laser frequency noise in a \MI{} is best suppressed when the arm lengths are matched, but the asymmetry introduced in this case is small compared to the length of the arms and so this does not have a significant impact on laser noise suppression in the audio band.

The choice can be made as to whether the second sideband should be resonant in both recycling cavities, or just one, and this is governed by the Schnupp asymmetry; the transmissivity of the power and signal recycling mirrors determines the Schnupp lengths corresponding to each case. A small offset of a few \SI{}{\centi\meter} between the arm lengths of the Michelson degree of freedom allows for the signal recycling cavity to be resonant for only one of the sideband frequencies, whereas a larger offset of a few tens of \SI{}{\centi\meter} makes both sideband frequencies resonant there \cite{Vajente2008}. Both methods of control are feasible, with the former being implemented in \ALIGO{} \cite{Abbott2010} and the latter in \KAGRA{} \cite{kagra2013}. We choose to follow the \ALIGO{} approach to provide good separation of the error signals in the signal recycling cavity to assist with its challenging control.

Figure\,\ref{fig:sideband-powers-vs-schnupp-detuned} shows the power of each sideband field in the recycling cavities of \ETLF{} given the Schnupp asymmetry and recycling cavity lengths with detuned signal recycling. Since the Schnupp asymmetry is a macroscopic length, it is not easily adjusted during operation and so it is necessary to designate this length in the design phase. Here we choose a Schnupp asymmetry that attempts to maximise the difference in power between the two sideband frequencies in the recycling cavities during detuned operation. This is around \SI{0.08}{\meter}. Also observe that a Schnupp asymmetry of \num{0} results in no sideband power in the signal recycling cavity, and that asymmetries of around \SI{0.5}{\meter} result in a situation where both sideband frequencies are resonant in both recycling cavities.

\begin{figure}
  \centering
  \includegraphics[width=\columnwidth]{graphics/generated/from-python/70-sideband-powers-vs-schnupp-detuned.pdf}
  \caption[Power of the control sidebands in the cavities of \ETLF{} in the detuned configuration]{\label{fig:sideband-powers-vs-schnupp-detuned}Power of the control sidebands in the cavities of \ETLF{} during detuned operation. A macroscopic offset, called the \emph{Schnupp asymmetry}, is intentionally introduced to the Michelson length in order to allow the coupling of the sidebands $f_1$ and $f_2$ into the signal recycling cavity for the purposes of control while maintaining the dark fringe condition for the carrier. Here, we choose to make $f_2$ resonant but not $f_1$, and the best Schnupp asymmetry in this case is one which provides the best decoupling of sideband error signals representing the lengths of the power and signal recycling cavities. The power is a reasonable estimate for what the relative sensitivity of the sidebands in the recycling cavities will be, and we choose an asymmetry which gives good separation of the power of the lower sidebands in each cavity, \SI{0.08}{\meter}.}
\end{figure}

\subsubsection{Optimisation of the signal recycling cavity length}
Note the discrepancy between two of the control sideband frequency constraints in Section\,\ref{sec:control-sideband-freqs}: the power and signal recycling cavities cannot both be simultaneously resonant and anti-resonant to $f_1$ and $f_2$ given that $f_2 = 5 f_1$. To resolve this discrepancy we can scan the length of the signal recycling cavity in order to find a position where $f_2$ is resonant and $f_1$ is not, as shown in Figure\,\ref{fig:sideband-powers-srcl-detuned}. This could otherwise have been achieved by changing $f_2$ by a fraction of the signal recycling cavity's \gls{FSR}. We can see that changing the signal recycling cavity length from \SI{310}{\meter} to \SI{311.585}{\meter} results in the desired sideband resonance condition for the upper $f_2$ sideband. The power of $f_2$ in the power recycling cavity drops as the lower and upper sidebands get critically coupled into the signal recycling cavity. As the detuning in \ETLF{} is so large, the signal recycling cavity is not resonant for both the upper and lower $f_2$ sidebands.

\begin{figure}
  \centering
  \includegraphics[width=\columnwidth]{graphics/generated/from-python/70-sideband-powers-vs-srcl-detuned.pdf}
  \caption[Power of the control sidebands in the signal recycling cavity as a function of length of \ETLF{} in the detuned configuration]{\label{fig:sideband-powers-srcl-detuned}The power in each sideband in each recycling cavity for detuned signal recycling. The nominal signal recycling cavity length defined by the design study, \SI{310}{\meter}, is not resonant for the second sideband as intended. As the sideband frequencies are different, we can fix this situation by scanning the signal recycling cavity to find a length that is resonant for $f_2$ but not for $f_1$. For the given choice of Schnupp asymmetry, this occurs at a length of \SI{311.585}{\meter} for the upper sideband.}
\end{figure}

In the power recycling cavity, $f_1$ provides an error signal for the power recycling cavity that is a factor of \num{13} larger than that of $f_2$. At the same time, $f_2$ provides an error signal for the length that is \num{650} times larger than the equivalent for $f_1$ in the signal recycling cavity.

\subsection{Dark fringe offset}
As described in Section\,\ref{sec:homodyne-readout}, \gls{DC} readout at the output port of a \DRFPMI{} requires carrier light to be present to act as a phase reference for the signal sidebands. In an interferometer with matched arms there is no classical light at the output (\emph{dark}) port and so this offset must be introduced by differentially detuning the arms by a small amount to create the appropriate \emph{dark fringe offset}. In practice, asymmetries which create differential detuning are already present in the arms, for example arising from mismatched arm cavity finesse or asymmetric beam splitter reflectivity.

For our model, we do not assume any asymmetries and so we can intentionally introduce a dark fringe offset with an offset in the \gls{MICH} or \gls{DARM} degrees of freedom. While a \gls{DARM} offset has the disadvantage that it involves the creation of an optical spring due to the high light power in the arms\textemdash mechanically coupling the \gls{CARM} and \gls{DARM} modes \cite{Heidmann2011, Vostrosablin2014}\textemdash it has favourable noise couplings compared to a \gls{MICH} offset \cite{Vajente2011}. We define the \gls{DARM} offset as a microscopic detuning of the arm cavity lengths, and it is differential such that one arm has length $L + \frac{\delta L_{\text{DARM}}}{2}$ while the other has length $L - \frac{\delta L_{\text{DARM}}}{2}$, where $L$ is the average length. This is depicted in Figure\,\ref{fig:schnupp-darm-offsets} alongside the Schnupp asymmetry.

Figure\,\ref{fig:total-power-vs-darm-offset-detuned} shows the power at the output port and in the arm cavities as a function of \gls{DARM} offset in the detuned configuration. Standard photodetectors used in \ALIGO{} and \AVIRGO{} can usually handle around \SI{100}{\milli\watt} and this should ideally be the power incident upon the photodetector at the output port to maximise the signal to dark noise ratio. In \ETLF{}, however, the \gls{DARM} offset required to reach this figure would create a significant mismatch in the power of each arm leading to a strong optical spring effect. For an offset of \SI{12}{\pico\meter}, the power at the output can be set to around \SI{10}{\milli\watt} with a difference of around 3\% in the power in the arms, which should be tolerable in terms of sensitivity and noise coupling.

\begin{figure}
  \centering
  \includegraphics[width=\columnwidth]{graphics/generated/from-python/70-total-power-vs-darm-offset-detuned.pdf}
  \caption[Carrier power at the output port of \ETLF{} in detuned configuration as a function of differential arm cavity offset]{\label{fig:total-power-vs-darm-offset-detuned}Carrier power at the output port of \ETLF{} in detuned configuration as a function of differential arm cavity length (\gls{DARM}) offset. The differential arm detuning required to allow carrier light to enter the dark port for \gls{DC} readout involves an increase or decrease in the microscopic length of each arm cavity, and this changes the circulating power. The compromise must be made between the power available to the photodetector for sensing while maintaining reasonably balanced arm cavities to prevent optical springs from influencing the sensitive band and creating additional noise coupling.}
\end{figure}

\subsection{Power in each light field}
The power in each field within each relevant space or cavity of the interferometer is shown in Tables\,\ref{tab:et-lf-detuned-dc-powers} for the detuned interferometer.

\begin{table}
  \centering
  \resizebox{16cm}{!}{%
    \begin{tabular}{r|ccccccccc|c}
      & \textbf{-68 MHz} & \textbf{-57 MHz} & \textbf{-45 MHz} & \textbf{-11 MHz} & \textbf{Carrier} & \textbf{11 MHz} & \textbf{45 MHz} & \textbf{57 MHz} & \textbf{68 MHz} & \textbf{Total} \\
      \hline
      \textbf{Input from laser} & 0 & 0 & 0 & 0 & \SI{3}{\watt} & 0 & 0 & 0 & 0 & \SI{3}{\watt} \\ 
      \textbf{After modulators} & \SI{19}{\micro\watt} & \SI{7}{\milli\watt} & \SI{19}{\micro\watt} & \SI{7}{\milli\watt} & \SI{3}{\watt} & \SI{7}{\milli\watt} & \SI{19}{\micro\watt} & \SI{7}{\milli\watt} & \SI{19}{\micro\watt} & \SI{3}{\watt} \\ 
      \textbf{Power recycling cavity} & \SI{220}{\nano\watt} & \SI{521}{\milli\watt} & \SI{220}{\nano\watt} & \SI{410}{\milli\watt} & \SI{65}{\watt} & \SI{407}{\milli\watt} & \SI{220}{\nano\watt} & \SI{34}{\milli\watt} & \SI{220}{\nano\watt} & \SI{66}{\watt} \\ 
      \textbf{Power recycling pick-off} & \SI{33}{\pico\watt} & \SI{78}{\micro\watt} & \SI{33}{\pico\watt} & \SI{61}{\micro\watt} & \SI{9}{\milli\watt} & \SI{61}{\micro\watt} & \SI{33}{\pico\watt} & \SI{4}{\micro\watt} & \SI{33}{\pico\watt} & \SI{10}{\milli\watt} \\ 
      \textbf{Michelson cavity} & \SI{131}{\nano\watt} & \SI{228}{\milli\watt} & \SI{112}{\nano\watt} & \SI{208}{\milli\watt} & \SI{33}{\watt} & \SI{204}{\milli\watt} & \SI{107}{\nano\watt} & \SI{27}{\milli\watt} & \SI{115}{\nano\watt} & \SI{33}{\watt} \\ 
      \textbf{Arm cavity X} & \SI{261}{\pico\watt} & \SI{557}{\micro\watt} & \SI{375}{\pico\watt} & \SI{9}{\milli\watt} & \SI{18}{\kilo\watt} & \SI{9}{\milli\watt} & \SI{357}{\pico\watt} & \SI{65}{\micro\watt} & \SI{230}{\pico\watt} & \SI{18}{\kilo\watt} \\ 
      \textbf{Arm cavity Y} & \SI{182}{\pico\watt} & \SI{722}{\micro\watt} & \SI{361}{\pico\watt} & \SI{9}{\milli\watt} & \SI{18}{\kilo\watt} & \SI{9}{\milli\watt} & \SI{379}{\pico\watt} & \SI{68}{\micro\watt} & \SI{210}{\pico\watt} & \SI{18}{\kilo\watt} \\ 
      \textbf{Signal recycling cavity} & \SI{3}{\nano\watt} & \SI{4}{\milli\watt} & \SI{379}{\pico\watt} & \SI{69}{\micro\watt} & \SI{15}{\milli\watt} & \SI{41}{\micro\watt} & \SI{411}{\pico\watt} & \SI{26}{\milli\watt} & \SI{928}{\pico\watt} & \SI{45}{\milli\watt} \\ 
      \textbf{Reflected back to laser} & \SI{19}{\micro\watt} & \SI{6}{\milli\watt} & \SI{19}{\micro\watt} & \SI{7}{\milli\watt} & \SI{62}{\micro\watt} & \SI{7}{\milli\watt} & \SI{19}{\micro\watt} & \SI{2}{\milli\watt} & \SI{19}{\micro\watt} & \SI{23}{\milli\watt} \\ 
      \textbf{Output} & \SI{558}{\pico\watt} & \SI{756}{\micro\watt} & \SI{76}{\pico\watt} & \SI{14}{\micro\watt} & \SI{3}{\milli\watt} & \SI{8}{\micro\watt} & \SI{82}{\pico\watt} & \SI{5}{\milli\watt} & \SI{186}{\pico\watt} & \SI{9}{\milli\watt}
    \end{tabular}
  }
  \caption[Light power in \ETLF{} in the detuned configuration]{\label{tab:et-lf-detuned-dc-powers}Powers in various parts of \ETLF{} in the detuned configuration. The input light is passed through \glspl{EOM} which impart control sidebands at two pairs of frequencies offset from the carrier. As the arm cavity \gls{FSR} is \checkme{higher} than that of the control sideband frequencies, the sideband light power in the arms is vastly smaller than the carrier power. The first sideband at $\SI{\pm11}{\mega\hertz}$ is resonant within the power recycling cavity. The second sideband at $\SI{\pm57}{\mega\hertz}$ is resonant within both the power and signal recycling cavities, with the \SI{-57}{\mega\hertz} sideband almost exactly resonant in the power recycling cavity and the +\SI{57}{\mega\hertz} sideband exactly resonant in the signal recycling cavity. The power reflected back towards the laser is composed mainly of light at the two sideband frequencies, as the transmissivity of the power recycling mirror minimises the reflected carrier light. At the output port, the carrier power is present due to the \gls{DARM} offset, and acts as a local oscillator to the signal sidebands there.}
\end{table}

\subsection{\label{sec:etlf-readout-ports}Readout ports}
Figure\,\ref{fig:dof-readouts} shows some available readout ports for \ETLF{} where the sidebands and carrier may be measured for the purposes of control:
\begin{itemize}
  \item \textbf{REFL} (\emph{reflected}) senses the light reflected from the interferometer back towards the input laser.
  \item \textbf{POP} (\emph{pick off \gls{PRCL}}) senses the light in the power recycling cavity. Although the diagram shows a pick-off mirror in the cavity, in reality this will probably be obtained via a secondary mirror reflection (such as the anti-reflective coating on the beam splitter).
  \item \textbf{AS} (\emph{asymmetric}) senses the light at the output port of the \DRFPMI{}.
\end{itemize}

\begin{figure}
  \centering
  \includegraphics[width=\columnwidth]{graphics/generated/from-svg/70-dof-readouts.pdf}
  \caption[Some available readout ports for sensing and control in \ETLF{}]{\label{fig:dof-readouts}Some available readout ports for sensing motion of the degrees of freedom in \ETLF{}. Phase modulation sidebands are imparted upon the input laser light at two primary frequencies before it enters the interferometer. Photodetectors are placed at three ports to sense the carrier and these sidebands for the purposes of sensing and control. The most important readout is \AS{}, which strongly senses \gls{DARM}, and this is placed in transmission of the signal recycling mirror. Light reflected back towards the laser is sensed via a Faraday isolator at the \REFL{} port, and a pick-off\textemdash \POP{}\textemdash senses light in the power recycling cavity.}
\end{figure}

For the purposes of the control simulations, each readout port contains incident fields at the carrier frequency and offsets of $\pm f_1$ (\SI{\pm11}{\mega\hertz}), $\pm f_2$ (\SI{\pm57}{\mega\hertz}), $-f_1 - f_2$ (\SI{-68}{\mega\hertz}), $f_1 - f_2$ and $-f_2 + f_1$ (\SI{-45}{\mega\hertz}), $-f_1 + f_2$ and $f_2 - f_1$ (\SI{45}{\mega\hertz}), and $f_1 + f_2$ (\SI{68}{\mega\hertz}). There are therefore \num{9} light fields propagated through the interferometer that are combined into signals at \num{5} frequencies at \num{3} ports. We furthermore demodulate with both \num{0} and \SI{90}{\degree} phases (termed \emph{I} and \emph{Q} quadratures, respectively) with respect to the excitation to be able to calculate the optimal readout phase at each port given the propagation delay between each mirror and each sensor. These are shown in Table\,\ref{tab:et-lf-probes}.

\begin{table}
  \centering
  {\renewcommand{\arraystretch}{1.2} % for extra vertical spacing between rows
    \begin{tabular}{r|ccc}
      \textbf{Frequency offset} & \textbf{Output port} & \textbf{Power recycling cavity pick-off} & \textbf{Reflected light pick-off} \\
      \hline
      \num{0} & \ASDC{} & \textemdash & \textemdash \\
      $\pm f_1$    & \ASFIRSTI{}, \ASFIRSTQ{} & \POPFIRSTI{}, \POPFIRSTQ{} & \REFLFIRSTI{}, \REFLFIRSTQ{} \\
      $\pm f_2$    & \ASSECONDI{}, \ASSECONDQ{} & \POPSECONDI{}, \POPSECONDQ{} & \REFLSECONDI{}, \REFLSECONDQ{} \\
      $\pm \left( f_2 - f_1 \right)$ & \ASDIFFI{}, \ASDIFFQ{} & \POPDIFFI{}, \POPDIFFQ{} & \REFLDIFFI{}, \REFLDIFFQ{} \\
      $\pm \left( f_1 + f_2 \right)$ & \ASSUMI{}, \ASSUMQ{} & \POPSUMI{}, \POPSUMQ{} & \REFLSUMI{}, \REFLSUMQ{}
    \end{tabular}
  }
  \caption[Probes that sense the light fields propagating within \ETLF{}]{\label{tab:et-lf-probes}Probes that sense the light fields propagating within \ETLF{} with respect to the carrier frequency. The carrier is sensed by \ASDC{} at the interferometer's output port, which is the light that propagates through the signal recycling mirror. The control sidebands, and the beats between the sidebands, are sensed at the same port demodulated at each relevant frequency, along with similar readouts sensing a small transmission of light through a folding mirror in the power recycling cavity (\POP{}) and the light reflected from the interferometer (\REFL{}).}
\end{table}

\subsection{Driving coefficients}
In order to calculate the response of each degree of freedom as defined in Section\,\ref{sec:dofs-of-drfpmi}, we must define the mirrors that we will actuate upon to create this motion. \gls{CARM} and \gls{DARM} are defined in terms of the motion of the \glspl{ETM}, with the difference between the two degrees of freedom being whether the motion of the two \gls{ETM} is in-phase or out-of-phase, respectively. The \gls{PRCL} and \gls{SRCL} degrees of freedom are defined as motion of the power and signal recycling mirrors, respectively. \gls{MICH} is a little more tricky, as simply moving the \glspl{ITM} results in a signal arising from the arms. We can instead either move the beam splitter and the recycling mirrors differentially, or both the \gls{ITM} and \gls{ETM} in each arm together, and differentially between the two cavities. We choose the latter to avoid the need to correct the beam splitter's motion for its angle of incidence. Note that in this particular case the driving coefficients appear at first glance to be for common Michelson motion, but this is due to the fact that the cavity mirrors' primary surfaces are facing each other and so driving must be applied with opposite sign to each \gls{ETM} and \gls{ITM} pair. The driving coefficients are shown in Table\,\ref{tab:et-lf-driving-coefficients}.

\begin{table}
  \centering
  \begin{tabular}{r|ccccccc}
    & \textbf{\gls{ITM} X} & \textbf{\gls{ETM} X} & \textbf{\gls{ITM} Y} & \textbf{\gls{ETM} Y} & \textbf{\gls{BS}} & \textbf{\gls{PRM}} & \textbf{\gls{SRM}} \\
    \hline
    \gls{CARM} & \num{0} & \num{0.5} & \num{0} & \num{0.5} & \num{0} & \num{0} & \num{0} \\
    \gls{DARM} & \num{0} & \num{0.5} & \num{0} & \num{-0.5} & \num{0} & \num{0} & \num{0} \\
    \gls{MICH} & \num{-0.5} & \num{0.5} & \num{0.5} & \num{-0.5} & \num{0} & \num{0} & \num{0} \\
    \gls{PRCL} & \num{0} & \num{0} & \num{0} & \num{0} & \num{0} & \num{1} & \num{0} \\
    \gls{SRCL} & \num{0} & \num{0} & \num{0} & \num{0} & \num{0} & \num{0} & \num{1} \\
  \end{tabular}
  \caption[Driving coefficients for each mirror and each degree of freedom of \ETLF{}]{\label{tab:et-lf-driving-coefficients}Driving coefficients for each mirror and each degree of freedom of \ETLF{}. \gls{CARM} and \gls{DARM} involve the driving of the \glspl{ETM} in- and out-of-phase, respectively. \gls{MICH} involves moving the \glspl{ITM} differentially, but to avoid sensing \gls{DARM} effects the \glspl{ETM} must be moved too. In this case, the \glspl{ETM} must be driven with an opposite sign to their respective \glspl{ITM} due to the fact that the mirrors' high-reflective surfaces are in opposite directions. \gls{PRCL} and \gls{SRCL} involve moving the power and signal recycling mirrors, respectively.}
\end{table}

\subsection{Control signals}
The response of each degree of freedom of the \DRFPMI{} to each of the readout ports at a certain frequency can be represented as a sensing matrix. This representation provides a clear view of the best readout ports to use for each degree of freedom, including the effect that other degrees of freedom would have on a particular degree of freedom's readout. We calculate the matrix by exciting each degree of freedom with driving coefficients shown in Table\,\ref{tab:et-lf-driving-coefficients} and taking transfer functions from these degrees of freedom to each probe shown in Table\,\ref{tab:et-lf-probes}. The sensing matrix therefore shows the gradient of the error signal for each degree of freedom as a function of their respective lengths.

At this early stage, a reasonable choice of error signals from the sensing matrix to use for control can be determined through a heuristic approach building upon knowledge gained from the control of \ALIGO{}. As $f_2$ resonates in both recycling cavities, it samples the motion of the mirrors that influence \gls{MICH} and \gls{SRCL} as well as \gls{PRCL}. Conversely, $f_1$ only strongly samples the motion of \gls{MICH} and \gls{PRCL}. Using demodulations at these frequencies, and combinations thereof, it should be possible to find a set of reasonably decoupled error signals for each degree of freedom.

\subsubsection{Combination of readout quadratures}
As the $I$ and $Q$ probes shown in Table\,\ref{tab:et-lf-probes} represent the same signals separated but demodulated with a phase difference of \SI{90}{\degree}, they can be combined electronically to produce an error signal with optimal gradient. The exact phase corresponding to the greatest magnitude is not important, as this can be influenced by technical factors such as the length of \gls{RF} transmission lines, but the relative phase between maximum error signals from different degrees of freedom on the same sensor is important. If error signals from different degrees of freedom have the same maximum signal at the same or opposite phase, one cannot be minimised with respect to the other through appropriate choice of demodulation phase. On the other hand, if two degrees of freedom couple to a pair of $I$ and $Q$ sensors with equal magnitude but separate phase, they can be used to sense both degrees of freedom. In practice due to temperature drifts and other time-varying effects it is difficult to maintain the demodulation phase of a set of sensors to a precision better than around \SI{1}{\degree} \cite{Effler2014}, and so the readout quadratures chosen for each of the degrees of freedom of \ETLF{} should ideally be separated by many degrees of phase.

\subsubsection{Sensing matrix for ET-LF in detuned configuration}
The sensing matrix for the detuned configuration given the readout ports defined in Table,\ref{tab:et-lf-probes} is shown in Table\,\ref{tab:et-lf-sensing-matrix-detuned} for perturbations at \gls{DC}. The suggested readouts for each degree of freedom are highlighted in bold and are described in the following text.

The gravitational wave signal will primarily affect \gls{DARM}, and this is by design sensed by \ASDC{}, the \gls{DC} readout. The common mode loop can be sensed at \REFLFIRST{} for the purposes of frequency stabilisation and very low frequency seismic noise that imparts common motion on both arms.

The \gls{MICH}, \gls{PRCL} and \gls{SRCL} cavity error signals are difficult to separate due to the cross-coupling of control sidebands via the Schnupp asymmetry and the sideband asymmetry created by the presence of a detuned signal recycling cavity \cite{Hild2007}. As the control system will be implemented in a \LIGO{} \gls{CDS}-style system \cite{Bork2010}, it will be possible to define in software error signals formed from linear combinations of different sensor signals that decouple other degrees of freedom from a particular readout. In Table\,\ref{tab:et-lf-sensing-matrix-detuned}, \gls{PRCL} is dominant over \gls{MICH} and \gls{SRCL} at \POPFIRST{} and represents a good extraction point for the motion of the power recycling cavity. \gls{MICH} couples strongly to a number of ports but alongside strong signals from the other degrees of freedom. From Equations \ref{eq:prcl-length} and \ref{eq:srcl-length} we see that \gls{PRCL} and \gls{SRCL} readouts contain a \gls{MICH} component, and as it is differential motion it is also seen by readouts sensitive to \gls{DARM} suppressed by the difference in finesse. Its strongest ports, \ASDC{} and \ASSECOND{}, contain much larger signals from \gls{DARM} which would be difficult to remove electronically. A better readout could be to use \POPSECOND{} with the dominant \gls{PRCL} signal suppressed with careful tuning of the demodulation phase. Residual \gls{PRCL} coupling can be subtracted electronically.

\gls{SRCL} is difficult to sense with $f_1$ because it is not resonant in the signal recycling cavity, nor $f_2$ because its error signal is dominated by those of \gls{MICH} or \gls{PRCL} at all ports. A possible sensing strategy could utilise the beats between sidebands found at \REFLDIFF{} or \REFLSUM{}, where the contribution from \gls{MICH} and \gls{PRCL} can be suppressed with suitable choice of demodulation phase. All \gls{SRCL} signals contain an offset from zero due to the detuning, \checkme{since the lower sideband is not resonant when the signal recycling mirror is at its operating point} (see Figure\,\ref{fig:sideband-powers-srcl-detuned}). To use \REFLSUM{} for \gls{SRCL} control an offset of \SI{-4.2}{\milli\watt} will be necessary.
% or is there an offset because the carrier is not nulled at the operating point?

\begin{table}
  \centering
  \resizebox{16cm}{!}{%
    {\renewcommand{\arraystretch}{1.2} % for extra vertical spacing between rows
      \begin{tabular}{r|ccccc}
	& \textbf{CARM} & \textbf{DARM} & \textbf{MICH} & \textbf{PRCL} & \textbf{SRCL} \\ 
	\hline
	\hline
	\textbf{\ASDC{}} & \num{1.44e+06} (\SI{180.00}{\degree}) & \textbf{\num{4.99e+08} (\SI{180.00}{\degree})} & \num{1.17e+06} (\SI{180.00}{\degree}) & \num{1.21e+05} (\SI{180.00}{\degree}) & \num{1.39e+04} (\SI{180.00}{\degree}) \\ 
	\textbf{\ASFIRST{}} & \num{2.25e+06} (\SI{-32.56}{\degree}) & \num{3.00e+07} (\SI{55.83}{\degree}) & \num{6.25e+04} (\SI{56.09}{\degree}) & \num{7.54e+03} (\SI{77.99}{\degree}) & \num{1.09e+03} (\SI{58.46}{\degree}) \\ 
	\textbf{\ASSECOND{}} & \num{1.76e+08} (\SI{-112.30}{\degree}) & \num{3.98e+08} (\SI{-162.68}{\degree}) & \num{8.39e+05} (\SI{-163.43}{\degree}) & \num{6.03e+05} (\SI{-78.31}{\degree}) & \num{6.73e+04} (\SI{94.85}{\degree}) \\ 
	\textbf{\ASDIFF{}} & \num{4.71e+04} (\SI{178.86}{\degree}) & \num{3.52e+03} (\SI{107.72}{\degree}) & \num{3.43e+04} (\SI{85.09}{\degree}) & \num{6.44e+04} (\SI{61.41}{\degree}) & \num{3.41e+03} (\SI{-146.01}{\degree}) \\ 
	\textbf{\ASSUM{}} & \num{1.00e+05} (\SI{-177.46}{\degree}) & \num{4.43e+04} (\SI{-83.81}{\degree}) & \num{5.97e+04} (\SI{43.44}{\degree}) & \num{5.13e+04} (\SI{103.97}{\degree}) & \num{4.67e+03} (\SI{-22.78}{\degree}) \\
	\hline
	\textbf{\POPFIRST{}} & \num{7.61e+07} (\SI{-144.47}{\degree}) & \num{6.88e+05} (\SI{-144.49}{\degree}) & \num{8.70e+02} (\SI{-26.41}{\degree}) & \textbf{\num{2.55e+05} (\SI{-36.40}{\degree})} & \num{3.08e+01} (\SI{-81.49}{\degree}) \\ 
	\textbf{\POPSECOND{}} & \num{5.27e+07} (\SI{99.38}{\degree}) & \num{4.72e+05} (\SI{92.38}{\degree}) & \textbf{\num{2.95e+04} (\SI{50.72}{\degree})} & \num{1.26e+05} (\SI{-130.50}{\degree}) & \num{1.08e+04} (\SI{120.38}{\degree}) \\ 
	\textbf{\POPDIFF{}} & \num{2.01e+02} (\SI{-51.56}{\degree}) & \num{9.34e+01} (\SI{-127.43}{\degree}) & \num{1.24e+03} (\SI{1.97}{\degree}) & \num{1.61e+04} (\SI{88.80}{\degree}) & \num{1.07e+03} (\SI{167.69}{\degree}) \\ 
	\textbf{\POPSUM{}} & \num{4.94e+02} (\SI{-22.60}{\degree}) & \num{9.93e+01} (\SI{-15.51}{\degree}) & \num{1.07e+03} (\SI{45.23}{\degree}) & \num{8.34e+03} (\SI{-145.33}{\degree}) & \num{1.07e+03} (\SI{-149.81}{\degree}) \\
	\hline
	\textbf{\REFLFIRST{}} & \textbf{\num{1.44e+10} (\SI{-0.01}{\degree})} & \num{1.31e+08} (\SI{-0.01}{\degree}) & \num{2.34e+05} (\SI{4.53}{\degree}) & \num{5.01e+07} (\SI{-0.41}{\degree}) & \num{2.25e+04} (\SI{45.05}{\degree}) \\ 
	\textbf{\REFLSECOND{}} & \num{4.63e+09} (\SI{-75.19}{\degree}) & \num{5.69e+07} (\SI{-86.52}{\degree}) & \num{1.47e+05} (\SI{-83.24}{\degree}) & \num{1.58e+07} (\SI{-76.05}{\degree}) & \num{1.67e+04} (\SI{30.68}{\degree}) \\ 
	\textbf{\REFLDIFF{}} & \num{3.90e+06} (\SI{0.94}{\degree}) & \num{1.31e+06} (\SI{0.04}{\degree}) & \num{2.19e+05} (\SI{163.21}{\degree}) & \num{1.28e+06} (\SI{-171.17}{\degree}) & \num{2.03e+05} (\SI{-25.15}{\degree}) \\
	\textbf{\REFLSUM{}} & \num{4.04e+06} (\SI{-101.97}{\degree}) & \num{1.32e+06} (\SI{-101.79}{\degree}) & \num{2.25e+05} (\SI{-76.80}{\degree}) & \num{1.40e+06} (\SI{-53.54}{\degree}) & \textbf{\num{2.03e+05} (\SI{89.52}{\degree})}
      \end{tabular}
    }
  }
  \caption[Gradients of the error signals from each degree of freedom to each probe in \ETLF{} at dc]{\label{tab:et-lf-sensing-matrix-detuned}Gradients of the error signals from each degree of freedom to each probe, in units of \SI{}{\watt\per\meter}, in \ETLF{} at \gls{DC}. The suggested readout probes for each degree of freedom are shown in bold. The $I$ and $Q$ quadratures of each readout have been combined into a single magnitude and the phase representing the greatest slope and the phase at which it is achieved. The \ASDC{} does not get demodulated but it has phase of \SI{0}{\degree} or \SI{180}{\degree} depending on the sign of the error signal. The \gls{RF} probes contain maximum slopes at phase angles determined by the propagation of the control sidebands through the interferometer. Probes can be optimised to sense the motion of a particular degree of freedom by adjusting the angle at which the sensor $I$ and $Q$ quadratures are combined, but signals on sensors that contain strong signals from other degrees of freedom at nearby phase angles are difficult to use. \note{Does the AS DC readout also need an offset applied, like SRCL?}}
\end{table}

Error signals corresponding to the suggested readouts for each degree of freedom are shown in Figure\,\ref{fig:sweeps-et-lf}. These are produced by calculating the power on each sensor at the relevant demodulation frequency as the mirrors are driven as shown in Table\,\ref{tab:et-lf-driving-coefficients} and represent the low-frequency limit of the transfer functions of each degree of freedom to each sensor.

\begin{figure}
  \centering
  \includegraphics[width=\columnwidth]{graphics/generated/from-python/70-sweeps-detuned.pdf}
  \caption[Sweeps through the zero-crossings of the chosen error signals in ET-LF]{\label{fig:sweeps-et-lf}Sweeps through the zero-crossings of the chosen error signals in \ETLF{} in the detuned configuration. Each error signal is linear about the operating point, which ensures a simple, bipolar error signal is available for the purposes of controlling each associated degree of freedom. This linearity has a different range for each readout, with \gls{CARM} and \gls{DARM} requiring the greatest precision.}
\end{figure}

\checkme{Figure\,B} shows the \ASDC{} readout, and the slope is zero when the arm cavities are tuned, consistent with Figure\,\ref{fig:total-power-vs-darm-offset-detuned}, showing that some classical carrier light power is required for \gls{DC} readout. Figure\,E shows the error signal crossing the operating point at a power of \SI{4.2}{\milli\watt}, showing the requirement for an offset.

\subsection{Sensitivity of the scheme}
The sensitivity for \ETLF{} shown in Figure\,\ref{fig:et-d-sensitivity} assumes that the interferometer contains squeezed vacuum input via two filter cavities in addition to the presence of seismic and other noise, and these features have not been modelled in this work. The ET-D study presented the quantum noise limited sensitivity of the interferometer in the absence of squeezing, and so a comparison is shown in Figure\,\ref{fig:et-lf-control-scheme-sensitivity} between this curve and the quantum noise limited sensitivity of the \ASDC{} readout in this scheme modelled with Optickle. Also shown is the curve generated with identical parameters in \emph{Finesse} (see Appendix\,\ref{sec:finesse-sim}), with excellent agreement. This shows that the choice of parameters in this work do not negatively impact upon the design sensitivity of the interferometer. The difference in sensitivity in the region of the cavity pole (\SI{7}{\hertz}) and the optical spring from the detuned signal recycling cavity (\SI{25}{\hertz}) is due to the assumed \gls{DARM} offset and the difference in the way in which quantum noise calculations are implemented between the tools.

\begin{figure}
  \centering
  \includegraphics[width=\columnwidth]{graphics/generated/from-python/70-et-lf-control-scheme-sensitivity-curve.pdf}
  \caption[ET-LF quantum noise limited sensitivity using the conceptual control scheme]{\label{fig:et-lf-control-scheme-sensitivity}\ETLF{} quantum noise limited sensitivity using the conceptual control scheme, with no squeezed light injection. The reference curve from the ET-D study is shown next to the sensitivity calculated with the Optickle model developed in this chapter, and also a Finesse curve generated using identical parameters. The curves broadly agree, showing that the chosen parameters do not negatively impact upon the sensitivity, though a difference in the noise calculations and the \gls{DARM} offset assumed in this work creates the slight mismatch in the most sensitive region.}
\end{figure}

The parameters used in the proposed control scheme are shown in Table\,\ref{tab:et-lf-updated-parameters} alongside the pre-existing parameters from the design study. The proposed scheme is not optimal, and does not consider a number of other possibilities such as the use of secondary reflections arising from the anti-reflective coatings on the beam splitter and \glspl{ITM} or the use of higher order combinations of $f_1$ and $f_2$ such as \emph{3f} signals used in the lock acquisition sequence of \VIRGO{} \cite{Acernese2008} and \ALIGO{} \cite{Staley2014}. It serves, however, as a first concept for the control of \ETLF{} proving that it can in theory be controlled with \SI{25}{\hertz} signal recycling cavity detuning.

\begin{table}
  \centering
  \resizebox{16cm}{!}{%
    \begin{tabular}{r|c|cc}
      \textbf{Parameter} & \textbf{Symbol in text} & \textbf{Design study value} & \textbf{Updated value} \\
      \hline
      Laser wavelength & & \multicolumn{2}{c}{\SI{1550}{\nano\meter}} \\
      Input power & & \multicolumn{2}{c}{\SI{3}{\watt}} \\
      Arm power         & & \multicolumn{2}{c}{\SI{18}{\kilo\watt}} \\
      \gls{ITM} transmissivity & & \multicolumn{2}{c}{\SI{7000}{\ppm}} \\
      \gls{ETM} transmissivity & & \multicolumn{2}{c}{\SI{6}{\ppm}} \\
      \gls{PRM} transmissivity & $T_{\text{PRM}}$ & \multicolumn{2}{c}{\SI{4.6}{\percent}} \\
      \gls{SRM} transmissivity & & \multicolumn{2}{c}{\SI{10}{\percent}} \\
      Signal recycling detuning & & \multicolumn{2}{c}{\SI{0.6}{\radian}} \\
      Arm cavity lengths & $L_{\text{X}}$, $L_{\text{Y}}$ & \multicolumn{2}{c}{\SI{10}{\kilo\meter}} \\
      Power recycling cavity length & $L_{\text{PRCL}}$  & \multicolumn{2}{c}{\SI{310}{\meter}} \\
      Signal recycling cavity length & $L_{\text{SRCL}}$  & \SI{310}{\meter} & \SI{311.585}{\meter} \\
      Schnupp asymmetry & $\Delta l_{\text{SCH}}$  & \textemdash & \SI{0.08}{\meter} \\
      \gls{DARM} offset & $\delta L_{\text{DARM}}$  & \textemdash & \SI{12}{\pico\meter} \\
      Control sideband frequencies & $f_1$, $f_2$ & \textemdash & \SI{11363101}{\hertz}, \SI{56815505}{\hertz} \\
      Demodulation frequencies & & \textemdash & $f_1$, $f_2$, $f_2 - f_1$, $f_1 + f_2$
    \end{tabular}
  }
  \caption[Updated parameters for \ETLF{} in the detuned configuration following the development of the conceptual sensing scheme]{\label{tab:et-lf-updated-parameters}Updated parameters for \ETLF{} in the detuned configuration following the development of the conceptual sensing scheme in this chapter.}
\end{table}

\section{Outlook and future work}
By utilising an \gls{RF} phase modulation scheme we have shown in this chapter that \ETLF{} can in principle be controlled in its detuned configuration. Control noise issues remain unaddressed, as have more exotic control schemes such as the use of additional modulation frequencies or \emph{sub-carriers}. This section describes some future work that will be necessary to refine and improve the results presented here towards a comprehensive technical design.

\subsection{Optimising the sensing matrix}
In \ALIGO{} the conceptual control scheme was first tested at the \CALTECHFORTYM{} and it is probable that any technical design for \ETLF{} will require a similar test. At this stage a quantitative assessment of the performance that a particular control scheme might provide might take the form of a technique presented by Mantovani and Freise \cite{Mantovani2008} developed for alignment control in \VIRGO{}, but suitable for longitudinal control. This involves the calculation of a \emph{quality parameter} representing the controllability of a given set of sensors and degrees of freedom. This approach only makes sense when $M$ represents the interferometer at its operating point, which means that the residual motion of controlled degrees of freedom does not create a significant cross-coupling. This approach requires hierarchical gain to be simulated as part of a lock acquisition sequence, and so some effort will be required to design some feedback servos.

\subsection{Switching between tuned to detuned operation}
Transition from tuned to detuned signal recycling operating points and vice versa involves a technique which can maintain control of the interferometer as it transitions between two desired set points. When dual recycling was first demonstrated in suspended optics in the \GARCHINGPROTOTYPE{}, it involved a varying frequency offset applied to the \gls{RF} modulation sidebands as the tuning was changed \cite{Freise2000}. This control technique was evolved in \GEO{} where a complicated sequence of actions \cite{Grote2004} including an uncontrolled ``jump'' between two operating points \cite{Hild2007} were performed to reached tuned mode from a detuned start point. In \ETLF{} it is expected that the signal recycling cavity finesse will be too high to allow for a previously demonstrated transition scheme, and so investigations are ongoing to model the impact that combinations of phase- and amplitude-modulated control sidebands added to the input light or \emph{subcarriers} added to the squeezing injection port have on the lock acquisition and control of the signal recycling cavity at arbitrary detunings. Another possibility is to adapt the \emph{arm length stabilisation} system developed for the lock acquisition of \ALIGO{} \cite{Mullavey2012, Staley2014}, whereby a second carrier at a different wavelength is used to lock cavities. This takes advantage of the lower finesse of the second carrier's wavelength in the cavities, allowing for a wider locking range. The cavities are first pre-stabilised using this second carrier before the main carrier is brought to resonance.

\subsection{Sensing and control of seismic and gravity gradient noise}
The \ET{} facility will be located and designed to minimise the impact of seismic noise, but due to the sensitivity requirement for \ETLF{} the microseism must be suppressed from around \num{e-6} to \SI{e-8}{\meter\per\sqrthz} at frequencies between \num{0.1} and \SI{1}{\hertz} to below \SI{e-18}{\meter\per\sqrthz} by \SI{2}{\hertz}. This represents a signal difference of around \num{e10}, and current sensor electronics can typically only provide dynamic range of around $\SI{130}{\deci\bel} \approx \num{3e6}$. To control seismic noise in addition to being able to sense displacements at the required level, a sensor hierarchy will need to be developed.

There is some precedent for seismic isolation from the work carried out in current and past detectors. In \VIRGO{}, the seismic coupling in the superattenuator was suppressed with a local controller in addition to the global feedback from the main interferometric readouts \cite{Acernese2004}. In \ALIGO{}, seismic pre-isolation is performed through the use of a series of displacement and velocity sensors \cite{Stochino2009}, and in the \AEIPROTOTYPE{} the addition of a \emph{suspension platform interferometer} \cite{Gossler2010} is able to reduce seismic noise to the level of around \SI{100}{\pico\meter\per\sqrthz} between \num{0.1} and \SI{1}{\hertz} \cite{Dahl2010}. Such a system would allow a high dynamic range global sensor to control the remaining motion. The particularly challenging aspect for \ETLF{} is that the suppression of this motion must occur over a bandwidth below \SI{2}{\hertz}, which makes the implementation of stable control filters extremely challenging. The results from the control of the \AEIPROTOTYPE{} and the advanced detectors will provide input to the technical design of the seismic isolation system for \ETLF{}.