\chapter{\label{c:et-lf-control}Conceptual longitudinal sensing scheme for the low frequency Einstein Telescope detector}
\chaptermark{Conceptual longitudinal sensing scheme for ET-LF}

% sinks
\newcommand{\AS}{AS}
\newcommand{\POP}{POP}
\newcommand{\REFL}{REFL}

% readout ports, I and Q
\newcommand{\ASDC}{$\text{\AS}_{\text{DC}}$}
\newcommand{\ASFIRSTI}{$\text{\AS}_{\num{11}}^{\text{I}}$}
\newcommand{\ASFIRSTQ}{$\text{\AS}_{\num{11}}^{\text{Q}}$}
\newcommand{\ASSECONDI}{$\text{\AS}_{\num{57}}^{\text{I}}$}
\newcommand{\ASSECONDQ}{$\text{\AS}_{\num{57}}^{\text{Q}}$}
\newcommand{\ASSUMI}{$\text{\AS}_{\num{68}}^{\text{I}}$}
\newcommand{\ASSUMQ}{$\text{\AS}_{\num{68}}^{\text{Q}}$}
\newcommand{\ASDIFFI}{$\text{\AS}_{\num{45}}^{\text{I}}$}
\newcommand{\ASDIFFQ}{$\text{\AS}_{\num{45}}^{\text{Q}}$}
\newcommand{\POPDC}{$\text{\POP}_{\text{DC}}$}
\newcommand{\POPFIRSTI}{$\text{\POP}_{\num{11}}^{\text{I}}$}
\newcommand{\POPFIRSTQ}{$\text{\POP}_{\num{11}}^{\text{Q}}$}
\newcommand{\POPSECONDI}{$\text{\POP}_{\num{57}}^{\text{I}}$}
\newcommand{\POPSECONDQ}{$\text{\POP}_{\num{57}}^{\text{Q}}$}
\newcommand{\POPSUMI}{$\text{\POP}_{\num{68}}^{\text{I}}$}
\newcommand{\POPSUMQ}{$\text{\POP}_{\num{68}}^{\text{Q}}$}
\newcommand{\POPDIFFI}{$\text{\POP}_{\num{45}}^{\text{I}}$}
\newcommand{\POPDIFFQ}{$\text{\POP}_{\num{45}}^{\text{Q}}$}
\newcommand{\REFLDC}{$\text{\REFL}_{\text{DC}}$}
\newcommand{\REFLFIRSTI}{$\text{\REFL}_{\num{11}}^{\text{I}}$}
\newcommand{\REFLFIRSTQ}{$\text{\REFL}_{\num{11}}^{\text{Q}}$}
\newcommand{\REFLSECONDI}{$\text{\REFL}_{\num{57}}^{\text{I}}$}
\newcommand{\REFLSECONDQ}{$\text{\REFL}_{\num{57}}^{\text{Q}}$}
\newcommand{\REFLSUMI}{$\text{\REFL}_{\num{68}}^{\text{I}}$}
\newcommand{\REFLSUMQ}{$\text{\REFL}_{\num{68}}^{\text{Q}}$}
\newcommand{\REFLDIFFI}{$\text{\REFL}_{\num{45}}^{\text{I}}$}
\newcommand{\REFLDIFFQ}{$\text{\REFL}_{\num{45}}^{\text{Q}}$}

% readout ports, collapsed
\newcommand{\ASFIRST}{$\text{\AS}_{\num{11}}$}
\newcommand{\ASSECOND}{$\text{\AS}_{\num{57}}$}
\newcommand{\ASSUM}{$\text{\AS}_{\num{68}}$}
\newcommand{\ASDIFF}{$\text{\AS}_{\num{45}}$}
\newcommand{\POPFIRST}{$\text{\POP}_{\num{11}}$}
\newcommand{\POPSECOND}{$\text{\POP}_{\num{57}}$}
\newcommand{\POPSUM}{$\text{\POP}_{\num{68}}$}
\newcommand{\POPDIFF}{$\text{\POP}_{\num{45}}$}
\newcommand{\REFLFIRST}{$\text{\REFL}_{\num{11}}$}
\newcommand{\REFLSECOND}{$\text{\REFL}_{\num{57}}$}
\newcommand{\REFLSUM}{$\text{\REFL}_{\num{68}}$}
\newcommand{\REFLDIFF}{$\text{\REFL}_{\num{45}}$}

\section{The \ET{} facility}
In 2011 a group of scientists primarily based in Europe completed a design study \cite{ET2011} to examine the infrastructure requirements for a gravitational wave observatory that pushes the \MI{} topology to its limits, while making any newly built facility generic enough to allow for the implementation of new topologies as the state of the art evolves. In this study they laid out the expected improvements in technologies to mitigate fundamental and technical noise sources currently limiting the sensitivity of the second generation of detectors, with a number of practical differences to existing facilities.

The design for the \ET{} targets an increase in sensitivity over \ALIGO{} and \AVIRGO{} by a factor of \num{10} over a wide bandwidth. In order to surpass the sensitivity of the current detectors which already expect to improve upon that of the first generation by an order of magnitude, a number of differences and improvements have been envisaged. The proposed \ET{} facility composes six \DRFPMI{}s split between the three corners of a triangle with \SI{10}{\kilo\meter} edges. The design exploits the geometry to implement interferometers with \SI{10}{\kilo\meter} arms meeting at the three vertices to benefit from the colocation of multiple interferometers.

Seismic noise limits the sensitivity of current generation detectors below \SI{10}{\hertz}, and there are astrophysical advantages to being able to achieve good sensitivity at these frequencies \cite{Sathyaprakash2012}, particularly in the ability to see the inspiral and merger of high mass black hole binary coalescences. The majority of spinning neutron stars discovered via optical techniques have also had orbital frequencies below \SI{10}{\hertz} where fundamental noise limits the ability for the current generation of detectors to see such signals. Signals evolving with frequency, such as the binary black hole merger witnessed as \GWFIRSTEVENT{} \cite{Abbott2016} which was seen for a few tens of \SI{}{\milli\second}, are present at frequencies around \SI{2}{\hertz} for hours. Having this extra observation time provides the possibility not only to better estimate the source parameters but also to track the signal evolution with corresponding changes in the signal recycling cavity tuning to provide optimal sensitivity \cite{Heinzel2002, Simakov2014}. The plan for the \ET{} is to have sufficient sensitivity in this band to provide access to this new science; low frequency sensitivity is difficult to achieve in Earth-based detectors, however.

As discussed in Chapter\,\ref{c:speedmeter-control}, the noise influencing gravitational wave interferometers can be split into groups arising from the sensing of signals and from sources that directly influence the test mass displacement. Sources of sensing noise are in general independent of the arm length and test mass parameters and instead rely on the amount of light power and the readout scheme being used. The effect on the strain sensitivity of displacement noise sources, however, typically scales inversely to arm length\footnote{This is not because the displacement noise decreases with arm length, but because the response of the interferometer is improved.}; but because other detector parameters must be re-optimised each new arm length this scaling is non-trivial \cite{Dwyer2015}.

As an example of both the benefits and challenges that longer arms can create, consider the arm cavities within a \FPMI{}. For a given readout technique, longer arms provide better strain sensitivity as shown by Equation\,\ref{eq:gw-mod-depth}. Some sources of noise are also reduced with respect to the increased signal \cite{aligocosmic2016}: quantum shot noise scales, for a fixed detector bandwidth, as the square root of the ratio of the lengths such that an arm twice as long as another has a factor $\sqrt{\frac{1}{2}} = 0.7$ of the shot noise with respect to the signal; quantum radiation pressure noise reduces even further, proportionally to the power $\frac{3}{2}$. Coating thermal noise scales, for fixed beam size, inversely to the arm length. The width of a Gaussian beam also scales inversely to the arm length, however, so the beam spreads over a wider surface area leading to clipping loss unless the mirror surface area is also scaled. While larger beams provide a further improvement in coating thermal noise, it also requires significantly larger test masses given that an increase in surface area corresponds to a significantly larger increase in mass, especially considering that the aspect ratio of the test masses should be kept close to that of the test masses employed in current detectors due to thermal noise \cite{Somiya2009a}. With larger test masses and longer arms come more complex suspension design requirements, particularly in the alignment of the cavity test masses. Significant research and development is ongoing to determine designs to mitigate these issues for the next generation of detectors.

\subsection{New facility}
As discussed in Section\,\ref{sec:seismic-noise}, seismic noise in current detectors limits the sensitivity at low frequencies and creates challenging control requirements due to the mirror motion created by ground vibrations. The \ET{} interferometers will be \num{100} to \SI{200}{\meter} underground to mitigate seismic noise. This location also helps to limit the impact of gravity gradient noise, as discussed in Section\,\ref{sec:gravity-gradient-noise}, which is expected to become a problem as seismic noise is mitigated.

There are a number of benefits to having multiple detectors located in the same facility. The noise properties across the facility will have a similar impact and so it should be possible to combine the signals from each detector in such a way as to generate a \emph{null stream} that contains noise but not signal \cite{Hewitson2005, Ajith2006}; this will be useful for the characterisation of noise sources and will be particularly beneficial for the new noise sources that may be interrogated due to the increased sensitivity. The arrangement of three detectors in a triangle also allows the facility to be optimally sensitive to gravitational waves from all directions \cite{Winkler1985}, whereas existing single-interferometer detectors are sensitive only to incident signals in the plane of the detector. The use of multiple detectors also allows upgrades to be made to some of the interferometers without losing sky coverage.

\subsection{Xylophone configuration}
To provide maximum astrophysical reach the facility is intended to provide sensitivity across an unprecedented bandwidth, from around \SI{2}{\hertz} to \SI{10}{\kilo\hertz}\textemdash a bandwidth significantly larger than that of existing detectors. It was realised that the most technically feasible option to obtain this bandwidth would be to implement two different types of detector, each optimised to provide good sensitivity in either low or high frequencies \cite{Hild2010}, an idea first proposed for \ALIGO{} \cite{Conforto2004}. In the proposed \emph{ET-D} configuration \cite{Hild2011}, a low power, cryogenic interferometer optimises sensitivity to reduce radiation pressure noise at the expense of shot noise, whilst a high power interferometer optimises high frequency sensitivity through the reduction of shot noise. The projected sensitivity of this arrangement is shown in black in Figure\,\ref{fig:et-d-sensitivity}.

% data from http://www.et-gw.eu/etsensitivities
\begin{figure}
  \centering
  \includegraphics[width=\columnwidth]{graphics/generated/from-python/70-et-d-sensitivity-curves.pdf}
  \caption[Sensitivity curves for the Einstein Telescope]{\label{fig:et-d-sensitivity}Sensitivity of the \ET{} detectors, based on the ET-D design \cite{Hild2011}. \ETLF{} is optimised for low frequencies, \ETHF{} is optimised for high frequencies, and the combination yields sensitivity between \SI{2}{\hertz} and \SI{10}{\kilo\hertz}.}
\end{figure}

\subsubsection{ET-LF}
The low frequency detector consists of a \DRFPMI{} configuration as introduced in Section\,\ref{sec:signal-recycling}, but with a detuned signal recycling cavity. This detuning allows for enhanced sensitivity at the signal recycling cavity pole where the optomechanical dynamics create an optical spring that provides sensitivity below the \gls{SQL} at the spring frequency. The cavities will have \SI{18}{\kilo\watt} of light power, which is considerably lower than that of \ALIGO{} at design sensitivity (\SI{800}{\kilo\watt}), leading to reduced quantum radiation pressure noise at low frequencies. Cryogenic test masses are to be used to facilitate a reduction in thermal noise, and the wavelength of the carrier will be changed from the standard \SI{1064}{\nano\meter} to \SI{1550}{\nano\meter} to utilise lower noise materials at such temperatures. Two filter cavities to facilitate frequency dependent squeezing for the further suppression of quantum noise are also included.

The suspension systems for the main test masses are based on those of the \emph{superattenuator} in \VIRGO{} \cite{Acernese2010}. The proposed \SI{17}{\meter} long pendulum system pushes the longitudinal resonant frequency down from around \SI{1}{\hertz} in existing detectors to \SI{170}{\milli\hertz}, providing better attenuation of seismic noise above \SI{2}{\hertz}.

The sensitivity of \ETLF{} is shown in blue in Figure\,\ref{fig:et-d-sensitivity}. The sensitive frequency band is between around \SI{2}{\hertz} and \SI{200}{\hertz}.

\subsubsection{ET-HF}
\ETHF{} takes the designs of \ALIGO{} and \AVIRGO{} and assumes improvements to the test mass coating loss, substrate absorption and available input laser power based on expectations for future research, and adds new technologies such as LG33 cavity modes \cite{Carbone2013} and frequency dependent squeezing \cite{Kimble2001} to reduce coating thermal and quantum noise. The combination of greater arm cavity power, heavier test masses, squeezing and improved coatings and materials will increase sensitivity at frequencies above a few hundred \SI{}{\hertz} beyond the current generation by a factor of around \num{10}, as shown in the \gls{ET} design study.
% sensitivity w.r.t. ALIGO is shown in Figure 5, p14 of design study

The sensitivity of \ETHF{} is shown in orange in Figure\,\ref{fig:et-d-sensitivity}.

\section{\label{sec:et-lf-control-challenges}Control challenges with the \ET{}}
Both \ETLF{} and \ETHF{} will present new challenges to the control of large-scale \DRFPMI{}s. Although \ETHF{} can to some extent be seen as a larger version of \ALIGO{} and \AVIRGO{}, and it may therefore be possible to adapt much of the advance detectors' strategies for both longitudinal and angular control, some aspects such as the use of LG33 modes on a large scale and the presence of parametric instabilities at such high arm cavity powers \cite{Evans2015} require extensive research to understand the implications they may have on control. \ETLF{} will also use a topology that resembles existing generation detectors, but it pushes the sensitivity at low frequencies further down and this presents additional challenges with sensing and noise. This chapter will discuss the longitudinal control of \ETLF{}, focusing in particular on the challenge of controlling the interferometer in its detuned state. The following subsection discusses the various configurations for tuned and detuned signal recycling.

\subsection{Signal recycling and resonant sideband extraction}
As introduced in Section\,\ref{sec:signal-recycling}, signal recycling cavities can be used to enhance the sensitivity of a detector over a particular frequency band \cite{Hild2006}. \emph{Tuned} signal recycling involves holding the signal recycling cavity resonant by ensuring that the carrier light's transmitted phase difference is zero, giving an enhancement below the signal recycling cavity's pole frequency. \emph{Detuned} signal recycling, meanwhile, involves configuring the signal recycling cavity's tuning in such a way as to provide the greatest sensitive bandwidth, by choosing to make the signal recycling cavity resonant for one of the signal sidebands instead of the carrier. This involves detuning the phase of the signal recycling mirror as seen by the carrier, $\phi$, within the range $0 < \phi < \frac{\pi}{4}$ \cite{Somiya2005}. Signal recycling techniques enhance the storage time of the signal within the interferometer. Another technique for changing the response of the interferometer is through the use of \emph{resonant sideband extraction} \cite{Mason2003}. This is typically used alongside topologies with arm cavities and the purpose is to decrease the storage time of the signal within the interferometer. The signal recycling cavity in this case is instead called the signal \emph{extraction} cavity. \emph{Tuned} resonant sideband extraction broadens the response of the interferometer beyond the bandwidth of the arm cavities, achieved by making the carrier anti-resonant ($\phi = \pi$) within the signal extraction cavity. The signal extraction cavity is then less reflective than the arm cavity \glspl{ITM}, effectively reducing the arm cavity finesse for the signal and therefore increasing the sensitive bandwidth. \emph{Detuned} resonant sideband extraction in contrast to the tuned variety provides greatest sensitivity at a non-zero signal frequency at which neither the arms nor the signal extraction cavity are resonant, and can be used to enhance the interferometer's sensitivity for a particular source. This involves tuning the signal extraction cavity slightly off anti-resonance, within the range $\frac{\pi}{4} < \phi < \pi$ \cite{Somiya2005}.
% See ``Development of a frequency-detuned interferometer as a prototype experiment for next-generation gravitational-wave detectors, https://core.ac.uk/download/pdf/4876334.pdf, for a very good comparison of SR vs RSE

Detuned signal recycling and resonant sideband extraction techniques involve the use of radiation pressure induced dynamics. So-called \emph{optical springs} are created when high laser power encounters optical cavities detuned from resonance, and the resulting optomechanical interactions can result in enhanced sensitivity at the spring frequency \cite{Buonanno2002}.

\subsection{Resonant sideband extraction in ET-LF}
The ability for a detuned signal recycling cavity to shift the most sensitive frequency has been demonstrated in \GEO{} between \SI{200}{\hertz} and \SI{1}{\kilo\hertz} \cite{Hild2006}, however the plan for \ETLF{} is to use detuned resonant sideband extraction with a detuning of around \SI{25}{\hertz}, a feat that has not been achieved before in a suspended audio-band detector. With tuned techniques, the sidebands used for control of the differential arm cavity mode are present within the signal recycling cavity with equal amplitude. In detuned operation, where the cavity is not resonant for the carrier, the sidebands have unequal amplitude and some of the interferometer's noise couplings that otherwise cancel at the output in the tuned case no longer cancel \cite{Hild2007}. The phase modulation of the control sidebands, created by \glspl{EOM} on the input path, also gets partially converted to amplitude modulation by the detuning, and this can lead to issues with the dynamic range of any photodetectors used to sense the readout \cite{Grote2007}. Often it can be difficult to find a port at which to sense the motion of the signal recycling cavity decoupled from other cavities, and this effect can be exacerbated by large detunings such as in \ETLF{} where noise cross-couplings can become more significant \cite{Hild2007}.

In the \gls{ET} design study the discussion for \ETLF{} stopped short of a control scheme. The rest of this chapter will discuss some control concepts, present a conceptual approach to the control of \ETLF{} and highlight the future work that must be undertaken before a technical design for the control of \ETLF{} can be produced.

\section{\label{sec:multi-dof-control}Longitudinal control of a \DRFPMI{}}
\sectionmark{Longitudinal control of a dual-recycled Fabry-Perot Michelson}
A successful control scheme for an interferometer must satisfy a number of requirements. When the interferometer is in its uncontrolled state, the control scheme must be able to bring it to the \emph{operating point} where it has the desired response to incident gravitational wave signals in a process called \emph{lock acquisition}. Once at the operating point, it must be robust against small perturbations by controlling the impact of noise and signal nonstationarities. Finally, the signal that represents the gravitational wave channel must have low noise, and therefore high sensitivity, to meet the design goals of the scheme.

The lock acquisition scheme is inextricably linked to the technical environment in which the interferometer will operate, and so it is inappropriate to discuss this while \gls{ET}'s technical design is subject to ongoing research. We will focus our efforts, therefore, on the second and third challenges above.

\subsection{The dc readout technique}
The standard readout technique for \GEO{} and the current generation detectors is \gls{DC} readout \cite{Hild2007, Ward2008, Fricke2012}, and the plan is for \ETLF{} and \ETHF{} to continue to use it. This technique is a form of homodyne readout that involves a compromise between the best sensitivity and technical complexity. The operating point is kept close to the \emph{dark fringe} (see Section\,\ref{sec:operating-point}) to optimise shot noise and reduce the coupling of technical noise, but a slight offset is introduced between the differential round trip phase of each arm in order to allow some of the carrier light to enter the readout port where it acts as a homodyne local oscillator to the signal sidebands (see Section\,\ref{sec:homodyne-readout}). In practice, this detuning\textemdash of the order \SI{}{\pico\meter} at the arm cavities in \ALIGO{}\textemdash is sufficiently small to prevent the sensitivity from being diminished. The light that does enter the output port is, however, pre-filtered by the arm cavities and the impact of laser noise is suppressed by the differential detuning. As the source of local oscillator is a fraction of the light from the cavities to be controlled with the readout, the stability of the local oscillator is guaranteed, as opposed to having a separate degree of freedom to control as with other homodyne techniques. The sensitivity of this readout is furthermore improved with respect to heterodyne techniques \cite{Fricke2012}, and in squeezed interferometers it avoids the need to inject squeezing at \gls{RF} sideband frequencies in addition to the carrier frequency.

\subsection{\label{sec:dofs-of-drfpmi}Degrees of freedom}
The \emph{degrees of freedom} of an interferometer are the non-degenerate ways in which the interferometer's mirrors may move away from the operating point. Each degree of freedom has a different precision requirement, with the most stringent typically being the degree of freedom corresponding to the gravitational wave channel. With \gls{DC} readout there is no local oscillator phase or alignment to independently control and so in a \DRFPMI{} the main degrees of freedom to consider are the arm cavity differential and common modes, the length between the beam splitter and the \glspl{ITM} and the recycling cavity lengths\footnote{Note that throughout this chapter we refer to the signal extraction cavity as a \emph{recycling} cavity to fit with convention.}. These are defined in the following sections and the lengths that compose each degree of freedom are shown in Figure\,\ref{fig:et-lf-cavity-lengths}.

\begin{figure}
  \centering
  \includegraphics[width=0.6\columnwidth]{graphics/generated/from-svg/70-cavity-lengths.pdf}
  \caption[Cavity lengths in a \DRFPMI{}]{\label{fig:et-lf-cavity-lengths}Cavity lengths in a \DRFPMI{}. The differential and common arm cavity modes compose $L_{\text{X}}$ and $L_{\text{Y}}$, while the auxiliary power and signal recycling and Michelson cavity lengths are composed of a subset of the distances between the beam splitter and each \gls{ITM} and the distances between each recycling mirror and the beam splitter.}
\end{figure}

The motion of each degree of freedom must be witnessed by a sensor and fed back to actuators to control the relevant length. This is called \emph{linear negative feedback} and the concept of shaping feedback dynamics is discussed in greater detail in the context of suspended interferometers in Chapter\,\ref{c:speedmeter-control}.

\subsubsection{Differential arm cavity length}
Gravitational waves change the length of the arms in a \MI{} differentially and due to the presence of the arm cavities in a \DRFPMI{} changes to the differential arm cavity length represents the motion the main readout is most sensitive to. This length is held close to the dark fringe but with a small offset for \gls{DC} readout. The differential arm cavity length signal can be fed back to the \glspl{ETM} differentially to hold the length at the desired operating point.

We define the differential arm cavity length, \emph{\gls{DARM}}, in terms of the average differential length of the arms:
\begin{equation}
  \label{eq:darm-length}
  \delta L_{\text{DARM}} = \frac{L_{\text{X}} - L_{\text{Y}}}{2}.
\end{equation}

\subsubsection{Common arm cavity length}
In-phase changes in the length of the arms of a \MI{} do not primarily couple to the gravitational wave channel, but it is crucial to control this degree of freedom in order to keep the arm cavities at their operating point which, in a \DRFPMI{}, is the state in which the power in the arm cavities is held near maximum. Noise common to the light entering both arms, particularly from the laser's amplitude and frequency fluctuations as discussed in Section\,\ref{sec:laser-noise}, can, unless corrected, change the resonant condition in the arms. Due to its speed a convenient actuator to control common arm length changes is the laser's crystal, whereupon the application of strain changes the laser's frequency as shown by Equation\,\ref{eq:freq-to-length}. It is typical to use feedback to an \gls{EOM} for corrections above around \SI{100}{\kilo\hertz}. It is also possible to correct slower, larger drifts with common feedback to the \glspl{ETM} or input optics.

We can define the common arm cavity length, \emph{\gls{CARM}}, in terms of the average length of the arms:
\begin{equation}
  \delta L_{\text{CARM}} = \frac{L_{\text{X}} + L_{\text{Y}}}{2}.
\end{equation}

Any arm cavity length change can be expressed in terms of a linear combination of \gls{DARM} and \gls{CARM}.

\subsubsection{Power and signal recycling cavity lengths}
The power recycling cavity should be resonant for the input light in order to optimally recycle light reflected from the beam splitter back towards the laser, to allow the arm cavity power with respect to the interferometer's input power to be maximised. The power recycling cavity length, \emph{\gls{PRCL}} can be defined in terms of the average distance between the power recycling mirror (\gls{PRM}) and the two \glspl{ITM}:
\begin{equation}
  \label{eq:prcl-length}
  \delta L_{\text{PRCL}} = l_{\text{P}} + \frac{l_{\text{X}} + l_{\text{Y}}}{2}.
\end{equation}

The signal recycling length, along with the signal recycling mirror (\gls{SRM}) transmissivity, determines the bandwidth of the signal extraction and therefore needs controlled in order to keep the interferometer's response stationary in time. This length, \emph{\gls{SRCL}}, is defined similarly to \gls{PRCL} in terms of the position of the signal recycling mirror:
\begin{equation}
  \label{eq:srcl-length}
  \delta L_{\text{SRCL}} = l_{\text{S}} + \frac{l_{\text{X}} + l_{\text{Y}}}{2}.
\end{equation}

Control of \gls{PRCL} and \gls{SRCL} can be achieved via corrective feedback to the position of the power and signal recycling mirrors, respectively.

\subsubsection{\label{sec:mich-length}Michelson length}
The length between the beam splitter and the \glspl{ITM} should be held in the dark fringe condition for the carrier to correctly couple the common and differential arm cavity modes to the input and output port of the beam splitter, respectively. In a \DRFPMI{} it must also be held constant to keep the amount of carrier and sideband power in the signal recycling cavity stable, which avoids the need for complicated time-varying control signals. This length, \emph{\gls{MICH}}, can be expressed as the differential length between the beam splitter and the \glspl{ITM}:
\begin{equation}
  \label{eq:mich-length}
  \delta L_{\text{MICH}} = \frac{l_{\text{X}} - l_{\text{Y}}}{2}.
\end{equation}

The \gls{MICH} length is kept constant by feeding back to the positions of the cavity mirrors. Alternatively this feedback can be applied to the beam splitter, but in this case the recycling mirrors must also be moved to avoid influencing the lengths of \gls{PRCL} and \gls{SRCL}.

There is an apparent degeneracy between the dark fringes produced by the \gls{MICH} and \gls{DARM} degrees of freedom, and so one might expect that the dark fringe offset required for \gls{DC} readout could be applied to the former. While a \gls{DARM} offset has the disadvantage that it involves the creation of an optical spring due to the high light power in the arms\textemdash mechanically coupling the \gls{CARM} and \gls{DARM} modes \cite{Heidmann2011, Vostrosablin2014}\textemdash it has favourable noise couplings compared to a \gls{MICH} offset \cite{Vajente2011}. The \gls{MICH} degree of freedom is filtered above the pole frequency of the power recycling cavity, which is typically at a high frequency, whereas the \gls{DARM} mode is additionally filtered by the arm cavity poles which are usually much lower in frequency, and so the noise coupling is reduced.

\subsubsection{Driving coefficients}
Table\,\ref{tab:et-lf-driving-coefficients} lists the coefficients to be applied to the test masses to produce the error signals representing each of the degrees of freedom. Corrections to \gls{CARM} and \gls{DARM} are defined in terms of the positions of the \glspl{ETM}, with the differences in coefficient sign between the two degrees of freedom being whether the correction is in-phase or out-of-phase, respectively. The \gls{PRCL} and \gls{SRCL} degrees of freedom are defined as corrections to the power and signal recycling mirrors, respectively. \gls{MICH} is a little more tricky, and as explained in Section\,\ref{sec:mich-length} it can be achieved with feedback either to all of the arm cavity mirrors or to the beam splitter and recycling mirrors. For the purposes of this study we choose the former to avoid the need to correct the unit drive amplitude applied to the beam splitter due to its angle of incidence.

\begin{table}
  \centering
  \begin{tabular}{r|ccccccc}
    & \textbf{\gls{ITM} X} & \textbf{\gls{ETM} X} & \textbf{\gls{ITM} Y} & \textbf{\gls{ETM} Y} & \textbf{\gls{BS}} & \textbf{\gls{PRM}} & \textbf{\gls{SRM}} \\
    \hline
    \gls{CARM} & \num{0} & \num{0.5} & \num{0} & \num{0.5} & \num{0} & \num{0} & \num{0} \\
    \gls{DARM} & \num{0} & \num{0.5} & \num{0} & \num{-0.5} & \num{0} & \num{0} & \num{0} \\
    \gls{MICH} & \num{-0.5} & \num{0.5} & \num{0.5} & \num{-0.5} & \num{0} & \num{0} & \num{0} \\
    \gls{PRCL} & \num{0} & \num{0} & \num{0} & \num{0} & \num{0} & \num{1} & \num{0} \\
    \gls{SRCL} & \num{0} & \num{0} & \num{0} & \num{0} & \num{0} & \num{0} & \num{1} \\
  \end{tabular}
  \caption[Driving coefficients for each mirror and each degree of freedom of \ETLF{}]{\label{tab:et-lf-driving-coefficients}Driving coefficients for each mirror and each degree of freedom of \ETLF{}. \gls{CARM} and \gls{DARM} involve the driving of the \glspl{ETM} in- and out-of-phase, respectively. \gls{MICH} involves moving the \glspl{ITM} differentially, but to avoid sensing \gls{DARM} effects the \glspl{ETM} must be moved too. The driving coefficients are defined with respect to the direction of the \gls{HR} surface of each optic, and as such the coefficients in \gls{MICH} for the \gls{ETM} and \gls{ITM} in each cavity have opposite sign. \gls{PRCL} and \gls{SRCL} involve moving the power and signal recycling mirrors, respectively.}
\end{table}

\subsection{\label{sec:decoupled-sidebands}Decoupled control signals}
With the exception of the main \gls{DC} readout for \gls{DARM}, the error signals representing the degrees of freedom are derived using heterodyne schemes, usually variations of the Pound-Drever-Hall technique discussed in Section\,\ref{sec:pdh}. For \MI{}s a number of techniques have been developed such as \emph{internal} and \emph{external} modulation, but the most prominent technique employed in all recent detectors is \emph{Schnupp} modulation. This uses control sidebands imposed on the carrier by means of phase modulation before the light is coupled into the arms by the beam splitter \cite{Heinzel1999}.

In a heterodyne sensing scheme, for the control sidebands to enter the output port their lengths must be macroscopically mismatched; this is called a \emph{Schnupp asymmetry}. For a Schnupp asymmetry $\Delta l_{\text{SCH}}$, the two arms in a \MI{} have length $l + \frac{\Delta l_{\text{SCH}}}{2}$ and $l - \frac{\Delta l_{\text{SCH}}}{2}$ where the nominal length is $l$. The Schnupp asymmetry differs from the previous microscipic distances defined in Equations \ref{eq:darm-length} to \ref{eq:mich-length} in that it is typically equivalent to many thousands of wavelengths. Although a \DRFPMI{} will typically employ \gls{DC} readout and therefore does not require heterodyne error signals for the control of \gls{DARM}, this asymmetry is required for the control of \gls{SRCL}. The application of a Schnupp asymmetry within a \DRFPMI{} is shown in Figure\,\ref{fig:schnupp-darm-offsets}.

\begin{figure}
  \centering
  \includegraphics[width=0.6\columnwidth]{graphics/generated/from-svg/70-schnupp-darm-offsets.pdf}
  \caption[Schnupp asymmetry and differential arm cavity offset in a \DRFPMI{}]{\label{fig:schnupp-darm-offsets}The Schnupp asymmetry and differential arm cavity offset in a \DRFPMI{}. The Schnupp asymmetry allows the control sidebands resonant in the \MI{} to leave the beam splitter's output port where they may sense the motion of the signal recycling cavity. This is typically an offset of the order of \SI{}{\centi\meter}. The differential arm cavity (\gls{DARM}) offset is a microscopic detuning of the arm cavities to allow a small amount of light to interfere constructively at the beam splitter in order for it to act as a local oscillator to the signal sidebands at the output port. This detuning is typically of the order of \SI{}{\pico\meter}.}
\end{figure}

In a simple \MI{} control sidebands at a single modulation frequency can be imposed upon the carrier to discriminate the phase of the arm cavities with respect to the input, and the modulation frequency is chosen such that the control sidebands will propagate in the interferometer but not enter the arm cavities so that they can act as a phase reference for the light that does. While the Schnupp modulation technique is applicable to \DRFPMI{}s \cite{Heinzel1998}, it is not trivial to decouple the five degrees of freedom when they are sensed by the carrier and a single modulation frequency. For example, the motion of the signal recycling mirror cannot be decoupled from the motion of the power recycling mirror in a single control sideband frequency; instead the signal will contain a linear combination of the motion of the two. By employing a second modulation frequency a suitable phase reference can be obtained for each of the degrees of freedom by carefully arranging for each modulation frequency to resonate in a subset of the cavities to control. Methods to find and create decoupled error signals in a \DRFPMI{} are discussed in the following subsections.

\subsubsection{\label{sec:drfpmi-gain-hierarchy}Gain hierarchy}
Some degrees of freedom require a wider control bandwidth in order for the interferometer to be held at its operating point. For example, laser frequency and intensity noise is present across the measurement band at the input of the interferometer and must be suppressed by many orders of magnitude to be compatible with the sensitivity requirements of a audio-band gravitational wave interferometer. In order to provide high gain within the detection band the bandwidth of the controller (see Appendix\,\ref{sec:control-bandwidth}) is high. Laser noise mainly couples to \gls{CARM} and not to the other degrees of freedom, but control of \gls{CARM} is necessary to maintain light power within the arm cavities, and hence sensitivity. In \ALIGO{} the control bandwidth for \gls{CARM} is \SI{65}{\kilo\hertz} \cite{Abbott2010}. On the other hand, control of \gls{DARM} and the auxiliary degrees of freedom tends to be driven by the presence of low frequency seismic noise and so typically requires much smaller control bandwidth (see for example the differential arm length control precision requirement for the \SSMEXPT{} in Section\,\ref{sec:ssm-required-control}). Through the appropriate selection of servos for each degree of freedom the cross-couplings present at each sensor from secondary degrees of freedom can be suppressed. This is called \emph{gain hierarchy} and has been previously demonstrated in \LIGO{} \cite{Fritschel2001}.

\subsubsection{\label{sec:sideband-beats}Combination of control sideband frequencies}
Error signals with greater decoupling, particularly for the inner degrees of freedom, can in some cases be found by demodulating the light at some combination of the two sideband frequencies \cite{Strain2003, Barr2006}. Some of the control signals used in \ALIGO{} involve demodulation at the sum or difference of the two sideband frequencies \cite{Abbott2010} or the use of double-demodulation \cite{Staley2014}.

\subsubsection{\label{sec:control-matrix-operations}Control matrix operations}
As the resonant conditions of the control sidebands can be different from the carrier, the magnitude and phase with which error signals representing each degree of freedom appear at each probe can be different.

The gradient of the error signal as witnessed by a sensor for motion of a given degree of freedom represents the interferometer's response for that degree of freedom, as shown for example in Figure\,\ref{fig:pdh-response} for the \SSM{} experiment. A \emph{sensing matrix} can be assembled with the collection of error signal slopes from each degree of freedom in an interferometer to each probe. With this matrix it is then possible to perform row and column operations to suppress cross-couplings in an operation that resembles Gaussian elimination. As the creation of linear combinations of signals can be performed in real time by the controller, the assembly of a \emph{control matrix} based on simulations and used to feed back combinations of error signals to the interferometer's actuators can serve as an approximation to the eventual implementation. This is the starting point for further analysis of the effects of noise coupling from each degree of freedom to the sensors.

Some signals from different degrees of freedom appear with similar magnitude at a particular port and cannot easily be decoupled with gain hierarchy or row and column operations. Instead, the phase of the readout can be used to discriminate between the two error signals. By demodulating each light field at two phases separated by \num{0} and \SI{90}{\degree} with respect to the control sideband modulation frequencies, so-called \emph{I} and \emph{Q} quadrature error signals can be obtained which can later be mixed to suppress the effect of one error signal with respect to another.

\section{Longitudinal sensing scheme for ET-LF}
The following section presents a concept for the longitudinal control of \ETLF{}. Using the techniques discussed in Section\,\ref{sec:multi-dof-control}, we will devise a control scheme based on the approach taken for the second generation detectors, namely through the use of \gls{DC} readout for \gls{DARM} and heterodyne readout for \gls{CARM}, \gls{PRCL}, \gls{SRCL} and \gls{MICH}.

\subsection{Scope and method}
Given the complexity of the interferometer we choose a numerical approach to the modelling and for this we will employ Optickle (see Appendix\,\ref{sec:optickle-sim}) using its plane wave mode. Angular control of the interferometer is expected to present its own challenges given the higher cavity power, larger beams and g-factors closer to unity compared to current generation detectors, but the longitudinal degrees of freedom must be shown to be controllable before angular degrees of freedom can be considered. Future angular sensing and control simulations will be possible with the model developed over the course of this work.

The sensing scheme assumes that the interferometer has been brought close to its operating point by a lock acquisition routine, and here we primarily consider the control of the interferometer in terms of its sensing matrix. It is well known that noise coupling between longitudinal degrees of freedom of detuned \DRFPMI{}s can be significant \cite{Hild2007}, but a crucial initial step before any control noise simulations can be undertaken is the development of a provisional sensing scheme. For \ETLF{} we loosely follow the approach taken for \ALIGO{} \cite{Abbott2010} and \AVIRGO{} \cite{Vajente2008} given that these represent the most sensitive \DRFPMI{}s built to date. To simplify the steps required to produce the scheme the parameters are defined in order from the least to the most constrained. With these parameters fixed we can then search for ports at which error signals representing each degree of freedom can be extracted, and this allows us to define a sensing matrix that highlights cross-couplings between each of the degrees of freedom. This is the starting point for control loop noise studies employing gain hierarchy as discussed in Section\,\ref{sec:drfpmi-gain-hierarchy}, which will be the subject of future work.

\subsubsection{Optimal input coupling}
Placing a power recycling mirror before a \MI{}, as discussed in Section\,\ref{sec:power-recycling}, creates an additional cavity between the input and the \glspl{ITM}. The intention of the power recycling mirror is to minimise the light reflected back towards the laser, and in order to do this the cavity it creates should be \emph{impedance matched} (see, for example, Section\,5.1 of \cite{Freise2010}). The loss within the interferometer in combination with the transmissivity of the power recycling mirror $T_{\text{PRM}}$ determines the impedance matching. In the \gls{ET} design study the loss per surface is assumed to be \SI{35}{\ppm} and the transmissivity of the \glspl{ETM} is \SI{6}{\ppm}; both contribute to the total loss. It also suggests the transmissivity of the power recycling mirror to be \SI{4.6}{\percent}. Figure\,\ref{fig:reflected-power-vs-prm-trans} shows the ratio of the light power leaving the power recycling mirror heading back towards the laser to the input light power. The minimum reflected power corresponds to a power recycling mirror transmissivity of \SI{4.6}{\percent}, validating the choice from the \gls{ET} design study. If the scatter or substrate loss of the optics is changed in the future, for instance due to the development of new coatings or the introduction of additional steering mirrors, this model can be used to recalculate the optimum power recycling mirror transmissivity.

\begin{figure}
  \centering
  \includegraphics[width=\columnwidth]{graphics/generated/from-python/70-reflected-power-vs-prm-transmissivity.pdf}
  \caption[Reflected power from \ETLF{} as a function of power recycling mirror transmissivity]{\label{fig:reflected-power-vs-prm-trans}Reflected power from \ETLF{} as a function of power recycling mirror transmissivity. For optimal coupling of the input laser light to the interferometer, the transmissivity of the power recycling mirror must be set to balance the input light with the total loss from the interferometer. For \ETLF{} with loss as per the \gls{ET} design study, this transmissivity should be \SI{4.6}{\percent}.}
\end{figure}

\subsection{\label{sec:control-sideband-freqs}Control sideband frequencies}
In the following subsections we define the constraints on the control sidebands before calculating appropriate frequencies. For a given interferometer it will be possible to find a set of control sideband frequencies any of which will be acceptable for a particular length, but in order for two sideband frequencies to be resonant in a combination of lengths it may be necessary to compromise the resonant conditions such that neither one is optimal for its particular purpose. The case is further complicated by the presence of a detuned signal recycling cavity in \ETLF{}, where the upper and lower control sidebands created with frequencies $-f$ and $+f$ offset from the carrier (see Figure\,\ref{fig:sideband-structure}) have different resonant conditions. These issues will be addressed in the following subsections.

\subsubsection{Control sideband resonance in the recycling cavities}
The \SI{310}{\meter} recycling cavity lengths defined in the \gls{ET} design study have free spectral range (\gls{FSR}, see Appendix\,\ref{sec:cavity-fom}):
\begin{equation}
  \text{FSR} = \frac{c_0}{2 L_{\text{PRCL}}} = \frac{c_0}{2 L_{\text{SRCL}}} = \SI{483.5}{\kilo\hertz}.
\end{equation}
To start, we can try to make the first sideband frequency resonate in the power recycling cavity. As the arm cavities at the operating point reflect the light back towards the power recycling cavity, the resonant condition is a half-integer multiple of the power recycling cavity \gls{FSR}, i.e.:
\begin{equation}
  \label{eq:prc-fsr}
  f_1 = \left(A + \frac{1}{2} \right) \frac{c_0}{2 L_{\text{PRC}}},
\end{equation}
for positive integer $A$.

We can repeat this step for the signal recycling cavity, instead making the second sideband frequency resonant and the first anti-resonant in order to create discrimination. In this case the resonant condition is an integer multiple of the cavity's \gls{FSR}, i.e.:
\begin{equation}
  \label{eq:src-fsr}
  \begin{split}
    f_1 &\neq B \frac{c_0}{2 L_{\text{SRC}}} \\
    f_2 &= C \frac{c_0}{2 L_{\text{SRC}}},
  \end{split}
\end{equation}
where $B$ and $C$ are again positive integers.

Any control sideband frequencies used in \ETLF{} must be outside the arm cavity resonances spaced by integer multiples of the arm cavity \gls{FSR} in order to allow them to act as a phase discriminant for \gls{CARM} and \gls{DARM}. The \SI{10}{\kilo\meter} arm cavity \gls{FSR} is \SI{14.99}{\kilo\hertz}. As an integer multiple of the arm cavity \gls{FSR} would allow optimal coupling of the sidebands into the arm cavities, one might assume that an odd half-integer multiple would be optimally anti-resonant; however, in this scenario the lower higher-order control sidebands, necessarily created by the phase modulation upon the \gls{EOM} (see Appendix\,\ref{eq:field-phase-bessel}), would become resonant, and so we choose to offset the sideband frequency slightly from the anti-resonant condition. We therefore stipulate two further requirements in addition to Equations \ref{eq:prc-fsr} and \ref{eq:src-fsr}:
\begin{align}
  \label{eq:arm-fsr}
  f_1 &= \left(D_{1} + \delta_{1} \right) \frac{c_0}{2 L_{\text{Arm}}} \\
  f_2 &= \left(D_{2} + \delta_{2} \right) \frac{c_0}{2 L_{\text{Arm}}},
\end{align}
for positive integers $D_{i}$ and small perturbations $\left| \delta_{i} \right| \ll \frac{1}{2}$.

\subsubsection{Control sideband frequencies}
The control sideband frequencies should ideally be \gls{RF}, at least around \SI{10}{\mega\hertz}, to benefit from the noise advantages discussed in Section\,\ref{sec:pdh}. An upper limit of \SI{100}{\mega\hertz} is reasonable given that quadrant photodetectors requiring large surface area will eventually be required for alignment control. Larger surface areas typically lead to greater stray capacitance, limiting the ability of the device to register signals at higher frequencies \cite{Freise2010}.

We chose the first sideband frequency $f_1$ to be \SI{11363101}{\hertz} which for $A = 23$ and $D_{1} = 758$ satisfy the requirements and falls within the suitable range. As the light in the signal recycling cavity must first pass through the power recycling cavity, we must ensure that $f_2$ also resonates in the power recycling cavity. This is achieved by choosing the second sideband to be an integer multiple of the first, which is already resonant in the power recycling cavity. To provide ample difference between the first and second sideband frequencies, we chose $f_2 = 5f_1 = \SI{56815505}{\hertz}$. This is separated far enough in frequency from $f_1$ that we can investigate the use of beats between $f_1$ and $f_2$ for control purposes as discussed in Section\,\ref{sec:sideband-beats}.

We assume a modulation depth of \SI{0.1}{\radian} for $f_1$ and $f_2$ to keep the power in higher modulation orders low while allowing for a reasonable amount of light power in the first order. This parameter has little impact on others and can be tuned later to provide for better separation between the two sideband signals on the sensors.

\subsection{Schnupp asymmetry}
In addition to facilitating the control of the signal recycling cavity, the Schnupp asymmetry governs whether the second control sideband frequency couples to both recycling cavities, or just one. A small offset of a few \SI{}{\centi\meter} between the Michelson lengths $l_X$ and $l_Y$ allows for the signal recycling cavity to be resonant for only one of the sideband frequencies, whereas a larger offset of a few tens of \SI{}{\centi\meter} makes both sideband frequencies resonant there \cite{Vajente2008}. Both methods of control are feasible, with the former being implemented in \ALIGO{} \cite{Abbott2010} and the latter in \KAGRA{} \cite{kagra2013}.

Figure\,\ref{fig:sideband-powers-vs-schnupp-detuned} shows the power of the upper and lower sideband fields with respect to the carrier in the recycling cavities of \ETLF{} given the Schnupp asymmetry and recycling cavity lengths with detuned signal recycling. Since the Schnupp asymmetry is a macroscopic length, it is not easily adjusted during operation and so it is necessary to designate this length in the design phase. Here we choose a Schnupp asymmetry that attempts to maximise the difference in power between the two sideband frequencies in the recycling cavities during detuned operation. This is around \SI{0.08}{\meter}. Also observe that a Schnupp asymmetry of \num{0} results in no sideband power in the signal recycling cavity, and that asymmetries of around \SI{0.5}{\meter} result in a situation where both sideband frequencies are resonant in both recycling cavities. In the latter case it would be possible to obtain decoupled control signals through control matrix operations as discussed in Section\,\ref{sec:control-matrix-operations}.

\begin{figure}
  \centering
  \includegraphics[width=\columnwidth]{graphics/generated/from-python/70-sideband-powers-vs-schnupp-detuned.pdf}
  \caption[Power of the control sidebands in the cavities of \ETLF{} in detuned configuration as a function of Schnupp asymmetry]{\label{fig:sideband-powers-vs-schnupp-detuned}Power of the control sidebands in the cavities of \ETLF{} during detuned operation as a function of Schnupp asymmetry. A macroscopic offset\textemdash the \emph{Schnupp asymmetry}\textemdash is intentionally introduced to the Michelson length in order to allow the coupling of the sidebands $\pm f_1$ and $\pm f_2$ into the signal recycling cavity for the purposes of control while maintaining the dark fringe condition for the carrier. Here, we choose to allow $\pm f_2$, but not $\pm f_1$, to couple to the signal recycling cavity. The power is a reasonable estimate for the relative sensitivity of the sidebands in the recycling cavities, and we choose an asymmetry which gives good separation of the power of the sidebands in each cavity, \SI{0.08}{\meter}.}
\end{figure}

\subsubsection{Optimisation of the signal recycling cavity length}
Note the discrepancy between two of the control sideband frequency constraints in Section\,\ref{sec:control-sideband-freqs}: the power and signal recycling cavities cannot both be simultaneously resonant and anti-resonant to $f_1$ and $f_2$ given that $f_2 = 5 f_1$. To resolve this discrepancy we can scan the length of the signal recycling cavity in order to find a position where $f_2$ is resonant and $f_1$ is not, as shown in Figure\,\ref{fig:sideband-powers-srcl-detuned}. This could otherwise have been achieved by changing $f_2$ by a fraction of the signal recycling cavity's \gls{FSR} given the relation between frequency and length shown in Equation\,\ref{eq:freq-to-length}. We can see that changing the signal recycling cavity length from \SI{310}{\meter} to \SI{311.585}{\meter} results in the desired sideband resonance condition for the $+f_2$ sideband. The power of $f_2$ in the power recycling cavity drops as the lower and upper sidebands get critically coupled into the signal recycling cavity. As the signal recycling cavity detuning in \ETLF{} is large, the signal recycling cavity is not resonant for both the upper and lower $f_2$ sidebands (in the tuned signal recycling cavity configuration the upper and lower sidebands are degenerate).

\begin{figure}
  \centering
  \includegraphics[width=\columnwidth]{graphics/generated/from-python/70-sideband-powers-vs-srcl-detuned.pdf}
  \caption[Power of the control sidebands in the signal recycling cavity of \ETLF{} in detuned configuration as a function of signal recycling cavity length]{\label{fig:sideband-powers-srcl-detuned}The power in each sideband in each recycling cavity as a function of signal recycling cavity length. The nominal signal recycling cavity length defined by the \gls{ET} design study, \SI{310}{\meter}, is not resonant for the second sideband as intended. As the sideband frequencies are different, we can fix this situation by scanning the signal recycling cavity to find a length that is resonant for $f_2$ but not for $f_1$. For the given choice of Schnupp asymmetry, this occurs at a length of \SI{311.585}{\meter} for the upper sideband. The choice to optimise either the upper or lower sideband is arbitrary since both sidebands contribute signal at the readout.}
\end{figure}

In the power recycling cavity, $f_1$ provides an error signal for the power recycling cavity that is a factor of \num{13} larger than that of $f_2$. At the same time, $f_2$ provides an error signal for the length that is \num{650} times larger than the equivalent for $f_1$ in the signal recycling cavity.

\subsection{Dark fringe offset}
As described in Section\,\ref{sec:homodyne-readout}, \gls{DC} readout at the output port of a \DRFPMI{} requires carrier light to be present to act as a phase reference for the signal sidebands. In an interferometer with matched arms there is no classical light at the output port and so a phase asymmetry must be introduced by differentially detuning the arms by a small amount to create the appropriate \emph{dark fringe offset}. In practice, asymmetries within the arms are already present, for example arising from mismatched arm cavity finesse or asymmetric beam splitter reflectivity. As these effects change the amplitude of the light, they appear in a different quadrature to the signal at the output port and so as long as the loss is small the effect on the sensitivity is minimal.
% The effect of imbalanced finesse, reflectivity etc. in the arms is that the AMPLITUDE of the light recombinging at the BS is different, which leads to amplitude fluctuations at the output port. Meanwhile, the GW effect is a change in phase, so this shows up as a phase fluctuation at the output. The resulting superpositions of the two effects are in different quadratures, i.e. 90 degrees out of phase. Loss like this does affect the optimal homodyne angle.

For our model we define the \gls{DARM} offset as a microscopic detuning of the arm cavity lengths, and it is differential such that one arm has length $L + \frac{\delta L_{\text{DARM}}}{2}$ while the other has length $L - \frac{\delta L_{\text{DARM}}}{2}$, where $L$ is the average length. This is depicted in Figure\,\ref{fig:schnupp-darm-offsets} alongside the Schnupp asymmetry.

Figure\,\ref{fig:total-power-vs-darm-offset-detuned} shows the power at the output port and in the arm cavities as a function of \gls{DARM} offset in the detuned configuration. Standard photodetectors used in \ALIGO{} and \AVIRGO{} can handle up to a few \num{10}s of \SI{}{\milli\watt} and this should ideally be the power incident upon the photodetector at the output port to maximise the signal to dark noise ratio. In \ETLF{}, however, the \gls{DARM} offset required to reach this figure would create a significant mismatch in the power of each arm leading to a strong optical spring effect. For an offset of \SI{12}{\pico\meter}, the power at the output can be set to around \SI{10}{\milli\watt} with a difference of around 3\% in the power in the arms, which should be tolerable in terms of sensitivity and noise coupling.

\begin{figure}
  \centering
  \includegraphics[width=\columnwidth]{graphics/generated/from-python/70-total-power-vs-darm-offset-detuned.pdf}
  \caption[Carrier power at the output port of \ETLF{} in detuned configuration as a function of differential arm cavity offset]{\label{fig:total-power-vs-darm-offset-detuned}Carrier power at the output port of \ETLF{} in detuned configuration as a function of differential arm cavity length (\gls{DARM}) offset. The differential arm detuning required to allow carrier light to enter the dark port for \gls{DC} readout involves an increase or decrease in the microscopic length of each arm cavity, and this changes the circulating power. The compromise must be made between the power available to the photodetector for sensing while maintaining reasonably balanced arm cavities to prevent optical springs from influencing the sensitive band and creating additional noise coupling.}
\end{figure}
% See "Stable optical spring in the Advanced LIGO detector with unbalanced arms and in the Michelson-Sagnac interferometer", equations 6f and 6g show a small part of the common mode entering the differential, and vice versa. Another way of looking at this is to say that the DARM offset at the point of maximum arm cavity power (for one arm) is compensating the detuned signal recycling such that the compound cavity formed by the SRM and (e.g.) the X arm is resonant. This explains the shape of the arm cavity power vs DARM offset.

\subsection{Power in each light field}
The power in each field within each relevant space or cavity of the interferometer is shown in Table\,\ref{tab:et-lf-detuned-dc-powers} for the detuned interferometer.

\begin{table}
  \centering
  \resizebox{16cm}{!}{%
    \begin{tabular}{r|ccccccccc|c}
      & \textbf{-68 MHz} & \textbf{-57 MHz} & \textbf{-45 MHz} & \textbf{-11 MHz} & \textbf{Carrier} & \textbf{11 MHz} & \textbf{45 MHz} & \textbf{57 MHz} & \textbf{68 MHz} & \textbf{Total} \\
      \hline
      \textbf{Input from laser} & 0 & 0 & 0 & 0 & \SI{3}{\watt} & 0 & 0 & 0 & 0 & \SI{3}{\watt} \\ 
      \textbf{After modulators} & \SI{19}{\micro\watt} & \SI{7}{\milli\watt} & \SI{19}{\micro\watt} & \SI{7}{\milli\watt} & \SI{3}{\watt} & \SI{7}{\milli\watt} & \SI{19}{\micro\watt} & \SI{7}{\milli\watt} & \SI{19}{\micro\watt} & \SI{3}{\watt} \\ 
      \textbf{Power recycling cavity} & \SI{220}{\nano\watt} & \SI{521}{\milli\watt} & \SI{220}{\nano\watt} & \SI{410}{\milli\watt} & \SI{65}{\watt} & \SI{407}{\milli\watt} & \SI{220}{\nano\watt} & \SI{34}{\milli\watt} & \SI{220}{\nano\watt} & \SI{66}{\watt} \\ 
      \textbf{Power recycling pick-off} & \SI{33}{\pico\watt} & \SI{78}{\micro\watt} & \SI{33}{\pico\watt} & \SI{61}{\micro\watt} & \SI{9}{\milli\watt} & \SI{61}{\micro\watt} & \SI{33}{\pico\watt} & \SI{4}{\micro\watt} & \SI{33}{\pico\watt} & \SI{10}{\milli\watt} \\ 
      \textbf{Michelson cavity} & \SI{131}{\nano\watt} & \SI{228}{\milli\watt} & \SI{112}{\nano\watt} & \SI{208}{\milli\watt} & \SI{33}{\watt} & \SI{204}{\milli\watt} & \SI{107}{\nano\watt} & \SI{27}{\milli\watt} & \SI{115}{\nano\watt} & \SI{33}{\watt} \\ 
      \textbf{Arm cavity X} & \SI{261}{\pico\watt} & \SI{557}{\micro\watt} & \SI{375}{\pico\watt} & \SI{9}{\milli\watt} & \SI{18}{\kilo\watt} & \SI{9}{\milli\watt} & \SI{357}{\pico\watt} & \SI{65}{\micro\watt} & \SI{230}{\pico\watt} & \SI{18}{\kilo\watt} \\ 
      \textbf{Arm cavity Y} & \SI{182}{\pico\watt} & \SI{722}{\micro\watt} & \SI{361}{\pico\watt} & \SI{9}{\milli\watt} & \SI{18}{\kilo\watt} & \SI{9}{\milli\watt} & \SI{379}{\pico\watt} & \SI{68}{\micro\watt} & \SI{210}{\pico\watt} & \SI{18}{\kilo\watt} \\ 
      \textbf{Signal recycling cavity} & \SI{3}{\nano\watt} & \SI{4}{\milli\watt} & \SI{379}{\pico\watt} & \SI{69}{\micro\watt} & \SI{15}{\milli\watt} & \SI{41}{\micro\watt} & \SI{411}{\pico\watt} & \SI{26}{\milli\watt} & \SI{928}{\pico\watt} & \SI{45}{\milli\watt} \\ 
      \textbf{Reflected back to laser} & \SI{19}{\micro\watt} & \SI{6}{\milli\watt} & \SI{19}{\micro\watt} & \SI{7}{\milli\watt} & \SI{62}{\micro\watt} & \SI{7}{\milli\watt} & \SI{19}{\micro\watt} & \SI{2}{\milli\watt} & \SI{19}{\micro\watt} & \SI{23}{\milli\watt} \\ 
      \textbf{Output} & \SI{558}{\pico\watt} & \SI{756}{\micro\watt} & \SI{76}{\pico\watt} & \SI{14}{\micro\watt} & \SI{3}{\milli\watt} & \SI{8}{\micro\watt} & \SI{82}{\pico\watt} & \SI{5}{\milli\watt} & \SI{186}{\pico\watt} & \SI{9}{\milli\watt}
    \end{tabular}
  }
  \caption[Light power in \ETLF{} in the detuned configuration]{\label{tab:et-lf-detuned-dc-powers}Powers in various parts of \ETLF{} in the detuned configuration. The input light is passed through \glspl{EOM} which impart control sidebands at frequencies offset from the carrier. As the arm cavity \gls{FSR} is almost optimally separated from the control sideband frequencies, the sideband light power in the arms is vastly smaller than the carrier power. The first sidebands at $\SI{\pm11}{\mega\hertz}$ are resonant within the power recycling cavity. The second sidebands at $\SI{\pm57}{\mega\hertz}$ is greatest within the power and signal recycling cavities, with the \SI{-57}{\mega\hertz} sideband almost exactly resonant in the power recycling cavity and the +\SI{57}{\mega\hertz} sideband exactly resonant in the signal recycling cavity. The power reflected back towards the laser is composed mainly of light at the two sideband frequencies, as the transmissivity of the power recycling mirror minimises the reflected carrier light. At the output port, the carrier power is present due to the \gls{DARM} offset, and acts as a local oscillator to the signal sidebands there.}
\end{table}

\subsection{\label{sec:etlf-readout-ports}Readout ports}
Figure\,\ref{fig:dof-readouts} shows some available readout ports for \ETLF{} where the sidebands and carrier may be measured for the purposes of control:
\begin{itemize}
  \item \textbf{REFL} (\emph{reflected}) senses the light reflected from the interferometer back towards the input laser, rejected for example by a Faraday isolator;
  \item \textbf{POP} (\emph{pick off \gls{PRCL}}) senses the light in the power recycling cavity using a small pick-off mirror with reflectivity \SI{150}{\ppm};
  \item \textbf{AS} (\emph{asymmetric}) senses the light at the output port of the \DRFPMI{}.
\end{itemize}

\begin{figure}
  \centering
  \includegraphics[width=\columnwidth]{graphics/generated/from-svg/70-dof-readouts.pdf}
  \caption[Some available readout ports for sensing and control in \ETLF{} in detuned configuration]{\label{fig:dof-readouts}Some available readout ports for sensing motion of the degrees of freedom in \ETLF{} in detuned configuration. Phase modulation sidebands are imparted upon the input laser light at two primary frequencies before it enters the interferometer. Photodetectors are placed at three ports to sense the carrier and these sidebands for the purposes of sensing and control. The most important readout is \AS{}, which strongly senses \gls{DARM}, and this is placed in transmission of the signal recycling mirror. Light reflected back towards the laser is sensed via a Faraday isolator at the \REFL{} port, and a pick-off\textemdash \POP{}\textemdash senses light in the power recycling cavity.}
\end{figure}

For the purposes of the control simulations, each readout port contains incident fields at the carrier frequency and offsets of $\pm f_1$ (\SI{\pm11}{\mega\hertz}), $\pm f_2$ (\SI{\pm57}{\mega\hertz}), $-f_1 - f_2$ (\SI{-68}{\mega\hertz}), $f_1 - f_2$ and $-f_2 + f_1$ (\SI{-45}{\mega\hertz}\footnote{This is not \SI{-46}{\mega\hertz} due to rounding.}), $-f_1 + f_2$ and $f_2 - f_1$ (\SI{45}{\mega\hertz}), and $f_1 + f_2$ (\SI{68}{\mega\hertz}). There are therefore \num{9} light fields propagated through the interferometer that are combined into signals at \num{5} frequencies at \num{3} ports. We furthermore demodulate the heterodyne sensors in the \emph{I} and \emph{Q} quadratures to be able to calculate the optimal readout phase at each port given the propagation delay between each mirror and each sensor. These are shown in Table\,\ref{tab:et-lf-probes}.

\begin{table}
  \centering
  {\renewcommand{\arraystretch}{1.2} % for extra vertical spacing between rows
    \begin{tabular}{r|ccc}
      \textbf{Offset} & \textbf{Output port} & \textbf{Power recycling cavity pick-off} & \textbf{Reflected light pick-off} \\
      \hline
      \num{0} & \ASDC{} & \textemdash & \textemdash \\
      $\pm f_1$    & \ASFIRSTI{}, \ASFIRSTQ{} & \POPFIRSTI{}, \POPFIRSTQ{} & \REFLFIRSTI{}, \REFLFIRSTQ{} \\
      $\pm f_2$    & \ASSECONDI{}, \ASSECONDQ{} & \POPSECONDI{}, \POPSECONDQ{} & \REFLSECONDI{}, \REFLSECONDQ{} \\
      $\pm \left( f_2 - f_1 \right)$ & \ASDIFFI{}, \ASDIFFQ{} & \POPDIFFI{}, \POPDIFFQ{} & \REFLDIFFI{}, \REFLDIFFQ{} \\
      $\pm \left( f_1 + f_2 \right)$ & \ASSUMI{}, \ASSUMQ{} & \POPSUMI{}, \POPSUMQ{} & \REFLSUMI{}, \REFLSUMQ{}
    \end{tabular}
  }
  \caption[Probes that sense the light fields propagating within \ETLF{}]{\label{tab:et-lf-probes}Probes that sense the light fields propagating within \ETLF{} with respect to the carrier frequency. The carrier is sensed by \ASDC{} at the interferometer's output port, which is the light that propagates through the signal recycling mirror. The control sidebands, and the beats between the sidebands, are sensed at the same port demodulated at each relevant frequency, along with similar readouts sensing a small transmission of light through a folding mirror in the power recycling cavity (\POP{}) and the light reflected from the interferometer (\REFL{}).}
\end{table}

\subsection{Control signals}
The sensing matrix, as introduced in Section\,\ref{sec:control-matrix-operations}, can be calculated for \ETLF{} by exciting each degree of freedom with the driving coefficients shown in Table\,\ref{tab:et-lf-driving-coefficients} and measuring the response at each probe shown in Table\,\ref{tab:et-lf-probes}.

At this preliminary stage, a reasonable choice of error signals from the sensing matrix to use for control can be determined through a heuristic approach building upon knowledge gained from the control of the second generation detectors. As $f_2$ resonates in both recycling cavities, it samples the motion of the mirrors that influence \gls{MICH} and \gls{SRCL} as well as \gls{PRCL}. Conversely, $f_1$ only strongly samples the motion of \gls{MICH} and \gls{PRCL}. Using demodulations at these frequencies, and combinations thereof, it should be possible to find a set of reasonably decoupled error signals for each degree of freedom.

\subsubsection{Combination of readout quadratures}
Table\,\ref{tab:et-lf-probes} shows signals at both the $I$ and $Q$ quadratures for all of the \gls{RF} sideband frequencies considered. These quadratures can be combined electronically to produce an error signal with optimal gradient. The exact phase corresponding to the greatest magnitude is not important, as this can be influenced by technical factors such as the length of \gls{RF} transmission lines, but the relative phase between maximum error signals from different degrees of freedom on the same sensor is. If error signals from different degrees of freedom have the same maximum gradient at the same or opposite phase, one cannot be minimised with respect to the other through appropriate choice of demodulation phase. On the other hand, if two degrees of freedom couple to a pair of $I$ and $Q$ sensors with equal magnitude but separate phase, they can be used to sense both degrees of freedom. In practice due to temperature drifts and other time-varying effects it is difficult to maintain the demodulation phase of a set of sensors to a precision better than around \SI{1}{\degree} \cite{Effler2014}, and so the readout quadratures chosen for each of the degrees of freedom of \ETLF{} should ideally be separated by many degrees.

\subsubsection{Sensing matrix for ET-LF in detuned configuration}
The sensing matrix for the detuned configuration given the readout ports defined in Table\,\ref{tab:et-lf-probes} is shown in Table\,\ref{tab:et-lf-sensing-matrix-detuned} for mirror perturbations at \gls{DC}. The suggested readouts for each degree of freedom are highlighted in bold and are described in the following text.

The gravitational wave signal will primarily affect \gls{DARM}, and this is by design sensed by \ASDC{}. The common mode can be sensed at \REFLFIRST{}. The \gls{MICH}, \gls{PRCL} and \gls{SRCL} cavity error signals are difficult to separate due to the cross-coupling of control sidebands via the Schnupp asymmetry and the sideband asymmetry created by the presence of a detuned signal recycling cavity. As the control system will be implemented in a \LIGO{} \gls{CDS}-style system \cite{Bork2010}, it will be possible to define in software error signals formed from linear combinations of different sensor signals that decouple other degrees of freedom from a particular readout, as discussed in Section\,\ref{sec:control-matrix-operations}. In Table\,\ref{tab:et-lf-sensing-matrix-detuned}, \gls{PRCL} is dominant over \gls{MICH} and \gls{SRCL} at \POPFIRST{} and so this represents a good extraction point for the motion of the power recycling cavity. \gls{MICH} couples strongly to a number of ports but alongside strong signals from the other degrees of freedom. Its strongest ports, \ASDC{} and \ASSECOND{}, contain much larger signals from \gls{DARM} due to the difference in finesse between the Michelson and the arm cavities, and this would be difficult to remove electronically given the phase degeneracy. A better readout could be to use \POPSECOND{} with the dominant \gls{PRCL} signal suppressed with careful tuning of the demodulation phase. Residual \gls{PRCL} coupling can be subtracted electronically.

\gls{SRCL} is difficult to sense with $\pm f_1$ because they are not resonant in the signal recycling cavity, nor a single demodulation at $\pm f_2$ because these error signals are contaminated with contributions from \gls{MICH} or \gls{PRCL} given the cross-coupling facilitated by the Schnupp asymmetry. A possible sensing strategy could utilise the beats between sidebands found at \REFLDIFF{} or \REFLSUM{} where the contribution from \gls{MICH} and \gls{PRCL} can be suppressed with suitable choice of demodulation phase. All \gls{SRCL} signals contain an offset from zero due to the detuning, since $-f_2$ is not resonant when the signal recycling mirror is at its operating point (see Figure\,\ref{fig:sideband-powers-srcl-detuned}). To use \REFLSUM{} for \gls{SRCL} control an offset equivalent to \SI{-4.2}{\milli\watt} will be necessary.

\begin{table}
  \centering
  \resizebox{16cm}{!}{%
    {\renewcommand{\arraystretch}{1.2} % for extra vertical spacing between rows
      \begin{tabular}{r|ccccc}
	& \textbf{CARM} & \textbf{DARM} & \textbf{MICH} & \textbf{PRCL} & \textbf{SRCL} \\ 
	\hline
	\hline
	\textbf{\ASDC{}} & \num{1.44e+06} & \red{\textbf{\num{4.99e+08}}} & \num{1.17e+06} & \num{1.21e+05} & \num{1.39e+04} \\ 
	\textbf{\ASFIRST{}} & \num{2.25e+06} (\SI{-32.56}{\degree}) & \num{3.00e+07} (\SI{55.83}{\degree}) & \num{6.25e+04} (\SI{56.09}{\degree}) & \num{7.54e+03} (\SI{77.99}{\degree}) & \num{1.09e+03} (\SI{58.46}{\degree}) \\ 
	\textbf{\ASSECOND{}} & \num{1.76e+08} (\SI{-112.30}{\degree}) & \num{3.98e+08} (\SI{-162.68}{\degree}) & \num{8.39e+05} (\SI{-163.43}{\degree}) & \num{6.03e+05} (\SI{-78.31}{\degree}) & \num{6.73e+04} (\SI{94.85}{\degree}) \\ 
	\textbf{\ASDIFF{}} & \num{4.71e+04} (\SI{178.86}{\degree}) & \num{3.52e+03} (\SI{107.72}{\degree}) & \num{3.43e+04} (\SI{85.09}{\degree}) & \num{6.44e+04} (\SI{61.41}{\degree}) & \num{3.41e+03} (\SI{-146.01}{\degree}) \\ 
	\textbf{\ASSUM{}} & \num{1.00e+05} (\SI{-177.46}{\degree}) & \num{4.43e+04} (\SI{-83.81}{\degree}) & \num{5.97e+04} (\SI{43.44}{\degree}) & \num{5.13e+04} (\SI{103.97}{\degree}) & \num{4.67e+03} (\SI{-22.78}{\degree}) \\
	\hline
	\textbf{\POPFIRST{}} & \num{7.61e+07} (\SI{-144.47}{\degree}) & \num{6.88e+05} (\SI{-144.49}{\degree}) & \num{8.70e+02} (\SI{-26.41}{\degree}) & \red{\textbf{\num{2.55e+05}} (\SI{-36.40}{\degree})} & \num{3.08e+01} (\SI{-81.49}{\degree}) \\ 
	\textbf{\POPSECOND{}} & \num{5.27e+07} (\SI{99.38}{\degree}) & \num{4.72e+05} (\SI{92.38}{\degree}) & \red{\textbf{\num{2.95e+04}} (\SI{50.72}{\degree})} & \num{1.26e+05} (\SI{-130.50}{\degree}) & \num{1.08e+04} (\SI{120.38}{\degree}) \\ 
	\textbf{\POPDIFF{}} & \num{2.01e+02} (\SI{-51.56}{\degree}) & \num{9.34e+01} (\SI{-127.43}{\degree}) & \num{1.24e+03} (\SI{1.97}{\degree}) & \num{1.61e+04} (\SI{88.80}{\degree}) & \num{1.07e+03} (\SI{167.69}{\degree}) \\ 
	\textbf{\POPSUM{}} & \num{4.94e+02} (\SI{-22.60}{\degree}) & \num{9.93e+01} (\SI{-15.51}{\degree}) & \num{1.07e+03} (\SI{45.23}{\degree}) & \num{8.34e+03} (\SI{-145.33}{\degree}) & \num{1.07e+03} (\SI{-149.81}{\degree}) \\
	\hline
	\textbf{\REFLFIRST{}} & \red{\textbf{\num{1.44e+10}} (\SI{-0.01}{\degree})} & \num{1.31e+08} (\SI{-0.01}{\degree}) & \num{2.34e+05} (\SI{4.53}{\degree}) & \num{5.01e+07} (\SI{-0.41}{\degree}) & \num{2.25e+04} (\SI{45.05}{\degree}) \\ 
	\textbf{\REFLSECOND{}} & \num{4.63e+09} (\SI{-75.19}{\degree}) & \num{5.69e+07} (\SI{-86.52}{\degree}) & \num{1.47e+05} (\SI{-83.24}{\degree}) & \num{1.58e+07} (\SI{-76.05}{\degree}) & \num{1.67e+04} (\SI{30.68}{\degree}) \\ 
	\textbf{\REFLDIFF{}} & \num{3.90e+06} (\SI{0.94}{\degree}) & \num{1.31e+06} (\SI{0.04}{\degree}) & \num{2.19e+05} (\SI{163.21}{\degree}) & \num{1.28e+06} (\SI{-171.17}{\degree}) & \num{2.03e+05} (\SI{-25.15}{\degree}) \\
	\textbf{\REFLSUM{}} & \num{4.04e+06} (\SI{-101.97}{\degree}) & \num{1.32e+06} (\SI{-101.79}{\degree}) & \num{2.25e+05} (\SI{-76.80}{\degree}) & \num{1.40e+06} (\SI{-53.54}{\degree}) & \red{\textbf{\num{2.03e+05}} (\SI{89.52}{\degree})}
      \end{tabular}
    }
  }
  \caption[Gradients of the error signals from each degree of freedom to each probe in \ETLF{} at dc]{\label{tab:et-lf-sensing-matrix-detuned}Gradients of the error signals from each degree of freedom to each probe, in units of \SI{}{\watt\per\meter}, in \ETLF{} at \gls{DC}. The suggested readout probes for each degree of freedom are shown in bold red. The $I$ and $Q$ quadratures of each heterodyne readout have been combined into a single magnitude and the phase representing the greatest slope and the phase at which it is achieved. They have maximum gradient at phase angles determined by the propagation of the control sidebands through the interferometer. Probes can be optimised to sense the motion of a particular degree of freedom by adjusting the angle at which the sensor $I$ and $Q$ quadratures are combined, but signals on sensors that contain strong signals from other degrees of freedom at nearby phase angles are difficult to use.}
\end{table}

Table\,\ref{tab:et-lf-sensing-matrix-detuned-normalised} shows the gradient of each degree of freedom's error signal at each of the suggested sensors. The values have been normalised with respect to the degree of freedom to be read out. This table shows that the hierarchical control techniques discussed in Section\,\ref{sec:decoupled-sidebands} will be necessary in order to decouple the individual degrees of freedom.

\begin{table}
  \centering
  {\renewcommand{\arraystretch}{1.2} % for extra vertical spacing between rows
    \begin{tabular}{r|ccccc}
      & \textbf{CARM} & \textbf{DARM} & \textbf{MICH} & \textbf{PRCL} & \textbf{SRCL} \\ 
      \hline
      \hline
      \textbf{\REFLFIRST{}} & \red{\textbf{\num{1.0}}} & \num{9.1e-3} & \num{1.6e-5} & \num{3.5e-3} & \num{1.6e-6} \\
      \textbf{\ASDC{}} & \num{2.8e-3} & \red{\textbf{\num{1.0}}} & \num{2.3e-3} & \num{2.4e-4}  & \num{2.8e-5} \\
      \textbf{\POPSECOND{}} & \num{1.8e3} & \num{1.6e1} & \red{\textbf{\num{1.0}}} & \num{4.3e0} & \num{3.7e-1} \\
      \textbf{\POPFIRST{}} & \num{3.0e2} & \num{2.7e0} & \num{3.4e-3} & \red{\textbf{\num{1.0}}} & \num{1.2e-4} \\
      \textbf{\REFLSUM{}} & \num{2.0e1} & \num{6.5e0} & \num{1.1e0} & \num{6.9e0} & \red{\textbf{\num{1.0}}}
    \end{tabular}
  }
  \caption[Normalised gradients of the suggested sensors to be used for the control of \ETLF{}'s degrees of freedom]{\label{tab:et-lf-sensing-matrix-detuned-normalised}Normalised gradients of the suggested sensors to be used for the control of \ETLF{}'s degrees of freedom. Each row from Table\,\ref{tab:et-lf-sensing-matrix-detuned} corresponding to a sensor used to control a degree of freedom has been scaled by the gradient of the error signal corresponding to the respective degree of freedom. This shows the prominence of the other degrees of freedom on each sensor without having applied any of the other control techniques as discussed in Section\,\ref{sec:decoupled-sidebands}.}
\end{table}

Error signals corresponding to the suggested readouts for each degree of freedom are shown in Figure\,\ref{fig:sweeps-et-lf}. These are produced by calculating the power on each sensor at the relevant demodulation frequency as the mirrors are driven as shown in Table\,\ref{tab:et-lf-driving-coefficients} and represent the low-frequency limit of the transfer functions of each degree of freedom to each sensor.

\begin{figure}
  \centering
  \includegraphics[width=\columnwidth]{graphics/generated/from-python/70-sweeps-detuned.pdf}
  \caption[Sweeps through the zero-crossings of the chosen error signals in ET-LF]{\label{fig:sweeps-et-lf}Sweeps through the zero-crossings of the chosen error signals in \ETLF{} in the detuned configuration. Each error signal is linear about the operating point, which ensures a simple, bipolar error signal is available for the purposes of controlling each associated degree of freedom. This linearity has a different range for each readout, with \gls{CARM} and \gls{DARM} requiring the greatest precision.}
\end{figure}

The gradient of the \ASDC{} readout is zero when the arm cavities are tuned, consistent with Figure\,\ref{fig:total-power-vs-darm-offset-detuned}, showing that some classical carrier light power is always present for functional \gls{DC} readout. The \gls{SRCL} error signal shows the operating point crossing at a power of \SI{4.2}{\milli\watt}, necessitating an offset.

\subsection{Sensitivity of the scheme}
The sensitivity for \ETLF{} shown in Figure\,\ref{fig:et-d-sensitivity} assumes that the interferometer contains squeezed vacuum input via two filter cavities in addition to the presence of seismic and other noise, and these features have not been modelled in this work. A comparison is shown in Figure\,\ref{fig:et-lf-control-scheme-sensitivity} between the quantum noise limited sensitivity for \ETLF{} in the absence of squeezing, as calculated by the tool used to present the results in the ET-D study \cite{Hild2011}, \gls{GWINC} \cite{gwinc}, and the quantum noise limited sensitivity of the \ASDC{} readout in this scheme modelled with Optickle. Also shown is the curve generated with identical parameters in \emph{Finesse} (see Appendix\,\ref{sec:finesse-sim}), with excellent agreement. This shows that the choice of parameters in this work do not negatively impact upon the design sensitivity of the interferometer. The difference in sensitivity in the region of the cavity pole (\SI{7}{\hertz}) and the optical spring from the detuned signal recycling cavity (\SI{25}{\hertz}) is due to the assumed \gls{DARM} offset and the difference in the way in which quantum noise calculations are implemented between the tools.

\begin{figure}
  \centering
  \includegraphics[width=\columnwidth]{graphics/generated/from-python/70-et-lf-control-scheme-sensitivity-curve.pdf}
  \caption[ET-LF quantum noise limited sensitivity using the conceptual control scheme]{\label{fig:et-lf-control-scheme-sensitivity}\ETLF{} quantum noise limited sensitivity using the conceptual control scheme, with no squeezed light injection. The reference curve from the ET-D study is shown next to the sensitivity calculated with the Optickle model developed in this chapter, and also a Finesse curve generated using identical parameters. The reference and simulation curves broadly agree, showing that the chosen parameters do not negatively impact upon the sensitivity, though a difference in the noise calculations and the \gls{DARM} offset assumed in this work creates the slight mismatch in the most sensitive region.}
\end{figure}

The parameters used in the proposed control scheme are shown in Table\,\ref{tab:et-lf-updated-parameters} alongside the pre-existing parameters from the \gls{ET} design study. It should be noted that the proposed scheme is not rigorously optimised, and does not consider a number of other control possibilities such as the use of secondary reflections arising from the anti-reflective coatings on the beam splitter and \glspl{ITM} or the use of higher order combinations of $f_1$ and $f_2$ such as \emph{3f} signals used in the lock acquisition sequence of \VIRGO{} \cite{Acernese2008} and \ALIGO{} \cite{Staley2014}. It serves, however, as a first concept for the control of \ETLF{} proving that it can in principle be controlled in its detuned state.

\begin{table}
  \centering
  \resizebox{16cm}{!}{%
    \begin{tabular}{r|c|cc}
      \textbf{Parameter} & \textbf{Symbol in text} & \textbf{Design study value} & \textbf{Updated value} \\
      \hline
      Laser wavelength & & \multicolumn{2}{c}{\SI{1550}{\nano\meter}} \\
      Input power & & \multicolumn{2}{c}{\SI{3}{\watt}} \\
      Arm power         & & \multicolumn{2}{c}{\SI{18}{\kilo\watt}} \\
      \gls{ITM} transmissivity & & \multicolumn{2}{c}{\SI{7000}{\ppm}} \\
      \gls{ETM} transmissivity & & \multicolumn{2}{c}{\SI{6}{\ppm}} \\
      \gls{PRM} transmissivity & $T_{\text{PRM}}$ & \multicolumn{2}{c}{\SI{4.6}{\percent}} \\
      \gls{SRM} transmissivity & & \multicolumn{2}{c}{\SI{10}{\percent}} \\
      Signal recycling detuning & & \multicolumn{2}{c}{\SI{0.6}{\radian}} \\
      Arm cavity lengths & $L_{\text{X}}$, $L_{\text{Y}}$ & \multicolumn{2}{c}{\SI{10}{\kilo\meter}} \\
      Power recycling cavity length & $L_{\text{PRCL}}$  & \multicolumn{2}{c}{\SI{310}{\meter}} \\
      Signal recycling cavity length & $L_{\text{SRCL}}$  & \SI{310}{\meter} & \SI{311.585}{\meter} \\
      Schnupp asymmetry & $\Delta l_{\text{SCH}}$  & \textemdash & \SI{0.08}{\meter} \\
      \gls{DARM} offset & $\delta L_{\text{DARM}}$  & \textemdash & \SI{12}{\pico\meter} \\
      Control sideband frequencies & $f_1$, $f_2$ & \textemdash & \SI{11363101}{\hertz}, \SI{56815505}{\hertz} \\
      Control sideband modulation depths & & \textemdash & \SI{0.1}{\radian} \\
      Demodulation frequencies & & \textemdash & $f_1$, $f_2$, $f_2 - f_1$, $f_1 + f_2$
    \end{tabular}
  }
  \caption[Updated parameters for \ETLF{} in the detuned configuration following the development of the conceptual sensing scheme]{\label{tab:et-lf-updated-parameters}Updated parameters for \ETLF{} in the detuned configuration following the development of the conceptual sensing scheme in this chapter.}
\end{table}

\section{Outlook and future work}
By utilising an \gls{RF} phase modulation scheme we have shown in this chapter that \ETLF{} can in principle be controlled with the presence of a \SI{25}{\hertz} signal recycling cavity detuning. Control noise issues remain unaddressed, as have more exotic control schemes such as the use of additional modulation frequencies or carriers. This section describes some future work that will be necessary to refine and improve the results presented here towards a comprehensive technical design.

\subsection{Optimising the sensing matrix}
In \ALIGO{} the conceptual control scheme was first tested at the \CALTECHFORTYM{} and it is probable that any technical design for \ETLF{} will require a similar test. At this stage a quantitative assessment of the performance that a particular control scheme might provide might take the form of a technique presented by Mantovani and Freise \cite{Mantovani2008} developed for alignment control in \VIRGO{}, but suitable for longitudinal control. This involves the calculation of a \emph{quality parameter} representing the controllability of a given set of sensors and degrees of freedom. This approach only makes sense when $M$ represents the interferometer at its operating point, which means that the residual motion of controlled degrees of freedom does not create a significant cross-coupling. This approach requires hierarchical gain to be simulated as part of a lock acquisition sequence, and so some effort will be required to design some feedback servos.

\subsection{Switching between tuned to detuned operation}
Transition from tuned to detuned signal recycling operating points and vice versa involves a technique which can maintain control of the interferometer as it transitions between two desired set points. When dual recycling was first demonstrated in suspended optics in the \GARCHINGPROTOTYPE{}, it involved a varying frequency offset applied to the \gls{RF} modulation sidebands as the tuning was changed \cite{Freise2000}. This control technique was evolved in \GEO{} where a complicated sequence of actions \cite{Grote2004} including an uncontrolled ``jump'' between two operating points \cite{Hild2007} were performed to reached tuned mode from a detuned start point. In \ETLF{} it is expected that the signal recycling cavity finesse will be too high to allow for a previously demonstrated transition scheme, and so investigations are ongoing to model the impact that combinations of phase- and amplitude-modulated control sidebands added to the input light or \emph{subcarriers} added to the squeezing injection port have on the lock acquisition and control of the signal recycling cavity at arbitrary detunings. Another possibility is to adapt the \emph{arm length stabilisation} system developed for the lock acquisition of \ALIGO{} \cite{Mullavey2012, Staley2014}, whereby a second carrier at a different wavelength is used to lock cavities. This takes advantage of the lower finesse of the second carrier's wavelength in the cavities, allowing for a wider locking range. The cavities are first pre-stabilised using this second carrier before the main carrier is brought to resonance.

\subsection{Sensing and control of seismic and gravity gradient noise}
The \ET{} facility will be located and designed to minimise the impact of seismic noise, but due to the sensitivity requirement for \ETLF{} the microseism must be suppressed from around \num{e-6} to \SI{e-8}{\meter\per\sqrthz} at frequencies between \num{0.1} and \SI{1}{\hertz} to below \SI{e-18}{\meter\per\sqrthz} by \SI{2}{\hertz}. This represents a signal difference of around \num{e10}, and current sensor electronics can typically only provide dynamic range in the region of $\SI{130}{\deci\bel} \approx \num{3e6}$ without significant design effort. To control seismic noise in addition to being able to sense displacements at the required level, a sensor hierarchy will need to be developed.

There is some precedent for seismic isolation from the work carried out in current and past detectors. In \VIRGO{}, the seismic coupling in the superattenuator was suppressed with a local controller in addition to the global feedback from the main interferometric readouts \cite{Acernese2004}. In \ALIGO{}, seismic pre-isolation is performed through the use of a series of displacement and velocity sensors \cite{Stochino2009}, and in the \AEIPROTOTYPE{} the addition of a \emph{suspension platform interferometer} \cite{Gossler2010} is able to reduce seismic noise to the level of around \SI{100}{\pico\meter\per\sqrthz} between \num{0.1} and \SI{1}{\hertz} \cite{Dahl2010}. Such a system would allow a high dynamic range global sensor to control the remaining motion. The particularly challenging aspect for \ETLF{} is that the suppression of this motion must occur over a bandwidth below \SI{2}{\hertz}, which makes the implementation of stable control filters extremely challenging. The results from the control of the \AEIPROTOTYPE{} and the advanced detectors will provide input to the technical design of the seismic isolation system for \ETLF{}.

\section{Summary}
The Einstein Telescope interferometers will create new challenges not previously encountered in the control of gravitational wave interferometers. In particular, the low frequency \ETLF{} detector, a \DRFPMI{}, will have significant cross-couplings between signals from each of the degrees of freedom due to the detuned signal recycling cavity. We have introduced some of the techniques employed in the state of the art \ALIGO{} and \AVIRGO{} detectors for the control of the five longitudinal degrees of freedom in a \DRFPMI{}, and we have shown with simulations that these methods can also be applied to the \ETLF{} interferometer. The presented sensing matrix for \ETLF{}'s longitudinal degree of freedom has reasonable error signals compatible with the experiment's sensitivity requirement. This result will be the basis for future work investigating the dynamics of the control system and the noise present at the site eventually selected for the facility.