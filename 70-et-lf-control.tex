\chapter{\label{c:et-lf-control}Conceptual control scheme for the low frequency Einstein Telescope detector}

% sinks
\newcommand{\AS}{AS}
\newcommand{\POP}{POP}
\newcommand{\REFL}{REFL}

% readout ports, I and Q
\newcommand{\ASDC}{$\text{\AS}_{\text{DC}}$}
\newcommand{\ASFIRSTI}{$\text{\AS}_{\num{11}}^{\text{I}}$}
\newcommand{\ASFIRSTQ}{$\text{\AS}_{\num{11}}^{\text{Q}}$}
\newcommand{\ASSECONDI}{$\text{\AS}_{\num{57}}^{\text{I}}$}
\newcommand{\ASSECONDQ}{$\text{\AS}_{\num{57}}^{\text{Q}}$}
\newcommand{\ASSUMI}{$\text{\AS}_{\num{68}}^{\text{I}}$}
\newcommand{\ASSUMQ}{$\text{\AS}_{\num{68}}^{\text{Q}}$}
\newcommand{\ASDIFFI}{$\text{\AS}_{\num{45}}^{\text{I}}$}
\newcommand{\ASDIFFQ}{$\text{\AS}_{\num{45}}^{\text{Q}}$}
\newcommand{\POPDC}{$\text{\POP}_{\text{DC}}$}
\newcommand{\POPFIRSTI}{$\text{\POP}_{\num{11}}^{\text{I}}$}
\newcommand{\POPFIRSTQ}{$\text{\POP}_{\num{11}}^{\text{Q}}$}
\newcommand{\POPSECONDI}{$\text{\POP}_{\num{57}}^{\text{I}}$}
\newcommand{\POPSECONDQ}{$\text{\POP}_{\num{57}}^{\text{Q}}$}
\newcommand{\POPSUMI}{$\text{\POP}_{\num{68}}^{\text{I}}$}
\newcommand{\POPSUMQ}{$\text{\POP}_{\num{68}}^{\text{Q}}$}
\newcommand{\POPDIFFI}{$\text{\POP}_{\num{45}}^{\text{I}}$}
\newcommand{\POPDIFFQ}{$\text{\POP}_{\num{45}}^{\text{Q}}$}
\newcommand{\REFLDC}{$\text{\REFL}_{\text{DC}}$}
\newcommand{\REFLFIRSTI}{$\text{\REFL}_{\num{11}}^{\text{I}}$}
\newcommand{\REFLFIRSTQ}{$\text{\REFL}_{\num{11}}^{\text{Q}}$}
\newcommand{\REFLSECONDI}{$\text{\REFL}_{\num{57}}^{\text{I}}$}
\newcommand{\REFLSECONDQ}{$\text{\REFL}_{\num{57}}^{\text{Q}}$}
\newcommand{\REFLSUMI}{$\text{\REFL}_{\num{68}}^{\text{I}}$}
\newcommand{\REFLSUMQ}{$\text{\REFL}_{\num{68}}^{\text{Q}}$}
\newcommand{\REFLDIFFI}{$\text{\REFL}_{\num{45}}^{\text{I}}$}
\newcommand{\REFLDIFFQ}{$\text{\REFL}_{\num{45}}^{\text{Q}}$}

% readout ports, collapsed
\newcommand{\ASFIRST}{$\text{\AS}_{\num{11}}$}
\newcommand{\ASSECOND}{$\text{\AS}_{\num{57}}$}
\newcommand{\ASSUM}{$\text{\AS}_{\num{68}}$}
\newcommand{\ASDIFF}{$\text{\AS}_{\num{45}}$}
\newcommand{\POPFIRST}{$\text{\POP}_{\num{11}}$}
\newcommand{\POPSECOND}{$\text{\POP}_{\num{57}}$}
\newcommand{\POPSUM}{$\text{\POP}_{\num{68}}$}
\newcommand{\POPDIFF}{$\text{\POP}_{\num{45}}$}
\newcommand{\REFLFIRST}{$\text{\REFL}_{\num{11}}$}
\newcommand{\REFLSECOND}{$\text{\REFL}_{\num{57}}$}
\newcommand{\REFLSUM}{$\text{\REFL}_{\num{68}}$}
\newcommand{\REFLDIFF}{$\text{\REFL}_{\num{45}}$}

\begin{itemize}
  \item Tobin's talk at G0900745 is really good for explaining DC readout's benefits
  
  \item Show plots of sideband powers not entering the cavities

	\begin{itemize}
	  \item Optical response: with and without mechanical TFs - show the difference it makes to the response at low frequencies, and why it is necessary to turn them off in the case when you're computing a sensing matrix
	\end{itemize}
\end{itemize}

\section{The Einstein Telescope Facility}
In \checkme{2011} a group of scientists primarily based in Europe completed a design study to examine the requirements for a gravitational wave observatory that pushes the \MI{} topology to its limits, while keeping any newly built facility generic enough to allow for the implementation of new topologies as the state of the art evolves. In this study they laid out the expected improvements in technologies to mitigate technical noise sources currently challenging the sensitivity of the third generation of detectors, with a number of practical differences to existing facilities.

The proposed \ET{} facility composes six interferometers split between the three corners of a triangle with length \SI{10}{\kilo\meter} long. The idea is to exploit the geometry to implement \MI{} with \SI{10}{\kilo\meter} at the three vertices of the triangle to benefit from the colocation of multiple interferometers. The extended arm length over existing facilities provides less susceptibility to displacement noise sources by a factor of \cite{Dwyer2015, aligocosmic2016},
\begin{equation}
  \frac{h_{\SI{10}{\kilo\meter}}}{h_{\SI{4}{\kilo\meter}}} = \sqrt{\frac{\SI{10}{\kilo\meter}}{\SI{4}{\kilo\meter}}} \approx 1.58,
\end{equation}
translating into an increase in the observable volume by around $3.95$, assuming that displacement noise sources stay at the same magnitude.

Seismic noise limits the sensitivity of current generation detectors below \SI{10}{\hertz}, and there are astrophysical advantages to being able to achieve good sensitivity at these frequencies \cite{Sathyaprakash2012}. The majority of spinning neutron stars discovered via optical techniques have had orbital frequencies $f$ below \SI{10}{\hertz}. With enhanced low frequency sensitivity, \gls{ET} would have greater sensitivity to such ``chirp'' signals, and as these signals increase in frequency with time proportional to $\frac{1}{f^{8/5}}$ the ability to see them above the noise from an earlier time would assist in being able to characterise the source parameters. Signals at \SI{2}{\hertz} stay in \ETLF{}'s sensitive band for hours instead of milliseconds as witnessed for instance with GW150914 \cite{ET2011} \note{cite detection paper}. Having extra time to observe chirps also provides the possibility to track the signal evolution with corresponding changes in the signal recycling cavity tuning \cite{Simakov2014}.

\subsection{New facility}
Seismic noise in current detectors limits the sensitivity at low frequencies, and creates challenging control requirements due to the \gls{RMS} motion of the mirrors due to the motion of the ground. The interferometer beam tubes will be underground \note{how far?} to mitigate seismic noise coupling.

The suspension systems for the main test masses are based on those of the superattenuator in \AVIRGO{}. The \checkme{\SI{17}{\meter}} long pendulum stage \note{which stage?} pushes the longitudinal resonant frequency down to \checkme{\SI{0.1}{\hertz}}, providing good attenuation of seismic noise above \SI{2}{\hertz}.

\subsection{Xylophone configuration}
To provide maximum astrophysical reach the facility is intended to provide sensitivity across an unprecedented bandwidth, from around \SI{2}{\hertz} to \SI{10}{\kilo\hertz}\textemdash a bandwidth significantly larger than that of existing detectors. It was realised that the most technically feasible option would be to implement a number of colocated, moderate bandwidth detectors, each optimised to provide good sensitivity in a certain band \cite{Hild2010}, an idea first proposed for \ALIGO{} \cite{Conforto2004}. In the \emph{ET-C xylophone} configuration \cite{Hild2010}, a low power, cryogenic interferometer optimises sensitivity to reduce radiation pressure noise at the expense of shot noise, whilst a high power interferometer optimises high frequency sensitivity through the reduction of shot noise.

\subsection{Colocated detectors}

\subsubsection{Antenna pattern}
As the configuration is arranged into a triangle with the interferometers at each of the three vertices, the combined readout signals from each of the interferometers can be combined in such a way as to achieve optimal sensitivity in all directions.

\subsubsection{Null streams}
The use of three separate xylophone interferometers sensing the

The facility will have a triangle of tunnels, each \SI{10}{\kilo\meter} long, with each of the six interferometers inhabiting two of the three;

Each corner of the triangle will contain one low and one high frequency interferometer in a pair to allow for 

\subsection{ET-LF}
The low frequency detector consists of a \DRFPMI{} configuration with a detuned signal recycling cavity. This detuning allows for enhanced sensitivity at the signal recycling cavity pole, where the light-mirror dynamics create an optical spring that provides sensitivity below the \gls{SQL} at the spring frequency (see Section\,\ref{sec:signal-recycling}). The cavities will have \SI{18}{\kilo\watt} of light power, which is considerably lower than that of \ALIGO{} at design sensitivity (approximately \SI{800}{\kilo\watt}); cryogenic test masses to facilitate a reduction in thermal noise and frequency dependent squeezing to reduce quantum noise.

\note{Two filter cavities... p255 design study}

The sensitivity of \ETLF{} is shown in Figure\,XXX

\subsection{ET-HF}

The sensitivity of \ETHF{} is shown in Figure\,XXX (same plot as ET-LF)

\note{One filter cavity... p255 design study}

\section{Longitudinal control of an interferometer with multiple degrees of freedom}
A successful control scheme for an interferometer must satisfy a number of requirements. When the interferometer is in its uncontrolled state, the control scheme must be able to bring it to the operating point (\emph{lock acquisition}). Once at the operating point, it must be able to keep it there for a long period of time by controlling the impact of noise. Finally, the signal that represents the gravitational wave channel must have low enough noise, and therefore high enough sensitivity, to meet the design goals of the scheme.

The final lock acquisition scheme is inextricably linked to the technical environment in which the interferometer will operate, and so it is inappropriate to discuss this while \gls{ET}'s technical design is unknown. We will focus our efforts, therefore, on the second and third challenges. The low-noise readout in both \ETLF{} and \ETHF{} will be \gls{DC} readout, which is the standard technique for \GEOHF{} and the advanced detectors \cite{Hild2007, Ward2008, Fricke2012}.

\subsection{Low-noise readout of the gravitational wave channel}
In Equation\,\ref{eq:mich-p-out} we see that a static field $\frac{P_{\text{in}}}{2}$ is present upon the photodetector, independent of the arm length change. In experiments where sensitivity can be sacrificed for simplicity, often it is practical to keep the interferometer at an operating point commonly referred to as ``half way up the fringe''. Here, the interferometer's mirrors are nominally positioned such that the output signal is oscillating about the midpoint between crest and trough (see Figure\,\ref{fig:optimal-operating-point}). As the gradient is steepest at this point, any small changes to the relative arm length of the Michelson interferometer result in a significant difference in power at the photodetector. This operating point, however, is not optimal in terms of \emph{sensitivity} to arm length fluctuations.

%% FIXME: change this plot's x-labels to use wavelength, to fit with the conclusion in the text.
\begin{figure}
  \centering
  \includegraphics[width=\columnwidth]{graphics/generated/from-python/70-optimal-operating-point.pdf}
  \caption[Fringe]{\label{fig:optimal-operating-point}Optimal operating point.}
\end{figure}

By inspecting Equation\,\ref{eq:mich-p-out}, it is clear to see that there must exist, in cases where there is a signal due to a difference in arm length, a static photodetector power independent of the arm length. This does not contribute any displacement information to the measurement, but does contribute shot noise:
\begin{equation}
  P_{\text{shot, out}} = \sqrt{2 h f_0 P_{\text{in}}},
\end{equation}
where $h$ is Planck's constant, $f_0$ is the light frequency and $P_{\text{in}} = A_{\text{in}}^2$, the power entering the interferometer at the beam splitter. The optimally sensitive operating point is therefore not simply one which maximises the signal gradient, but rather one which maximises the \gls{SNR}. The \gls{SNR} is:
\begin{equation}
  \text{SNR} = \frac{P_{\text{out}}}{P_{\text{shot, out}}} = \sqrt{\frac{P_{\text{in}}}{4 h f_0}} \left( 1 + \cos \left(k \Delta x \right) \right).
\end{equation}

The $\Delta x$ term in Equation\,\ref{eq:mich-p-out} is a combination of a static arm length \emph{detuning}\textemdash representing the arm length mismatch required to reach the desired operating point\textemdash and a differential gravitational wave signal $\Delta x_{\text{GW}}$. A suitable choice of $ x_{\text{tune}}$ can remove the majority of the static power present at the output. Setting the slope of the \gls{SNR} with respect to the tuning to zero,
\begin{equation}
  \frac{\Delta \text{SNR}}{\Delta x_{\text{GW}}} = -k \sqrt{\frac{P_{\text{in}}}{4 h f_0}} \sin \left(k \Delta x\right) = 0,
\end{equation}
we find that maximum \gls{SNR} is achieved for static tunings 
\begin{equation}
  \Delta x \text{ mod } \lambda = 0.
\end{equation}
This result shows that the optimal operating point in terms of \gls{SNR} is at the point where the light from the two arms interferes destructively. While any multiple of $\lambda$ will satisfy the \gls{SNR} condition as defined, in reality we have not considered laser noise coupling. The more matched the arm lengths are, the lower the laser noise couples to the output port. In reality there are also mismatches in the reflectivities of the mirrors in the arms: this creates an asymmetry called a \emph{contrast defect} which leads to additional shot noise at the output port.

\subsubsection{The dc readout technique}
The \gls{DC} readout technique is a form of homodyne readout (see Section\,\ref{sec:homodyne-readout}) that involves a compromise between the best sensitivity and the technical complexity. The operating point is kept close to the dark fringe to optimise shot noise, but a slight offset is introduced between the differential round trip phase of the arms in order to allow some of the carrier light to enter the readout port where it acts as a homodyne local oscillator to the signal sidebands. In practice, this detuning \textemdash of the order \SI{}{\pico\meter} at the arm cavities in \ALIGO{}\textemdash is sufficiently small to allow the required sensitivities to be reached. This provides improved sensitivity over heterodyne techniques \cite{Fricke2012}, and in squeezed interferometers such as \ETLF{} and \ETHF{} it avoids the need to inject squeezing at \gls{RF} sideband frequencies in addition to the carrier frequency.

\subsection{\label{sec:dofs-of-drfpmi}Degrees of freedom of the \DRFPMI{}}
As \ETLF{} and \ETHF{} will employ \gls{DC} readout, there is no local oscillator phase or alignment to control (see Section\,\ref{sec:homodyne-readout}). The degrees of freedom that must be controlled in order to reach the design sensitivity are the arm cavity differential and common modes, the length between the beam splitter and the \glspl{ITM} and the recycling cavity lengths. These are defined in the following sections.

\begin{figure}
  \centering
  \includegraphics[width=\columnwidth]{graphics/generated/from-svg/70-darm-schnupp-offsets.pdf}
  \caption[Differential arm and Schnupp offsets in a \DRFPMI{}]{\label{fig:darm-schnupp-offsets}\note{Make this a generic diagram of ETLF with the ports and readouts defined: output port in particular, as it is mentioned in text}}
\end{figure}

\subsubsection{Differential arm cavity length}
\checkme{Should be held at the dark fringe by means of differential feedback to the end test masses...}

\subsubsection{Power recycling cavity length}
The power recycling cavity should be resonant for the input light in order to optimally recycle light reflected from the beam splitter back towards the laser. The power recycling cavity length can be defined in terms of the average distance between the power recycling mirror and the two \glspl{ITM}:

equation...

\subsubsection{Michelson length}
The length between the \glspl{ITM} and the beam splitter should be held constant to keep the amount of carrier and sideband power in the signal recycling cavity stable, which avoids the need for complicated time-varying control signals. This is kept constant by feeding back to the position of the beam splitter or with differential actuation upon to the \glspl{ITM}.

\note{Lots of useful info in Heinzel 2002}

\subsubsection{Signal recycling cavity length}
The signal recycling length, along with the signal recycling mirror transmissivity, determines the bandwidth of the signal extraction and therefore needs controlled in order to keep the signal to noise stable...

\section{\label{sec:et-lf-control-challenges}Control challenges with the \ET{}}
Both \ETLF{} and \ETHF{} will present new challenges to the control of large-scale \DRFPMI{}s. \ETHF{} can to some extent be seen as a version of \ALIGO{} and \AVIRGO{} that pushes the state of the art of high power, heavy mirror, room temperature detectors, and so it may be possible to adapt much of the advance detectors' strategies for both longitudinal and angular control. Some aspects such as the use of LG33 modes and the presence of parametric instabilities, however, require extensive research to understand the implications they may have on control, and so much work remains to be done before a technical design for the \ET{} detectors will be ready. \ETLF{} also uses a topology that resembles existing generation detectors, but it pushes the sensitivity at low frequencies further down and this presents additional challenges. This chapter will discuss the longitudinal control of \ETLF{}, focusing in particular on the challenge of controlling the interferometer in its detuned state.

\ETLF{} employs a detuned signal recycling cavity to increase the sensitivity of the interferometer at low frequencies. The ability for a detuned signal recycling to shift the most sensitive frequency has been demonstrated in \GEO{} between \SI{200}{\hertz} and \SI{1}{\kilo\hertz} \cite{Hild2006}. \ETLF{} plans to detune the signal recycling cavity to around \SI{25}{\hertz}, which has never before been demonstrated in suspended audio-band detectors. With tuned signal recycling, the \gls{RF} sidebands used for control of the differential arm cavity mode are present within the signal recycling cavity with equal amplitude. In detuned operation, the sidebands no longer have equal amplitude and so some of the interferometer's noise couplings that otherwise cancel at the output in the tuned case no longer cancel \cite{Hild2007}. The phase modulation of the control sidebands (created by an \gls{EOM} on the input path) also gets partially converted to amplitude modulation by the detuning, and this can lead to issues with the dynamic range of the photodetector to be used to sense the readout \cite{Grote2007}. Finally, the asymmetric sidebands create additional zero crossings in the error signal from the signal recycling cavity which for the tuned case do not appear. Often the control of a detuned signal recycling cavity involves locking to the zero crossing of a sideband instead of the carrier, but for low detunings it can be difficult to find a port at which to sense the motion of the signal recycling cavity decoupled from other cavities.

The rest of this chapter will discuss a conceptual sensing scheme for \ETLF{} and discuss the future work that must be undertaken before a technical design for the control of \ETLF{} can be completed.

\section{Longitudinal sensing scheme for ET-LF}

\subsection{Scope and method}
The simulations required to devise a longitudinal sensing scheme are conducted with Optickle and involve the plane wave approximation. Angular control of the interferometer is not expected to present a challenge greater than that already seen in the advanced detectors, although future angular sensing and control simulations will be possible with the models developed over the course of this work.

The sensing scheme assumes that the interferometer has been brought close to its operating point by a lock acquisition routine. A complete control scheme would require effort to devise a lock acquisition scheme, and for a \DRFPMI{} this is not straightforward and can involve the transition through many intermediate control states before the interferometer is brought to the operating point.

We only consider the control of the interferometer in terms of its response. It is well known that noise coupling between longitudinal degrees of freedom of detuned \DRFPMI{}s can be significant \cite{Hild2007}, but a crucial initial step before any control noise simulations can be undertaken is the development of a sensing scheme. For \ETLF{}'s sensing scheme at the operating point we loosely follow the approach taken for \ALIGO{} \cite{Abbott2010} and \AVIRGO{} \cite{Vajente2008} given that these represent the most sensitive \DRFPMI{}s built to date. To simplify the steps required to produce the scheme the least constrained parameters are defined first and parameters which depend on previously defined parameters are then determined sequentially. Given the constraints from the design study, we begin with the transmissivity of the power recycling mirror and the frequency of the control sidebands, and then use these to define auxiliary lengths within the interferometer and finally calculate the arm cavity offset required for the \gls{DC} readout to work. With these parameters set, we can determine ports at which sensible error signals representing each degree of freedom can be extracted and we can then define a sensing matrix that shows the cross-couplings between each of the degrees of freedom. This is the starting point for control loop noise studies which will be the subject of future work.

\subsection{Optimal input coupling}
As discussed in Section\,\ref{sec:power-recycling}, placing a power recycling mirror in an \MI{} creates an additional cavity between the input light and the arms. The intention of the power recycling mirror is to minimise the light reflected back towards the laser, and in order to do this the cavity it creates should be \emph{impedance matched} (see, for example, Section\,5.1 of \cite{Freise2010}). The transmissivity of the mirror determines the impedance matching, and it can be determined using knowledge of the light lost within the interferometer. The design study defines the loss per optic to be \SI{35}{\ppm}, and the transmissivity of the \glspl{ETM} is \SI{6}{\ppm}; both contribute to the impedance matching. Figure\,\ref{fig:reflected-power-vs-prm-trans} shows the light power leaving the power recycling mirror heading back towards the laser. The minimum reflected power corresponds to a power recycling mirror transmissivity of \num{0.046}, which is comparable to that of the \glspl{ITM}.

\begin{figure}
  \centering
  \includegraphics[width=\columnwidth]{graphics/generated/from-python/70-reflected-power-vs-prm-transmissivity.pdf}
  \caption[Reflected power from \ETLF{} as a function of power recycling mirror transmissivity]{\label{fig:reflected-power-vs-prm-trans}Reflected power from \ETLF{} as a function of power recycling mirror transmissivity. For optimal coupling of the input laser light to the interferometer, the transmissivity of the power recycling mirror must be set to balance the input light with the total loss from the interferometer. For \ETLF{} with loss as per the design study, this transmissivity should be \num{0.046}.}
\end{figure}

For a future technical design study, additional allowance should be made for scattering loss and other mechanisms which will affect impedance matching. A slight overestimate of the transmissivity of the power recycling mirror is often beneficial to allow for uncertainty in the manufacturing of optical substrates and coatings.

\subsection{\label{sec:decoupled-sidebands}Control of the interferometer's degrees of freedom}
As with any actively controlled interferometer, error signals must be extracted from \ETLF{} that can be used to correct perturbations of its mirrors from the interferometer's operating point. In gravitational wave detectors these error signals are usually derived from an extension of the Pound-Drever-Hall scheme discussed in Section\,\ref{sec:pdh}, which was generalised to \MI{}s with the advent of \emph{Schnupp} modulation. In prototype facilities this control method was extended to power- \cite{Regehr1995} and \DRFPMI{}s \cite{Heinzel1998}.

\subsubsection{Gain hierarchy}
It is not necessary to obtain error signals that contain solely information representing one degree of freedom as long as the individual constituents can be separated after detection. By employing gain hierarchy it can be possible to suppress signal content at a particular readout from another degree of freedom, and this has been demonstrated in \LIGO{} \cite{Fritschel2001}. The most problematic extraneous signal content in signals demodulated at the main sideband modulation frequencies tend to arise from the common and differential arm cavity modes (\gls{CARM} and \gls{DARM}, respectively); however these signals tend to be suppressed in the error signals after these degrees of freedom are controlled. The common mode fluctuations tend to arise from the laser's noise, and because this occurs across the sensitive band of the detector this requires a wide bandwidth servo. Control of \gls{DARM}, and the auxiliary modes representing the power and signal recycling cavity lengths (\gls{PRCL} and \gls{SRCL}, respectively) and the inner Michelson degree of freedom (\gls{MICH}), tends to require a much smaller servo bandwidth. Through the appropriate choice of servo hierarchy the cross-couplings present at each sensor from secondary degrees of freedom can be minimised.

\subsubsection{Control sidebands}
\ETLF{} will require at least two sideband frequencies in order to control the common mode arm (\gls{CARM}), power and signal recycling (\gls{PRCL} and \gls{SRCL}) and Michelson (\gls{MICH}) cavity lengths because it is not possible to decouple the motion of multiple cavities when they are sensed by the same light. Suitably decoupled sideband resonances provide error signals representing the cavities to be controlled, though it is not necessary to use a different frequency for each cavity; indeed it is beneficial from the point of view of quantum shot noise on the sensors to try to find unique combinations of the two sideband frequencies to sense each of the five degrees of freedom.

The control sideband frequencies must be chosen to be outside an integer multiple of the free spectral range (\gls{FSR}, see Appendix\,\ref{sec:cavity-fom}) of the arm cavities in order to prevent them from entering the arm cavities and getting resonantly enhanced by the arm cavity finesse, and should ideally be \gls{RF} to benefit from the noise advantages presented in Section\,\ref{sec:pdh}. An upper limit of \SI{100}{\mega\hertz} is reasonable, as higher frequency demodulations are difficult to implement in photodetector electronics.

Often, demodulation at the sideband frequency at a particular port of the interferometer contains useful signal content representing one of the degrees of freedom with minimal cross-couplings from other cavities; however, sometimes a signal with greater decoupling, particularly for the inner degrees of freedom, can be found by demodulating the light at some combination of sideband frequencies \cite{Strain2003, Barr2006}. In \ALIGO{}, for example, some of the control signals used in the operating mode involve demodulation at the sum or difference of the two sideband frequencies \cite{Abbott2010}.

\subsubsection{Control sideband frequencies}

\paragraph{Control sideband resonance in the recycling cavities}
The \SI{310}{\meter} recycling cavity lengths are have \gls{FSR} \SI{483.5}{\kilo\hertz}. We therefore want to find a frequency $f_2$ which approximately satisfies:
\begin{equation}
  \label{eq:src-fsr}
  \begin{split}
    L_{\text{SRC}} &= A \frac{c_0}{2 f_2} \\
                   &\neq B \frac{c_0}{2 f_1},
  \end{split}
\end{equation}
where $A$ and $B$ are positive integers. We must also ensure that the sidebands are resonant within the power recycling cavity. As the arm cavities at the operating point reflect the light back towards the power recycling cavity, the resonant condition is instead a half-integer multiple of the power recycling cavity \gls{FSR}, i.e.:
\begin{equation}
  \label{eq:prc-fsr}
  L_{\text{PRC}} = \left(C + \frac{1}{2} \right) \frac{c_0}{2 f_1},
\end{equation}
for positive integer $C$.

\paragraph{Avoiding resonance of the control sidebands in the arm cavities}
In addition to optimising the control sideband frequencies using Equations \ref{eq:src-fsr} and \ref{eq:prc-fsr}, we must prevent the sidebands from entering into the arm cavities. The arm cavity \gls{FSR} is given by Equation\,\ref{eq:fsr} to be \SI{14.99}{\kilo\hertz} for the \SI{10}{\kilo\meter} arms. As an integer multiple of the arm cavity \gls{FSR} would allow optimal coupling of the sidebands into the arm cavities, one might assume that an odd half-integer multiple would be optimally anti-resonant; however, in this scenario the lower higher-order control sidebands, necessarily created by the phase modulation upon the \gls{EOM} (see Appendix\,\ref{eq:field-phase-bessel}), would become resonant, and so we choose to offset the sideband frequency slightly from the anti-resonant condition. We therefore stipulate two further requirements:
\begin{equation}
  \label{eq:arm-fsr}
  \begin{split}
    L_{\text{Arm}} &= \left(D + \frac{1}{2} \right) \frac{c_0}{2 f_1} \\
                   &= \left(D + \frac{1}{2} \right) \frac{c_0}{2 f_2},
  \end{split}
\end{equation}
for positive integer $D$.

\paragraph{Control sideband frequencies}
For \ETLF{} we chose the first sideband frequency $f_1$ to be \SI{11363101}{\hertz} which satisfies Equations \ref{eq:src-fsr}, \ref{eq:prc-fsr} ($C = 23$) and \ref{eq:arm-fsr}. We can then choose the second sideband to be an integer multiple of the first in order to allow both to be resonant within the power recycling cavity. We chose the second sideband frequency to be $f_2 = 5f_1 = \SI{56815505}{\hertz}$, which is a high enough multiple of $f_1$ that we can investigate the use of different beats between $f_1$ and $f_2$ for control purposes (see Section\,\ref{sec:etlf-readout-ports}). We also assume a modulation depth of \num{0.1} for $f_1$ and $f_2$ to minimise higher order modulation sidebands while keeping a reasonable amount of light power in the sidebands.

\subsubsection{Schnupp asymmetry}
In a \DRFPMI{} the arm lengths must be slightly mismatched to allow for the control sidebands to enter the signal recycling cavity and therefore provide an error signal for the signal recycling cavity length. This is called the \emph{Schnupp asymmetry}, which comes from the Schnupp \emph{modulation} scheme developed for \MI{}s as an evolution upon the Pound-Drever-Hall technique in \FP{} cavities \note{cite something}. As described in Section\,\ref{sec:laser-noise}, laser frequency noise in a \MI{} is best suppressed when the arm lengths are matched, but the asymmetry introduced in this case is small compared to the length of the arms and so this does not have a significant impact on laser noise suppression.

For a Schnupp asymmetry $L_{\text{asy}}$, the two arms have length $L + \frac{L_{\text{asy}}}{2}$ and $L - \frac{L_{\text{asy}}}{2}$ where the nominal length is $L$. This mismatch is ideally as small as possible, to allow for the greatest suppression of laser noise, but large enough to allow for sufficient sideband power in the desired recycling cavities.

The choice can be made as to whether both sidebands should be resonant in both recycling cavities, or to have one cavity without a sideband resonance, and this is governed by the Schnupp asymmetry $L_{\text{asy}}$. The transmissivity of the power and signal recycling mirrors determine the Schnupp lengths corresponding to each case. A small offset of a few \SI{}{\centi\meter} between the Michelson arm lengths allows for the coupling of both sidebands into both recycling cavities, whilst a larger offset of a few tens of \SI{}{\centi\meter} prevents one sideband from entering the signal recycling cavity \cite{Vajente2008}. Both methods of control are possible, with the former being implemented in \KAGRA{} \note{reference for KAGRA?} and the latter in \ALIGO{} \cite{Abbott2010}. We choose to follow the \ALIGO{} approach so that we can optimise one sideband frequency ($f_1$) to be anti-resonant in the signal recycling cavity, to assist with lock acquisition. As the signal recycling cavity will probably be the last cavity controlled in the lock acquisition sequence, the anti-resonance of $f_1$ there prevents the signal recycling mirror's motion from having a significant effect on the cavities controlled using $f_1$.

Figure\,\ref{fig:sideband-powers-vs-schnupp-detuned} shows the power of each sideband field in the recycling cavities of \ETLF{} given the Schnupp asymmetry and recycling cavity lengths with detuned signal recycling. Since the Schnupp asymmetry is a macroscopic length, it is not easily adjusted during operation, and so it is necessary to set this length in the design phase. Here we choose a Schnupp asymmetry that attempts to maximise the difference in power in the signal recycling cavity between the two sideband frequencies in detuned operation. This is around \SI{0.08}{\meter}.

\begin{figure}
  \centering
  \includegraphics[width=\columnwidth]{graphics/generated/from-python/70-sideband-powers-vs-schnupp-detuned.pdf}
  \caption[Power of the control sidebands in the cavities of \ETLF{} in the detuned configuration]{\label{fig:sideband-powers-vs-schnupp-detuned}Power of the control sidebands in the cavities of \ETLF{} during detuned operation. A small offset, called the \emph{Schnupp asymmetry}, is intentionally introduced to the Michelson length in order to allow the sidebands $f_1$ and $f_2$ to couple to the signal recycling cavity for the purposes of control. Here, $f_2$ is resonant but $f_1$ is not, and so there are discriminating signals for the power and signal recycling cavity. The optimal Schnupp asymmetry is one which provides the best decoupling of sideband error signals representing the lengths of the power and signal recycling cavities. The power is a good estimate for what the relative sensitivity of the sidebands in the recycling cavities will be.}
\end{figure}

\subsubsection{Optimisation of the signal recycling cavity length}
Note the discrepancy between two of the control sideband frequency constraints: the power and signal recycling cavities cannot both be simultaneously resonant and anti-resonant to $f_1$ and $f_2$ given that $f_2 = 5 f_1$. To resolve this discrepancy we can scan the length of the signal recycling cavity in order to find where the next \gls{FSR} is encountered for $f_2$, and this is shown in Figure\,\ref{fig:sideband-powers-srcl-detuned}. We can see that changing the signal recycling cavity length from \SI{310}{\meter} to \SI{311.585}{\meter} results in the desired sideband resonance condition for the upper $f_2$ sideband. The power of $f_2$ in the power recycling cavity drops as the lower and upper sidebands get critically coupled into the signal recycling cavity. As the detuning in \ETLF{} is so large, the signal recycling cavity is not resonant for both the upper and lower $f_2$ sidebands and so the control of the power recycling cavity must be achieved before that of the signal recycling cavity to minimise the signal content in $f_2$ arising from the former.

In the signal recycling cavity, $f_2$ provides an error signal for the length that is \num{650} times larger than the equivalent for $f_1$. Meanwhile, $f_1$ provides an error signal for the power recycling cavity that is a factor of \num{13} larger than that of $f_2$. If this difference proves not to be high enough, it can be increased at the expense of the difference between the two in the signal recycling cavity by scaling the modulation depths of $f_1$ and $f_2$.

\begin{figure}
  \centering
  \includegraphics[width=\columnwidth]{graphics/generated/from-python/70-sideband-powers-vs-srcl-detuned.pdf}
  \caption[Power of the control sidebands in the signal recycling cavity as a function of length of \ETLF{} in the detuned configuration]{\label{fig:sideband-powers-srcl-detuned}Sideband powers for detuned signal recycling.}
\end{figure}

\subsection{Dark fringe offset}
As described in Section\,\ref{sec:homodyne-readout}, \gls{DC} readout at the output port of a \DRFPMI{} requires carrier light to be present to act as a local oscillator (phase reference) for the signal sidebands. In a perfectly controlled interferometer there is no classical light at the output port and so this offset must be introduced by differentially detuning the interferometer's arms by a small amount to create the appropriate dark fringe offset. In practice, asymmetries which create differential detuning are already present in the arms, for example arising from mismatched arm cavity finesse or asymmetric beam splitter reflectivity.

For our simulations, we do not assume any asymmetries and so we can intentionally introduce a dark fringe offset with an offset in the \gls{MICH} or \gls{DARM} degrees of freedom. While a \gls{DARM} offset has the disadvantage that it involves the creation of an optical spring due to the high light power in the arms \cite{Heidmann2011}, it has favourable noise couplings compared to a \gls{MICH} offset \cite{Vajente2011}.

Figure\,\ref{fig:total-power-vs-darm-offset-detuned} shows the power at the output port and in the arm cavities as a function of \gls{DARM} offset in the detuned configuration. For an offset of \SI{12}{\pico\meter}, the power at the output can be set to around \SI{10}{\milli\watt} with a difference of about 3\% in the power in the arms. Standard photodetectors used in \ALIGO{} and \AVIRGO{} can usually handle around \SI{100}{\milli\watt} but in \ETLF{} the \gls{DARM} offset required to reach this figure would reduce the power in the Y arm cavity significantly, affecting sensitivity. The optical spring in this case occurs at a frequency of around \checkme{\SI{0.02}{\hertz}}, which is \checkme{well} outside the sensitive band.

\begin{figure}
  \centering
  \includegraphics[width=\columnwidth]{graphics/generated/from-python/70-total-power-vs-darm-offset-detuned.pdf}
  \caption[Carrier power at the output port of \ETLF{} in detuned configuration as a function of differential arm cavity offset]{\label{fig:total-power-vs-darm-offset-detuned}Carrier power at the output port of \ETLF{} in detuned configuration as a function of differential arm cavity length (\gls{DARM}) offset. In \gls{DC} readout, as with any readout scheme, a steeper error signal slope results in better sensitivity to small changes in position. The differential arm detuning required to allow carrier light to enter the dark port for \gls{DC} readout involves an increase or decrease in the microscopic length of each arm cavity, and this changes the circulating power. The compromise must be made between maximising the slope while maintaining reasonably balanced arm cavities to prevent optical springs from entering the sensitive band and additional noise coupling.}
\end{figure}

\subsection{Power in each light field}
The power in each field within each relevant space or cavity of the interferometer is shown in Tables\,\ref{tab:et-lf-detuned-dc-powers} for the detuned interferometer.

\begin{table}
  \centering
  \resizebox{16cm}{!}{%
    \begin{tabular}{r|ccccccccc|c}
      & \textbf{\SI{-68}{\mega\hertz}} & \textbf{\SI{-57}{\mega\hertz}} & \textbf{\SI{-45}{\mega\hertz}} & \textbf{\SI{-11}{\mega\hertz}} & \textbf{Carrier} & \textbf{\SI{11}{\mega\hertz}} & \textbf{\SI{45}{\mega\hertz}} & \textbf{\SI{57}{\mega\hertz}} & \textbf{\SI{68}{\mega\hertz}} & \textbf{Total} \\
      \hline
      Input & 0 & 0 & 0 & 0 & \SI{3}{\watt} & 0 & 0 & 0 & 0 & \SI{3}{\watt} \\
      After modulators & \SI{19}{\micro\watt} & \SI{7.4}{\milli\watt} & \SI{19}{\micro\watt} & \SI{7.4}{\milli\watt} & \SI{3}{\watt} & \SI{7.4}{\milli\watt} & \SI{19}{\micro\watt} & \SI{7.4}{\milli\watt} & \SI{19}{\micro\watt} & \SI{3}{\watt} \\
      Power recycling cavity & \SI{1}{\milli\watt} & \SI{521}{\milli\watt} & \SI{1}{\milli\watt} & \SI{410}{\milli\watt} & \SI{65}{\watt} & \SI{407}{\milli\watt} & \SI{1}{\milli\watt} & \SI{34}{\milli\watt} & \SI{1}{\milli\watt} & \SI{66}{\watt} \\
      Michelson cavity & \SI{1}{\milli\watt} & \SI{228}{\milli\watt} & \SI{1}{\milli\watt} & \SI{208}{\milli\watt} & \SI{33}{\watt} & \SI{204}{\milli\watt} & \SI{1}{\milli\watt} & \SI{27}{\milli\watt} & \SI{1}{\milli\watt} & \SI{33}{\watt} \\
      Arm cavity X & \SI{1}{\nano\watt} & \SI{557}{\micro\watt} & \SI{1}{\nano\watt} & \SI{9}{\milli\watt} & \SI{18}{\kilo\watt} & \SI{8.8}{\milli\watt} & \SI{1}{\nano\watt} & \SI{65}{\micro\watt} & \SI{1}{\nano\watt} & \SI{18}{\kilo\watt} \\
      Arm cavity Y & \SI{1}{\nano\watt} & \SI{772}{\micro\watt} & \SI{1}{\nano\watt} & \SI{8.7}{\milli\watt} & \SI{18}{\kilo\watt} & \SI{8.8}{\milli\watt} & \SI{1}{\nano\watt} & \SI{68}{\micro\watt} & \SI{1}{\nano\watt} & \SI{18}{\kilo\watt} \\
      Signal recycling cavity & \SI{1}{\nano\watt} & \SI{3.8}{\milli\watt} & \SI{1}{\nano\watt} & \SI{69}{\micro\watt} & \SI{15}{\milli\watt} & \SI{41}{\micro\watt} & \SI{1}{\nano\watt} & \SI{26}{\milli\watt} & \SI{1}{\nano\watt} & \SI{45}{\milli\watt} \\
      Output & \SI{1}{\nano\watt} & \SI{756}{\micro\watt} & \SI{1}{\nano\watt} & \SI{14}{\micro\watt} & \SI{3}{\milli\watt} & \SI{8.3}{\micro\watt} & \SI{1}{\nano\watt} & \SI{5.1}{\milli\watt} & \SI{1}{\nano\watt} & \SI{9}{\milli\watt} \\
    \end{tabular}
  }
  \caption{\label{tab:et-lf-detuned-dc-powers}Powers in various parts of \ETLF{} in the detuned configuration. The input light is passed through \glspl{EOM} which impart control sidebands at two pairs of frequencies offset from the carrier. As the arm cavity \gls{FSR} is \checkme{higher} than that of the control sideband frequencies, the sideband light power in the arms is vastly smaller than the carrier power.}
\end{table}

\subsection{\label{sec:etlf-readout-ports}Readout ports}
\note{Figure xx} shows some available readout ports for \ETLF{} where the sidebands and carrier may be measured for the purposes of control:
\begin{itemize}
  \item \textbf{REFL} (\emph{reflected}) senses the light reflected from the interferometer back towards the input laser.
  \item \textbf{POP} (\emph{pick off \gls{PRCL}}) senses the light in the power recycling cavity. Although the diagram shows a beam splitter, in reality this will probably be obtained via a secondary mirror reflection (such as the anti-reflective coating on the power recycling mirror).
  \item \textbf{AS} (\emph{asymmetric}) senses the light at the output port of the \DRFPMI{}.
\end{itemize}

For the purposes of the control simulations, each readout port contains photodetectors demodulating the signals at \gls{DC}, $\pm f_1$ (\SI{\pm11}{\mega\hertz}), $\pm f_2$ (\SI{\pm57}{\mega\hertz}), $-f_1 - f_2$ (\SI{-68}{\mega\hertz}), $-f_1 + f_2$ (\SI{-45}{\mega\hertz}), $f_2 - f_1$ (\SI{45}{\mega\hertz}) and $f_2 + f_1$ (\SI{68}{\mega\hertz}). There are therefore \num{9} classical light fields propagated through the interferometer, and these are shown in Table\,\ref{tab:et-lf-probes}.

\begin{table}
  \centering
  {\renewcommand{\arraystretch}{1.2} % for extra vertical spacing between rows
    \begin{tabular}{r|ccc}
      \textbf{Frequency offset} & \textbf{Output port} & \textbf{Power recycling cavity pick-off} & \textbf{Reflected light pick-off} \\
      \hline
      \num{0} & \ASDC{} & \textemdash & \textemdash \\
      $f_1$    & \ASFIRSTI{}, \ASFIRSTQ{} & \POPFIRSTI{}, \POPFIRSTQ{} & \REFLFIRSTI{}, \REFLFIRSTQ{} \\
      $f_2$    & \ASSECONDI{}, \ASSECONDQ{} & \POPSECONDI{}, \POPSECONDQ{} & \REFLSECONDI{}, \REFLSECONDQ{} \\
      $f_2 - f_1$ & \ASDIFFI{}, \ASDIFFQ{} & \POPDIFFI{}, \POPDIFFQ{} & \REFLDIFFI{}, \REFLDIFFQ{} \\
      $f_1 + f_2$ & \ASSUMI{}, \ASSUMQ{} & \POPSUMI{}, \POPSUMQ{} & \REFLSUMI{}, \REFLSUMQ{}
    \end{tabular}
  }
  \caption[Probes that sense the light fields propagating within \ETLF{}]{\label{tab:et-lf-probes}Probes that sense the light fields propagating within \ETLF{} with respect to the carrier frequency. The carrier is sensed by \ASDC{} at the interferometer's output port, which is the light that propagates through the signal recycling mirror. The control sidebands, and the beats between the sidebands, are sensed at the same port demodulated at each relevant frequency, along with similar readouts sensing a small transmission of light through a folding mirror in the power recycling cavity (\POP{}) and the light reflected from the interferometer (\REFL{}).}
\end{table}

\subsection{Driving coefficients}
In order to calculate the response of each degree of freedom as defined in Section\,\ref{sec:dofs-of-drfpmi}, we must define the mirrors that we will actuate upon to create this motion. \gls{CARM} and \gls{DARM} are defined in terms of the motion of the \glspl{ETM}, with the difference between the two degrees of freedom being whether the motion of the two \gls{ETM} is in-phase or out-of-phase, respectively. The \gls{PRCL} and \gls{SRCL} degrees of freedom are defined as motion of the power and signal recycling mirrors, respectively. \gls{MICH} is a little more tricky, as simply moving the \glspl{ITM} results in a signal arising from the arms. We can instead either move the beam splitter and the recycling mirrors differentially, or both the \gls{ITM} and \gls{ETM} in each arm together, and differentially between the two cavities. We choose the latter to avoid the need to correct the beam splitter's motion for its angle of incidence. Note that in this particular case the driving coefficients appear at first glance to be for common Michelson motion, but this is due to the fact that the cavity mirrors' primary surfaces are facing each other and so driving must be applied with opposite sign to each \gls{ETM} and \gls{ITM} pair. The driving coefficients are shown in Table\,\ref{tab:et-lf-driving-coefficients}.

\begin{table}
  \centering
  \begin{tabular}{r|ccccccc}
    & \textbf{\gls{ITM} X} & \textbf{\gls{ETM} X} & \textbf{\gls{ITM} Y} & \textbf{\gls{ETM} Y} & \textbf{\gls{BS}} & \textbf{\gls{PRM}} & \textbf{\gls{SRM}} \\
    \hline
    \gls{CARM} & \num{0} & \num{0.5} & \num{0} & \num{0.5} & \num{0} & \num{0} & \num{0} \\
    \gls{DARM} & \num{0} & \num{0.5} & \num{0} & \num{-0.5} & \num{0} & \num{0} & \num{0} \\
    \gls{MICH} & \num{-0.5} & \num{0.5} & \num{0.5} & \num{-0.5} & \num{0} & \num{0} & \num{0} \\
    \gls{PRCL} & \num{0} & \num{0} & \num{0} & \num{0} & \num{0} & \num{1} & \num{0} \\
    \gls{SRCL} & \num{0} & \num{0} & \num{0} & \num{0} & \num{0} & \num{0} & \num{1} \\
  \end{tabular}
  \caption[Driving coefficients for each mirror and each degree of freedom of \ETLF{}]{\label{tab:et-lf-driving-coefficients}Driving coefficients for each mirror and each degree of freedom of \ETLF{}. \gls{CARM} and \gls{DARM} involve the driving of the \glspl{ETM} in- and out-of-phase, respectively. \gls{MICH} involves moving the \glspl{ITM} differentially, but to avoid sensing \gls{DARM} effects the \glspl{ETM} must be moved too. In this case, the \glspl{ETM} must be driven with an opposite sign to their respective \glspl{ITM} due to the fact that the mirrors' high-reflective surfaces are in opposite directions. \gls{PRCL} and \gls{SRCL} involve moving the power and signal recycling mirrors, respectively.}
\end{table}

\subsection{Control signals}
The response of each degree of freedom of the \DRFPMI{} to each of the readout ports at a certain frequency can be represented as a \emph{sensing matrix}. This representation provides a clear view of the best readout ports to use for each degree of freedom, including the effect that other degrees of freedom would have on a particular degree of freedom's readout. We calculate the matrix by exciting each degree of freedom with driving coefficients as shown in Table\,\ref{tab:et-lf-driving-coefficients} and taking transfer functions from these degrees of freedom to each probe shown in Table\,\ref{tab:et-lf-probes}.

At this early stage, a reasonable control matrix can be determined through a heuristic approach building upon knowledge gained from the control of \ALIGO{}. As $f_2$ resonates in both recycling cavities, it samples the motion of the mirrors that influence \gls{MICH} and \gls{SRCL} as well as \gls{PRCL}. Conversely, $f_1$ only strongly samples the motion of \gls{PRCL}. Using demodulations at these frequencies, and combinations thereof, it should be possible to construct a reasonably decoupled sensing matrix for each degree of freedom.

% remove? move to appendix?
Care must be taken with the calculation of the response functions. As we are interested in the \emph{gradient} of the error signal, and not the sensor's response to \emph{test mass motion}, we must make sure that the calculation is performed with free body approximations for each test mass. The presence of suspension transfer functions, for instance, will produce incorrect results (see Appendix\,\ref{sec:sims-obtaining-tfs}).

\subsubsection{Combination of readout quadratures}
As the $I$ and $Q$ probes shown in Table\,\ref{tab:et-lf-probes} represent the same signals separated but demodulated with a phase difference of \SI{90}{\degree}, they can be combined electronically to produce an error signal with optimal magnitude. The exact phase corresponding to the greatest magnitude is not important, as this can be influenced by technical factors such as the length of \gls{RF} transmission lines, but the relative phase between maximum error signals from different degrees of freedom on the same sensor is important. If error signals from different degrees of freedom have the same maximum signal at the same or opposite phase, one cannot be minimised with respect to the other through appropriate choice of demodulation phase. On the other hand, if two degrees of freedom couple to a pair of $I$ and $Q$ sensors with equal magnitude but separate phase, they can be used to sense both degrees of freedom. In practice due to temperature drifts and other time-varying effects it is difficult to maintain the demodulation phase of a set of sensors to a precision better than around \SI{1}{\degree} \cite{Effler2014}, and so the readout quadratures chosen for each of the degrees of freedom of \ETLF{} should ideally be separated by many degrees of phase.

\subsubsection{Sensing matrix for ET-LF in detuned configuration}
The sensing matrix for the detuned configuration given the readout ports defined in Section\,\ref{sec:etlf-readout-ports} is shown in Table\,\ref{tab:et-lf-sensing-matrix-detuned} for \gls{DC}. The suggested readouts for each degree of freedom are highlighted in bold. The gravitational wave signal will primarily affect \gls{DARM}, and this is by design sensed by \ASDC{}, the \gls{DC} readout. The common mode loop can be sensed at \REFLFIRST{} for the purposes of frequency stabilisation and very low frequency seismic noise that imparts common motion on both arms.

The \gls{MICH}, \gls{PRCL} and \gls{SRCL} cavity error signals are difficult to separate due to the cross-coupling of control sidebands via the Schnupp asymmetry and the sideband asymmetry created by the presence of a detuned signal recycling cavity \cite{Hild2007}. As the control system will be implemented in a \LIGO{} \gls{CDS}-style system, it will be possible to create error signals formed from linear combinations of different sensor signals that decouple other degrees of freedom from a particular readout. In Table\,\ref{tab:et-lf-sensing-matrix-detuned}, \gls{PRCL} is dominant over \gls{MICH} and \gls{SRCL} at \POPFIRST{} and represents a good extraction point for the motion of the power recycling cavity. \gls{MICH} couples strongly to a number of ports but alongside strong signals from the other degrees of freedom. Its strongest ports, \ASDC{} and \ASSECOND{}, contain much larger signals from \gls{DARM} which would be difficult to remove electronically. A better readout could be to use \POPSECOND{} with the dominant \gls{PRCL} signal suppressed with careful tuning of the demodulation phase. Residual \gls{PRCL} coupling can be subtracted electronically.

\gls{SRCL} is difficult to sense with $f_1$ because it is not resonant in the signal recycling cavity, nor $f_2$ because its error signal is dominated by those of \gls{MICH} or \gls{PRCL} at all ports. A possible sensing strategy could utilise the beats between sidebands found at \REFLDIFF{} or \REFLSUM{}, where the contribution from \gls{MICH} and \gls{PRCL} can be suppressed with suitable choice of demodulation phase. All \gls{SRCL} signals contain an offset from zero due to the detuning, since the lower sideband is not resonant when the signal recycling mirror is at its operating point (see Figure\,\ref{fig:sideband-powers-srcl-detuned}). To use \REFLSUM{} for \gls{SRCL} control an offset of \SI{-4.2}{\milli\watt} will be necessary.

\begin{table}
  \centering
  \resizebox{16cm}{!}{%
    {\renewcommand{\arraystretch}{1.2} % for extra vertical spacing between rows
      \begin{tabular}{r|ccccc}
	& \textbf{CARM} & \textbf{DARM} & \textbf{MICH} & \textbf{PRCL} & \textbf{SRCL} \\ 
	\hline
	\hline
	\textbf{\ASDC{}} & \num{1.44e+06} (\SI{180.00}{\degree}) & \textbf{\num{4.99e+08} (\SI{180.00}{\degree})} & \num{1.17e+06} (\SI{180.00}{\degree}) & \num{1.21e+05} (\SI{180.00}{\degree}) & \num{1.39e+04} (\SI{180.00}{\degree}) \\ 
	\textbf{\ASFIRST{}} & \num{2.25e+06} (\SI{-32.56}{\degree}) & \num{3.00e+07} (\SI{55.83}{\degree}) & \num{6.25e+04} (\SI{56.09}{\degree}) & \num{7.54e+03} (\SI{77.99}{\degree}) & \num{1.09e+03} (\SI{58.46}{\degree}) \\ 
	\textbf{\ASSECOND{}} & \num{1.76e+08} (\SI{-112.30}{\degree}) & \num{3.98e+08} (\SI{-162.68}{\degree}) & \num{8.39e+05} (\SI{-163.43}{\degree}) & \num{6.03e+05} (\SI{-78.31}{\degree}) & \num{6.73e+04} (\SI{94.85}{\degree}) \\ 
	\textbf{\ASDIFF{}} & \num{4.71e+04} (\SI{178.86}{\degree}) & \num{3.52e+03} (\SI{107.72}{\degree}) & \num{3.43e+04} (\SI{85.09}{\degree}) & \num{6.44e+04} (\SI{61.41}{\degree}) & \num{3.41e+03} (\SI{-146.01}{\degree}) \\ 
	\textbf{\ASSUM{}} & \num{1.00e+05} (\SI{-177.46}{\degree}) & \num{4.43e+04} (\SI{-83.81}{\degree}) & \num{5.97e+04} (\SI{43.44}{\degree}) & \num{5.13e+04} (\SI{103.97}{\degree}) & \num{4.67e+03} (\SI{-22.78}{\degree}) \\
	\hline
	\textbf{\POPFIRST{}} & \num{7.61e+07} (\SI{-144.47}{\degree}) & \num{6.88e+05} (\SI{-144.49}{\degree}) & \num{8.70e+02} (\SI{-26.41}{\degree}) & \textbf{\num{2.55e+05} (\SI{-36.40}{\degree})} & \num{3.08e+01} (\SI{-81.49}{\degree}) \\ 
	\textbf{\POPSECOND{}} & \num{5.27e+07} (\SI{99.38}{\degree}) & \num{4.72e+05} (\SI{92.38}{\degree}) & \textbf{\num{2.95e+04} (\SI{50.72}{\degree})} & \num{1.26e+05} (\SI{-130.50}{\degree}) & \num{1.08e+04} (\SI{120.38}{\degree}) \\ 
	\textbf{\POPDIFF{}} & \num{2.01e+02} (\SI{-51.56}{\degree}) & \num{9.34e+01} (\SI{-127.43}{\degree}) & \num{1.24e+03} (\SI{1.97}{\degree}) & \num{1.61e+04} (\SI{88.80}{\degree}) & \num{1.07e+03} (\SI{167.69}{\degree}) \\ 
	\textbf{\POPSUM{}} & \num{4.94e+02} (\SI{-22.60}{\degree}) & \num{9.93e+01} (\SI{-15.51}{\degree}) & \num{1.07e+03} (\SI{45.23}{\degree}) & \num{8.34e+03} (\SI{-145.33}{\degree}) & \num{1.07e+03} (\SI{-149.81}{\degree}) \\
	\hline
	\textbf{\REFLFIRST{}} & \textbf{\num{1.44e+10} (\SI{-0.01}{\degree})} & \num{1.31e+08} (\SI{-0.01}{\degree}) & \num{2.34e+05} (\SI{4.53}{\degree}) & \num{5.01e+07} (\SI{-0.41}{\degree}) & \num{2.25e+04} (\SI{45.05}{\degree}) \\ 
	\textbf{\REFLSECOND{}} & \num{4.63e+09} (\SI{-75.19}{\degree}) & \num{5.69e+07} (\SI{-86.52}{\degree}) & \num{1.47e+05} (\SI{-83.24}{\degree}) & \num{1.58e+07} (\SI{-76.05}{\degree}) & \num{1.67e+04} (\SI{30.68}{\degree}) \\ 
	\textbf{\REFLDIFF{}} & \num{3.90e+06} (\SI{0.94}{\degree}) & \num{1.31e+06} (\SI{0.04}{\degree}) & \num{2.19e+05} (\SI{163.21}{\degree}) & \num{1.28e+06} (\SI{-171.17}{\degree}) & \num{2.03e+05} (\SI{-25.15}{\degree}) \\
	\textbf{\REFLSUM{}} & \num{4.04e+06} (\SI{-101.97}{\degree}) & \num{1.32e+06} (\SI{-101.79}{\degree}) & \num{2.25e+05} (\SI{-76.80}{\degree}) & \num{1.40e+06} (\SI{-53.54}{\degree}) & \textbf{\num{2.03e+05} (\SI{89.52}{\degree})}
      \end{tabular}
    }
  }
  \caption[Gradients of the error signals from each degree of freedom to each probe in \ETLF{} at dc]{\label{tab:et-lf-sensing-matrix-detuned}Gradients of the error signals from each degree of freedom to each probe, in units of \SI{}{\watt\per\meter}, in \ETLF{} at \gls{DC}. The suggested readout probes for each degree of freedom are shown in bold. The $I$ and $Q$ quadratures of each readout have been combined into a single magnitude and the phase representing the greatest slope and the phase at which it is achieved. The \ASDC{} does not get demodulated but it has phase of \SI{0}{\degree} or \SI{180}{\degree} depending on the sign of the error signal. The \gls{RF} probes contain maximum slopes at phase angles determined by the propagation of the control sidebands through the interferometer. Probes can be optimised to sense the motion of a particular degree of freedom by adjusting the angle at which the sensor $I$ and $Q$ quadratures are combined, but signals on sensors that contain strong signals from other degrees of freedom at nearby phase angles are difficult to use.}
\end{table}

Error signals corresponding to the suggested readouts for each degree of freedom are shown in Figure\,\ref{fig:sweeps-et-lf}. These are produced by calculating the power on each sensor at the relevant demodulation frequency as the mirrors are driven as shown in Table\,\ref{tab:et-lf-driving-coefficients} and represent the low-frequency limit of the transfer functions of each degree of freedom to each sensor.

\begin{figure}
  \centering
  \includegraphics[width=\columnwidth]{graphics/generated/from-python/70-sweeps-detuned.pdf}
  \caption[Sweeps through the zero-crossings of the chosen error signals in ET-LF]{\label{fig:sweeps-et-lf}Sweeps...}
\end{figure}

Figure\,B shows the \ASDC{} readout, and the slope is zero when the arm cavities are tuned, consistent with Figure\,\ref{fig:total-power-vs-darm-offset-detuned}, showing that some classical carrier light power is required for \gls{DC} readout. Although a higher slope can be found at higher offsets, this results in reduced sensitivity due to a loss in arm power. Figure\,E shows the error signal crossing the operating point at a power of \SI{4.2}{\milli\watt}, showing the requirement for an offset.

The parameters used in the proposed control scheme are shown in Table\,\note{XXX} alongside the pre-existing parameters from the design study. This scheme is not optimal, and does not consider a number of other possibilities such as the use of secondary reflections arising from the anti-reflective coatings on the beam splitter and \glspl{ITM} or the use of higher order combinations of $f_1$ and $f_2$ such as \emph{3f} signals used in the lock acquisition sequence of \VIRGO{} \cite{Acernese2008} and \ALIGO{} \cite{Staley2014}. It serves, however, as a first concept for the control of \ETLF{} with \SI{25}{\hertz} signal recycling cavity detuning. The next section will discuss some noise implications of \ETLF{}'s low frequency control strategy.

\note{Put updated parameters in a table here}

\section{Control noise}
\note{Highlight the dynamic range problem with photodetectors sensing the low frequency motion (poster from Florence), and introduce the basic control loop developed with SimulinkNb - discuss the noises included, the assumptions made, etc. Finish with the list from the poster of what has to be done: further modelling of suspension SPIs, or better sensors, or both, etc...}

\subsection{Seismic noise in ET-LF}
\note{Take transfer function through S.A. from lowest noise site measured...}
\note{Assume optimal worst contribution of noise to ETMs...}

\section{Outlook and future work}
The work presented in this chapter shows that \ETLF{} can in principle be controlled with an \ALIGO{} style scheme, though with increased cross-coupling of degrees of freedom particularly for \gls{MICH} and \gls{SRCL}. Some sensing noise issues remain unaddressed, as have more exotic control schemes such as the use of additional modulation frequencies or \emph{sub-carriers}.

\subsection{Assessing the controllability of the sensing matrix}
In \ALIGO{} the conceptual control scheme was first tested at the \CALTECHFORTYM{} and it is probable that any technical design for \ETLF{} will require a similar test. At this stage a quantative assessment of the performance that a particular control scheme might provide might take the form of a technique presented by Mantovani and Freise \cite{Mantovani2008} developed for alignment control in \VIRGO{}, but suitable for longitudinal control. This involves the calculation of a \emph{quality parameter} representing the controllability of a given set of sensors and degrees of freedom. This approach only makes sense when $M$ represents the interferometer at its operating point, which means that the residual motion of controlled degrees of freedom does not create a significant cross-coupling. This approach requires hierarchical gain to be applied in a lock acquisition sequence, and will be the subject of future work.

\subsection{Twin signal recycling and control with sub-carriers}

\subsection{Switching between tuned to detuned operation}
Transition from tuned to detuned signal recycling operating points and vice versa involves a technique which can maintain control of the interferometer as it transitions between two desired set points. When dual recycling was first demonstrated in suspended optics in the Garching prototype, it involved a varying frequency offset applied to the \gls{RF} modulation sidebands as the tuning was changed \cite{Freise2000}. In \GEO{} the control technique involved a complicated sequence of events \cite{Grote2004} including an uncontrolled ``jump'' between two operating points \cite{Hild2007} to reached tuned mode. In \ETLF{} it is expected that the signal recycling cavity finesse will be too high to allow for a previously demonstrated transition scheme. Instead, investigations are underway to model the impact that single control sidebands added to the input light or subcarriers added to the squeezing injection port have on the controllability of the signal recycling cavity at arbitrary detunings. Another possibility is to adapt the \emph{arm length stabilisation} system developed for the lock acquisition of \ALIGO{} \cite{Mullavey2012, Staley2014}, whereby a second carrier at a different wavelength is used to lock cavities. This takes advantage of the lower finesse of the second carrier's wavelength in the cavities, allowing for a wider locking range. The cavities are first pre-stabilised using this second carrier before the main carrier is brought to resonance.

The transmissivity of the signal recycling mirror in \ETLF{} allows for a reasonably wide capture range as the signal recycling mirror swings through the resonant condition and so tuned signal recycling is straightforward to control given the prior experience from the advanced detectors... however...

Future work will address the problem of transitioning \ETLF{} from tuned to detuned operation.

\subsection{Sensing and control of seismic and Newtonian noise}
\note{Discuss dynamic range issues...}

\subsection{Testing of control concepts}
A possible test for \GEOHF{} to conduct in the future is to detune its signal recycling cavity to \SI{25}{\hertz} to test the ability to control the interferometer at such an extreme.

\subsection{Future upgrades to ET}
\note{SSM, Stefan D's idea for triangular speed meter, etc...}