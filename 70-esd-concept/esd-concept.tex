\chapter{Demonstration of a Plate Capacitor Electrostatic Drive Actuator}
\label{c:esd-concept}

* Introduce Holger's paper~\cite{Wittel2015} \etal, discuss need for high voltages
* HV power supply design considerations (bleed resistor, voltage requirements, etc.)
* HV power supply current limiting circuit explanation (called ``foldback limiting'', see Horowitz and Hill p694)
* HV power supply heatsink considerations: power produced by MOSFETs at maximum current rating (100 mA, produces 25 W of heat - see datasheets) - chose heatsinks to dissipate this heat and make installation of the board easy (can screw TO-220 casing to L-shaped brackets)
* HV amplifier design: pressure and temperature cut-offs (explanation of how it works), input/output signals, choice of connectors, routing of signals for ease of assembly (front panel disconnect, etc.)
* Protective earthing
* HV amplifier transfer function, usign both HV output (via voltage divider) and monitor output
* HV amplifier monitor noise measurement
* This experiment will inform the main SSM experiment.

\section{High voltage power supply}

\note{Describe indicator LEDs: their brightness is proportional to current, so while they are still lit the experiment is live.}

% ESD force gradient, from esd-ansys.pdf figure
\newcommand{\ESDFORCEGRAD}{\SI{-3.68}{\nano\newton\per\volt}}

% ESD maximum voltage
\newcommand{\ESDMAXVOLTAGE}{\SI{750}{\volt}}

% ESD maximum force, read from esd-ansys.pdf figure (gradient * voltage doesn't work because there's a non-zero y-intercept)
\newcommand{\ESDMAXFORCE}{\SI{-1.48e-6}{\micro\newton}}

There are a number of requirements for the plate capacitor's power supply. Due to the nature of the load--a capacitor formed from parallel plates--the power supply does not need to drive a significant current. On the other hand, the power supply will be modulated by the amplifier and sent to the plates, and the plates will actuate directly upon the test masses, and so any noise present upon the power supply output can potentially limit the sensitivity of the experiment. In order to achieve significant actuation, it is also necessary to provide a high voltage supply to the amplifier. Simulations conducted in the development of this experiment by group colleagues with the finite element modelling package \emph{ANSYS} have found that, for a mirror geometry resembling that of the \SSMEXPT's ETMs, the force gradient will be \ESDFORCEGRAD, as shown in Figure\,\ref{fig:esd-ansys}. For a sufficient level of force actuation within the voltage isolation limit of our vacuum tank feedthroughs, a maximum feasible voltage across the capacitor appears to be in the region of \ESDMAXVOLTAGE, leading to a maximum mirror actuation force of \ESDMAXFORCE.
% ESD volts -> force from https://arran.physics.gla.ac.uk/wp/speedmeter/?p=4507

As this experiment is somewhat a technology demonstration for the main \SSMEXPT, it is worth keeping in mind its goals. For this reason, the power supply should provide suitably low output noise and sufficient channels for the purposes of the control of the full experiment.

\begin{figure}
  \begin{center}
    \includegraphics[width=\columnwidth]{70-esd-concept/graphics/dynamic/esd-ansys.pdf}
    \caption{Simulations of the actuation force produced by the proposed ESD design upon a \SI{100}{\gram} cylindrical test mass of diameter \SI{48.6}{\milli\meter} and depth \SI{24.5}{\milli\meter} resembling that of the \SSM experiment's ETMs. The plate separation and the position of the mirror with respect to the plates influence the level of force produced. \checkme{In practice it is most beneficial to have the mirror centre of mass aligned to the edge of the plates and the plates as close as possible to the mirror without touching.}}
    \label{fig:esd-ansys}
  \end{center}
\end{figure}

\section{High voltage amplifier}

A means of controlling the \AC{} component of the high voltage supply is necessary to perform frequency-dependent corrections upon the mirror. Although the ESD, and indeed the \SSM{} experiment, primarily require corrections at lower frequencies where seismic noise is dominant, it is beneficial to utilise an amplifier which can provide actuation up to many tens, if not hundreds, of \SI{}{\kilo\hertz}. This is to provide a transfer function which is flat across the vast majority of each experiment's measurement band, avoiding the roll-off at high frequencies due to the integrated circuits utilised within the high voltage amplifier. A flat transfer function makes the calibration of the actuator plant as simple as possible. Another benefit of having a high bandwidth amplifier is the possibility to use it for common mode control loops, where laser frequency stabilisation can be split between feedback to the laser's piezoelectric transducer and actuators on the test masses.

The key component of a high bandwidth amplifier is the power op-amp. This class of op-amps typically utilises a \MOSFET{} design, and can provide high voltage output given a low voltage input. As an example, the Apex PA95 op-amps used in Advanced LIGO's \ESDs{} provide up to \SI{900}{\volt} output up to a frequency of around \SI{15}{\kilo\hertz}, and up to \SI{50}{\volt} at \SI{250}{\kilo\hertz}. The PA98 op-amp, also from Apex, provides an output of \SI{450}{\volt} nominally up to \SI{60}{\kilo\hertz}, potentially up to \SI{500}{\kilo\hertz} for a low capacitive load.

% claim about voltage output of PA95 comes from figure ``Power Response'' on p3 of PA95 datasheet (https://www.apexanalog.com/resources/products/pa95u.pdf)
% claim about voltage output of PA98 comes from figure 8: ``power Response'' on p7 of PA98 datasheet (https://www.apexanalog.com/resources/products/pa98u.pdf)

For the \ESD{} experiment an amplifier circuit utilising the PA98 was built based on a design used for the \AEIPROTOTYPE{}. Two notable modifications have been made, both of which are safety enhancements.

The first modification is the use of alternative high voltage connectors. The \AEIPROTOTYPE{} design called for two SHV connectors carrying the separate +HV and -HV rails to the vacuum feedthrough, where the vacuum system acts as the virtual ground between the two rails. While both cables are connected, this system is safe; however, if one cable is disconnected, a fault with the amplifier circuit can cause the vacuum system to become live. To avoid the possibility of this situation ocurring in the \SSM{}, a Bulgin-type connector was used to combine the +HV and -HV rails in a single connector and cable. With this style of connector it is not possible to disconnect from the vacuum system any single HV rail while leaving the other connected.

The second modification made to the amplifier circuit was the addition of a safety interlock mechanism. The breakdown voltage of the plate capacitors as a function of pressure, given by Paschen's Law, has a minimum in the region of \SI{e3}{} to \SI{e7}{\milli\bar} depending on the separation of the anode and cathode. In addition, related effects such as surface tracking can lead to arcing at voltages above \SI{50}{\volt} in low vacuum. Although the use of high voltage plate capacitors is in general safe at both atmospheric pressure and high vacuum (below \SI{e-6}{\milli\bar}), the act of pumping gas out of the vacuum system necessarily passes through pressures at which arcing can occur.

To prevent the possibility of arcing, a cut-off mechanism was added to the design to prevent high voltage output unless a voltage above \SI{5}{\volt} was applied across terminals on the enclosure. This threshold was chosen based on the monitor output voltage provided by the pressure gauge attached to the vacuum system. This varies logarithmically from \SI{0}{} to \SI{10}{\volt} for pressures between standard atmosphere and ultra-high vacuum. A voltage above \SI{5}{\volt} indicates a pressure below \SI{e-5}{\milli\bar}.

% pressure vs voltage claim from speedmeter labbook, at https://arran.physics.gla.ac.uk/wp/speedmeter/2015/01/30/esd-hv-amplifier-pressure-cutoff/

\begin{figure}
  \begin{center}
    \includegraphics[width=\columnwidth]{70-esd-concept/graphics/dynamic/esd-paschen.pdf}
    \caption{The minimum breakdown voltage between the two plates of the \ESD{} for different separations. This is calculated using Paschen's Law, assuming nitrogen gas and a flat plate geometry. The effect is a lot more complicated for real systems, where different plate geometries will have different relationships, but the steep slope at lower pressures shown here indicates that the apparatus used in the \ESD{} experiment will very likely avoid any problems with arcing as long as a suitable pressure-based interlock is utilised in the design of the electronics.}
    \label{fig:esd-paschen}
  \end{center}
\end{figure}

\subsection{Transfer functions}
\note{Describe measurement: use of voltage divider, need to take into account impedance of the send box in its design, bandwidth of each measurement: CDS up to 10 kHz, SR785 up to 100 kHz, Agilent up to MHz. Describe monitor and HV rail outputs, and how we avoid ground loops (isolated case from double LEMO and monitor outputs.}

\section{Outlook}