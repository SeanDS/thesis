\chapter{\label{c:instrumentation}Sensitivity and noise in gravitational wave interferometers}
\chaptermark{Sensitivity and noise in interferometers}

To achieve maximum sensitivity in an interferometric gravitational wave detector to a particular type of signal the parameters of the optics, arm lengths and light fields must be considered alongside the characteristics of the signals and noise and the controllability and robustness of the resultant design. This chapter describes some of the considerations to be made in the design of detectors to provide the basis on which the rest of this thesis will build. \Cref{sec:ifo-foundations} details the state in which an interferometer must be brought in order to be sensitive to gravitational waves and the means of keeping it there; \cref{sec:ifo-noise} introduces the limiting noise sources in ground-based gravitational wave detectors and the physical processes at play; \cref{sec:ifo-response} discusses ways to improve the sensitivity of interferometers in the frequency bands of interest; and \cref{sec:sub-sql-techniques} introduces concepts in order to reduce the most challenging noise source arising from the quantum nature of light.

\section{\label{sec:ifo-foundations}Interferometer foundations}
The effect that the output light from an interferometer has on a sensor (e.g. a photodetector) as some variable is modulated is termed its \emph{response}. As discussed in \cref{c:gw-detection} the most important response to consider in gravitational wave interferometry from an astrophysical perspective is that of the differential motion of the arms (\gls{DARM}) to the sensor at the output port. The response has a dependence on the input light power but it varies as a function of frequency due to the presence of additional \emph{cavities} used to enhance or suppress the response in a given frequency band.

\subsection{Measurement of interferometer length fluctuations}
The complex-valued electric field amplitude of an electromagnetic wave propagating in time and space, $E$, can be expressed as
\begin{equation}
  \label{eq:em-propagation}
  E = E_0 \text{e}^{\text{i} \left( \omega t - kx \right)},
\end{equation}
where $i$ is the imaginary unit, $\omega$ is the wave's angular frequency, $t$ is time, $k = \frac{2 \pi}{\lambda}$ is the wave number and $x$ is the coordinate in the direction in which $E$ is measured. An arbitrary phase offset defined with respect to some point is contained within the complex-valued maximum field amplitude, $E_0$.

Typically the underlying amplitude of a particular interferometer signal can only be inferred from the light power measured by a sensor. A simple example is the measurement of mirror displacement in a \MI{} via the photocurrent output of a photodetector. The measured power $P$ in this case would be
\begin{equation}
  P = E^{\ast} E,
\end{equation}
where $\ast$ denotes the complex conjugate.

\Cref{eq:em-propagation} can be simplified to a sinusoidal function with real maximum field amplitude $E'_0$ and phase offset $\phi$:
\begin{equation}
  \label{eq:em-propagation-real}
  E' = E'_0 \cos \left( \omega t - kx + \phi \right),
\end{equation}
and in this way we can express the measured power as the square of the real field amplitude, i.e. $P = E'^2$.

Assuming that laser light with amplitude described by \cref{eq:em-propagation-real} is incident upon the beam splitter shown in the \MI{} in \cref{fig:mi}, the light returning to the beam splitter having reflected from the north and east arms, $n$ and $e$, respectively, would be
\begin{align}
  E'_n &= -\frac{E'_0}{\sqrt{2}} \cos \left( \omega t - 2kL_n \right) \\
  E'_e &= \frac{E'_0}{\sqrt{2}} \cos \left( \omega t - 2kL_e \right),
\end{align}
where $L_n$ and $L_e$ are the two arm lengths. We employ a particular reflection phase convention such that a negative coefficient is gained on the light reflected from one side of the beam splitter, to conserve energy (see \cref{a:reflection-phase}). We can also express $L_n$ and $L_e$ in terms of the average arm length $L = \frac{L_n + L_e}{2}$ and differential length $\delta L = L_n - L_e$:
\begin{align}
  E'_n &= -\frac{E'_0}{\sqrt{2}} \cos \left( \omega t - 2k \left( L + \frac{\delta L}{2} \right) \right) \\
  E'_e &= \frac{E'_0}{\sqrt{2}} \cos \left( \omega t - 2k \left( L - \frac{\delta L}{2} \right) \right).
\end{align}

The superpositions of the light from the arms leaving the beam splitter towards the input laser, $E'_{\text{in}}$, and the light leaving at the output port, $E'_{\text{out}}$, are then
\begin{equation}
  \begin{split}
    E'_{\text{in}} &= \frac{E'_{e} - E'_{n}}{\sqrt{2}} \\
                      &= E'_0 \cos \left( \omega t - 2kL \right) \cos \left( k \delta L \right)
  \end{split}
\end{equation}
\begin{equation}
  \begin{split}
    E'_{\text{out}} &= \frac{E'_{e} + E'_{n}}{\sqrt{2}} \\
                       &= -E'_0 \sin \left( \omega t - 2kL \right) \sin \left( k \delta L \right).
  \end{split}
\end{equation}
A real photodetector is not quick enough to measure changes in intensity at the frequency of the light. Instead, it sees the time averaged square of the field. The photodetector power at the output as a function of $\delta L$, $P_{\text{out}}$, is
\begin{equation}
  \label{eq:mich-p-out}
  P_{\text{out}} \left( \delta L \right) = \frac{P_0}{2} \left( 1 - \cos \left( 2k \delta L \right) \right),
\end{equation}
where $P_0$ is the power of the incident laser light, showing that the signal from the differential arm length is encoded in the power of the light present at the output port. Note that implicit in this derivation is the assumption that the arms are both perfectly reflective. When the optics within the interferometer have different reflectivity, the calculation becomes more complicated and it is sometimes more practical to use a simulation tool, as discussed in \cref{a:simulation-tools}.

\subsection{\label{sec:operating-point}Optimal operating point}
The phase change created by the difference in the lengths of the arms shown in \cref{eq:mich-p-out} as $k \delta L$ can be expressed as a combination of a \emph{static} tuning $\alpha$ and the phase change created by incident gravitational waves, $\delta \phi_{\text{GW}}$, i.e.
\begin{equation}
  k \delta L = \alpha + \delta \phi_{\text{GW}}.
\end{equation}

The static tuning $\alpha$ is the differential arm phase at which the interferometer is nominally held. In experiments where sensitivity can be sacrificed for simplicity, often it is practical to keep the interferometer in the state commonly referred to as ``half way up the fringe''. Here, the interferometer's arms are nominally tuned \SI{45}{\degree} out of phase such that the output signal oscillates about the midpoint between crest and trough of the superposition waveform at the output. As the gradient is steepest at this point, any small changes to the relative arm length of the \MI{} result in a significant difference in power at the photodetector. This operating point, however, is not optimal in terms of \emph{sensitivity} to arm length fluctuations. The static light power at this operating point contributes significant \emph{shot noise} at the output. Its power spectral density is defined as~\cite{Meers1991}
\begin{equation}
  S_{\text{shot}} = 2 h f_0 P_{\text{out}},
\end{equation}
where $h$ is Planck's constant and $f_0$ is the light frequency. The optimally sensitive operating point is therefore not simply one which maximises the signal gradient, but rather one which maximises the \emph{signal-to-noise ratio} (\gls{SNR}). It turns out that the reduced signal in the case of the operating point close to the \emph{dark fringe}, where light from the two arms interferes destructively, is more than compensated for by the lack of shot noise such that the overall sensitivity is better. Interferometer operation near the dark fringe is the basis of \emph{\gls{DC} readout}, described in \cref{sec:homodyne-readout}, which is the standard measurement technique for all current generation detectors.

\subsection{\label{sec:readout}Readout}
% chapter 4 points here
Note that the light power at the output shown in \cref{eq:mich-p-out} as a sinusoidal function of the change in arm length is symmetrical and so displacements $\pm \delta L$ yield identical changes in light power. As the interferometer must be held at the dark fringe in order to maximise sensitivity, the controller requires a \emph{bipolar} error signal providing a different response for motion in different directions. The purpose of the \emph{readout} technique is to facilitate a bipolar error signal. Another benefit certain types of readout can provide is access to the displacement information in a particular \emph{quadrature} of the output light. Signal (and noise) can in general be encoded in both the amplitude and phase of the light, representing the light's quadratures. When the ratio between optical power and mirror mass is high this information is primarily contained within the phase quadrature, and when significant optomechanical interactions are present either with lighter mirrors or higher laser power the information can be encoded as a linear combination of the phase and amplitude quadratures. Some readout techniques facilitate arbitrary readout quadratures where the signal can be maximised with respect to the noise, while others have a quadrature fixed by the interferometer parameters.

There are two common types of readout technique. \emph{Heterodyne} readout involves the use of a second light frequency used as a \emph{local oscillator} for the primary light frequency that is resonant within the interferometer. This was one of the first techniques used to control laser interferometric gravitational wave detectors~\cite{Willke2002}, but due to the presence of \emph{cyclostationary} noise~\cite{Niebauer1991} and challenges related to the creation of a stable local oscillator frequency this has largely been superseded by \emph{homodyne} techniques as the \gls{DARM} sensor. Homodyne readout involves the use of the carrier as both a signal field and local oscillator, and this leads to some cancellation of noise sources common to the carrier at the expense of additional technical complexity. These techniques are discussed in greater detail in the following subsections.

\subsubsection{Heterodyne readout}
When light with multiple frequency components is incident upon a photodetector the resulting electrical signal shows the \emph{beat signal} between the two components. Assuming that a photodetector has an incident electric field amplitude composed of two frequency components, we get~\cite{Freise2010}
\begin{equation}
  E' = E'_0 \cos \left( \omega_1 t \right) + E'_0 \cos \left( \omega_2 t \right),
\end{equation}
where $\omega_1$ and $\omega_2$ are the two frequencies and $t$ is time. The photodetector measures the power of the field, $P$:
\begin{equation}
  \label{eq:photodetector-beat-power}
  P = {E'}^2 = {E'}_0^2 \left( \cos^2 \left( \omega_1 t \right) + \cos^2 \left( \omega_2 t \right) + \cos \left( \left( \omega_1 + \omega_2 \right) t \right) + \cos \left( \left( \omega_1 - \omega_2 \right) t \right) \right).
\end{equation}
If the difference frequency $\omega_1 - \omega_2$ in \cref{eq:photodetector-beat-power} is small, it can be measured by the photodetector. Most heterodyne techniques involve the phase modulation of a single carrier which creates a series of sidebands offset in frequency (see \cref{fig:sideband-structure}) from the carrier that beat together at the photodetector. Different resonant conditions for the sidebands with respect to the carrier allow some to act as phase discriminants for others, and with suitable \emph{demodulation} at the photodetector these can be used to sense displacement in the interferometer arms.

\subsubsection{\label{sec:homodyne-readout}Homodyne readout}
One way in which to picture homodyne readout is as a heterodyne readout with $\omega_1 = \omega_2$. It is possible to create a homodyne local oscillator by using a second laser with identical frequency to the first, though it is usually beneficial to use the same laser to benefit from coherent noise cancellation.

The first large scale application of homodyne readout in gravitational wave detectors was \emph{\gls{DC} readout}~\cite{Fricke2012}, where a detuning (\emph{\gls{DC} offset}) is intentionally created within the interferometer's arms to allow for some of the carrier light to appear at the output port where it acts as a local oscillator to the rest of the carrier that contains the gravitational wave signal. This technique has the benefit that the local oscillator is filtered by the interferometer which suppresses certain types of noise, but it involves the intentional introduction of a classical light field at the output port. Another homodyne technique, \emph{balanced homodyne detection}, involves the picking off of a fraction of the interferometer's input for use as a local oscillator. In this case, signal encoded in the light leaving the interferometer can be mixed with the local oscillator without the need for a \gls{DC} offset. The \gls{DC} readout technique is used in current generation detectors but for future interferometers it is possible that balanced homodyne detection will become the norm~\cite{Gard2016}.

The technical implementation of \gls{DC} readout is discussed in more detail in \cref{c:et-lf-control}, and balanced homodyne readout forms the basis for the experiment introduced in \cref{c:speedmeter-intro}.

\subsection{Control}
In order for the interferometer to be kept at its operating point the readout signal representing the positions of the mirrors (\emph{test masses}) must be fed back to actuators. These actuators typically take the form of voice coils, piezoelectric stacks and other mechanical transducers. The ubiquitous technique for the control of the positions of mirrors is \emph{linear negative feedback}, where the readout signal is passed through a \emph{servo} which applies frequency-dependent filtering to enhance or suppress particular components and invert the signal before it is sent to the actuators. If the control system is designed to react quickly to test mass motion, the interferometer can be held almost exactly at the operating point where the error signal from the readout is nulled. \Cref{eq:mich-p-out} shows that small arm displacements lead to linear changes in the output power, as is the case for other readout techniques. Effective control of the interferometer holds it within this linear region to ensure that the readout is maximally sensitive to the displacement of the arms.

Control strategies are discussed in greater detail in \cref{c:waveguides,c:speedmeter-control,c:et-lf-control}. \Cref{a:control} also introduces some background concepts useful for the understanding of the control strategies presented in these chapters.

\section{\label{sec:ifo-noise}Measurement noise in interferometers}
The ``signal'' in an interferometer is the collection of electrical oscillations representing the particular variable of interest which, in most cases, represents the motion of the test masses in the arms. ``Noise'', on the other hand, refers to the unwanted oscillations that appear in the measurement independent of such a variable. The sensitivity of an interferometer is represented by the magnitude of the signal with respect to the noise, the \gls{SNR}, introduced in the context of the operating point in \cref{sec:operating-point}.

Gravitational wave interferometers are limited by a plethora of noise sources across the spectrum. The knowledge of the limiting noise sources gained from the science runs undertaken by the initial generation of interferometric detectors (\LIGO{}, \VIRGO{}, \GEO{} and \TAMA{}) has fed in to the design of the current second generation.

The creation of \emph{noise budgets} from theoretical descriptions and measurements of sources is a useful way to examine how noise influences the sensitivity of an experiment. The noise budget for \ALIGO{}'s design configuration is shown in \cref{fig:aligo-noise-budget}. At its most sensitive frequencies, \ALIGO{} is limited by \emph{quantum} and \emph{thermal} noise, while at lower frequencies the motion of the ground from seismic sources sets the limit. Careful design involving specially selected materials and techniques has reduced thermal noise arising from the mirror coatings and suspensions and technical noise associated with electronics and facilities. Quantum noise sets the fundamental limit given the available light power and mirror masses utilised in the detectors. In order to improve the sensitivity beyond the limit set by quantum noise, approaches that involve changing the nature of the quantum interactions within the interferometer have to be implemented. Important noise sources useful for the rest of this thesis are discussed throughout this section.

\begin{figure}
  \centering
  \input{graphics/generated/from-python/20-aligo-noise-budget.pgf}
  \caption[Advanced LIGO noise budget]{\label{fig:aligo-noise-budget}\ALIGO{} noise budget calculated with \gls{GWINC}~\cite{gwinc}. Greater sensitivity to gravitational waves is achieved by having lower residual strain noise. The incoherent sum of the noise sources leads to the overall sensitivity of the interferometer, and this is shown in black. All of the noise sources shown have some frequency dependence, and optimal sensitivity in a detector is reached by designing the experiment in such a way as to minimise the noise sources in the frequency band of interest. The creation of budgets like this from theoretical descriptions of noise sources is a useful way in which to understand how they affect the sensitivity.}
\end{figure}

\subsection{\label{sec:noise-via-loss}Noise arising from loss and uncertainty}
Quantum theory showed that the universe contains a continuous spectrum of quantum fluctuations at all frequencies. Virtual photons are constantly created and annihilated in all space, albeit with an average energy of zero, producing the measurement uncertainty predicted by quantum mechanics. Collections of mirrors within interferometers create local filters of this quantum spectrum which allow a subset of vacuum modes to circulate. Virtual photons are able to enter the interferometer via its \emph{loss points}, where light can escape the interferometer, just as virtual photons created within the interferometer are allowed to leave. As the vacuum fluctuations are uncorrelated with the motion of the test masses, non-unity reflectivity of optics, scattering and other photon loss effects within an interferometer lead to the intrusion of vacuum noise.

In lasers, a pumped electromagnetic field creates a state which can be used as the input light for an interferometer. This typically involves pumping the field into the \emph{coherent} state in which the average laser amplitude and phase quadratures are matched, and noise arises from the presence of virtual photons with arbitrary amplitude and phase in the pumped field leading to an uncertainty in the number of photons output from the laser.

Noise in interferometers does not arise solely from the quantum vacuum, however. In general, noise was shown by Callen and Welton to arise from loss processes quantified by their \emph{fluctuation dissipation theorem} developed in the early \nth{20} century~\cite{Callen1951}, which showed that the noise power spectral density created due to fluctuations,
\begin{equation}
  S_{\text{fluc}} \propto \frac{T}{Q},
\end{equation}
is related not only to the energy quantified by the temperature $T$, but also to the quality factor $Q$ related to the lossiness of the material.

The effect of noise on the interferometer can be calculated by quantifying the magnitude of noise entering at a loss point and propagating this noise to the signal detection point where it can potentially mask the signal. The noise at a photodetector is then the sum of noise propagated from each point of loss to the readout point. The way in which some forms of noise can enter a \MI{} and propagate to the output port is shown in \cref{fig:modelling-noise}.

\begin{figure}
  \centering
  \includegraphics[width=\columnwidth]{graphics/generated/from-svg/20-modelling-noise.pdf}
  \caption[Some entry points for noise in a \MI{}]{\label{fig:modelling-noise}Some entry points for noise in a \MI{}.}
\end{figure}

\subsection{Thermal noise}
Thermal noise arises from loss in materials used to reflect and focus light and to suspend test masses, where photons given a phase change due to thermal excitations are able to propagate to the sensors.

Thermal noise is quantified by a material's \emph{loss angle}, which is the imaginary part of the Young's modulus relating applied stress to the corresponding strain of the material. Material with a high loss angle results in an applied stress creating an associated strain at a different time, and during this time the incident light can accumulate noise via thermal fluctuations of the material. The most significant thermal noise contributions in current generation gravitational wave detectors arise from the test mass optical coatings and suspensions.

\subsubsection{\label{sec:coating-thermal-noise}Coating thermal noise}
In the conceptual design for the first generation of gravitational wave detectors such as GEO-600~\cite{Willke2002} the designers were not aware that thermal noise associated with the reflective coatings on mirrors would play a significant role in the sensitivity of the interferometers. For a long time it was known that thermal noise would contribute to the sensitivity of the detectors, particularly from the bulk material forming the test masses, but it soon became clear as the detectors were being commissioned that thermal noise arising from the reflective mirror coatings would dominate the thermal noise associated with the test masses in the frequency band of interest despite forming only a tiny fraction of the test masses by volume. Investigations conducted by Harry \etal{}~\cite{Harry2002, Harry2007}, among others, concluded that mechanical loss present in the numerous dielectric coating stacks on the test masses required for high reflectivity led to Brownian noise creating a limit to the sensitivity of detectors across a wide range of frequencies. Contributions from thermoelastic noise, arising from the thermal expansion coefficient of the materials of the coatings~\cite{Braginsky1999a}, and thermorefractive noise, arising from the change in refractive index caused by fluctuations in the material's temperature~\cite{Braginsky2000a}, produce further noise which will become more important as coatings with improved Brownian noise are developed.

Over the past two decades, efforts have been made to both quantify and reduce coating thermal noise. Particular interest is being paid to the study of coatings for cryogenically cooled mirrors, such as the sapphire (\ce{Al_2O_3}) test masses to be used in \gls{KAGRA}~\cite{Somiya2012}. A loss peak in the mirror material silica (\ce{SiO_2}), for detectors until recently ubiquitous, occurs at low temperature. This makes the material unsuitable for cryogenic use, as mechanical loss will couple into the light within the interferometer and make its way to the detection port. Other materials such as sapphire do not feature this loss peak and provide lower thermal noise than silica at room temperature for a given mirror design. Coating noise is also proportional to temperature, so cryogenically cooled mirrors can offer better performance. Additionally, crystalline coatings made from compounds such as \ce{AlGaAs} can offer future detectors a coating thermal noise reduction of up to \num{3} over the current state of the art~\cite{Cole2013} if technical challenges in their manufacturing can be overcome.

The dominant contribution to coating noise in current generation gravitational wave detectors, Brownian noise, has a power spectral density given by~\cite{Harry2002}
\begin{equation}
  \label{eq:coating-brownian-psd}
  S_{\text{coating}} = \frac{2 k_B T}{\pi^{3/2} f} \frac{1}{w Y} \left( \phi_{\text{sub}} + \frac{1}{\sqrt{\pi}} \frac{d}{w} \left( \frac{Y'}{Y} \phi_{\text{para}} + \frac{Y}{Y'} \phi_{\text{perp}} \right) \right),
\end{equation}
for Boltzmann constant $k_B$, temperature $T$, frequency $f$, beam size $w$, Young's modulus $Y$, loss angle $\phi$ and coating thickness $d$. The Young's moduli are split into components representing the coatings and substrate, $Y$ and $Y'$, and the loss angles are split into parallel and perpendicular components in the coatings, $\phi_{\text{para}}$ and $\phi_{\text{perp}}$, and substrate, $\phi_{\text{sub}}$, respectively. The measurement and interaction between these components is an active area of research. \Cref{fig:aligo-noise-budget} shows coating Brownian noise jointly dominating the noise in \ALIGO{} at frequencies around \SI{70}{\hertz}.

A mirror topology which avoids the use of many alternating coating layers can potentially offer an improvement in noise performance. Mirrors employing grating structures can resonantly reflect light with less coating material than similarly performing dielectric mirrors~\cite{Mashev1985}, though at the expense of additional technical complexity in their utility in gravitational wave detectors~\cite{Leavey2015}. \Cref{c:waveguides} discusses a form of grating mirror for use in interferometers.

\subsubsection{\label{sec:sus-thermal-noise}Suspension thermal noise}
The test masses in audio-band gravitational wave detectors must be suspended from pendulum systems to filter ground vibrations, and current generation observatories (with the notable exception of \KAGRA{}) utilise fused silica fibres, a technique pioneered for \GEO{}~\cite{Barr2002}. The reason for the use of this material is that the thermal noise present within the previously used steel loops was high enough to impart significant displacement noise to the test mass in the gravitational wave channel, with the noise becoming dominant at frequencies around \SI{100}{\hertz} where the interferometer would otherwise be most sensitive~\cite{Hammond2012}. Due to its high quality factor, fused silica has reduced mechanical loss and therefore lower noise. \Cref{fig:aligo-noise-budget} shows that suspension thermal noise is no longer a dominant noise source, unlike in \ILIGO{}.

As \KAGRA{} will be a cryogenic detector, it does not gain the same noise benefit from using fused silica. Instead, it will use crystalline sapphire which offers similar noise performance at low temperatures.

At higher frequencies, suspension \emph{violin modes} have a significant influence on the measured noise~\cite{Robertson2002}. A violin mode with high quality factor can resonantly enhance noise such that it dominates all other sources in a narrow band at frequencies starting around a few hundred \SI{}{\hertz}\footnote{\Cref{fig:aligo-noise-budget} appears to show that violin modes are not dominant, however the narrow linewidth of the noise is such that the resolution is insufficient to show the effect.}. This is reduced through the use of heavier test masses, which push the violin mode frequencies higher, away from the detection band, and with special monitoring techniques~\cite{Sorazu2010}.

\subsection{\label{sec:quantum-noise}Quantum noise and the Standard Quantum Limit}
Arising from the Heisenberg Uncertainty Principle, the quantum noise present within a classical interferometer\footnote{Note the misnomer: a \emph{classical} interferometer can still be limited by \emph{quantum} noise. The name refers to the readout technique, namely the measurement of classical light intensity to determine displacement.} limits its sensitivity.

Classical laser light in the coherent state, approximating what a standard laser will output, contains equal fractional amplitude and phase uncertainties. The phase fluctuations appearing at the sensor used to measure the output of the interferometer and amplitude fluctuations interacting with the interferometer's test masses create noise at the measurement ports\footnote{A derivation of both effects can be found in, for example, ref.~\cite{Danilishin2012}.}. A fundamental limit to the sensitivity of classical interferometers arises from the combination of these two effects; this is described in more detail in \cref{sec:sql}. The following subsections summarise results from ref.~\cite{Danilishin2012}.

\subsubsection{\label{sec:quantum-shot-noise}Quantum shot noise}
As described in \cref{sec:noise-via-loss}, open ports in the interferometer allow vacuum noise to enter, and when this noise is measured by a photodetector it appears as \emph{quantum shot noise}. The phase fluctuations upon the light produce a varying photocurrent due to the stochastic arrival of photons at the sensor. The displacement-equivalent power spectral density of shot noise in an interferometer is
\begin{equation}
  \label{eq:shot-noise-psd}
  \tilde{x}^2_{\text{shot}} = \frac{\hbar c^2}{P \omega_0},
\end{equation}
in units of \SI{}{\meter^2\per\hertz}, for power $P$ and laser angular frequency $\omega_0$. As this noise arises from  spontaneous creation and annihilation of photons in space, it is a statistical random process and so the spectral density has equal power at all frequencies; it is \emph{white}. The strain-equivalent power spectral density is \cref{eq:shot-noise-psd} normalised to the arm length $L$:
\begin{equation}
  \tilde{h}^2_{\text{shot}} = \frac{\hbar c^2}{P \omega_0 L^2}.
\end{equation}

Since it scales with input power, the detrimental effect on the sensitivity due to phase uncertainty is mitigated by an increase in the classical light power injected into the interferometer.

\subsubsection{\label{sec:quantum-rp-noise}Quantum radiation pressure noise}
Despite being massless, photons impart momentum to mirrors upon reflection inversely proportional to their wavelength. The strongest effect this has on an interferometer is via \emph{\gls{DC} radiation pressure}, which arises from the classical light power circulating within the interferometer. In a suspended interferometer this radiation pressure effect extends the microscopic arm cavity length, with the equilibrium point being defined by the equivalence of the radiation pressure force to the suspension's restoring force.

\emph{Quantum} radiation pressure, on the other hand, arises from the fluctuating momentum imparted onto the test masses by fluctuations in the number of photons present within the interferometer from the laser and loss points. As with quantum shot noise this effect is related to the input power of the interferometer, but in this case fluctuations in the number of input photons creates amplitude noise that is transformed into equivalent strain noise via the dynamics of the mirror. Amplitude fluctuations upon the light beat with the classical field, creating a force noise. This fluctuating force changes the position of the mirror microscopically via its mechanical susceptibility and this appears as phase noise at the output port. As the spectrum of noise from virtual photons is white the energy imparted to the mirror is the same at all frequencies. The mechanical susceptibility of a suspended mirror follows an inverse square law in frequency above the resonant frequency, and so in terms of strain this noise source is most important at low frequencies. The radiation pressure noise power spectral density is given in this case by
\begin{equation}
  \label{eq:rp-noise-psd}
  \tilde{x}^2_{\text{rp}} = \frac{P \hbar \omega_0}{c^2 m^2 \omega^4},
\end{equation}
with reduced mirror mass $m$ and angular frequency of mirror oscillation $\omega = 2 \pi f$. The reduced mirror mass is the effective mass of the mechanical mode, given in the case of a \FPMI{} as
\begin{equation}
  m = \frac{m_1 m_2}{m_1 + m_2},
\end{equation}
where $m_1$ and $m_2$ denote the individual cavity test masses.

The strain-equivalent power spectral density is
\begin{equation}
  \tilde{h}^2_{\text{rp}} = \frac{P \hbar \omega_0}{c^2 m^2 \omega^4 L^2}.
\end{equation}
The strain amplitude noise, $\tilde{h}_{\text{rp}}$, is proportional to $\frac{1}{\omega^2}$ as expected from a free mass.

\subsubsection{\label{sec:sql}The Standard Quantum Limit}
Note that \cref{eq:rp-noise-psd} is proportional to power while \cref{eq:shot-noise-psd} is inversely proportional to power. This implies the existence of a lower bound on the achievable sensitivity at a given observation frequency $f$ in the case of uncorrelated shot and radiation pressure quantum noise sources. This bound, known as the \emph{standard quantum limit} (\gls{SQL})~\cite{Braginsky1967}, is a direct consequence of the Heisenberg Uncertainty Principle in a continuous measurement of a test mass.

The \gls{SQL} is the point at which the sum power spectral density of shot and radiation pressure noise is minimised, and this occurs when the individual components are equal. For each laser power there exists a single frequency at which the \gls{SQL} can be reached. The \gls{SQL} forms a sensitivity limit with amplitude spectral density proportional to $\frac{1}{f}$ which can only be surpassed with special, \emph{sub}-\gls{SQL} techniques. The presence of cavities in the arms of a \MI{} (formed by placing an additional, partially reflecting mirror in each arm) can enhance the power available to be able to reach the \gls{SQL}. In terms of the strain-equivalent power spectral density, the \gls{SQL} is specified for two free test masses separated by a distance $L$ by~\cite{Braginsky1996}
\begin{equation}
  \label{eq:strainsql}
  \tilde{h}^2_{SQL} = \frac{8 \hbar}{m \omega^2 L^2},
\end{equation}
with units of \SI{}{\per\hertz}.

The strain-equivalent power spectral density noise for a \MI{} with arm cavities can be written with respect to the \gls{SQL}~\cite{Kimble2001} as
\begin{equation}
  \label{eq:classicalifospectrum}
  S^{h}_{MI} = \frac{\tilde{h}^{2}_{SQL}}{2} \left( \frac{1}{\kappa} + \kappa \right),
\end{equation}
where the \gls{SQL} is reached only at a single frequency. The term $\kappa$ is the (dimensionless) \emph{opto-mechanical coupling factor}~\cite{Kimble2001}:
\begin{equation}
 \kappa = \frac{P_0}{P_{SQL}} \frac{2 \gamma^4}{\omega^2 \left( \gamma^2 + \omega^2 \right)},
 \label{eq:optomechanicalcoupling}
\end{equation}
with $P_0$ the laser power at the test masses, $P_{SQL}$ the laser power required to reach the \gls{SQL} at the cavity pole frequency and $\gamma$ the arm cavity half-bandwidth. $P_{SQL}$ is given as~\cite{Kimble2001}
\begin{equation}
 P_{SQL} = \frac{m L^2 \gamma^4}{4 \omega_0}.
\end{equation}
The effect of $\kappa$ is described in more detail in \cref{sec:position-meter-measurement}.

The \gls{SQL} is a locus defined at all frequencies, while the spectral density of a quantum noise limited interferometer touches the \gls{SQL} at only one frequency. By injecting more photons into the interferometer to carry more information regarding the motion of the mirrors, we see a smaller shot noise spectral density while we see a larger radiation pressure noise spectral density~\cite{Caves1981}. This situation is illustrated in \cref{fig:sql-vs-input-power} for different input powers.

\begin{figure}
  \centering
  \input{graphics/generated/from-python/20-sql-vs-power.pgf}
  \caption[Standard quantum limit and quantum noise with various input powers]{\label{fig:sql-vs-input-power}The \gls{SQL} for a Michelson interferometer with arm cavities of length \SI{1}{\kilo\meter}, mirrors with reduced mass \SI{50}{\kilo\gram} and optimal frequency \SI{100}{\hertz}, along with quantum noise limited sensitivity curves for three different intracavity powers. The effect of the cavity pole frequency is visible in the case of the blue curve. The higher the intracavity power, the higher the strain sensitivity can be pushed, but at the expense of higher radiation pressure noise and thus higher optimal frequency for a given interferometer configuration. Quantum non-demolition techniques can be used to surpass the \gls{SQL} (see \cref{sec:sub-sql-techniques}).}
\end{figure}

An important distinction to make here is that the \gls{SQL} is defined for \emph{uncorrelated} shot and radiation pressure noise. Techniques exist in theory and practice to reduce overall noise by introducing correlations between the two noise components with so-called \emph{quantum non-demolition} interferometry, and this is discussed in greater detail in \cref{sec:sub-sql-techniques} and \cref{c:speedmeter-intro}.

\subsection{Other fundamental noise}

\subsubsection{\label{sec:seismic-noise}Seismic noise}
The Earth's surface vibrates with a large amplitude and low frequency~\cite{ET2011}. At around \SI{10}{\micro\hertz} tidal forces due to the gravitational interaction between the Earth and Moon\footnote{\SI{10}{\micro\hertz} is about one cycle per day, the same as the Earth's rotation.} dominate the spectrum producing displacements of up to \SI{100}{\micro\meter}~\cite{Adhikari2004}. At around \SI{0.15}{\hertz} the swell of the ocean can be measured almost anywhere on the Earth, even far from coasts. These effects produce a large amount of displacement noise at low frequencies which must be filtered.

As seismic noise is large in amplitude, it is able to move test masses in interferometers far enough that they no longer fulfil the resonant condition and lose light power. In almost all audio-band interferometric experiments a large degree of isolation must be utilised to mitigate this seismic noise. In Advanced \gls{LIGO}, active platforms sitting atop passive damping materials are used to reduce this noise. Test masses are also suspended from many pendulum stages to isolate higher frequencies such that by \SI{10}{\hertz} the ground motion is suppressed by more than \num{10} orders of magnitude~\cite{Aston2012}.

Homogeneous, vertical surfaces do not couple vertical seismic noise into the gravitational wave channel horizontal to each test mass. Real suspended optics, however, contain imperfections in their manufacturing and couple a small amount of vertical motion into the horizontal direction. In addition, the curvature of the Earth over distances like the \SI{4}{\kilo\meter} arms in \ALIGO{} mean that the local gravitational fields at the \glspl{ETM} are not entirely aligned to those of the \glspl{ITM}, and so to achieve cavity resonance the operating point requires a slight off-horizontal tilt which creates seismic noise coupling. In \ALIGO{} the requirement for vertical to horizontal coupling is to be below \num{1e-3}.

\subsubsection{\label{sec:gravity-gradient-noise}Gravity-gradient noise}
Changes in the density of the ground near the test masses created by seismic noise can couple to the gravitational wave channel via \emph{gravity-gradient} noise, also referred to as \emph{Newtonian} noise. No experiment has successfully been able to decouple this subtle effect from other sources of noise, but it is believed from extensive modelling effort that this noise source will represent a problem particularly for low frequency detectors such as \ETLF{}~\cite{ET2011, Hild2011}. Simulations have shown promise in subtraction of gravity-gradient noise inferred from a series of auxiliary witness sensors~\cite{Harms2015} as well as a benefit to shaping the profile of the ground near test masses~\cite{Harms2014}.

Gravity gradient noise will be discussed in the context of \ETLF{} in \cref{c:et-lf-control}.

\subsection{Technical noise}

\subsubsection{\label{sec:laser-noise}Laser frequency and intensity noise}
A perfect laser would provide output at a single, well defined frequency. In reality such lasers do not exist and their outputs contain spectral impurities. As the laser wavelength is the ``metre stick'' by which we make displacement measurements in interferometers, it is very important to ensure that the laser's wavelength, and therefore frequency, is well defined. Frequency stabilisation control loops involving optics and electronics are usually necessary in high precision interferometric experiments.

Laser frequency noise affects the phase of the output light by creating beats between waves with different frequencies, created via thermal effects in the laser material. This can be expressed as a time-varying shift $\phi \left( t \right)$ in the underlying wave's phase. This phase transforms into frequency noise via the relation
\begin{equation}
  \Delta f = \frac{1}{2 \pi} \frac{d \phi}{dt}.
\end{equation}

The spectral density of frequency noise can be calculated from the autocorrelation between a frequency fluctuation at time $t$ and another at time $t + \Delta t$, but a simpler method is to realise that the laser is used to measure the cavity length by relating its change in frequency to the change in length via \cref{eq:freq-to-length}. Multiplying the relative frequency noise by the difference in arm lengths $\delta L$ gives a first order estimate of the displacement noise created by fluctuations in the laser:
\begin{equation}
  \label{eq:laser-freq-noise}
  \tilde{x}_{\text{freq. noise}} = \delta L \frac{\Delta f}{f}.
\end{equation}

The laser's intensity fluctuates due to similar mechanisms. Thermally driven misalignments within the laser can lead to scattering and the production of higher order modes, which reduce the intensity of the fundamental mode. This has a similar effect on the output as frequency noise, coupling relative intensity noise $\frac{\Delta P}{P}$ directly to the output via the microscopic offset from the dark fringe condition $\delta l$:
\begin{equation}
  \label{eq:laser-int-noise}
  \tilde{x}_{\text{int. noise}} = \delta l \frac{\Delta P}{P}.
\end{equation}

\Cref{eq:laser-freq-noise,eq:laser-int-noise} show that the level to which the dark fringe condition is satisfied determines the laser noise witnessed at the output. This is because of the cancellation at the beam splitter from noise in the two arms. Laser noise propagates to the beam splitter, where it is split between the arms. Matching the arms macroscopically cancels frequency noise and matching the arms microscopically cancels intensity noise.

\subsubsection{\label{sec:johnson-nyquist-noise}Electronic noise}
Johnson-Nyquist noise arises from loss within electronic conductors. The noise scales with resistance and is characterised in units of \SI{}{\volt^2\per\hertz} by the equation
\begin{equation}
  \tilde{v}^2_{\text{john. noise}} = 4 k_B T R,
\end{equation}
where $\tilde{v}^2_{\text{john. noise}}$ is the voltage noise power spectal density, $k_B$ is the Boltzmann constant and $R$ is the electrical resistance. The Johnson-Nyquist noise from a resistor in the \SI{}{\kilo\ohm} to \SI{}{\mega\ohm} range is comparable to the noise of some low-noise operational amplifiers at room temperature, and so care must be taken in the choice of passive and active components in the design of electronics to avoid introducing excess fluctuations.

Other electronic noise can arise in integrated circuits used as part of readout electronics in detectors. Current and voltage noise present at the input and output of devices such as operational amplifiers (op-amps) can become larger than the signals being amplified without careful selection of the device for the intended application. This is examined in more detail in \cref{sec:op-amp-noise,sec:hv-amplifier}.

\subsubsection{\label{sec:quantisation-noise}Quantisation noise}
The conversion of analogue signals to digital and vice versa for sensing and control involves the use of analogue-to-digital and digital-to-analogue converters (\glspl{ADC} and \glspl{DAC}, respectively). Noise in \glspl{ADC} and \glspl{DAC} arises from \emph{quantisation error} $\epsilon_{\text{ADC}}$, which is the mismatch between the underlying signal input or output and the level determined by the \gls{ADC} or \gls{DAC}~\cite{Allen1997}. By averaging over successive cycles, \glspl{ADC} and \glspl{DAC} can make good approximations of the underlying signals, reducing the quantisation error in the case of the \gls{ADC} to the interval $\epsilon_{\text{ADC}} \in \left( \frac{-\Delta}{2}, \frac{\Delta}{2} \right]$~\cite{Allen1997}, where $\Delta$ is the smallest voltage reference in an \gls{ADC} with $2^b$ codes:
\begin{equation}
  \begin{split}
    \Delta &= \frac{V_{\text{max}} - V_{\text{min}}}{2^{b}} \\
           &= \frac{V_{\text{range}}}{2^{b}}.
  \end{split}
\end{equation}

Noise from \glspl{ADC} and \glspl{DAC} is minimised through selection of hardware with appropriate dynamic range for the signals to be sensed. The use of \emph{whitening} techniques can also prevent the prevalence of quantisation noise; this is discussed in greater detail in \cref{sec:whitening-dewhitening}.

\section{\label{sec:ifo-response}Sensitivity of the \MI{}}
The phase change due to gravitational waves appearing at the output is proportional to the power in the arms, and so greater input power leads to greater response at the output (given the caveats regarding quantum noise as discussed in \cref{sec:quantum-noise}).

As shown in \cref{sec:gw-interferometry}, the arm length of a \MI{} to provide optimal modulation upon the light field from a passing gravitational wave can be many hundreds of \SI{}{\kilo\meter} for audio band signals. Furthermore, given the standard wavelength $\lambda_0 = \SI{1064}{\nano\meter}$ for which low noise lasers exist, and a gravitational wave strain $h_0 = \num{e-21}$ similar to that of \GWFIRSTEVENT{}, the modulation index will be of the order $\frac{\omega_0}{\omega_g} \approx \num{e12}$ and so the phase of the light will have to be measured at the output port to a precision of around $h_0 \frac{\omega_0}{\omega_g} \approx \SI{e-9}{\radian}$, a difficult feat.

When the interferometer is held close to the dark fringe the light from each arm containing common phase changes exits the beam splitter back towards the input, and so this light would otherwise be lost.

Improvements to the \MI{} design have been made over the past decades in order to address these issues, and these are discussed in the following subsections.

\subsection{\label{sec:fabry-perot-cavities}\FP{} arm cavities}
One way to simultaneously reduce the phase measurement requirement and the \emph{effective} arm length is to use \FP{} cavities. \FP{} cavities increase a photon's path length by reflecting it many times between two partially transmissive mirrors. By placing \FP{} cavities in the arms of a \MI{}, as shown in \cref{fig:fpmi}, the response can be enhanced in a particular frequency band defined by the cavity parameters. The effect of the cavity on the sensitivity can be characterised by the \emph{finesse} as discussed in \cref{sec:cavity-fom}. Increased cavity finesse leads to a greater number of stored photons, allowing for greater response to incident gravitational waves, but over a narrower bandwidth than the simple \MI{}. The reflectivity of the mirrors in \FP{} cavities must be chosen to allow for sufficient sensitivity in the desired band; the objective is not simply to maximise the light storage time.

\begin{figure}
  \centering
  \includegraphics[width=0.5\columnwidth]{graphics/generated/from-svg/20-fabry-perot-michelson.pdf}
  \caption[\FPMI{}]{\label{fig:fpmi}\FPMI{} topology.}
\end{figure}

The arm cavities within a \FPMI{} must be held at the operating point just as with a \MI{} to maintain maximum sensitivity. Angular misalignments allow higher order modes of the light field to resonate which can complicate the longitudinal control of the interferometer and can introduce additional noise coupling from mirror surface defects.

\subsection{\label{sec:power-recycling}Power recycling}
As shown in \cref{sec:operating-point} gravitational wave interferometers are typically held at the dark fringe where the carrier light is rejected by the beam splitter back towards the input laser. It is typical to place a Faraday isolator in the input path to prevent the interferometer from sampling the positions of the input optics used to steer the laser light towards the beam splitter, and so this light is dumped. Once lost this light is not available to sample the positions of the test masses.

To compensate for interferometer light lost towards the input port it is possible to increase the input laser power, but in general appropriate input lasers are already used at the maximum output power that satisfies an experiment's laser noise requirement, and this doesn't solve the underlying loss mechanism: some light will still be dumped by the Faraday isolator. Another approach is to place a \emph{power recycling mirror} at the input to the interferometer which reflects the returning light back into the interferometer by forming a cavity between the recycling mirror and the arms, effectively increasing the power stored there. Used in combination with \FP{} arm cavities this technique can achieve enhanced sensitivity over the standard \MI{}. The power recycling mirror can be calibrated to enhance the carrier power in a band wider than the intended detector bandwidth, and the \FP{} mirrors can be calibrated to set the detector bandwidth. The first generation \ILIGO{} and \IVIRGO{} detectors were \PRFPMI{}s.

\subsection{\label{sec:signal-recycling}Signal recycling}
Signal recycling is a similar concept to power recycling, whereby an additional mirror is placed within the interferometer to selectively enhance light in a particular frequency band~\cite{Meers1988}. In this case the \emph{signal recycling mirror} is placed at the output port of the beam splitter to create an additional cavity between the output and the arms. This mirror enhances the signal power at the expense of the bandwidth of the arms, as opposed to the carrier enhanced by the use of a power recycling mirror. The signal recycling mirror's transmissivity can be set to determine the frequency range over which this enhancement occurs, and the position of the signal recycling mirror can be tuned to focus this enhancement in either a narrow or broad frequency band~\cite{Buonanno2001}. This is discussed in more detail in \cref{sec:et-lf-control-challenges}.

\subsection{Dual recycling with \FP{} arm cavities}
The natural combination of power and signal recycling with the \FPMI{} leads to the \emph{\DRFPMI{}} shown in \cref{fig:drfpmi}. This is the topology that provides the greatest sensitivity in a given band of interest, either broadband or narrowband depending on the tuning of the signal recycling cavity, for a given laser power, arm length and mirror mass; it is therefore the topology employed in current generation detectors. The use of \emph{dual recycling} was initially demonstrated in both table-top and suspended prototype experiments~\cite{Strain1991, Heinzel1998, Freise2000}, and later a full-scale dual-recycled \MI{} detector was demonstrated at \GEO{}~\cite{Heinzel2002, Grote2004}. The \ALIGO{} interferometers were the first to fully implement the \DRFPMI{} topology in detectors capable of sensing gravitational waves.

\begin{figure}
  \centering
  \includegraphics[width=0.5\columnwidth]{graphics/generated/from-svg/20-dual-recycled-fabry-perot-michelson.pdf}
  \caption[Dual-recycled \FPMI{}]{\label{fig:drfpmi}Dual-recycled \FPMI{}.}
\end{figure}

\section{\label{sec:sub-sql-techniques}Surpassing the Standard Quantum Limit}
Predictions for the population of sources within the range of the advanced detectors show that it is beneficial to improve the sensitivity at low frequencies~\cite{Sathyaprakash2012}. The sensitivity of a \MI{} at low frequencies can be increased through the use of heavier masses as shown by \cref{eq:strainsql}, scaling proportionally to $\sqrt{m}$. The use of mirrors larger and heavier than the \SI{40}{\kilo\gram} mirrors used in \ALIGO{} is a considerable technical challenge. The availability of test mass material of suitable quality in such dimensions is not clear, as is the ability for the suspension systems to isolate noise from such large masses.

To improve sensitivity at higher frequencies, \cref{eq:shot-noise-psd} shows that laser power can be increased. As with heavier mirrors, this presents technical challenges in laser stability~\cite{Hildebrandt2007}, the control of \emph{parametric instabilities}~\cite{Evans2015} and the thermal effects associated with absorption in materials~\cite{Steinlechner2016}.

To bypass the problems associated with the use of heavier mirrors and more powerful lasers, a number of techniques have been proposed in the literature to increase the sensitivity of interferometers beyond the \gls{SQL} through the use of \emph{quantum non-demolition} (\gls{QND})~\cite{Braginsky1995}. These include the modification of the optics of the interferometer~\cite{Kimble2001}, such as through the injection of squeezed light~\cite{Caves1981}, variational readout~\cite{Vyatchanin1995, Vyatchanin1996} or \SM{}s~\cite{Braginsky1990}; and the creation of new light-mirror interactions to increase the response of the interferometer to differential motion of the test masses~\cite{Chen2011}.

\subsection{\label{sec:squeezing}Squeezing}
The use of squeezing is an attempt to reduce the quantum noise at the output of the interferometer by injecting vacuum light with \emph{correlated} noise. By choosing a suitable injection \emph{quadrature}, it is possible to remove some of the quantum noise impinging upon the signal at a frequency of interest, instead moving the noise terms into the orthogonal, unobserved readout quadrature. Squeezing is particularly favourable in combination with \gls{DC} readout, a combination currently implemented in \GEO{}~\cite{Willke2006, Affeldt2014}.

To reduce the effect of frequency-dependent \emph{ponderomotive} squeezing arising from the mechanical susceptibility of the test masses, and to achieve broadband reduction of quantum noise, it is necessary to inject the squeezed light via filter cavities to provide a frequency-dependent phase shift to the vacuum field~\cite{Kimble2001}. These cavities are typically high finesse, which makes the squeezed light particularly susceptible to filter cavity loss~\cite{Kwee2014}.

Squeezing has been demonstrated in \GEO{} with high duty cycle~\cite{Grote2013} and is a planned upgrade for \ALIGO{} in the near future~\cite{Miller2015}. The proposed designs for the \ET{} and \LIGOCE{} assume \SI{10}{\deci\bel} effective squeezing.

\subsection{Variational readout}
Instead of modifying the input noise at the output port of the interferometer with frequency-dependent squeezing, variational readout achieves sub-\gls{SQL} sensitivity through the use of a homodyne detector with a homodyne angle chosen to create coherent cancellation of quantum noise between the local oscillator and the signal. Frequency dependent variational readout can be achieved in a similar way to squeezing: the output light can be passed through filter cavities in the same way as squeezed input.

Variational readout can in theory be combined with squeezing either with a fixed squeezing angle~\cite{Buonanno2004} or through a complicated frequency dependence of both squeezing and homodyne phase filter cavities~\cite{Harms2003}. The use of homodyne readout in gravitational wave detectors, however, is not considered mature enough for upgrades to existing or future facilities, primarily due to the stability requirements for the local oscillator field~\cite{Steinlechner2015}.

\subsection{Light-mirror interactions}
\emph{Optical springs}~\cite{Braginsky1999, Buonanno2002, Corbitt2007, Rehbein2008, Gordon2015}, \emph{optical inertia}~\cite{Khalili2011, Voronchev2012} and \emph{intracavity schemes}~\cite{Braginsky1997, Khalili2002, Danilishin2006} have been proposed for use in gravitational wave detectors to improve sensitivity beyond the \gls{SQL} through modification of the mechanical response of the interferometer's mirrors with light using optomechanical interaction.

The creation of optical springs requires complicated control arrangements. The use of two optical springs to remove the instabilities created by a single spring can relax some of the control requirements~\cite{Rehbein2008} but full studies of the effect of noise and the sensitivity on this type of interferometer are not at a stage to be able to predict their use in future detectors.

\subsection{Modification of the interferometer design}
First proposed in 1990~\cite{Braginsky1990}, the measurement of the speed of test masses instead of displacement can lead to a reduction in quantum noise. Proposals for experiments to measure speed were made later and involved the use of an additional optical cavity at the output port of a \MI{}~\cite{Braginsky2000, Purdue2002}, termed a \emph{sloshing} cavity, creating an interaction between the main interferometer and the sloshing cavity that samples the test mass coordinates in a way that resembles speed.

In 2003, Chen showed that the Sagnac interferometer contained the necessary characteristics of a \SM{}~\cite{Chen2003} and estimated the sensitivity that such an interferometer might achieve when the corner mirrors are replaced with arm cavities to resemble a \FPMI{}. This work was later expanded to include the effect of losses~\cite{Danilishin2004, Danilishin2015}.

The \SSM{} is being considered as a potential upgrade for the \ET{} beyond its initial configuration~\cite{Wang2013, Huttner2016}, albeit following a polarising topology with linear arm cavities~\cite{Danilishin2004} due to the sensitivity degradation from back-scattering in triangular arm cavities~\cite{Pascucci2016}.

\section{The future of ground-based gravitational wave interferometry}
Plans are in place for upgrades to \ALIGO{} after the science run in 2016, when squeezed light injection will be implemented. In the medium term, the interferometer may be adapted to run with cryogenic optics to provide sensitivity at lower frequencies. The \ALIGO{} and \AVIRGO{} detectors already push their current facilities to their limits, however, and so in the long term the goal is to build new facilities with significantly improved sensitivity. A conceptual design study for the \ET{} was completed in 2011~\cite{ET2011}, a new European facility in a triangular, \SI{10}{\kilo\meter} arm configuration, and similar studies for a new \SI{40}{\kilo\meter} \LIGO{} facility are ongoing~\cite{Dwyer2015, aligocosmic2016}. These facilities are planned for the late-2020s to early-2030s, and the ongoing research and development work will help to determine the technologies that become part of these detectors. The reduction of quantum and thermal noise and the control of such interferometers will be crucial areas of investigation, and some potential solutions are presented in the rest of this work.