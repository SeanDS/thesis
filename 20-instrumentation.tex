\chapter{Sensitivity and Noise in Gravitational Wave Detectors}
\label{c:instrumentation}

\section{Interferometer Response}

The effect that an interferometer has on a means of readout (e.g. a photodetector) as some variable is modulated is termed the \emph{response}. The most important response to consider in gravitational wave interferometry is the arm cavity differential degree of freedom's response to the beam splitter's output port. The response varies as a function of frequency depending on the interferometer topology and its light storage time. For instance, in a Michelson interferometer the dARM response is typically flat for frequencies below the arm cavity pole, and decaying \checkme{at a rate proportional to frequency} above. At higher frequencies, less light can be stored as the cavity cannot build up sufficient light power due to mirror loss and round-trip time. This effect is shown in \note{Figure XXX} for different end test mass reflectivities and in \note{Figure XXX} for different arm cavity lengths.

In order to calculate a response function for a set of optics, it is necessary to understand the propagation of a light field through vacuum and through optics.

\subsection{Interferometer Fields}
* Intro to field equations of Fabry-Perot. Perhaps generalise to Michelson? Point reader towards analytical calculation of DRFPMI (Ken Strain paper).

We define the field amplitude of light across dimension $x$ as:
\begin{equation}
  \label{eq:field-amplitude}
  A = A_0 \text{e}^{\text{i} kx},
\end{equation}
for initial amplitude $A_0$ and wavenumber $k = \frac{2 \pi}{\lambda}$, where $\lambda$ denotes wavelength. Upon transmission through an optic, the transmitted field is equal to the input field, $A_{\text{in}}$, scaled by the optic's field transmissivity, $t$:
\begin{equation}
  A_{\text{t}} = tA_{\text{in}}.
\end{equation}
The reflected field is similarly scaled by the optic's field reflectivity, $r$:
\begin{equation}
  \label{eq:reflected-field}
  A_{\text{r}} = \text{i}rA_{\text{in}}.
\end{equation}
Note the presence of complex coefficient $\text{i}$, equivalent to a multiplication by $\text{e}^{\text{i} \frac{\pi}{2}}$. This shows that the phase difference between reflected and transmitted fields is $\frac{\pi}{2}$, which is required for energy conservation. We choose to apply this phase shift to the reflected field by convention (see Appendix\,\ref{a:reflection-phase}).

The field due to motion of a mirror at a distance $x$ can be determined by combining Equations \ref{eq:field-amplitude} and \ref{eq:reflected-field}. As the light must travel to and from the optic, the field picks up a factor of $\text{e}^{2\text{i}kL}$ in phase, and a factor of $\text{i}r$ in amplitude. Distance $x$ may in general be a time varying function, so the time varying field is then:
\begin{equation}
  A_{\text{r}} \left( t \right) = \text{i} r \text{e}^{\text{i} k x \left( t \right)}.
\end{equation}
If we assume the motion of the mirror is sinusoidal (a very good approximation, given that mirrors are typically attached to a rigid body), then the field can be expressed in terms of the the amplitude $x_0$ and frequency $f$ of oscillation:
\begin{equation}
  A
\end{equation}
\note{See G.V. p216, derive equivalent version}

\subsubsection{Transmission and reflection}

\subsubsection{Signal sidebands}
\note{Talk about signal sidebands ON THE CARRIER ONLY (see G.H. thesis p58)}

\subsubsection{Heterodyne detection}
Add RF sidebands that don't resonate, schnupp asymmetry, etc...

\subsubsection{Homodyne detection}
pick off a bit of carrier light with a DARM offset...

\section{Limiting noise sources in detectors}
Quantum noise, thermal noise (since if you push QN low enough, you run into thermal), possibly mention others (Newtonian noise - why ET will be under ground, etc.)

\subsection{Measurement noise in gravitational wave interferometry}
Gravitational wave interferometry is ultimately a matter of signals and noise. The term ``signal'' simply refers to a wanted pattern of oscillations representing a particular variable of interest. ``Noise'', on the other hand, is unwanted oscillations that appear at the signal measurement device. The measurement of any signal necessarily entails the measurement of some noise, and so it does not make sense to talk about signal without talking about noise.

\subsubsection{\label{sec:snr}Signal to noise ratio}

Part of scientists' jobs, both experimental and theoretical, is to design experiments and apparatus in such a way as to maximise the ratio of signal power, $S$, to noise power, $N$, in the region of interest. This is the \emph{signal-to-noise} ratio (SNR),
\begin{equation}
  \text{SNR} = \frac{S}{N}.
\end{equation}
In engineering circles the standard representation of signal to noise is in units of \emph{decibels} (\SI{}{\deci\bel}), defined for signal power as:
\begin{equation}
  \text{SNR}_{\SI{}{\deci\bel}} = 10 \log_{10} \left( \frac{s}{n} \right).
\end{equation}
As this representation is logarithmic, it is a useful for expressing both small and large signals. It is borrowed from engineering circles, and is particularly popular in dicussions of electronic filtering.

\subsubsection{Optimal operating point}

In precision measurement, typically an experimentalist can only infer the quantity of an underlying amplitude from a measured power. A simple example is the measurement of mirror displacement in a simple Michelson interferometer \emph{via} the photocurrent output of the photodetector. The wave is composed of the electric and magnetic fields, $E$ and $B$, respectively, each of which can be expressed as a travelling wave:
\begin{align}
  E &= E_0 \text{e}^{\text{i} \left( kx - \omega t \right)}, \\
  B &= B_0 \text{e}^{\text{i}  \left( kx - \omega t \right)},
\end{align}
where $E_0$ and $B_0$ are the initial field amplitudes, $k = \frac{2 \pi}{\lambda}$ is the wave vector, $x$ is the displacement, $\omega$ is the angular frequency and $t$ is time. The intensity $S$ is the product of the two, with the magnetic field weighted by the permittivity of free space $\epsilon_0$ and speed of light in vacuum $c_0$ \checkme{check this is consistent with Living Review p31}:
\begin{equation}
  S = \frac{1}{2} \epsilon_0 \left( E^2 + c_0^2 B^2 \right).
\end{equation}
As the $E$ and $B$ fields are orthogonal, their sum at any point in time and space remains constant. We can therefore state that the wave's intensity is proportional to the square of an ``amplitude'' $A$, expressing the combination of $E$ and $B$:
\begin{equation}
  S \propto A^2.
\end{equation}
Leaving the beam splitter towards the end of each arm, each wave in the Michelson interferometer propagates with amplitude
\begin{equation}
  A = A_{\text{in}} \text{e}^{\text{i} \left( kx - \omega t \right)},
\end{equation}
where $A_{\text{in}}$ represents the field at the beam splitter's input.

A difference in path length between the arms $\Delta x$ leads to a difference in round-trip phase between light returning to the beam splitter. With light of constant amplitude injected into the interferometer, and assuming that the light round-trip time is much quicker than the transient causing a change to the arm lengths, we can look at the superposition of returning light at the beam splitter to determine the change in path length. At the beam splitter's output port, the field superposition becomes:
\begin{equation}
  A_{\text{out}} = \frac{A_{\text{in}}}{2} \text{e}^{\text{i} \left( k \left( x + \frac{\Delta x}{2} \right) - \omega t \right)} + \frac{A_{\text{in}}}{2} \text{e}^{\text{i} \left( k \left( x - \frac{\Delta x}{2} \right) - \omega t \right)},
\end{equation}
where we assume that the path length difference caused by the transient is evenly distributed, differentially, between the two arms. The power measured by a photodetector, $P_{\text{out}}$, is then the field multiplied by its complex conjugate:
\begin{equation}
  \label{eq:mich-p-out}
  \begin{split}
    P_{\text{out}} &\propto \langle A_{\text{out}}^*A_{\text{out}} \rangle \\
                   &= \frac{P_{\text{in}}}{2} \left( 1 + \cos \left( k \Delta x \right) \right),
  \end{split}
\end{equation}
using the fact that $P_{\text{in}} = A_{\text{in}}^2$.

Here we see that a static field $\frac{P_{\text{in}}}{2}$ is present upon the photodetector, independent of the arm length change. In simple experiments, often it is practical to keep the interferometer at an operating point commonly referred to as ``half way up the fringe''. Here, the interferometer's mirrors are nominally positioned such that the output signal is oscillating about the midpoint between crest and trough (see Figure\,\ref{fig:fringe}). As the gradient is steepest at this point, any small changes to the relative arm length of the Michelson interferometer result in a significant difference in power at the photodetector. This operating point, however, is not optimal in terms of \emph{sensitivity} to arm length fluctuations. As discussed in Section\,\ref{sec:snr}, the noise level is just as important as the signal.

%% FIXME: change this plot's x-labels to use wavelength, to fit with the conclusion in the text.
\begin{figure}
  \centering
  \includegraphics[width=\columnwidth]{graphics/generated/from-python/20-fringe.pdf}
  \caption{Fringe.}
  \label{fig:fringe}
\end{figure}

By inspecting Equation\,\ref{eq:mich-p-out}, it is clear to see that there must exist, in cases where there is a signal due to a difference in arm length, a static photodetector power independent of the arm length. This does not contribute any displacement information to the measurement, but does contribute shot noise:
\begin{equation}
  P_{\text{shot, out}} = \sqrt{2 h f_0 P_{\text{in}}},
\end{equation}
where $h$ is Planck's constant, $f_0$ is the light frequency and $P_{\text{in}} = A_{\text{in}}^2$, the power entering the interferometer at the beam splitter. The optimally sensitive operating point is therefore not simply one which maximises the signal gradient, but rather one which maximises the SNR. The SNR is:
\begin{equation}
  \text{SNR} = \frac{P_{\text{out}}}{P_{\text{shot, out}}} = \sqrt{\frac{P_{\text{in}}}{4 h f_0}} \left( 1 + \cos \left(k \Delta x \right) \right).
\end{equation}

The $\Delta x$ term in Equation\,\ref{eq:mich-p-out} is a combination of a static arm length \emph{detuning}\textemdash representing the arm length mismatch required to reach the desired operating point\textemdash and a differential gravitational wave signal $\Delta x_{\text{GW}}$. A suitable choice of $ x_{\text{tune}}$ can remove the majority of the static power present at the output. Setting the slope of the SNR with respect to the tuning to zero,
\begin{equation}
  \frac{\Delta \text{SNR}}{\Delta x_{\text{GW}}} = -k \sqrt{\frac{P_{\text{in}}}{4 h f_0}} \sin \left(k \Delta x\right) = 0,
\end{equation}
we find that maximum SNR is achieved for static tunings 
\begin{equation}
  \Delta x = 0 \text{ mod } \lambda.
\end{equation}
This result shows that the optimal operating point in terms of SNR is at the point where the light from the two arms interferes destructively. While any multiple of $\lambda$ will satisfy the SNR condition as defined, in reality we have not considered laser noise coupling. The more matched the arm lengths are, the lower the laser noise couples to the output port. In reality there are also mismatches in the reflectivities of the mirrors in the arms: this creates an asymmetry called a \emph{contrast defect} which leads to additional shot noise at the output port.

%CHECKME At the output port, light from one arm is transmitted through the beam splitter while the light from the other arm is reflected, and so a reflection phase convention applies (see Appendix\,\ref{a:reflection-phase}). The arm lengths are therefore offset by $\frac{\lambda}{4}$ with respect to one another.

% Section 1.3.1 of Gabriele Vajente's thesis covers this in more detail.

\subsubsection{Root mean square amplitude}

\subsubsection{Spectral density}
In gravitational wave interferometry, a lot of thought and planning typically goes in to the design of the apparatus in order to minimise noise \emph{transients}\textemdash pulses of unwanted noise with finite energy\textemdash such as a door slam\footnote{And, of course, some forms of gravitational wave signal; though it is not the intention of experimentalists to minimise the effect of this particular source.}. Such transients have finite energy over a finite time, and can be entirely characterised\textemdash within measurement error\textemdash by a Fourier transform of the time series in which the event occurred. With sufficient isolation from unwanted transients, the remaining noise sources within the interferometer tend to arise from stationary, random processes, with energy approaching infinity as measurement time approaches infinity. In this circumstance, the Fourier transform of the underlying time-domain signal does not strictly exist. An alternative representation of a noise process is to represent the amount of work it performs per unit time: its power. The \emph{power spectral density} is a representation of the power present within each frequency of a signal in the steady state. This distribution can then be integrated back into a finite, steady state power.

Typically the time evolution of a particular signal is the only data an experimentalist has at their disposal in order to characterise a signal in terms of its spectral density. A true representation of the power spectral density requires, like the measurement of energy spectral density, infinite time. A compromise can be made, however, in order to estimate the power spectral density from a set of Fourier transforms of the time series. By averaging Fourier transforms of various subsets of the data $\left[ t, t + \Delta t \right]$, a power spectral density can be approximated for a particular frequency range, determined by the bounds established by the inverse of the longest and shortest subsets of the time series transformed. While this compromise is typically reasonable in most cases, and indeed essential for any finite measurement, it is unable to exclude the possibility that part of the power spectral density is formed from signal content in frequencies outwith the measured range. Longer time series measurements allow for more averaging of frequency bands, and thus a better approximation to the true power spectral density.

\subsubsection{Bandlimiting}
\note{Nyquist-Shannon sampling theorem, https://en.wikipedia.org/wiki/Bandlimiting}

\subsection{Technical Noise}

\subsubsection{Laser Frequency Noise}
A perfect laser provides output at a single, well defined frequency, or in other words, it has an infinitely narrow linewidth. In reality, such lasers do not exist and instead the output contains spectral impurities. As the laser wavelength is the ``metre stick'' by which we make measurements of length in interferometers, it is very important to ensure that the laser's wavelength, and therefore frequency, is well defined. A well designed laser will get an experimentalist part of the way there, but in most cavity experiments a frequency stabilisation control loop involving optics and electronics is necessary.

Apart from its linewidth, it is possible to represent a laser's frequency noise in terms of its noise spectral density, measurable via some heterodyne technique with a standard spectrum analyser. The effect of laser frequency noise is for a beat signal $V \left( t \right)$ to occur due to overlapping sinusoidal waves:
\begin{equation}
  V \left( t \right) = V_0 \left( t \right) \sin \left[ 2 \pi f_0 + \phi \left( t \right) \right],
\end{equation}
where $V_0 \left( t \right)$ is the beat amplitude and $\phi \left( t \right)$ is the phase difference at time $t$. We can rewrite this in terms of frequency $f \left( t \right)$,
\begin{equation}
  \begin{split}
    f \left( t \right) &= f_0 + \frac{1}{2 \pi} \frac{\text{d} \phi \left( t \right)}{\text{d} t} \\
                       &= f_0 + \Delta f \left( t \right),
  \end{split}
\end{equation}
with $\Delta f \left( t \right)$ the instantaneous frequency fluctuation. The power spectral density arises from the autocorrelation between a frequency fluctuation at time $t$ and another at time $t + \Delta t$, which can be expressed in the form:
\begin{equation}
  S_{\Delta f} \left( f \right) = 2 \int^{\infty}_{0} \langle \Delta f \left( t \right) \Delta f \left( t + \Delta t \right) \rangle \text{e}^{\left( -\text{i2}\pi f \Delta t \right) \text{d}\Delta t}
\end{equation}



\note{See Neil's thesis p14, https://arran.physics.gla.ac.uk/wp/speedmeter/2016/03/24/update-on-linear-cavity-work, Optics Communications 201 (2002) 391-397, and Applied Optics 49, issue 25, 4801-4807, 2010}

\subsubsection{Laser Intensity Noise}
\note{describe RIN, etc...}

\subsection{Thermal Noise}

\subsection{Quantum Noise}

\subsubsection{The Standard Quantum Limit}

\section{Overview of Current Efforts}
* aLIGO, aVirgo, KAGRA, GEO
* Sensitivity curves for all?

\section{The Future of Gravitational Wave Interferometry}

\subsection{Planned upgrades and new facilities}
* Worldwide network of interferometric detectors
* Plans to build/upgrade more (KAGRA, ET, LIGO Voyager, LIGO CE, etc.)
* Space based detectors

\subsection{Surpassing the Standard Quantum Limit}
\subsubsection{Quantum Non-Demolition}
Give general overview of technique, then explain the Sagnac speedmeter in more detail.

\begin{figure}
  \centering
  \includegraphics[width=\columnwidth]{graphics/generated/from-python/20-sideband-structure.pdf}
  \caption{Sideband structure}
  \label{fig:sideband-structure}
\end{figure}