\chapter{Sensitivity and noise in gravitational wave detectors}
\label{c:instrumentation}

\note{Signal extraction cavity vs signal recycling cavity?}

\section{Interferometer foundations}

The effect that an interferometer has on a means of readout (e.g. a photodetector) as some variable is modulated is termed the \emph{response}. The most important response to consider in gravitational wave interferometry is the arm cavity differential degree of freedom's response to the beam splitter's output port. The response varies as a function of frequency depending on the interferometer topology and its light storage time. For instance, in a Michelson interferometer the dARM response is typically flat for frequencies below the arm cavity pole, and decaying \checkme{at a rate proportional to frequency} above. At higher frequencies, less light can be stored as the cavity cannot build up sufficient light power due to mirror loss and round-trip time. This effect is shown in \note{Figure XXX} for different end test mass reflectivities and in \note{Figure XXX} for different arm cavity lengths.

In order to calculate a response function for a set of optics, it is necessary to understand the propagation of a light field through vacuum and through optics.

\note{Reference the appendix on interferometry fields etc.}

\subsection{Numerical simulation tools}
Recall the reflection term $r$ we ignored earlier. Light propagating within an interferometer will collect reflection and transmission terms from each mirror it encounters. A photodetector placed at a port of the interferometer will then see light with amplitude and phase information representing the path it has taken through the interferometer. We have shown this effect in a trivial example involving a \FP{} cavity with one input, but the same mathematics can be used to represent any interferometer. For anything beyond the simplest of examples, we typically employ numerical simulation tools to perform the tedious calculations involved in obtaining the output signals from interferometers, with the packages Finesse \cite{Freise2004} (not to be confused with \emph{cavity} finesse) and Optickle \cite{Evans2012} being popular choices within the field of gravitational wave interferometry.

\section{Limiting noise sources in current detectors}
Quantum noise, thermal noise (since if you push QN low enough, you run into thermal), possibly mention others (Newtonian noise - why ET will be under ground, etc.)

PUT ALIGO NOISE BUDGET HERE

\subsection{Measurement noise in gravitational wave interferometry}
Gravitational wave interferometry is ultimately a matter of signals and noise. The term ``signal'' simply refers to a wanted pattern of oscillations representing a particular variable of interest. ``Noise'', on the other hand, is unwanted oscillations that appear at the signal measurement device. The measurement of any signal necessarily entails the measurement of some noise, and so it does not make sense to talk about signal without talking about noise.

\subsubsection{\label{sec:snr}Signal to noise ratio}

Part of scientists' jobs, both experimental and theoretical, is to design experiments and apparatus in such a way as to maximise the ratio of signal power, $S$, to noise power, $N$, in the region of interest. This is the \emph{signal-to-noise} ratio (SNR),
\begin{equation}
  \text{SNR} = \frac{S}{N}.
\end{equation}
In engineering circles the standard representation of signal to noise is in units of \emph{decibels} (\SI{}{\deci\bel}), defined for signal power as:
\begin{equation}
  \text{SNR}_{\SI{}{\deci\bel}} = 10 \log_{10} \left( \frac{s}{n} \right).
\end{equation}
As this representation is logarithmic, it is a useful for expressing both small and large signals. It is borrowed from engineering circles, and is particularly popular in discussions of electronic filtering.

\subsubsection{Optimal operating point}

In precision measurement, typically an experimentalist can only infer the quantity of an underlying amplitude from a measured power. A simple example is the measurement of mirror displacement in a simple Michelson interferometer \emph{via} the photocurrent output of the photodetector. The wave is composed of the electric and magnetic fields, $E$ and $B$, respectively, each of which can be expressed as a travelling wave:
\begin{align}
  E &= E_0 \text{e}^{\text{i} \left( kx - \omega t \right)}, \\
  B &= B_0 \text{e}^{\text{i}  \left( kx - \omega t \right)},
\end{align}
where $E_0$ and $B_0$ are the initial field amplitudes, $k = \frac{2 \pi}{\lambda}$ is the wave vector, $x$ is the displacement, $\omega$ is the angular frequency and $t$ is time. The intensity $S$ is the product of the two, with the magnetic field weighted by the permittivity of free space $\epsilon_0$ and speed of light in vacuum $c_0$ \checkme{check this is consistent with Living Review p31}:
\begin{equation}
  S = \frac{1}{2} \epsilon_0 \left( E^2 + c_0^2 B^2 \right).
\end{equation}
As the $E$ and $B$ fields are orthogonal, their sum at any point in time and space remains constant. We can therefore state that the wave's intensity is proportional to the square of an ``amplitude'' $A$, expressing the combination of $E$ and $B$:
\begin{equation}
  S \propto A^2.
\end{equation}
Leaving the beam splitter towards the end of each arm, each wave in the Michelson interferometer propagates with amplitude
\begin{equation}
  A = A_{\text{in}} \text{e}^{\text{i} \left( kx - \omega t \right)},
\end{equation}
where $A_{\text{in}}$ represents the field at the beam splitter's input.

A difference in path length between the arms $\Delta x$ leads to a difference in round-trip phase between light returning to the beam splitter. With light of constant amplitude injected into the interferometer, and assuming that the light round-trip time is much quicker than the transient causing a change to the arm lengths, we can look at the superposition of returning light at the beam splitter to determine the change in path length. At the beam splitter's output port, the field superposition becomes:
\begin{equation}
  A_{\text{out}} = \frac{A_{\text{in}}}{2} \text{e}^{\text{i} \left( k \left( x + \frac{\Delta x}{2} \right) - \omega t \right)} + \frac{A_{\text{in}}}{2} \text{e}^{\text{i} \left( k \left( x - \frac{\Delta x}{2} \right) - \omega t \right)},
\end{equation}
where we assume that the path length difference caused by the transient is evenly distributed, differentially, between the two arms. The power measured by a photodetector, $P_{\text{out}}$, is then the field multiplied by its complex conjugate:
\begin{equation}
  \label{eq:mich-p-out}
  \begin{split}
    P_{\text{out}} &\propto \langle A_{\text{out}}^*A_{\text{out}} \rangle \\
                   &= \frac{P_{\text{in}}}{2} \left( 1 + \cos \left( k \Delta x \right) \right),
  \end{split}
\end{equation}
using the fact that $P_{\text{in}} = A_{\text{in}}^2$.

Here we see that a static field $\frac{P_{\text{in}}}{2}$ is present upon the photodetector, independent of the arm length change. In simple experiments, often it is practical to keep the interferometer at an operating point commonly referred to as ``half way up the fringe''. Here, the interferometer's mirrors are nominally positioned such that the output signal is oscillating about the midpoint between crest and trough (see Figure\,\ref{fig:fringe}). As the gradient is steepest at this point, any small changes to the relative arm length of the Michelson interferometer result in a significant difference in power at the photodetector. This operating point, however, is not optimal in terms of \emph{sensitivity} to arm length fluctuations. As discussed in Section\,\ref{sec:snr}, the noise level is just as important as the signal.

%% FIXME: change this plot's x-labels to use wavelength, to fit with the conclusion in the text.
\begin{figure}
  \centering
  \includegraphics[width=\columnwidth]{graphics/generated/from-python/20-fringe.pdf}
  \caption[Fringe]{\label{fig:fringe}Fringe.}
\end{figure}

By inspecting Equation\,\ref{eq:mich-p-out}, it is clear to see that there must exist, in cases where there is a signal due to a difference in arm length, a static photodetector power independent of the arm length. This does not contribute any displacement information to the measurement, but does contribute shot noise:
\begin{equation}
  P_{\text{shot, out}} = \sqrt{2 h f_0 P_{\text{in}}},
\end{equation}
where $h$ is Planck's constant, $f_0$ is the light frequency and $P_{\text{in}} = A_{\text{in}}^2$, the power entering the interferometer at the beam splitter. The optimally sensitive operating point is therefore not simply one which maximises the signal gradient, but rather one which maximises the SNR. The SNR is:
\begin{equation}
  \text{SNR} = \frac{P_{\text{out}}}{P_{\text{shot, out}}} = \sqrt{\frac{P_{\text{in}}}{4 h f_0}} \left( 1 + \cos \left(k \Delta x \right) \right).
\end{equation}

The $\Delta x$ term in Equation\,\ref{eq:mich-p-out} is a combination of a static arm length \emph{detuning}\textemdash representing the arm length mismatch required to reach the desired operating point\textemdash and a differential gravitational wave signal $\Delta x_{\text{GW}}$. A suitable choice of $ x_{\text{tune}}$ can remove the majority of the static power present at the output. Setting the slope of the SNR with respect to the tuning to zero,
\begin{equation}
  \frac{\Delta \text{SNR}}{\Delta x_{\text{GW}}} = -k \sqrt{\frac{P_{\text{in}}}{4 h f_0}} \sin \left(k \Delta x\right) = 0,
\end{equation}
we find that maximum SNR is achieved for static tunings 
\begin{equation}
  \Delta x = 0 \text{ mod } \lambda.
\end{equation}
This result shows that the optimal operating point in terms of SNR is at the point where the light from the two arms interferes destructively. While any multiple of $\lambda$ will satisfy the SNR condition as defined, in reality we have not considered laser noise coupling. The more matched the arm lengths are, the lower the laser noise couples to the output port. In reality there are also mismatches in the reflectivities of the mirrors in the arms: this creates an asymmetry called a \emph{contrast defect} which leads to additional shot noise at the output port.

%CHECKME At the output port, light from one arm is transmitted through the beam splitter while the light from the other arm is reflected, and so a reflection phase convention applies (see Appendix\,\ref{a:reflection-phase}). The arm lengths are therefore offset by $\frac{\lambda}{4}$ with respect to one another.

% Section 1.3.1 of Gabriele Vajente's thesis covers this in more detail.

\subsection{Thermal noise}

\subsubsection{Coating thermal noise}
In the conceptual design for the first generation of gravitational wave detectors such as GEO-600 \cite{Willke2002} the designers were not aware that thermal noise associated with the reflective coatings on mirrors would play a significant role in the sensitivity of the interferometers. For a long time it was known that thermal noise would contribute to the sensitivity of the detectors, particularly from the bulk material forming the test masses, but it soon became clear as the detectors were being commissioned that thermal noise arising from the reflective mirror coatings would dominate the thermal noise associated with the test masses in the frequency band of interest despite forming only a tiny fraction of the test masses by volume. Investigations conducted by Harry \etal{} \cite{Harry2002, Harry2007}, among others, concluded that mechanical loss present in the dielectric coatings on the test masses led to Brownian noise creating a limit to the sensitivity of detectors across a wide range of frequencies. Contributions from thermoelastic noise, arising from the thermal expansion coefficient of the materials of the coatings \cite{Braginsky1999a}, and thermorefractive noise, arising from the change in refractive index caused by such expansion \cite{Braginsky2000a}, produce further noise contributions which will become more important as coatings with improved thermal noise contributions are developed.

Over the past two decades, efforts have been made to both quantify and reduce coating thermal noise. Particular interest is being paid to the study of coatings for cryogenically cooled mirrors, such as
the sapphire (\ce{Al_2O_3}) test masses to be used in the Japanese detector \gls{KAGRA} \cite{Somiya2012}. A loss peak in the mirror material silica, for detectors until recently ubiquitous, occurs at low temperature. This makes the material unsuitable for cryogenic use, as mechanical loss will couple into the light within the interferometer and make its way to the detection port. Other materials such as sapphire do not feature this loss peak and provide lower thermal noise than silica at room temperature for a given mirror design. Coating noise is also proportional to temperature, so cryogenically cooled mirrors can offer better performance. Additionally, crystalline coatings made from compounds such as \ce{AlGaAs} can offer future detectors a coating thermal noise reduction of up to 3 over current state of the art \cite{Cole2013} if technical challenges in their manufacturing can be overcome.

A mirror topology which avoids the use of many alternating coating layers can potentially offer an improvement in noise performance. Mirrors employing grating structures can resonantly reflect light with less coating material than similarly performing dielectric mirrors \cite{Mashev1985}, though at the expense of additional technical complexity in their utility in gravitational wave detectors \cite{Leavey2015}.

\subsubsection{\label{sec:sus-thermal-noise}Suspension thermal noise}
blah

\note{Mention violin modes here}

\subsection{Quantum noise}
A major concern for designers of metrological experiments is that of quantum noise. These are signals which appear in instruments uncorrelated with the signals intended to be measured. As such, the presence of quantum noise in an instrument will degrade the precision to which an intended measurement can be made.

One of the results of quantum mechanics is that two non-commuting observables cannot be simultaneously known to full precision. For two operators $\hat{O}_+$ and $\hat{O}_-$, there exists an error $\epsilon$:
\begin{equation}
 \left[ \hat{O}_+, \hat{O}_- \right] = \epsilon.
\end{equation}
This result means that the observables contain correlated components, and as such they cannot be considered entirely separate entities. Measuring one form automatically influences the other, leading to the well-known \emph{Heisenberg Uncertainty Principle}:
\begin{equation}
 \left| \Delta \hat{O}_+ \right| \cdot \left| \Delta \hat{O}_- \right| \geq
\frac{1}{2} \left| \epsilon \right|.
\end{equation}
For interferometry there are two pairs of operators of particular importance, namely the position and momentum operators, $x$ and $p_x$, and the photon number and phase operators, $n$ and $\phi$, respectively. The canonical commutation relation between $x$ and $p_x$ is given as:
\begin{equation}
 | \Delta x\left( t \right) | \cdot |\Delta p\left( t \right) | \geq
\frac{\hbar}{2},
 \label{eq:heisenburguncertainty}
\end{equation}
where $\hbar$ is the reduced Planck constant. Consider a position measurement of a free mirror of mass $m$ being oscillated by a signal of frequency $f$. If a measurement is made at time $t$ and then again at time $t + \tau$, where $\tau = \frac{1}{\pi f}$, the uncertainty on the latter can be expressed in terms of the uncertainty in momentum of the mirror multiplied by the time difference:
\begin{equation}
 \Delta x \left( t + \tau \right) = \Delta x \left( t \right) + \Delta p_x
\left( t \right) \frac{\tau}{m}.
 \label{eq:heisenburgtime}
\end{equation}
This result shows that the momentum at time $t$ influences the position at a later time. Since momentum is position scaled by velocity, this leads to a minimum value on which $x \left( t \right)$ can take:
\begin{equation}
 \Delta x \left( t \right) \geq \sqrt{\frac{\hbar \tau}{2m}}.
\end{equation}
This shows that even with an otherwise unperturbed mirror, the momentum imparted by the measurement at time $t$ influences the later measurement at $t + \tau$ in such a way that it adds noise. If the wavefunction representing the measurement is known too accurately, then the wavefunction representing the mirror will narrow enough to make position estimation a process with high
uncertainty.

Equation\,\ref{eq:heisenburguncertainty} expresses that the smaller the error on the position of an observable in a quantum mechanical system, the greater the error on the momentum; and vice-versa.

In current and next-generation gravitational wave detectors, the quantum noise sources considered are \emph{shot} noise and \emph{radiation pressure} noise. The former effect arises from counting statistics in the signal readout methods used in the interferometers. Lasers do not produce entirely equally-spaced photons, and photodiodes have a finite sampling rate. Naturally, each sample period will not necessarily count the same number of photons as its neighbours, even when average power levels remain constant. This represents a noise source that is present at all frequencies. Radiation pressure noise, on the other hand, arises from the effect of momentum transfer between photons and reflective surfaces within interferometers, and this is illustrated by Equation\,\ref{eq:heisenburgtime}. Upon reflection, a photon imparts momentum to mirrors, and this moves the mirrors such that the next photons to hit them undergo a different optical path length to the initial photon. This, like shot noise, represents a noise source across all frequencies.

\subsubsection{Quantum noise at loss points}
Sources of loss in an interferometer introduce uncorrelated vacuum\textendash photons created and annihilated spontaneously\textendash which propagates through the interferometer to the output port where it is sensed by the readout as shot noise and radiation pressure noise. Open ports, present for example as non-unity reflectivity in a mirror, are sources of loss where vacuum fluctuations may enter an interferometer.

\subsubsection{\label{sec:sql}The Standard Quantum Limit}
The \emph{standard quantum limit} (\gls{SQL}) is a manifestation of the Heisenberg Uncertainty Principle in interferometry. It is the point at which the quadrature sum of shot and radiation pressure noise is minimised, and this occurs when the individual components are equal. For each laser power level there exists a single frequency at which the \gls{SQL} can be reached, since shot and radiation pressure noise scale inversely to one another with frequency. The \gls{SQL} is commonly plotted across a given frequency range, where it forms a line which can only be crossed with special \emph{sub-\gls{SQL}} techniques.

For a given classical interferometer, that is to say, an interferometer lacking special readout techniques aimed at reducing quantum noise sources, it can be shown \cite{Braginsky1996} that the spectral density follows the relation:
\begin{equation}
 S_h = \frac{h^{2}_{SQL}}{2} \left( \frac{1}{\kappa} + \kappa \right)
 \label{eq:classicalifospectrum}.
\end{equation}
In above equation, $\kappa$ is defined as the \emph{opto-mechanical coupling constant}:
\begin{equation}
 \kappa = \frac{I_0}{I_{SQL}} \frac{2 \gamma^4}{\Omega^2 \left( \gamma^2 +
\Omega^2 \right)},
 \label{eq:optomechanicalcoupling}
\end{equation}
with $I_0$ the laser power, $I_{SQL}$ the laser power required to reach the \gls{SQL}, $\gamma$ the cavity half-bandwidth, and $\Omega$ the gravitational wave frequency. The interferometer's opto-mechanical coupling constant arises from fluctuations in pressure exerted by light on the mirrors due to shot noise. $I_{SQL}$ can be itself defined as:
\begin{equation}
 I_{SQL} = \frac{m L^2 \gamma^4}{4 \omega_0},
\end{equation}
with $m$ the mirror mass, $L$ the arm length and $\omega_0$ the light's angular frequency.

Equation\,\ref{eq:classicalifospectrum} can be minimised by setting $I_0 = I_{SQL}$ in Equation~\ref{eq:optomechanicalcoupling}. This yields the \gls{SQL}:
\begin{equation}
 h^{2}_{SQL} = \frac{8 \hbar}{m \Omega^2 L^2}.
 \label{eq:strainsql}
\end{equation}

An important distinction to make here is that the \gls{SQL} is defined for \emph{uncorrelated} shot and radiation pressure noise. Techniques exist in theory and practice to reduce overall noise by producing correlations between the two effects such as squeezing and variational readout, so-called \emph{quantum non-demolition} interferometry.

Equation\,\ref{eq:strainsql} tells us that the \gls{SQL} can be reduced with heavier mirrors and longer arms. In gravitational wave observatories, however, the cost and technical requirements mean that other techniques to reduce the \gls{SQL} are being considered.

Various methods exist to achieve sensitivity in an interferometer beyond the \gls{SQL}. One such approach is to use \emph{DC readout} and \emph{squeezing}, a combination currently implemented in the detector GEO-HF \cite{Willke2006, Affeldt2014}.

\subsection{Other fundamental noise}

\subsubsection{Seismic noise}
\note{Microseism, rms motion, etc.}

\subsubsection{Gravity-gradient noise}
\note{See Hild 2011 for some basic info, and Jan's living review}

The changes in the density of the ground near the test masses created by seismic noise can couple to the gravitational wave channel via \emph{gravity-gradient} noise.

\note{Typically limits sensitivity at low frequencies, so only causes a problem to ET and LIGO Voyager...}

\note{Subtraction techniques are a possibility but still an early stage of research}

\subsection{Technical noise}

\subsubsection{Laser frequency noise}
A perfect laser provides output at a single, well defined frequency, or in other words, it has an infinitely narrow linewidth. In reality, such lasers do not exist and instead the output contains spectral impurities. As the laser wavelength is the ``metre stick'' by which we make measurements of length in interferometers, it is very important to ensure that the laser's wavelength, and therefore frequency, is well defined. A well designed laser will get an experimentalist part of the way there, but in most cavity experiments a frequency stabilisation control loop involving optics and electronics is necessary.

Apart from its linewidth, it is possible to represent a laser's frequency noise in terms of its noise spectral density, measurable via some heterodyne technique with a standard spectrum analyser. The effect of laser frequency noise is for a beat signal $V \left( t \right)$ to occur due to overlapping sinusoidal waves:
\begin{equation}
  V \left( t \right) = V_0 \left( t \right) \sin \left[ 2 \pi f_0 + \phi \left( t \right) \right],
\end{equation}
where $V_0 \left( t \right)$ is the beat amplitude and $\phi \left( t \right)$ is the phase difference at time $t$. We can rewrite this in terms of frequency $f \left( t \right)$,
\begin{equation}
  \begin{split}
    f \left( t \right) &= f_0 + \frac{1}{2 \pi} \frac{\text{d} \phi \left( t \right)}{\text{d} t} \\
                       &= f_0 + \Delta f \left( t \right),
  \end{split}
\end{equation}
with $\Delta f \left( t \right)$ the instantaneous frequency fluctuation. The power spectral density arises from the autocorrelation between a frequency fluctuation at time $t$ and another at time $t + \Delta t$, which can be expressed in the form:
\begin{equation}
  S_{\Delta f} \left( f \right) = 2 \int^{\infty}_{0} \langle \Delta f \left( t \right) \Delta f \left( t + \Delta t \right) \rangle \text{e}^{\left( -\text{i2}\pi f \Delta t \right) \text{d}\Delta t}
\end{equation}



\note{See Neil's thesis p14, https://arran.physics.gla.ac.uk/wp/speedmeter/2016/03/24/update-on-linear-cavity-work, Optics Communications 201 (2002) 391-397, and Applied Optics 49, issue 25, 4801-4807, 2010}

\subsubsection{Laser intensity noise}
\note{describe RIN, etc...}

\subsubsection{Johnson-Nyquist noise}
\note{Just quote the equation, explain it is due to thermal effects. Or, go into more detail about how it relates to the fluc. diss. theorem.}

\section{Control considerations}

\subsection{Root mean square amplitude}

\subsection{Spectral density}
In gravitational wave interferometry, a lot of thought and planning typically goes in to the design of the apparatus in order to minimise noise \emph{transients}\textemdash pulses of unwanted noise with finite energy\textemdash such as a door slam\footnote{And, of course, some forms of gravitational wave signal; though it is not the intention of experimentalists to minimise the effect of this particular source.}. Such transients have finite energy over a finite time, and can be entirely characterised\textemdash within measurement error\textemdash by a Fourier transform of the time series in which the event occurred. With sufficient isolation from unwanted transients, the remaining noise sources within the interferometer tend to arise from stationary, random processes, with energy approaching infinity as measurement time approaches infinity. In this circumstance, the Fourier transform of the underlying time-domain signal does not strictly exist. An alternative representation of a noise process is to represent the amount of work it performs per unit time: its power. The \emph{power spectral density} is a representation of the power present within each frequency of a signal in the steady state. This distribution can then be integrated back into a finite, steady state power.

Typically the time evolution of a particular signal is the only data an experimentalist has at their disposal in order to characterise a signal in terms of its spectral density. A true representation of the power spectral density requires, like the measurement of energy spectral density, infinite time. A compromise can be made, however, in order to estimate the power spectral density from a set of Fourier transforms of the time series. By averaging Fourier transforms of various subsets of the data $\left[ t, t + \Delta t \right]$, a power spectral density can be approximated for a particular frequency range, determined by the bounds established by the inverse of the longest and shortest subsets of the time series transformed. While this compromise is typically reasonable in most cases, and indeed essential for any finite measurement, it is unable to exclude the possibility that part of the power spectral density is formed from signal content in frequencies outwith the measured range. Longer time series measurements allow for more averaging of frequency bands, and thus a better approximation to the true power spectral density.

\subsection{Bandlimiting}
\note{Nyquist-Shannon sampling theorem, https://en.wikipedia.org/wiki/Bandlimiting}

\subsection{\label{sec:gain-phase-margin}Stable loops}
\note{Discuss briefly the stability requirement for the unity gain frequency - leads to oscillation. See Freise thesis for some useful content.}

\subsection{Spectral densities and root-mean-square figures}
\note{Explain where spectral densities come from, i.e. FFTs of time series}
\note{Explain calculation of rms noise from spectral density}

\subsection{Parametric instabilities}
\note{see DCC P1500163}
Although not strictly a noise source, parametric instabilities are nevertheless a significant control problem...

\section{Overview of current efforts}
* aLIGO, aVirgo, KAGRA, GEO
* Space based detectors: reference Section\,\ref{sec:gw-interferometry} that arm length of 750 km is impractical on ground, and how this is not impractical in space.
* Sensitivity curves for all?

\section{The future of gravitational wave interferometry}

\subsection{Planned upgrades and new facilities}
* Worldwide network of interferometric detectors
* Plans to build/upgrade more (KAGRA, ET, LIGO Voyager, LIGO CE, etc.)
* Space based detectors

PUT ET NOISE BUDGET HERE

LIGO CE citation: P1600143 (not yet published), also Dwyer et al, \cite{Dwyer2015}

\subsection{Surpassing the Standard Quantum Limit}
\note{Various techniques exist...}

\subsubsection{Squeezing}
The use of \emph{squeezing} is an attempt to instead introduce vacuum with \emph{correlated} noise. By choosing a suitable \emph{readout quadrature}, it is possible to avoid quantum noise impinging upon observables, instead moving the noise terms into the orthogonal, unobserved quadrature. Squeezing is particularly favourable in combination with DC readout, since it is not necessary to squeeze additional \gls{RF} sidebands in addition to the carrier light.

\subsubsection{Quantum non-demolition}
Give general overview of technique, then explain the Sagnac speedmeter in more detail.