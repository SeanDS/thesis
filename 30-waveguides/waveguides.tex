\chapter{Measuring transverse-to-longitudinal phase coupling in waveguide mirrors}
\label{c:waveguides}

\emph{The following chapter has been adapted from \emph{Upper limit to the transverse to longitudinal motion coupling of a waveguide mirror \cite{Leavey2015}}, published in Classical and Quantum Gravity in 2015. The material from the article has been expanded as appropriate for this thesis, but the results presented are identical.}

As shown in the introduction \note{need link}, waveguide mirrors have been shown to offer a reduction in thermal noise over a dielectric mirror offering equivalent reflectivity, at cryogenic temperatures.

\note{Show plot of aLIGO mirror thermal noise from, and discuss, the Heinert paper.}

\begin{figure}
  \begin{center}
    \includegraphics[width=\columnwidth]{30-waveguides/graphics/dynamic/coating-vs-grating-noise.pdf}
    \caption{Coating vs grating noise}
    \label{fig:coating-vs-grating-noise}
  \end{center}
\end{figure}

\begin{figure}
  \begin{center}
    \includegraphics[width=\columnwidth]{30-waveguides/graphics/dynamic/individual-factors.pdf}
    \caption{Individual factors}
    \label{fig:individual-factors}
  \end{center}
\end{figure}

\section{Transverse to Longitudinal Phase Coupling}
\note{Pillage the paper for a description of this effect.}

\section{Experiment}
\note{Describe experiment}
\subsection{The Pound-Drever-Hall Technique}
As discussed in \note{instrumentation chapter}, heterodyne locking...

\subsection{Suspended Michelson interferometers}
\note{Say why they didn't work.}

\subsection{Off-Axis Voice Coil}
\note{Put results from test of off-axis voice coil force measurements.}

\section{Analysis and Results}
\note{Bayesian stuff...}

\section{Outlook}

The work presented in this chapter shows that waveguide mirrors potentially offer a competitive alternative to dielectric mirrors in future gravitational wave detectors.

\note{Put plot of equivalent angular noise in aLIGO, showing that better suspensions are needed if it were to be used?}
