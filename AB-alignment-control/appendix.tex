\chapter{Alignment Control}
\label{a:alignment-control}
As shown throughout this thesis, the use of optical cavities can greatly enhance interferometric length measurement techniques. \note{Whilst} the initial alignment of most ``simple'' Michelson interferometers is relatively straightforward, the inclusion of optical cavities to the arms of the Michelson, and indeed other interferometer topologies, provides an additional challenge in obtaining and maintaining lock, both in the angular and longitudinal degrees of freedom.

Recent interferometric experiments in the \GLASGOWTENM have required a high degree of sensitivity and thus high finesse optical cavities \note{(cite Neil's thesis, paper)}. Test masses, suspended from multiple pendulum stages, are constrained in angular and longitudinal degrees of freedom by voice coil and magnet pairs (see, for example, \note{Figure X from Waveguide chapter}). As initial alignment of optics is a task undertaken infrequently, and not part of any closed control loop, it is safe to use actuators on stages above the test mass. This has two advantages:
\begin{itemize}
 \item the main control loops to produce fast corrections to the test mass positions during data acquisition need not share the same actuators as the initial alignment;
 \item and the dc alignment signal is filtered by the pendulum system so as not to contaminate the (typically audio frequency) measurement band.
\end{itemize}
As the task of alignment is an open loop system, it is not an efficient use of resources to dedicate high dynamic range ADC channels. It is therefore beneficial to develop a system for initial alignment using separate control software and hardware to the main system.

\section{Requirements}
Given the experience of lab colleagues from previous experiments, the requirements of such an alignment control system boil down to the following:
\begin{itemize}
 \item ability to align a given optic across the face of its neighbours;
 \item ability to control the alignment of all suspensions from a central location;
 \item relatively low cost compared to full dynamic range experimental ADC channels.
\end{itemize}
From previous experience with other control systems, it is also desirable to have the following features:
\begin{itemize}
 \item reproduceability of prior states;
 \item ability to control suspension alignment from any location.
\end{itemize}
Previous experience with commercial control equipment has led to some issues. Commercial hardware is typically at the mercy of proprietary software from the same vendor sometimes designed as an afterthought. Once a line of lab equipment is no longer popular it is also at the discretion of the manufacturer to maintain and support the equipment still in use. Due to these issues, along with the lists of requirements above, the seemingly obvious solution pointed towards the use of a distributed, networked system of individual suspension controllers using open hardware and software.

\section{Open Hardware and Software}
In recent years there has been a trend towards the production of low cost, open hardware. One such line of devices is the \emph{Arduino}, with hobby-level units available to the general public with open programmable interfaces, and a vibrant community of volunteers offering support.