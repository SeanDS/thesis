\documentclass[]{standalone}
\usepackage[siunitx,european]{circuitikz}
\usetikzlibrary{positioning}
\usepackage{libertine}
\renewcommand{\familydefault}{\sfdefault} % so Tikz uses the above font
\begin{document}
  \begin{tikzpicture}[scale=2]
    \draw[]
    % V+ voltage
    (0,2) node[above=1mm]{$\text{V}+$}
    % V- voltage
    (0,0) node[below=1mm]{$\text{V}-$}
	  % connect to BHD B
	  to[empty photodiode, o-*, n=bhdb] (0,1)
	  % connect to BHD A
	  to[empty photodiode, *-o, n=bhda] (0,2)
    % op-amp
    (1,1) node[op amp,anchor=-](opamp1){}
    % connect difference to op-amp 1 negative input
    (0,1) to[*-] (opamp1.-)
    % connect op-amp 1 positive input to ground
    (opamp1.+) to[-] ++(0,-0.5) node[rground]{}
    % give op-amp 1 nodes some names and markers
    (opamp1.-) node(node1)[circ]{}
    (opamp1.out) node(node2)[circ]{}
    % connect the op-amp 1 feedback resistor via coordinates perpendicular to the op-amp I/Os
    (node1) to[-] ++(0,0.5) coordinate (l1) to[R, l=$\text{R}_{\text{T}}$] (l1 -| opamp1.out) to (node2)
    % connection to second op-amp
    (node2) to[R, l=$\text{R}$] ++(1,0) node[op amp,anchor=-](opamp2){}
    % connect op-amp 2 positive input to ground
    (opamp2.+) to[R, l_=$\text{R}$] ++(0,-1) node[rground]{}
    % give op-amp 2 nodes some names and markers
    (opamp2.-) node(node3)[circ]{}
    (opamp2.out) node(node4)[circ]{}
    % connect the op-amp 2 feedback resistor via coordinates perpendicular to the op-amp I/Os
    (node3) to[-] ++(0,0.5) coordinate (l2) to[R, l=$100\text{R}$] (l2 -| opamp2.out) coordinate (r2) to (node4)
    % floating output
    (node4) to[short,-o] ++(0.5,0) node[right=1mm,align=center]{To DAQ}
    % label for op-amp 1
    (opamp1) node[label={[label distance=4mm]-75:$\text{N}_1$}]{}
    % label for op-amp 2
    (opamp2) node[label={[label distance=4mm]-75:$\text{N}_2$}]{}
    ;
  \end{tikzpicture}
\end{document}