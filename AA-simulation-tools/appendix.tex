\chapter{Simulation Tools}
How to build simulations in both tools...

\note{Mention about difference in homodyne angles between Optickle and Finesse - empirical formula on speedmeter labbook post from end of 2015/early 2016}

\section{Finesse}

\section{Optickle}
Optickle is a tool primarily designed to produce a series of matrices as outputs. As it is based in \MATLAB, it is possible to study the behaviour of the code to determine its capabilities and limitations. This section attempts to explain the basic operation of the primary feature that Optickle provides: the \code{tickle} function.

In order to simulate an interferometer, an optical environment is created into which optics, sensors and links between optics may be defined. The primary output of Optickle is a transfer matrix mapping \emph{drives} of each optic in the system to probes within that system. A drive is simply a degree of freedom of an optic, such as longitudinal motion of a mirror. Probes are the simulation equivalent of photodetectors, but can be lossless and can perform arbitrary RF demodulations. Another useful output is the quantum noise spectral density at a probe. Dividing the noise spectral density at a probe by a suitable linear combination of optical transfer functions--a degree of freedom of the interferometer---results in the probe's noise-to-signal ratio, or sensitivity, to that degree of freedom.

Whilst Optickle's code is extensive, the vast majority can be categorised into a few sets of routines:

\begin{itemize}
  \item defining parameters and matrices for various types of optic, including the field, reaction and noise transfer matrices;
  \item defining manipulations of sets of matrices to produce transfer functions and noise spectral densities for optics and sensors within the system.
\end{itemize}

Once an optical system has been defined by the user, a call to the \code{tickle} function results in a number of operations being performed:
\begin{itemize}
  \item the construction of matrices mapping the drives of an optic to the fields in the interferometer, and each field to each other field;
  \item the constructi
\end{itemize}

\subsection{Drive and Field Maps}
The drive-to-field matrix produced by \code{tickle} is the mapping between a \emph{drive} of an optic---a degree of freedom such as longitudinal motion of a mirror---to the corresponding fields in the interferometer. A \emph{field} is in this case the amplitude of the light between a particular pair of optics or sensors for a given wavelength. The individual transfer functions are themselves products of each optic's drive matrix (which converts a force input at the test mass to a phase change in the reflected and transmitted light, via the optic's mechanical transfer function) and a phase matrix defining the phase change light would have travelling between each optic in the system. The field-to-drive matrix is also produced in a similar way, but this time maps the interaction that light fields have with the optical drives--i.e. the effect of radiation pressure.

The field-to-field matrix is a similar mapping to the drive-to-field matrix, but this instead maps each field to each other field. As such, with this matrix the propagation of input light from lasers or vacuum injection to an arbitrary part of the interferometer can be calculated, albeit without the effect of radiation pressure.

The effect of field amplitudes propagating to other fields, optics modifying incident fields, and fields modifying optics can be collected together into a \emph{propagation matrix} $\mathbf{M}_{\text{AC}}$.

\subsection{Calculation of Field Amplitudes}
The field amplitudes within the interferometer $\vec{v}_{\text{AC}}$ can be determined by calculating the steady-state solution of the optical system to a given excitation. An excitation is, for instance, the injection of light. The field amplitudes within the interferometer are of course determined by the excitation of the interferometer by external light injection, but in general they are also influenced by the signal sidebands produced by the modulation of optics within the interferometer at non-zero frequencies. Furthermore, the presence of optics will alter the injected excitation. The field amplitudes within the interferometer therefore depend not only on the excitation but also on the existing field amplitudes, analogous to feedback systems. In the initial state these fields are zero and so the interferometer's field amplitude vector is simply equal to the excitation vector, i.e. $\vec{v}_{\text{AC}} = \vec{v}_{\text{exc}}$. The stored light will increase until eventually the injected excitation is equal to the light power lost in the interferometer. Once this condition is reached the interferometer is in its steady-state.

The steady state condition can be solved numerically using matrix inversion. As described above, the field amplitudes depend not only on the input but also on the field amplitudes themselves, i.e.
\begin{equation}
  \vec{v}_{\text{AC}} = \mathbf{M}_{\text{AC}} \vec{v}_{\text{AC}} + \vec{v}_{\text{exc}},
\end{equation}
where $\mathbf{M}_{\text{AC}}$ is the phase matrix specified earlier. This equation can be solved as such:
\begin{equation}
  \vec{v}_{\text{AC}} = \frac{\vec{v}_{\text{exc}}}{1 - \mathbf{M}_{\text{AC}}}.
\end{equation}
Since $\mathbf{M}_{\text{AC}}$ is a matrix and $\vec{v}_{\text{AC}}$ and $\vec{v}_{\text{exc}}$ are vectors, the problem can be represented as the matrix equation:
\begin{equation}
  \vec{v}_{\text{AC}} = \left( \mathbb{I} - \mathbf{M}_{\text{AC}} \right)^{-1} \vec{v}_{\text{exc}},
\end{equation}
where $\mathbb{I}$ is the identity matrix. The calculation of the field amplitudes in the interferometer therefore becomes a task of finding the inverse of $\mathbb{I} - \mathbf{M}_{\text{AC}}$.

\subsection{Probe Signals}
With the steady-state field amplitudes, the signals produced by the interferometer can be determined with the application of a \emph{probe matrix} $\mathbf{M}_{\text{probe}}$ which maps the fields in the interferometer to probes contained within the interferometer. Since the fields amplitudes are determined for every wavelength under consideration, it is possible to calculate the signals that would appear on photodetector circuits implementing RF demodulation. The probe matrix contains complex amplitudes to transform the fields at the location of the probe by the required amount given the demodulation frequencies and phase angles. The probe signals are therefore defined as:
\begin{equation}
  \vec{v}_{\text{probe}} = \mathbf{M}_{\text{probe}} \left( \mathbb{I} - \mathbf{M}_{\text{AC}} \right)^{-1} \vec{v}_{\text{exc}}.
\end{equation}

\subsection{Calculation of Transfer Functions}
The operation of calculating the probe signals from the field amplitudes in the interferometer can be repeated for arbitrary frequencies of excitation to produce a three-dimensional drive-to-probe transfer matrix. This represents the transfer function from each optic's degree of freedom to each probe. As such, the signal from a particular \emph{interferometer} degree of freedom can be constructed via a linear combination of optic degrees of freedom transfer functions. The differential arm degree of freedom transfer function for a \MI to its asymmetric port, for instance, can be calculated by extracting the transfer function of each end test mass to a probe situated at the asymmetric port and taking the difference of the two \note{, as shown in Equation x in Section y.}

\subsection{Probe Quantum Noise}
...

\section{Reflection Phase Convention}
\label{a:reflection-phase}
\note{Difference between Optickle and Finesse sign conventions for transmission and reflection. See footnote 1 on p2 of T1100110 for more details.}
