\chapter{Control of the Low Frequency Einstein Telescope Detector}
\label{c:et-lf-control}

  * Build upon the PDH stuff laid out in Chapter 3: describe the evolution of the ET-LF interferometer in order to control it: the need for a Schnupp asymmetry (to provide the equivalent of a PDH-style controller), DARM offset, etc...
  
  See documentation from ET-LF repository, and aLIGO design study
  
  Also see scribblings / emails from Ken etc. around the time of the Florence meeting where we discussed the low frequency sensing problem
  
  See noise budget from before Florence

  * The Einstein telescope facility: ET-LF and ET-HF, the xylophone, etc.
    * Unprecedented LF sensitivity: opens up universe
    * Pushing warm technology to the limit with ET-HF
    * Challenges: control at low frequencies with such detuning
  * ET-LF layout
  * Predicted sensitivity vs Optickle calculated sensitivity
  * ISC stuff...
    * Consider only plane waves - justification: only want to control lengths for now. Angles are not considered a challenging aspect as nothing much has changed since aLIGO.
    * Optical response: with and without mechanical TFs - show the difference it makes to the response at low frequencies, and why it is necessary to turn them off in the case when you're computing a sensing matrix
    * From sensing matrix to control matrix (control loops, locking order, bandwidth, etc.)
    * Dynamic range of sensors and actuators in ET-LF
      * Problem with dynamic range, need local control or SPI or similar

\section{Outlook}