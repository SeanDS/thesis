\chapter{The Sagnac-Speedmeter Experiment}
\label{c:speedmeter-intro}

\section{Concept}

\section{Implementation}

Some technical challenges in the implementation of the speedmeter are discussed in this chapter. Certain topics involve substantially more scientific endeavour and are thus presented as discrete chapters: the control of the primary degree of freedom of the speedmeter, presented in Chapter X; and a proof-of-principle experiment to demonstrate a new type of actuator, presented in Chapter Y.

\subsection{Wiring}
\note{Avoidance of ground loops, interfacing with CDS, avoiding plugging the wrong things in (why we use D-sub-9 and D-sub-15), in-vacuum wiring: why it needs careful thought, and what we did (octopus cables, etc.).}
\note{Avoidance of ground loops}
\note{Auxiliary coil driver subrack wiring design / motivation / assembly}
\note{In-vacuum cabling design, strain relief housing}
\note{Backplane board design - talk about rationale, show Eagle diagrams, etc.}