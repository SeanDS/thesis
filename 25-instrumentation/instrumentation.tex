\chapter{Instrumentation}
\label{c:instrumentation}

\section{Development of the Gravitational Wave Detector}
The field of experimental gravitational wave detection began with Joseph Weber's studies in the 1960s \cite{Weber1960}. His \emph{Weber bar} was developed to act as a strain meter, with piezoelectric sensors placed on the surface of an aluminium cylinder to convert changes in length into electrical signals. Whilst the expected change in length of such a cylinder from gravitational radiation would in most cases be tiny---\checkme{of the order \SI{1e-22}{\meter} for a particularly loud source}---the resonant frequency of the cylinder, typically in the kilohertz range, acts to enhance the amplitude of the length change. The sensitivity of such a bar as a function of frequency is determined in part by its quality factor (Q), with a necessary trade-off being made between peak sensitivity (high Q) and detection bandwidth (low Q). As sources of gravitational radiation are almost universally weak, the only reasonable hope of making such a detection is to choose a high Q material and hope for a favourable signal frequency.

The original resonant bar detectors were evolved over time to become cryogenic, to decrease the effect of thermal noise; and spherical, to maximise the test mass's Q. Despite such improvements the peak sensitivity of state-of-the-art resonant bar detectors was surpassed by interferometric gravitational wave detectors in 2003 \cite{Pitkin2011} after it was shown that second generation detectors would offer superior sensitivity across a much wider bandwidth\footnote{Interestingly, a Weber bar had a particularly high profile opportunity to make the first detection. One was contained in the scientific payload of Apollo 17 with the intention to observe gravitational radiation from the low seismic noise environment of the Moon. Unfortunately a manufacturing error led to a failure in the experiment.} \cite{Harry2002a}. The interferometer was first suggested as a means for gravitational wave detection shortly after the introduction of the Weber bar\footnote{The first known example being by Gertsenshtein and Pustovoit in the Soviet \emph{Journal of Experimental and Theoretical Physics} in 1962.}, but efforts to build prototypes and understand the significant sources of noise only gained momentum in the 1970s (see for example Moss \etal \cite{Moss1971} from 1971 or Weiss \cite{Weiss1972} from 1972).

% Search 'Lunar Surface Gravimeter' for Moon bar detector details

\subsection{The Gravitational Wave Interferometer}
Over the course of the 1980s the \MI (see Figure\,\ref{fig:mi}) was developed into a very respectable gravitational wave detection apparatus.

\begin{figure}
  \begin{center}
    \begin{subfigure}{.3\textwidth}
      \includegraphics[width=\columnwidth]{20-detection/graphics/dynamic/michelson.pdf}
      \caption{Simple \MI}
      \label{fig:mi}
    \end{subfigure}
    \hfill
    \begin{subfigure}{.3\textwidth}
      \includegraphics[width=\columnwidth]{20-detection/graphics/dynamic/fabry-perot-michelson.pdf}
      \caption{\FPMI}
      \label{fig:fpmi}
    \end{subfigure}
    \hfill
    \begin{subfigure}{.3\textwidth}
      \includegraphics[width=\columnwidth]{20-detection/graphics/dynamic/dual-recycled-fabry-perot-michelson.pdf}
      \caption{\DRFPMI}
      \label{fig:drfpmi}
    \end{subfigure}
    \caption[The evolution of the gravitational wave detector]{The evolution of the gravitational wave detector. Figure\,\ref{fig:mi} shows the simple \MI used since the famous Michelson and Morley experiments of the 1880s, and proposed for gravitational wave detection in early literature. Figure\,\ref{fig:fpmi} shows a \MI with the addition of \FP arm cavities to enhance sensitivity. Figure\,\ref{fig:drfpmi} shows a \FPMI with the addition of recycling mirrors.}
  \end{center}
\end{figure}

\subsection{Pulsar Timing}
\note{Short note on pulsar timing arrays.}

\section{Limiting noise sources in future detectors}
Quantum noise, thermal noise (since if you push QN low enough, you run into thermal), possibly mention others (Newtonian noise - why ET will be under ground, etc.)

\subsection{Measurement noise in gravitational wave interferometry}
Gravitational wave interferometry is ultimately a matter of signals and noise. The term ``signal'' simply refers to a wanted pattern of oscillations representing a particular variable of interest. ``Noise'', on the other hand, is unwanted oscillations that appear at the signal measurement device. The measurement of any signal necessarily entails the measurement of some noise. Part of scientists' jobs, both experimental and theoretical, is to design experiments and apparatus in such a way as to maximise the ratio of signal to noise in the region of interest.

\subsubsection{Spectral density}
In gravitational wave interferometry, a lot of thought and planning typically goes in to the design of the apparatus in order to minimise noise \emph{transients}\textemdash pulses of unwanted noise with finite energy\textemdash such as a door slam\footnote{And, of course, some forms of gravitational wave signal; though it is not the intention of experimentalists to minimise the effect of this particular source.}. Such transients have finite energy over a finite time, and can be entirely characterised\textemdash within measurement error\textemdash by a Fourier transform of the time series in which the event occurred. With sufficient isolation from unwanted transients, the remaining noise sources within the interferometer tend to arise from stationary, random processes, with energy approaching infinity as measurement time approaches infinity. In this circumstance, the Fourier transform of the underlying time-domain signal does not strictly exist. An alternative representation of a noise process is to represent the amount of work it performs per unit time: its power. The \emph{power spectral density} is a representation of power present within each frequency of a signal in the steady state. This distribution can then be integrated back into a finite, steady state power.

Typically the time evolution of a particular signal is the only data an experimentalist has at their disposal in order to characterise a signal in terms of its spectral density. A true representation of the power spectral density requires, like the measurement of energy spectral density, infinite time. A compromise can be made, however, in order to estimate the power spectral density from a set of Fourier transforms of the time series. By averaging Fourier transforms of various subsets of the data $\left[ t, t + \Delta t \right]$, a power spectral density can be approximated for a particular frequency range, determined by the bounds established by the inverse of the longest and shortest subsets of the time series transformed. While this compromise is typically reasonable in most cases, and indeed essential for any finite measurement, it is unable to exclude the possibility that part of the power spectral density is formed from signal content in frequencies outwith the measured range. Longer time series measurements allow for more averaging of frequency bands, and thus a better approximation to the true power spectral density.

\subsubsection{Bandlimiting}
\note{Nyquist-Shannon sampling theorem, https://en.wikipedia.org/wiki/Bandlimiting}

\subsection{Technical Noise}

\subsubsection{Laser Frequency Noise}
A perfect laser provides output at a single, well defined frequency, or in other words, it has an infinitely narrow linewidth. In reality, such lasers do not exist and instead the output contains spectral impurities. As the laser wavelength is the ``metre stick'' by which we make measurements of length in interferometers, it is very important to ensure that the laser's wavelength, and therefore frequency, is well defined. A well designed laser will get an experimentalist part of the way there, but in most cavity experiments a frequency stabilisation control loop involving optics and electronics is necessary.

Apart from its linewidth, it is possible to represent a laser's frequency noise in terms of its noise spectral density, measurable via some heterodyne technique with a standard spectrum analyser. The effect of laser frequency noise is for a beat signal $V \left( t \right)$ to occur due to overlapping sinusoidal waves:
\begin{equation}
  V \left( t \right) = V_0 \left( t \right) \sin \left[ 2 \pi f_0 + \phi \left( t \right) \right],
\end{equation}
where $V_0 \left( t \right)$ is the beat amplitude and $\phi \left( t \right)$ is the phase difference at time $t$. We can rewrite this in terms of frequency $f \left( t \right)$,
\begin{equation}
  \begin{split}
    f \left( t \right) &= f_0 + \frac{1}{2 \pi} \frac{\text{d} \phi \left( t \right)}{\text{d} t} \\
                       &= f_0 + \Delta f \left( t \right),
  \end{split}
\end{equation}
with $\Delta f \left( t \right)$ the instantaneous frequency fluctuation. The power spectral density arises from the autocorrelation between a frequency fluctuation at time $t$ and another at time $t + \Delta t$, which can be expressed in the form:
\begin{equation}
  S_{\Delta f} \left( f \right) = 2 \int^{\infty}_{0} \langle \Delta f \left( t \right) \Delta f \left( t + \Delta t \right) \rangle \text{e}^{\left( -\text{i2}\pi f \Delta t \right) \text{d}\Delta t}
\end{equation}



\note{See https://arran.physics.gla.ac.uk/wp/speedmeter/2016/03/24/update-on-linear-cavity-work, Optics Communications 201 (2002) 391-397, and Applied Optics 49, issue 25, 4801-4807, 2010}

\subsubsection{Laser Intensity Noise}
\note{describe RIN, etc...}

\subsection{Thermal Noise}

\subsection{Quantum Noise}

\section{Overview of Current Efforts}
* Worldwide network of interferometric detectors
* Plans to build/upgrade more (KAGRA, ET, LIGO Voyager, LIGO CE, etc.)
* Space based detectors

\section{Quantum Non-Demolition}
Give general overview of technique, then explain the Sagnac speedmeter in more detail.

\begin{figure}
  \begin{center}
    \includegraphics[width=\columnwidth]{20-detection/graphics/dynamic/sideband-structure.pdf}
    \caption{Sideband structure}
    \label{fig:sideband-structure}
  \end{center}
\end{figure}