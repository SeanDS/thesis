\chapter{Instrumentation}
\label{c:instrumentation}

\section{Development of the Gravitational Wave Detector}
The field of experimental gravitational wave detection began with Joseph Weber's studies in the 1960s \cite{Weber1960}. His \emph{Weber bar} was developed to act as a strain meter, with piezoelectric sensors placed on the surface of an aluminium cylinder to convert changes in length into electrical signals. Whilst the expected change in length of such a cylinder from gravitational radiation would in most cases be tiny---\checkme{of the order \SI{1e-22}{\meter} for a particularly loud source}---the resonant frequency of the cylinder, typically in the kilohertz range, acts to enhance the amplitude of the length change. The sensitivity of such a bar as a function of frequency is determined in part by its quality factor (Q), with a necessary trade-off being made between peak sensitivity (high Q) and detection bandwidth (low Q). As sources of gravitational radiation are almost universally weak, the only reasonable hope of making such a detection is to choose a high Q material and hope for a favourable signal frequency.

The original resonant bar detectors were evolved over time to become cryogenic, to decrease the effect of thermal noise; and spherical, to maximise the test mass's Q. Despite such improvements the peak sensitivity of state-of-the-art resonant bar detectors was surpassed by interferometric gravitational wave detectors in 2003 \cite{Pitkin2011} after it was shown that second generation detectors would offer superior sensitivity across a much wider bandwidth\footnote{Interestingly, a Weber bar had a particularly high profile opportunity to make the first detection. One was contained in the scientific payload of Apollo 17 with the intention to observe gravitational radiation from the low seismic noise environment of the Moon. Unfortunately a manufacturing error led to a failure in the experiment.} \cite{Harry2002a}. The interferometer was first suggested as a means for gravitational wave detection shortly after the introduction of the Weber bar\footnote{The first known example being by Gertsenshtein and Pustovoit in the Soviet \emph{Journal of Experimental and Theoretical Physics} in 1962.}, but efforts to build prototypes and understand the significant sources of noise only gained momentum in the 1970s (see for example Moss \etal \cite{Moss1971} from 1971 or Weiss \cite{Weiss1972} from 1972).

% Search 'Lunar Surface Gravimeter' for Moon bar detector details

\subsection{The Gravitational Wave Interferometer}
Over the course of the 1980s the \MI (see Figure\,\ref{fig:mi}) was developed into a very respectable gravitational wave detection apparatus.

\begin{figure}
  \begin{center}
    \begin{subfigure}{.3\textwidth}
      \includegraphics[width=\columnwidth]{20-detection/graphics/dynamic/michelson.pdf}
      \caption{Simple \MI}
      \label{fig:mi}
    \end{subfigure}
    \hfill
    \begin{subfigure}{.3\textwidth}
      \includegraphics[width=\columnwidth]{20-detection/graphics/dynamic/fabry-perot-michelson.pdf}
      \caption{\FPMI}
      \label{fig:fpmi}
    \end{subfigure}
    \hfill
    \begin{subfigure}{.3\textwidth}
      \includegraphics[width=\columnwidth]{20-detection/graphics/dynamic/dual-recycled-fabry-perot-michelson.pdf}
      \caption{\DRFPMI}
      \label{fig:drfpmi}
    \end{subfigure}
    \caption[The evolution of the gravitational wave detector]{The evolution of the gravitational wave detector. Figure\,\ref{fig:mi} shows the simple \MI used since the famous Michelson and Morley experiments of the 1880s, and proposed for gravitational wave detection in early literature. Figure\,\ref{fig:fpmi} shows a \MI with the addition of \FP arm cavities to enhance sensitivity. Figure\,\ref{fig:drfpmi} shows a \FPMI with the addition of recycling mirrors.}
  \end{center}
\end{figure}

\subsection{Pulsar Timing}
\note{Short note on pulsar timing arrays.}

\section{Limiting noise sources in future detectors}
Quantum noise, thermal noise (since if you push QN low enough, you run into thermal), possibly mention others (Newtonian noise - why ET will be under ground, etc.)

\subsection{Thermal Noise}

\subsection{Quantum Noise}

\section{Overview of Current Efforts}
* Worldwide network of interferometric detectors
* Plans to build/upgrade more (KAGRA, ET, LIGO Voyager, LIGO CE, etc.)
* Space based detectors

\section{Quantum Non-Demolition}
Give general overview of technique, then explain the Sagnac speedmeter in more detail.

\begin{figure}
  \begin{center}
    \includegraphics[width=\columnwidth]{20-detection/graphics/dynamic/sideband-structure.pdf}
    \caption{Sideband structure}
    \label{fig:sideband-structure}
  \end{center}
\end{figure}