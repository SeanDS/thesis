\begin{center} {\bf Abstract}\end{center}

The first direct detection of gravitational waves last year was the beginning of a new field of astronomy. While we have already learned a great deal from the signals sensed by the LIGO interferometers in their first observation run, research is already underway to improve upon the sensitivity of the state of the art detectors. Novel mirror designs, new interferometer topologies and larger, more advanced detectors are all being considered as future improvements, and these topics form the focus of this thesis.

A reduction in the thermal noise arising from the mirrors within gravitational wave detectors will enhance sensitivity near their most sensitive frequencies, and this can potentially be achieved through the use of \emph{waveguide} mirrors employing gratings. It has been shown that the thermal noise is reduced in waveguide mirrors compared to standard dielectric mirrors whilst retaining the required reflectivity, but an open question regarding their suitability remains due to the potential for increased technical noise coupling created by the substructure. We place an upper limit on this coupling with a suspended cavity experiment, showing that this approach to the design of grating mirrors has promise.

While the use of higher classical laser input initially increases interferometer sensitivity, eventually the \MI{} topology employed in existing detectors reaches the \emph{standard quantum limit} preventing further enhancement. Efforts are being made to test the suitability of so-called \emph{quantum non-demolition} (QND) technologies able to surpass this limit, one of which involves the use of a new interferometer topology altogether. An experiment to demonstrate a reduction in quantum radiation pressure noise in a QND-compatible \SSM{} topology is underway in Glasgow, and we introduce novel techniques to control this suspended, audio-band interferometer to inform the technical design of future detectors wishing to measure beyond the standard quantum limit. In particular, the problem of controlling the interferometer at low frequencies is discussed. Due to the nature of the speed meter topology, the response of the interferometer vanishes towards zero frequency, while the interferometer's noise does not. This creates a control problem at low frequencies where test mass perturbations arising from, for example, seismic and electronic noise, can lead to loss of interferometer sensitivity over the course of minutes to hours. We present a solution involving the blending of signals from different readout ports of the interferometer, facilitating measurements with almost arbitrary integration times.

The longer, larger \ET{} facility planned as part of the next generation of detectors will push the \MI{} topology to the limit. The low frequency interferometer will utilise optomechanical interactions to enhance its sensitivity at low frequencies, and the control problems associated with this technique have not been investigated in detail. Following the approach taken in the current generation of detectors we show that the interferometer can be controlled without adversely affecting its sensitivity to gravitational waves, paving the way for a future technical design.