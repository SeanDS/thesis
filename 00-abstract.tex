\vspace*{1.75in}
\begin{center} {\bf Abstract}\end{center}

Until last year, the two methods with which humanity could observe the universe were through the use of the electromagnetic spectrum and via particle interactions. With the first direct detection of gravitational waves by the LIGO Scientific Collaboration\textemdash the missing experimental proof in Einstein's General Relativity\textemdash a new, \emph{gravitational} window has been opened into the universe. Events previously invisible to us, such as the coalescence of binary black hole systems, are now aspects of our universe that we can study with gravitational wave detectors.

This is only the beginning. Work now focuses on the improvement in sensitivity of the current detectors, which are already encountering fundamental limits difficult to surpass without significant research and development. Novel mirror designs, techniques for the control of complex interferometers and new topologies to twist the quantum nature of light are developments aimed at improving sensitivity in future detectors, and these topics form the primary focus of this thesis.

\note{This must be much more detailed regarding the work}