\chapter{Control of the low frequency Einstein Telescope detector}
\label{c:et-lf-control}

  * Build upon the PDH stuff laid out in Chapter 3: describe the evolution of the ET-LF interferometer in order to control it: the need for a Schnupp asymmetry (to provide the equivalent of a PDH-style controller), DARM offset, etc...
  
  See documentation from ET-LF repository, and aLIGO design study
  
  Also see scribblings / emails from Ken etc. around the time of the Florence meeting where we discussed the low frequency sensing problem
  
  See noise budget from before Florence

  * The Einstein telescope facility: ET-LF and ET-HF, the xylophone, etc.
    * Unprecedented LF sensitivity: opens up universe
    * Pushing warm technology to the limit with ET-HF
    * Challenges: control at low frequencies with such detuning
  * ET-LF layout
  * Predicted sensitivity vs Optickle calculated sensitivity
  * ISC stuff...
    * Consider only plane waves - justification: only want to control lengths for now. Angles are not considered a challenging aspect as nothing much has changed since aLIGO.
    * Optical response: with and without mechanical TFs - show the difference it makes to the response at low frequencies, and why it is necessary to turn them off in the case when you're computing a sensing matrix
    * From sensing matrix to control matrix (control loops, locking order, bandwidth, etc.)
    * Dynamic range of sensors and actuators in ET-LF
      * Problem with dynamic range, need local control or SPI or similar
  
  See Bryan's paper for sidebands on sidebands in the context of the Caltech prototype for aLIGO, to describe the use of beats between sidebands: https://iopscience.iop.org/article/10.1088/0264-9381/23/18/010/pdf

\section{The Einstein Telescope Facility}

\subsection{ET-LF}

\subsection{ET-HF}

\section{Control challenges}
\note{Focus on ET-LF, but discuss challenges with ET-HF too (parametric instabilities, etc.) - basically stress that there's a lot of work to be done before the technical design}

\note{Discuss that we need to model both tuned and detuned ET-LF states, as the locking will have to start with tuned}

\section{Modelling ET-LF}
\note{See email sent to Andreas about outcomes of ASPERA. Basically, we had to model the interferometer controls and higher order modes, so we did it simultaneously with two tools. Discuss how this was achieved, via a shared parameter set, etc.}

\section{Conceptual control scheme}
\note{Discuss how we chose to follow Advanced LIGO as much as possible, but had to define some as-yet undefined parameters first}

\note{Put the control matrices here: show the slopes of the error signals for tuned and detuned operation}

\section{Control noise}
\note{Highlight the dynamic range problem with photodetectors sensing the low frequency motion (poster from Florence), and introduce the basic control loop developed with SimulinkNb - discuss the noises included, the assumptions made, etc. Finish with the list from the poster of what has to be done: further modelling of suspension SPIs, or better sensors, or both, etc...}

\section{Outlook}