\chapter{\label{a:simulation-tools}Interferometer modelling}
This appendix describes the process of modelling an interferometer in terms of its optical response and noise and introduces the two numerical simulation tools used throughout this thesis, Finesse and Optickle, introduced in Section\,\ref{sec:numerical-software}. Section\,\ref{sec:ifo-modelling} gives an introduction into how these tools work, and Section\,\ref{sec:optickle-field-tfs} describes modifications made to Optickle in order to calculate a particular type of noise amplitude.

\section{\label{sec:numerical-software}Software}
\subsection{\label{sec:finesse-sim}Finesse}
Finesse is an extensive tool for simulating complex optical environments in terms of their transverse electromagnetic (\gls{TEM}) modes. It was developed originally for use in \GEO{} \cite{Freise2004} but has since been used for checks and tests in the design and commissioning of the advanced detectors \cite{Mueller2015, Kumeta2015} and a variety of other experiments\footnote{A list of known uses of Finesse for scientific research is maintained at \url{http://www.gwoptics.org/finesse/impact.php}.}. Finesse's syntax allows for a number of different outputs to be generated for given excitations, for instance the transfer function from a set of optics to a set of sensors given an arbitrary excitation. Finesse's strengths are its numerous supported optical components, speed of computation, and support for arbitrary numbers of paraxial modes.

\subsection{\label{sec:optickle-sim}Optickle}
Optickle was originally created for the design of \ALIGO{} but has been extended to support a feature set capable of simulating arbitrary topologies. It is implemented in \MATLAB{} and is primarily intended to simulate plane wave behaviour of interferometers, however it also contains extensions to its code base for first order modes necessary to simulate misalignment effects. The output from Optickle is a series of matrices representing the transfer functions from optical degrees of freedom to sensors placed within the system, and the quantum noise upon those sensors.

\section{\label{sec:ifo-modelling}Modelling an interferometer}
The analytical calculation of the behaviour of interferometers beyond all but the most trivial examples is a complicated process and has to be performed with a particular configuration in mind. For example, adding or removing an optic from an analytical model of an interferometer may involve the addition of many new terms to the equations describing the main readout signals. Models for \DRFPMI{}s have been available for a number of years \cite{Strain2003, Mueller2003, Mason2003} but cannot be easily modified to account for optics beyond the ones considered in the model and the equations representing the readout signals have only been developed for the most important ports of the interferometer. In order to be able to calculate the signals present at any optic or probe within an interferometer, the most straightforward approach for an experimentalist is to use a numerical simulation tool.

Both Finesse and Optickle use broadly the same approach to compute results based on a technique used to model electronic circuits as for example with \gls{LISO}, where the interferometer is described by an often large set of simultaneous equations. The tools take advantage of decades of development of numerical linear algebra tools to quickly solve these systems, with the results typically available in seconds.

The primary output from the tools is the calculation of field amplitudes or powers at ports of the interferometer at its operating point given a set of connected optics. In order to calculate these signals the simulation must map the effect that each optic's motion has onto the light fields within the interferometer, propagate these fields to each output and then calculate the corresponding electrical signals. These processes are described in more detail below in the context of Optickle, but the approach taken in Finesse is similar.

\subsection{Optics}
An \emph{optic} refers to any component within the interferometer which has an effect on the light's amplitude, phase or frequency. Apart from mirrors and beam splitters, components such as lasers, \glspl{EOM} and Faraday isolators can all be handled in the same way via matrices which translate inputs to outputs. Such \emph{transfer matrices} define the physical process taking place as the light propagates through the component. For example, the transfer matrix of a simple mirror can be defined as \cite{Freise2010}:
\begin{equation}
  M =
  \begin{bmatrix}
    it & r \\
    r & it
  \end{bmatrix},
\end{equation}
where $r$ and $t$ represent the amplitude reflectivity and transmissivity of the mirror, with the condition $r^2 + t^2 = 1$, assuming no loss.

Using the first mirror in Figure\,\ref{fig:fabry-perot} as an example, the inputs $a_0$ and $a_5$ map to outputs $a_1$ and $a_6$ as:
\begin{equation}
  \begin{bmatrix}
    a_1 \\
    a_6
  \end{bmatrix}
  =
  M
  \begin{bmatrix}
    a_0 \\
    a_5
  \end{bmatrix}
  ,
\end{equation}
which can be re-expressed as individual transfer functions identical to those shown in Equation\,\ref{eq:fabry-perot-coefficients-1}:
\begin{equation}
  \begin{split}
    a_1 &= it a_0 + r a_5 \\
    a_6 &= r a_0 + it a_5.
  \end{split}
\end{equation}

\subsubsection{\label{a:reflection-phase}Reflection phase convention}
To conserve energy, a phase change must be applied to either the reflected or transmitted fields at an optical surface. In Finesse, the convention is such that this phase is added to the transmitted beam in the form of an imaginary coefficient. Optickle uses the convention that the reflection coefficient from the front of a mirror is $-r$, and from the rear it is $r$, and transmission is always $t$. Both conventions are valid, with Finesse's definition closer to the real effect that dielectric coatings would have on the transmitted light, and Optickle's more consistent with an infinitely thin optic.

This difference has no effect on the fidelity of the simulations because only the relative phase between reflected and transmitted fields is important for the calculation of transfer functions, noise and control signals. One practical difference is that in Finesse one arm of a \MI{} must be tuned by \SI{90}{\degree} with respect to the other to achieve cancellation of the carrier at the output port of the beam splitter. In Optickle, due to the reflection phase convention this happens without special tuning.

\subsection{Propagation}
Propagation through free space is in general defined by Equation\,\ref{eq:field-amplitude}, and in matrix form it is:
\begin{equation}
  M = \text{e}^{-ikD}
  \begin{bmatrix}
    1 & 0 \\
    0 & 1
  \end{bmatrix}
  ,
\end{equation}
for wave vector $k$ and distance $D$. In Finesse and Optickle, however, this behaviour is slightly different. Interferometers have path lengths of many metres, whereas the wavelength of the light being modelled is typically microscopic. For a cavity to be resonant its mirrors must be separated by an integer number of half-wavelengths, and so the \SI{4}{\kilo\meter} \FP{} cavities of \ALIGO{} would actually have to be defined with length $\num{3759398497}\lambda = \SI{4000.00000081}{\meter}$ given its $\SI{1064}{\nano\meter}$ wavelength. To make the creation of interferometer configurations easier, the simulation tools instead take the macroscopic propagation length and round it to the nearest integer number of wavelengths for the carrier, whereupon the phase difference from propagation is zero\footnote{In \emph{most} cases. This is different for certain modes in which Finesse can run, controlled by the \emph{phase} command.}. To model the effects of non-zero phase propagation, such as a detuned cavity, the optics have additional phase tuning factors present within their transfer matrices. By separating these macroscopic and microscopic phase effects, issues with numerical precision can be avoided.

\subsection{Fields}
As spaces are defined as zero-phase propagation, the light between optics can be modelled with a single amplitude for each mode within the interferometer. In addition to the carrier, any signal sideband or control sideband present within the interferometer is considered a mode, as well as vacuum fields entering at points of loss as described in Section\,\ref{sec:noise-via-loss}, and so there can be many tens of figures representing the light between any two optics. These can be considered as the interferometer's \emph{degrees of freedom}, and the propagation of each field through the interferometer can be modelled individually. A point in the interferometer represented by a field is termed a \emph{field evaluation point}.

\subsection{Drive and field maps}
The calculation of signals at photodetectors requires the calculation of the field amplitudes within the interferometer, which can be determined by calculating the steady-state solution of the optical system defined within a matrix mapping each field to each other field.

The process of calculating this \emph{interferometer} matrix starts with the creation of the \emph{field to field} matrix, which maps the transfer function between each of the fields within the interferometer with stationary optics without the presence of signal or control sidebands. This matrix allows the propagation of input light from lasers or vacuum injection to an arbitrary part of the interferometer to be calculated.

The effects of the mechanical degrees of freedom of a mirror or the electrical degrees of freedom of for instance an \gls{EOM} on the light can be described by a \emph{drive to field} map. Encompassed within this map are the amplitude and phase effects upon the carrier and sidebands cause by for example the motion of a mirror in the longitudinal direction. Similarly, the \emph{field to drive} map, encompassing the effect of fields on the mechanical degrees of freedom of optics, allows the effect of radiation pressure to be handled properly.

Once the various maps between the light inputs, the mechanical and electrical drives and the optics have been calculated, they can be combined together in the form of a block-diagonal matrix $\mathbf{M}_{\text{AC}}$ representing the transfer functions between the carrier, signal and control sidebands at each field evaluation point to each other field evaluation point.

\subsection{Calculation of field amplitudes}
The field amplitudes within the interferometer are of course determined by the excitation of the interferometer by external light injection, but in general they are also influenced by the signal sidebands produced by the modulation of optics within the interferometer at non-zero frequencies. The field amplitudes within the interferometer therefore depend not only on the excitation but also on the existing field amplitudes, analogous to feedback systems. In the initial state these fields are zero and so the interferometer's field amplitude vector is simply equal to the excitation vector, i.e. $\vec{v}_{\text{AC}} = \vec{v}_{\text{exc}}$. The stored light will increase until eventually the injected excitation is equal to the light power lost in the interferometer. Once this condition is reached the interferometer is in its steady-state, and the matrix of field equations this represents is the required input to the calculation of readout signals.

The steady state condition can be solved numerically using matrix inversion. As described above, the field amplitudes can be described as the current amplitudes plus the input:
\begin{equation}
  \vec{v}_{\text{AC}} = \mathbf{M}_{\text{AC}} \vec{v}_{\text{AC}} + \vec{v}_{\text{exc}},
\end{equation}
where $\mathbf{M}_{\text{AC}}$ is the interferometer matrix specified earlier. This equation can be solved as such:
\begin{equation}
  \vec{v}_{\text{AC}} = \frac{\vec{v}_{\text{exc}}}{1 - \mathbf{M}_{\text{AC}}}.
\end{equation}
Since $\mathbf{M}_{\text{AC}}$ is a matrix and $\vec{v}_{\text{AC}}$ and $\vec{v}_{\text{exc}}$ are vectors, the problem can be represented as the equation:
\begin{equation}
  \vec{v}_{\text{AC}} = \left( \mathbb{I} - \mathbf{M}_{\text{AC}} \right)^{-1} \vec{v}_{\text{exc}},
\end{equation}
where $\mathbb{I}$ is the identity matrix. The calculation of the field amplitudes in the interferometer therefore becomes a task of finding the inverse of $\mathbb{I} - \mathbf{M}_{\text{AC}}$, which is a problem for which many optimised algorithms have been developed.

\subsection{Probe signals}
With the steady-state field amplitudes, the signals produced by the interferometer can be determined with the application of a \emph{probe matrix} $\mathbf{M}_{\text{probe}}$ which maps the fields in the interferometer to probes contained within the interferometer. Since the fields amplitudes are determined for every wavelength under consideration, it is possible to calculate the signals that would appear on photodetector circuits implementing \gls{RF} demodulation. The probe matrix contains complex amplitudes to transform the fields at the location of the probe by the required amount given the demodulation frequencies and phase angles. The probe signals are therefore defined as:
\begin{equation}
  \vec{v}_{\text{probe}} = \mathbf{M}_{\text{probe}} \left( \mathbb{I} - \mathbf{M}_{\text{AC}} \right)^{-1} \vec{v}_{\text{exc}}.
\end{equation}

\subsection{Calculation of transfer functions}
The operation of calculating the probe signals from the field amplitudes in the interferometer can be repeated for arbitrary frequencies of excitation to produce a three-dimensional drive-to-probe transfer matrix. This represents the transfer function from each optic's degree of freedom to each probe. As such, the signal from a particular set of mirror movements can be constructed via a linear combination of the transfer functions representing the degrees of freedom of individual optics. The differential arm degree of freedom transfer function for a \MI{} to its asymmetric port, for instance, can be calculated by extracting the transfer function of each end test mass to a probe situated at the asymmetric port and taking the difference of the two, as shown in Equation\,\ref{eq:darm-length}.

\subsection{Probe quantum noise and sensitivity}
The quantum noise calculations are similar in Finesse and Optickle, and they are both based on the work by Corbitt \etal{} \cite{Corbitt2005}, which ultimately derives from the two-photon formalism by Caves and Schumaker \cite{Caves1985, Schumaker1985}. Quantum noise is calculated using this technique by propagating two photons through the interferometer from each point of noise entry, in much the same way as signals are propagated. One photon represents the amplitude quadrature and the other represents the phase quadrature, and so appropriate quantum noise limited signals can be properly derived for any readout quadrature. A simpler, ``one-photon'' calculation involving the propagation of a single complex number representing the amplitude and phase photons is used in Optickle and Finesse \cite{Evans2013, Brown2016}.

The sensitivity of a probe within an interferometer model is simply the noise on that probe divided by the relevant transfer function. When only quantum noise is considered, the sensitivity is \emph{quantum noise limited}.

\section{\label{sec:optickle-field-tfs}Calculation of field transfer matrices in Optickle}
In the process of determining a probe signal in Optickle, the quadrature sum of the field amplitudes immediately in front of the probe is computed and the phase information contained within these fields is lost. Similarly, transfer functions from drives to probes are provided, but not transfer functions from drives to fields.

In order to calculate the cross-correlation spectral density required for the calculation of the optimal filter in Section\,\ref{sec:optimal-filter}, the complex field and drive transfer matrices, $\mathbf{M}^{\textrm{ff}}$ and $\mathbf{R}$, respectively, must be extracted from Optickle indirectly. Optickle's calculation of the quantum noise at each probe within the interferometer uses field to field and drive to field matrices, but because the quantum noise and drive excitations are not necessarily unity, these matrices are not transfer matrices. In order to obtain $\mathbf{M}^{\textrm{ff}}$ the code which computes the quantum noise at each probe has to be modified to instead inject quantum noise at open ports with unity amplitude. Similarly, $\mathbf{R}$ can be computed by setting the drive amplitudes to unity. The modified source code is publicly available \cite{controlspaperdata}.