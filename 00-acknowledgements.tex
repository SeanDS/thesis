No thesis represents solely the work of its author. The presented results are the culmination of the effort of many individuals, so this is my attempt to give thanks to the brilliant people I've had the pleasure of working with.

\bigskip

To my supervisor Stefan (Hild), for your constant positivity and enthusiasm for all things, and for having faith that this programmer could achieve something useful in the lab. Your broad research interests seem to have rubbed off on me, if this eclectic thesis is anything to go by.

To Ken, for your approachable wisdom and uncanny knack for spotting problems before they arise, and for having patience to teach this boy who failed Circuits and Systems all about electronics.

To Bryan, for all of the help on the front line and for accepting the constant pestering from yet another student (almost like you're used to it). Maybe one day I'll get you those pints.

To Stefan (Danilishin), for the discourse and the cycle trips, and the heroic feat of translating the algebra of interferometers into the verbiage I so desired without resorting to finger puppets.

To the other marvellous postdocs I've had the pleasure of working with: Christian, Sebastian, Conor and Tobin. Thanks for the endless patience and help; I've learned so much from all of you.

Office buddies of 427a past and present, thanks for the absurd chat and the welcome distractions. The same sentiments go to my fellow denizens of the common room coffee club; the jitteringly strong coffee provided the lucid moments of clarity I sometimes needed.

To the rest of the IGR: it has been a pleasure to work in the wonderful environment you all help to create. Keep up the open and relaxed atmosphere!

To the Science and Technology Facilities Council for funding my PhD\footnote{Grant number ST/k502005/01.} and secondment to Hanover, and the Institute of Physics and Leibniz Universit\"{a}t Hannover for generous travel grants.

Mum, Andy, Jill and Jessica: thanks for the constant love and support. I promise I'll stop having fun and get a real job now.

\paragraph{Chapter 1}
Figure\,\ref{fig:gravitational-wave-polarisation} is a reproduction of a figure in Jun Mizuno's thesis. The data for the plot of \GWFIRSTEVENT{} were provided by the \LSC{} through its open science center.

\paragraph{Chapter 3}
The experimental work was conducted in close cooperation with Bryan Barr. Stefan Hild and Ken Strain assisted with conceptual and technical development of the experiment. Neil Gordon, John Macarthur, Angus Bell, Borja Sorazu and Sabina Huttner assisted with construction and measurements. Russell Jones and Steven Craig provided help with the design and manufacturing of the reaction masses. Stephanie Kroker produced the waveguide mirror. Stefan Hild, Chris Messenger and Matt Pitkin assisted with the analysis of the results.

\paragraph{Chapter 4}
The conceptual design of the experiment was developed by Stefan Hild, Christian Gr\"{a}f, Sebastian Steinlechner and Ken Strain. The optical layout for the experiment was developed by Jan Hennig and Roland Schilling.

\paragraph{Chapter 5}
The control design was developed in cooperation with Stefan Danilishin, Andreas Gl\"{a}fke, Stefan Hild, Ken Strain, Bryan Barr, Kentaro Somiya and Harald L\"{u}ck. The definition of degrees of freedom was jointly conducted in cooperation with Andreas Gl\"{a}fke, Christian Gr\"{a}f and Stefan Hild. The quantum noise model was developed with the help of Stefan Danilishin. The state-space ETM suspension model was developed by Ken Strain and Liam Cunningham, and based on the mechanical design developed by Russell Jones. The suspension control scheme was developed in cooperation with Ken Strain. The control software package \emph{SimulinkNb} was developed by Christopher Wipf, who generously provided email support. The necessary modifications to the simulation tool \emph{Optickle} were made with the help of Matt Evans.

\paragraph{Chapter 6}
The ANSYS experiments were conducted by Christian Gr\"{a}f and Christopher Mullen. The PA95-based high voltage amplifier and digital signalling design was made in cooperation with Ken Strain. The whitening filter frequency response was calculated with \emph{LISO} by Gerhard Heinzel. Steven O'Shea, Colin Craig and Nicholas Scott provided assistance with the assembly of the amplifier.

\paragraph{Chapter 7}
The interferometer simulations with \emph{Finesse} were designed in collaboration with Vaishali Adya, Stefan Hild and Harald L\"{u}ck, with technical support from Daniel Brown and Andreas Freise. The simulations with \emph{Optickle} were conducted with the help of Tobin Fricke.

\paragraph{Other credits}
Most of the plots in this thesis were produced with \emph{Python} using the Matplotlib, Numpy and Scipy packages, all provided open source by a generous community of volunteers. Data and simulations were, in places, generated with \emph{Matlab}, from Mathworks. The optical component graphics used in numerous figures throughout this work were provided by Alexander Franzen.