\chapter{\label{c:conclusion}Conclusions and future work}

At their most sensitive frequencies, current generation detectors are limited in sensitivity primarily by quantum and thermal noise. Improvements beyond the level already achieved in \ALIGO{} and soon to be achieved in \AVIRGO{} and \KAGRA{} will require significant research and development, with the ``low hanging fruit'' all but gone. This thesis presented techniques to potentially reduce the thermal and quantum noise for future detector facilities.

In Chapter\,\ref{c:waveguides} we discussed the thermal noise arising from dielectric mirrors used in all previous and current gravitational wave detectors. This is a consequence of the presence of Brownian noise within the many layers of material forming the reflective coatings of the test masses. While reduction of this noise can be made through the development of new coating materials, progress in recent years has been frustrating. We introduced a grating mirror with a resonant waveguide structure as a possible alternative to dielectric coating layers, with calculations showing that the thermal noise these mirrors would produce at cryogenic temperatures would offer a factor \num{10} improvement over the coatings employed in existing detectors. One potential show-stopper with this type of mirror is, however, the potential for phase noise coupling arising from motion of the mirror transverse to the beam axis as seen with previous grating mirrors. The addition in this design of a resonant waveguide is intended to mitigate this phase coupling. We developed an experiment to verify that the phase noise coupling was suppressed. The results showed that, if indeed the coupling is present, the measurement errors place an upper limit on the coupling at one part in \num{17000}. While further experiments will be necessary to fully assess their suitability, these results suggest that waveguide mirrors are a potential option for thermal noise reduction in future detectors.

Chapters \ref{c:speedmeter-intro}, \ref{c:speedmeter-control} and \ref{c:esd-concept} discussed the \emph{\SM{}} topology as a way in which to reduce quantum noise in future gravitational wave detectors. Theory has shown that this design can provide a vast reduction in quantum radiation pressure noise beyond that of ubiquitous position meters. Alongside some theory highlighting the reduced noise in this design, the ongoing proof-of-concept \SSMEXPT{} in Glasgow was introduced in Chapter\,\ref{c:speedmeter-intro} with the goal of the experiment being to demonstrate a reduction in quantum radiation pressure noise over an equivalent \MI{}.

One of the key experimental challenges faced with the \SSM{} is the control of its longitudinal degree of freedom given its lack of sensitivity at low frequencies. Due to the \SM{}'s velocity response, the lack of a signal towards \gls{DC} leads to long term drift from the set point due to seismic motion, creating a significant drop in sensitivity. This problem will only become worse in full-scale detectors utilising this topology. A solution to the control of this drift was presented in Chapter\,\ref{c:speedmeter-control} by blending a displacement signal with the main velocity signal at low frequencies, and a realistic simulation of the full control loop was presented to show that the drift was effectively removed without harming the reduction in radiation pressure noise. The implementation of this control system in the experiment once commissioned will serve as a test for its effectiveness in the correction of long term drifts, informing the technical implementation of detector-scale \SM{}s in the future.

Another aspect of research connected to the \SSM{} was discussed in Chapter\,\ref{c:esd-concept} with regards to the use of a novel electrostatic drive for direct actuation upon the test masses within the experiment. This design has a number of advantages over the approach taken in current detectors, the foremost being the potential for reduced seismic noise coupling to the gravitational wave channel. This chapter outlined the development of the control apparatus for the actuator, with a particular focus on the high voltage electronics. The development of an amplifier with adequate voltage, dynamic range and noise and safety standards was presented, and calculations of the expected maximum displacement noise in the \SSMEXPT{} due to the noise of the electronics showed that the design meets the requirements. Tests with the electronics in-situ will help to characterise this actuator's performance.

The last chapter, \ref{c:et-lf-control}, discussed the challenges faced with the control of the low frequency interferometer, \ETLF{}, as part of the \ET{} detector. This is a proposed facility utilising two colocated \emph{\DRFPMI{}s}, each respectively optomised for low and high frequencies, to be built by a European collaboration in the distant future. The detector will have a longer baseline and will use advanced materials and special techniques to reduce thermal and quantum noise beyond the current generation. The conceptual design for \ETLF{} recommended that its signal recycling cavity be detuned in order to enhance the sensitivity within a particular frequency band, but stopped short of discussing the control strategy. The use of detuning within \DRFPMI{}s has been shown to create control challenges not present with the equivalent tuned varieties, and so the purpose of the work presented in this chapter was to show that the interferometer can indeed be controlled. Using an approach similar to that of the \ALIGO{} and \AVIRGO{} detectors, we showed using numerical simulations that the interferometer is indeed controllable, albeit with the need for a number of control procedures to obtain suitable error signals. Future work will investigate the effect that environmental and technical noise sources will have on the controllability of the interferometer, and the investigate more exotic control approaches.