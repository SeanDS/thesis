No thesis represents solely the work of its author. Whilst every word in this thesis was written by me, the results are the culmination of the effort of many people, and the list below is an attempt to provide credit to as many of them as possible.

\paragraph{Chapter 1}
Figure\,\ref{fig:gravitational-wave-polarisation} is a direct reproduction of a figure in Jun Mizuno's thesis. The plot of the GW150914 signals were provided by the \LSC{}.

\paragraph{Chapter 3}
The experimental work was conducted in close cooperation with Bryan Barr. Stefan Hild and Ken Strain assisted with conceptual and technical development of the experiment. Neil Gordon, John Macarthur, Angus Bell, Borja Sorazu and Sabina Huttner assisted with measurements. Russell Jones and Steven Craig provided help with the design and manufacturing of the reaction masses. Stephanie Kroker produced the waveguide mirror. Stefan Hild, Chris Messenger and Matt Pitkin assisted with the analysis of the results.

\paragraph{Chapter 4}
The conceptual design of the experiment was developed by Christian Gr\"{a}f, Stefan Hild, Sebastian Steinlechner and Ken Strain. The ETM suspension design was developed by Ken Strain, Liam Cunningham and Russell Jones. The auxiliary coil drivers were developed in cooperation with Sebastian Steinlechner and Jan Hennig. The DB50 enclosure was developed by Russell Jones, based on a design from Conor Mow-Lowry. The wiring diagrams are based on a concept by Conor Mow-Lowry and Tobias Westphal.

\paragraph{Chapter 5}
The control design was developed in cooperation with Stefan Danilishin, Andreas Gl\"{a}fke, Stefan Hild, Ken Strain, Bryan Barr, Kentaro Somiya and Harald L\"{u}ck. The definition of degrees of freedom was jointly conducted in cooperation with Andreas Gl\"{a}fke, Christian Gr\"{a}f and Stefan Hild. The quantum noise model was developed with the help of Stefan Danilishin. The state-space suspension model was developed by Ken Strain and Liam Cunningham. The suspension control scheme was developed in cooperation with Ken Strain. The control software package ``SimulinkNb'' was provided by Christopher Wipf, who also provided email support. The modification to the simulation tool ``Optickle'' was conducted with the help of Matt Evans.

\paragraph{Chapter 6}
The experimental design was developed in cooperation with Christian Gr\"{a}f and Jan Hennig. The ANSYS experiments were conducted by Christain Gr\"{a}f and Christopher Mullen. The PA98-based high-voltage amplifier design was provided by Andreas Weidner. The high-voltage power supply was based on a design by Henning Vahlbruch, with modifications made in cooperation with Ken Strain and Sebastian Steinlechner. The PA95-based high voltage design was made in cooperation with Ken Strain. The whitening filter frequency response was produced with ``LISO'' by Gerhard Heinzel. Steven O'Shea, Colin Craig and Nicholas Scott provided assistance with the assembly of the power supply and amplifiers.

\paragraph{Chapter 7}
The interferometer simulations with ``FINESSE'' presented in Chapter\,\ref{c:et-lf-control} were designed in collaboration with Vaishali Adya, Stefan Hild and Harald L\"{u}ck, with technical support from Daniel Brown and Andreas Freise. The simulations with ``Optickle'' were conducted with the help of Tobin Fricke.

\paragraph{Other credits}
Plots in this thesis were produced with the open-source programming language Python using the Matplotlib, Numpy and Scipy packages, all provided free of charge by a generous community of volunteers. Data and simulations were, in places, generated with Matlab, from Mathworks, and LISO, from Gerhard Heinzel. The optical component graphics used in numerous figures throughout this work were provided by Alexander Franzen.