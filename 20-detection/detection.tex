\chapter{Gravitational Waves}
\label{c:gw-detection}

* Section on DOFs, both for a DRMI and angular DOFs (for benefit of alignment appendix)

\section{Gravitational Wave Event GW150914}
\checkme{One billion years ago}, a pair of black holes, one with 36 solar masses and another with 29 solar masses, merged into a single black hole with 62 solar masses. The missing energy equivalent to 3 solar masses was radiated away in the form of gravitational waves.

At 09:50:45 \gls{UTC} on \nth{14} September 2015, gravitational waves from the black hole passed through the LIGO Livingston detector, perturbing the mirrors by \checkme{\SI{e-18}{\meter}} and creating a signal large enough for the electronics controlling the interferometer to detect the ripple in space time more than \checkme{23} times above the noise. Seven milliseconds later, the same wavefront passed the LIGO Hanford detector and moved the mirrors in the opposite direction. At that moment, for the first time in human history, a gravitational wave had been detected.

From the gravitational waveform witnessed at the LIGO sites it was possible to determine the type and parameters of the waves' source. \emph{GW150914}'s waveform can be observed in Figure\,\note{ADD FIGURE}. This waveform, consistent with a binary black hole merger, swept up in frequency (the \emph{inspiral}) before combining (the \emph{merger}) and creating an audible ``chirp''. The signal was only above each detector's noise level from a frequency of around \checkme{x}. Not only did LIGO make the first observation of a gravitational wave, it also made the first detection of a binary black hole system. The window into the universe opening up due to the LIGO\textemdash and the worldwide network of gravitational wave detectors in operation and under assembly, GEO-600, Virgo and KAGRA\textemdash represents a new opportunity to study the universe in a completely different way. Some secrets have already been learned, but there are surely many more to be uncovered. Although this is just the beginning of gravitational astronomy, work is already under way to produce bigger and better gravitational wave detectors both on the Earth and in Space.

\section{Gravitational Waves as Predicted by General Relativity}
* Why do we care about GWs? See \href{https://github.com/rxa254/StatisticalPhysics/blob/master/LectureNotes/ph2c_2015.pdf}{Rana's lecture notes}  (Early universe section might be useful)

Also see ET design study chapter 2

\section{Brief History}
The first observations of the universe were made visually: at night we looked up at the sky and found it to be a dark firmament dotted with bright, twinkling specks; during the day we observed the energy of the Sun. With time scientists discovered multispectral imaging techniques, opening up a vast new spectrum of discovery. Physical processes not visible to the naked eye were now manifested in the spots on photographic plates, signals pushing above the background in electronic instruments and vapour trails in cloud chambers.

Until this year, the two methods with which we could observe the universe were through the use of the electromagnetic spectrum and via particle interactions. The first direct detection of gravitational waves by the LIGO Scientific Collaboration has opened up to us the gravitational universe. Occurrences previously invisible to us, such as the coalescence of binary black hole systems, are now events we can study through the use of gravitational wave detectors.

\subsection{General Relativity}
One hundred years ago, the landscape in the field of physics was dominated by two physical theories: that of universal gravitation, by Newton, and that of electromagnetism, by Maxwell. Although Newton's theory explained with great precision the motion of objects from apples to planets, it did not reconcile the prediction from Maxwell that the speed of light is constant: a prediction hinted experimentally in the late 19th Century by Michelson and Morley. In 1905, Albert Einstein published his Special Theory of Relativity, implying alongside a wide range of other consequences that space and time are linked in a single continuum known as spacetime. The finite speed of light and the existence of spacetime, however, did not accommodate the idea from Newton that gravity was instantaneous attraction between matter across universal distances, nor did it fix the age-old problem with Newtonian mechanics in explaining the apparent precession of the planet Mercury's orbit. Einstein's later work, his General Theory of Relativity, extended Special Relativity to include strong gravitational attraction, and in doing so he solved the problem with Mercury's orbit and predicted the existence of gravitational waves that dictate how matter should move in spacetime.

\subsection{Gravitational Waves}
blah blah blah

* Bar detectors
* First interferometric detectors:
  * Drever, Hough, Glasgow and Munich stuff
  * Glasgow 10m, Caltech 40m, Garching 30m, etc.
  * Formation of LIGO and GEO collaborations
  * Foundational papers (Weiss noise analysis, conference presentations from Schilling and Drever, etc.)
  * Meers paper on signal recycling
  * Initial LIGO, GEO-600, TAMA, Virgo, etc.
  * Enhanced LIGO (and Enhanced Virgo?)

\subsection{Sources}
Gravitational waves from any Earth-bound object, including the Earth itself, are not even remotely detectable. The strain in spacetime produced by such objects is so weak that there is no hope for us to make such a detection with any known technology. A good estimate for the strain produced by a pair of rotating objects is given in \cite{Sathyaprakash2009} as\footnote{Note that the equation in question, (9) in \cite{Sathyaprakash2009}, has been converted here into SI units.}:
\begin{equation}
  \label{eq:happrox}
  h \lesssim \frac{2 G \left( M v^{2} \right)_{\text{nonspherical}}}{c^4 r},
\end{equation}
where $G$ is the gravitational constant, $\left( M v^{2} \right)_{\text{nonspherical}}$ is the kinetic energy associated with the non-spherical parts of the source, $c$ is the speed of light and $r$ is the distance to the source. To get an idea of what the spacetime strain would be for man-made sources, we can consider as in \cite{Sathyaprakash2009} the case of two cars of mass $M = \SI{e3}{\kilo\gram}$ attached to opposite ends of a rod of length $d = \SI{10}{\meter}$, spinning about its centre in a centrifuge at a frequency of $f = \SI{10}{\hertz}$. The tangential velocity of the cars will be around $2 \pi f d \approx \SI{600}{\meter\per\second}$. Placing the detector one wavelength away, and using Equation\,\ref{eq:happrox}, the strain turns out to be around $\SI{4e-43}{}$. To be able to detect such a strain, Advanced LIGO would require an improvement in sensitivity of \SI{20}{} orders of magnitude, which is clearly ludicrous.

It is only the waves produced by the most massive objects in the universe which we have any chance of detecting: black holes, neutron stars and supernovae, amongst others. Even then, gravitational radiation is only produced by \checkme{the presence of a quadropolar moment} within the source, which means that in general only a subset of sources that happen to be in coalescence or contain surface asymmetries will produce waves.

\subsubsection{Compact Binary Coalescence}
As detected by LIGO in 2016...

\note{Show a picture of the waveform, with annotations - it might be available from https://wiki.ligo.org/viewauth/EPO/DiscoveryTalks.}

\subsubsection{Core Collapse}
\note{Look up numbers from that rates paper.}

\subsubsection{Continuous Wave}
\note{Pulsars}

\subsubsection{Stochastic Background}
CMB...

\note{Make a table of the rules of thumb for strains produced by each source, a la Sathya's article.}

\subsection{Source Localisation}
\note{See Section 2.3 of Sathya's Living Review. Don't go into detail about the type of detector, just the position on Earth.}