\chapter{Gravitational Wave Detection}
\label{c:gw-detection}

Blah blah blah

\section{Gravitational Waves as Predicted by General Relativity}
* Why do we care about GWs? See \href{https://github.com/rxa254/StatisticalPhysics/blob/master/LectureNotes/ph2c_2015.pdf}{Rana's lecture notes}  (Early universe section might be useful) 

\section{Brief History}
The first observations of the universe were made visually: at night we looked up at the sky and found it to be a dark firmament dotted with bright, twinkling specks; during the day we observed the energy of the Sun. With time scientists discovered multispectral imaging techniques, opening up a vast new spectrum of discovery. Physical processes not visible to the naked eye were now manifested in the spots on photographic plates, signals pushing above the background in electronic instruments and vapour trails in cloud chambers.

Until this year, the two methods with which we could observe the universe were through the use of the electromagnetic spectrum and via particle interactions. The first direct detection of gravitational waves by the LIGO Scientific Collaboration has opened up to us the gravitational universe. Occurrences previously invisible to us, such as the coalescence of binary black hole systems, are now events we can study through the use of gravitational wave detectors.

\subsection{General Relativity}
One hundred years ago, the landscape in the field of physics was dominated by two physical theories: that of universal gravitation, by Newton, and that of electromagnetism, by Maxwell. Although Newton's theory explained with great precision the motion of objects from apples to planets, it did not reconcile the prediction from Maxwell that the speed of light is constant: a prediction hinted experimentally in the late 19th Century by Michelson and Morley. In 1905, Albert Einstein published his Special Theory of Relativity, implying alongside a wide range of other consequences that space and time are linked in a single continuum known as spacetime. The finite speed of light and the existence of spacetime, however, did not accommodate the idea from Newton that gravity was instantaneous attraction between matter across universal distances. Einstein's later work, his General Theory of Relativity, extended Special Relativity to include strong gravitational attraction, and in doing so he predicted the existence of gravitational waves carrying information of how matter should move in spacetime.

\subsection{Gravitational Waves}
blah blah blah

* Bar detectors
* First interferometric detectors:
  * Drever, Hough, Glasgow and Munich stuff
  * Glasgow 10m, Caltech 40m, Garching 30m, etc.
  * Formation of LIGO and GEO collaborations
  * Foundational papers (Weiss noise analysis, conference presentations from Schilling and Drever, etc.)
  * Meers paper on signal recycling
  * Initial LIGO, GEO-600, TAMA, Virgo, etc.
  * Enhanced LIGO (and Enhanced Virgo?)

\section{Overview of Current Efforts}
* Worldwide network of interferometric detectors
* Plans to build/upgrade more (KAGRA, ET, LIGO Voyager, LIGO CE, etc.)
* Space based detectors

\section{Limiting noise sources in future detectors}
Quantum noise, thermal noise (since if you push QN low enough, you run into thermal), possibly mention others (Newtonian noise - why ET will be under ground, etc.)

\section{Quantum Non-Demolition}
Give general overview of technique, then explain the Sagnac speedmeter in more detail.