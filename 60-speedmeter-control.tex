\chapter{Control of the Sagnac Speedmeter interferometer}
\label{c:speedmeter-control}

\note{Pillage paper}

    * Definition of degrees of freedom, powers, sidebands, etc...
    * Control loop design / noise budget
      * Inclusion of other noises, like seismic, thermal, electronic, etc. Calculation of noise budget.
      * Suspension hierarchical gain (handling ESD / coil ranges) (see labbook posts)
        * Crossover at ~200 Hz, show calculation of how high a frequency we need to control the arms to keep the power in there (see labbook)
      * Photodetector transimpedance
        * Set transimpedance so that signals are ~1 V...?
      * Mixing displacement and velocity signals
      * CDS overall gain setting
        * Set overall gain to not max the actuators on the suspensions, and provide ~1 V on the input to CDS
      * Whitening/dewhitening design (see labbook posts)
        * ADC input noise and effective number of bits (see labbook, Borja's email from 2015-08-28)
        * Set input voltage to CDS to be ~1/10th range, i.e. 1 V.
        * Make sure input noise from ADCs is 1/10th below the signal at all frequencies
    * Dark noise measurements of op-amp over long timescales (explain additional low frequency noise)
    * Note in ``outlook'' section on how we might verify how effective the optimal filter is compared to naive addition or a ``bad'' filter. Perhaps make a TF from an open port? Choose an optimally bad filter?