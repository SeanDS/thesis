\chapter{\label{a:hv-power-supply}HV power supply design}

\emph{Note: the power supply design presented in this section was eventually shown not to work with the amplifier introduced in Section\,\ref{sec:hv-amplifier}. The text in this section exists to discuss why a high voltage supply is necessary, as a reference for how the experiment's power supply was built and to provide the possibility that the design be adapted in the future.}

\section{Outline}

Power op-amps don't provide significant power supply noise rejection so it is necessary to provide filtering to remove unwanted \gls{AC} components on the supply rails. As a particular supply voltage is also desired for amplifier circuits, the obvious solution is to use a \gls{DC} power supply. The most appropriate source of current for such a supply is from the mains, where the \gls{AC} sinusoidal waveform must be converted to \gls{DC}. An \gls{HV} power supply can be as simple as a mains voltage transformer (to change the peak voltage), a bridge rectifier (to remove current reversal) and filter capacitors (to suppress the sinusoidal change in voltage). This approach can be dangerous, however, due to the lack of current limit protection; and as capacitors cannot provide infinite filtering at all frequencies it leads to a compromise between having \gls{AC} components present upon the intended \gls{DC} output or dangerously high energy stored in large capacitors. Instead, a common approach is to use a power \gls{MOSFET} as a regulator. Such a circuit uses the \gls{MOSFET} as a switch to open and close the gate to keep the current flowing through it constant. As this \gls{MOSFET} operates at very high frequency (\SI{}{\mega\hertz}), the device is more than capable of removing much of the \gls{AC} component of the mains supply. Additional smoothing capacitors help to provide further filtering before and after the regulator.

\section{Design}

Commercial \gls{HV} \gls{DC} power supplies are expensive and are typically more suited to high current applications. It was thought that the power supply for the \gls{HV} amplifier could be built from low cost components following the aforementioned \gls{MOSFET} approach, which led to the circuit shown in Figure\,\ref{fig:hv-power-supply}. The zener diodes D3-D8 set the output voltage of the positive and negative rails to be +\SI{360}{\volt} and \SI{-360}{\volt}, respectively. This design also features current limiting circuitry by means of the bipolar junction transistors T3 and T4. These use the voltage drop across \emph{witness} resistors R7 and R8 to clamp the output to the ground rail in the event of a short. If the voltage drop across R7 or R8 reaches \SI{0.7}{\volt}, corresponding to a current of \SI{70}{\milli\ampere}, each respective transistor opens up a low resistance path to ground for the field normally facilitating current flow through the \gls{MOSFET}, effectively halting the output current.

The addition of \glspl{LED} (D11 and D12) allows for visual identification of current output. Each \gls{LED}'s brightness is proportional to the current flow, giving an indication of whether the output is live. In the event that the overcurrent protection is tripped, these \glspl{LED} will extinguish.

\begin{figure}
  \centering
  \includegraphics[width=\columnwidth]{graphics/generated/from-svg/AC-hv-power-supply.pdf}
  \caption[High voltage power supply electronic schematic]{\gls{HV} power supply design. This circuit uses a transformer, bridge rectifier, \gls{MOSFET} regulator, zener diodes and smoothing capacitors. It is an evolution on a single-sided power supply by Henning Valbruch (AEI, Hannover), following the \emph{de-facto} standard design \cite{Horowitz2015}, with the addition of a negative output rail and current limiting circuitry. This design \emph{does not} work with the \gls{HV} amplifier as discussed in Section\,\ref{sec:hv-amplifier}\textemdash see Section\,\ref{sec:hv-psu-amp-measurements} for more details.}
  \label{fig:hv-power-sumpply}
\end{figure}

The power dissipated in each \gls{MOSFET} in certain circumstances can reach \SI{50}{\watt}, so sufficient heat sinks are required. For this design the choice was made to mount L-shaped angle brackets to the enclosure and the circuit board, onto which the \gls{MOSFET} casings were fixed. These brackets were further connected via the enclosure to large metal fins. \note{Photo?}
% 50W claim: see Ken's email, 2016-01-19 11:56. Device can withstand 100mA @ 500V for brief periods.

\section{\label{sec:hv-psu-tests}Tests}

A series of parallel resistors were connected to the output of the completed power supply and the output current was inferred from the voltage across these resistors. High power resistors were necessary in order to test the current limit circuitry: for current exceeding \SI{70}{\milli\ampere} at full output (\SI{720}{\volt} potential difference), the resistance must be approximately \SI{10.3}{\kilo\ohm} and so dissipated power is around \SI{50}{\watt}.

A \SI{680}{\kilo\ohm}, \SI{1}{\watt} resistor was connected to check normal operating conditions, and this was observed to operate at the full output voltage for many minutes with negligible heating of the heat sink. For load \SI{<10.3}{\kilo\ohm}, the voltage output was observed to drop in order to maintain the current limit, as intended. This load was assembled from numerous \SI{25}{\watt} rated \SI{22}{\kilo\ohm} and \SI{50}{\kilo\ohm} resistors connected in parallel to provide \SI{10}{\kilo\ohm} or less. This parallel configuration allows the total power dissipation to be shared to avoid exceeding any one resistor's rating.

\section{Use with amplifier}

\note{Unfortunately, this didn't work with reactive loads...}