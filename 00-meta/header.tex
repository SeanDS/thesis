%%%
% Packages

\usepackage{graphicx}
\usepackage{amsmath}
\usepackage{amssymb}
\usepackage{color}
\usepackage{bookmark}
\usepackage{hyphenat} % for \hyp{} instead of "-".

%%%
% Bibliography

% import package and set some options
\usepackage[minnames=1,maxnames=5,style=numeric-comp,sorting=none,backend=bibtex]{biblatex}

% set path to database
\bibliography{bibliography/bibliography.bib}

%%%
% SI units

% import package
\usepackage[separate-uncertainty=true]{siunitx}

% for nice display of negative symbol in equations
\sisetup{
   detect-mode,
   detect-family,
   detect-inline-family=math,
}

%%%
% List of symbols, definitions, etc.

% import package
\usepackage[acronym,nomain]{glossaries}

% make the list
\makeglossaries

%%%
% superscript th, st, nd, rd, etc.
\usepackage[super]{nth}

%%%
% Hyperlinks
%
% (This should be as close to the end of the package imports as possible - see https://www.tug.org/applications/hyperref/manual.html)

% import package
\usepackage{hyperref}
\hypersetup{
  colorlinks=true,
  linkcolor=blue,
  filecolor=magenta,      
  urlcolor=cyan
}

%%%
% Fix silly LaTeX space/punctuation behaviour with custom commands
%
% The command '\xspace' should be added to the end of custom commands which lack arguments, to make LaTeX put a space after the command's output.
\usepackage{xspace}

%%%
% Smaller text in captions
\usepackage[font=small]{caption}

%%%
% Subcaptions
\usepackage{subcaption}

%%%
% Custom commands

% command to make plural versions of commands (http://tex.stackexchange.com/questions/259287/how-to-define-a-newcommand-that-expands-into-another-newcommand)
% e.g.
% 	\makename[es]{veggie}{potato}
% makes the commands \veggie (=potato) and \veggies (=potatoes)
\newcommand{\makename}[3][s]{%
  \expandafter\newcommand\csname #2\endcsname{#3\xspace}%
  \expandafter\newcommand\csname #2s\endcsname{#3#1\xspace}%
}

% red, italic, bold - for notes
\newcommand{\note}[1]{\textcolor{red}{\emph{[\textbf{Note: }#1]}}}

% blue, normal text - for statements that need fact-checked
\newcommand{\checkme}[1]{\textcolor{blue}{#1}}

% inline code, monospace font
\newcommand{\code}[1]{\texttt{#1}}

% naive with diaeresis
\newcommand{\naive}{na\"ive\xspace}

% et al.
% (and a backslash to prevent TeX thinking a new sentence starts afterwards)
\newcommand{\etal}{et~al.\@}

% Michelson interferometer
\newcommand{\MI}{Michelson interferometer\xspace}

% Fabry-Perot with accent
\newcommand{\FP}{Fabry-P\'{e}rot\xspace}

% Fabry-Perot Michelson interferometer
\newcommand{\FPMI}{Fabry-P\'{e}rot Michelson interferometer\xspace}

% Dual-recycled Fabry-Perot Michelson interferometer
\newcommand{\DRFPMI}{Dual-recycled Fabry-P\'{e}rot Michelson interferometer\xspace}

% Sagnac Speedmeter
\newcommand{\SSM}{Sagnac speedmeter\xspace}

% Sagnac Speedmeter experiment
\newcommand{\SSMEXPT}{Sagnac speedmeter experiment\xspace}

% AEI 10m prototype
\newcommand{\AEIPROTOTYPE}{AEI \SI{10}{\meter} prototype\xspace}

% Glasgow 10m prototype
\newcommand{\GLASGOWTENM}{Glasgow \SI{10}{\meter} prototype\xspace}

% Matlab
\newcommand{\MATLAB}{Matlab\xspace}

% AC
\newcommand{\AC}{ac\xspace}

% ESD/ESDs
\makename[s]{ESD}{ESD}

% MOSFET
\newcommand{\MOSFET}{MOSFET\xspace}

%%%
% Document

\begin{document}

%\pagestyle{empty}
%\pagenumbering{gobble}

\title{Enhancing the Sensitivity of Future Interferometric Gravitational Wave Detectors}
\author{Sean Stephan Leavey}
\date{Month Year}

\maketitle

\cleardoublepage

\vspace*{1.75in}
\begin{center} {\bf Abstract}\end{center}

This is a dissertation outline using the style guidelines defined by
the University of Glasgow. 
%\addcontentsline{toc}{chapter}{Abstract}

\newpage
% Position text on same point of page as the pre-defined Abstract style.
\vspace*{1.75in}

\begin{center} {\bf Acknowledgements}\end{center}

ACK
 
%\addcontentsline{toc}{chapter}{Acknowledgments}

\cleardoublepage
\vspace*{1.75in}
\begin{flushright} Dedication. (Is what you need.)\end{flushright}


%\pagenumbering{roman}
%\setcounter{page}{1}

% -----------------------------------------------------------------------------
\newpage

\vspace*{1.75in}
\begin{center} {\bf Creative Commons Licence}\end{center}
\noindent I hereby grant permission under the following terms...

\begin{description}
\item Sean Stephan Leavey\ \\ \ \\ \ \\
\hrule
\end{description}
\addcontentsline{toc}{chapter}{Licence}
\newpage

% -----------------------------------------------------------------------------

% import glossary terms
%% Glossary/acronym definitions

\newglossaryentry{utc}{name=UTC, description={Coordinated Universal Time}} 

% -----------------------------------------------------------------------------
\newpage
%\thispagestyle{empty}
\tableofcontents
\listoftables
\listoffigures

\addcontentsline{toc}{chapter}{List of Acronyms and Symbols}
\printglossaries

% -----------------------------------------------------------------------------
\cleardoublepage

% Number chapter boomarks
\bookmarksetup{numbered}

\section{Thesis Statement}
\label{c:intro:thesisstatement}

This is a thesis statement. There are many like it, but this one is mine.

